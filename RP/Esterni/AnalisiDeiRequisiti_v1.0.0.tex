\documentclass{scalatekids-article}
\begin{document}
\lfoot{Analisi dei Requisiti 1.0.0}
\newgeometry{top=3.5cm}
\begin{titlepage}
  \begin{center}
    \begin{center}
      \includegraphics[width=10cm]{sklogo.png}
    \end{center}
    \vspace{1cm}
    \begin{Huge}
      \begin{center}
        \textbf{Analisi dei Requisiti}
      \end{center}
    \end{Huge}
    \vspace{11pt}
    \bgroup
    \def\arraystretch{1.3}
    \begin{tabular}{r|l}
      \multicolumn{2}{c}{\textbf{Informazioni sul documento}} \\
      \hline
      \setbox0=\hbox{0.0.1\unskip}\ifdim\wd0=0pt
      \\
      \else
      \textbf{Versione} & 1.0.0\\
      \fi
      \textbf{Redazione} & \multiLineCell[t]{Davide Trevisan\\Marco Boseggia\\Michael Munaro}\\
      \textbf{Verifica} & \multiLineCell[t]{Alberto De Agostini\\Giacomo Vanin}\\
      \textbf{Approvazione} & \multiLineCell[t]{Andrea Giacomo Baldan}\\
      \textbf{Uso} & Esterno\\
      \textbf{Lista di Distribuzione} & \multiLineCell[t]{ScalateKids\\Prof. Tullio Vardanega\\Prof. Riccardo Cardin}\\
    \end{tabular}
    \egroup
    \vspace{22pt}
  \end{center}
\end{titlepage}
\restoregeometry
\clearpage
\pagenumbering{Roman}
\setcounter{page}{1}
\begin{flushleft}
  \vspace{0cm}
         {\large\bfseries Diario delle modifiche \par}
\end{flushleft}
\vspace{0cm}
\begin{center}
  \begin{tabular}{| l | l | l | l | p{5cm} |}
    \hline
    Versione & Autore & Ruolo & Data & Descrizione \\
    \hline
    1.1.0 & Marco Boseggia & Verificatore & 2016-02-23 & Verifica sezione 3.2\\
    \hline
    1.0.1 & Francesco Agostini & Analista & 2016-02-22 & Stesura sezione 3.2\\
    \hline
    1.0.0 & Andrea Giacomo Baldan & Responsabile & 2016-01-20 & Approvazione documento\\
    \hline
    0.4.0 & Giacomo Vanin & Verificatore & 2016-01-19 & Verifica sezione Tracciamento\\
    \hline
    0.3.1 & Marco Boseggia & Analista & 2016-01-18 & Stesura sezione Tracciamento\\
    \hline
    0.3.0 & Alberto De Agostini & Verificatore & 2016-01-16 & Verifica sezione Classificazione Requisiti\\
    \hline
    0.2.3 & Michael Munaro & Analista & 2016-01-15 & Integrazione sezione Classificazione Requisiti\\
    \hline
    0.2.1 & Marco Boseggia & Analista & 2016-01-14 & Stesura sezione Classificazione Requisiti\\
    \hline
    0.2.2 & Davide Trevisan & Analista & 2016-01-14 & Correzione sezione Casi d'uso\\
    \hline
    0.2.0 & Giacomo Vanin & Verificatore & 2016-01-12 & Verifica sezione Casi d'uso\\
    \hline
    0.1.1 & Michael Munaro & Analista & 2016-01-10 & Integrazione sezione Casi d'uso\\
    \hline
    0.1.0 & Alberto De Agostini & Verificatore & 2016-01-07 & Verifica sezioni stese in precedenza\\
    \hline
    0.0.4 & Davide Trevisan & Analista & 2016-01-05 & Integrazione sezione Casi d'uso\\
    \hline
    0.0.3 & Marco Boseggia & Analista & 2016-01-04 & Inizio stesura sezione Casi d'uso\\
    \hline
    0.0.2 & Michael Munaro & Analista & 2016-01-03 & Stesura sezione Descrizione Generale\\
    \hline
    0.0.1 & Andrea Giacomo Baldan & Amministratore & 2015-12-16 & Creazione scheletro del documento\\
    \hline
  \end{tabular}
\end{center}
\tableofcontents
\newpage
\pagenumbering{arabic}

\section{Sommario}

\subsection{Scopo del documento}

Il seguente documento ha lo scopo di presentare le funzionalità che il prodotto
\textbf{Actorbase} esporrà all'utilizzatore finale. Inoltre elenca e descrive i
requisiti derivanti dalle suddette funzionalità, emersi durante le riunioni
interne ed esterne con il \textit{Proponente}.
\prodPurpose{}\glossExpl{}

\subsection{Riferimenti}

\subsubsection{Normativi}

\begin{itemize}
\item\textbf{Capitolato d'appalto C1:} \textit{Actorbase: a NoSQL DB based on the Actor model}\\
  \url{http://www.math.unipd.it/~tullio/IS-1/2015/Progetto/C1.pdf}
\item\textbf{Verbale esterno:}\\
  \href{run:../RR/Interni/VerbaleEsterno20160112\_v1.0.0.pdf}{Verbale Esterno 20160112 v1.0.0}\\
  \href{run:../RR/Interni/VerbaleEsterno20160119\_v1.0.0.pdf}{Verbale Esterno 20160119 v1.0.0}
\item\textbf{Norme di Progetto:}\\
  \href{run:../Interni/NormeDiProgetto\_v1.0.0.pdf}{Norme di Progetto v1.0.0}
\end{itemize}

\subsubsection{Informativi}

\begin{itemize}
\item\textbf{Dispense fornite dall'insegnamento Ingegneria del Software mod. A:}\\
  \url{http://www.math.unipd.it/~tullio/IS-1/2015/Dispense/L06.pdf}
\item\textbf{Dispense fornite dall'insegnamento Ingegneria del Software mod. A:}\\
  \url{http://www.math.unipd.it/~tullio/IS-1/2015/Dispense/E02.pdf}
\item\textbf{CAP theorem:}\\
  \url{https://en.wikipedia.org/wiki/CAP_theorem}
\item\textbf{Reactive Manifesto:}\\
  \url{http://www.reactivemanifesto.org/}\\
  \url{https://en.wikipedia.org/wiki/Reactive_programming}
\item\textbf{Amazon DynamoDB:}\\
  \url{http://docs.aws.amazon.com/amazondynamodb/latest/developerguide/Introduction.html}
\item\textbf{Leader election:}\\
  \url{https://en.wikipedia.org/wiki/Leader_election}
\end{itemize}

\section{Descrizione generale}

\subsection{Prospettive del prodotto}

l'obiettivo del prodotto è fornire un \gloss{database} \gloss{NoSQL} di tipo
\gloss{key-value}, quindi senza schemi predefiniti e tabelle; che utilizzi il
modello ad \gloss{attori} per garantire un alto grado di \gloss{concorrenza} e
\gloss{scalabilità orizzontale} idealmente illimitata.

\subsection{Funzioni}

Il software fornirà un sistema di interazione con l'utente basata su \gloss{UI}
testuale direttamente da riga di comando. Questa permetterà di effettuare le operazioni
basilari che ogni \gloss{database} fornisce:
\begin{itemize}
\item Creazione di una o più \gloss{collezioni};
\item Cancellazione di una o più \gloss{collezioni};
\item Modifica del nome di una \gloss{collezione};
\item Ricerca all'interno del \gloss{database} o all'interno di una o più \gloss{collezioni};
\item Inserimento \gloss{item} con o senza sovrascrittura;
\item Cancellazione \gloss{item};
\item Connessione ad altre istanze \textbf{Actorbase}.
\end{itemize}
Il programma inoltre permetterà la richiesta di un aiuto esplicativo sull'uso
dei comandi.

\subsection{Caratteristiche utenza}

Il prodotto è orientato all'utilizzo da parte di clientela interessata a
sviluppo di applicazioni \gloss{reactive}, che trattino grandi moli di dati e
debbano fornire brevissimi tempi di risposta sacrificando dunque le
funzionalità relazionali tipiche dei tradizionali \gloss{database} \textit{SQL} in
favore di un sistema fortemente \gloss{concorrente} e senza l'\gloss{overhead} generato
dagli schemi \gloss{SQL}.

\subsection{Vincoli}

Per far funzionare \textbf{Actorbase} sarà necessario disporre di un computer con
installata la \textit{\gloss{JVM} versione 8}. Non ci sono ulteriori richieste hardware o software.

\section{Casi d'uso}

Le aspettative di esperienza utente derivano dall'utilizzo da parte dei
componenti del gruppo di \gloss{Amazon DynamoDB}, un \gloss{database} \gloss{key-value}
sviluppato da \textit{Amazon}, utilizzato come modello di riferimento per lo sviluppo. I casi d'uso seguono le norme di stesura elencate nel documento \href{run:../Interni/NormeDiProgetto\_v1.0.0.pdf}{Norme di Progetto v1.0.0}.

\subsection{Identificazione attori}

L'analisi del capitolato e le riunioni con il \textit{Proponente} hanno fatto emergere
alcune considerazioni sull'obiettivo che \textbf{Actorbase} si pone di raggiungere.
Trattandosi di un'applicazione distribuita il prodotto finale sarà costituito
da due \gloss{macro} componenti: un lato server e un lato client fornito di interfaccia
comandi testuale per poter dialogare con la base di dati.\\ In particolare la
scalabilità che il prodotto deve fornire ha portato alla suddivisione degli
attori primari nel modo seguente:\\
\textbf{Primari}
\begin{itemize}
\item\textbf{Utente non autenticato:}\\
  Rappresenta l'utente generico che avvia la \gloss{CLI} (Command Line Interface) di \textbf{Actorbase} ma non è ancora connesso al lato server del prodotto;
\item\textbf{Utente autenticato:}\\
  Rappresenta l'utente generico connesso al lato server del prodotto;
\item\textbf{Amministratore:}\\
  Rappresenta un superutente con privilegi di \gloss{visibilità} maggiori rispetto a utenti generici e poteri di modifica sulla \gloss{visibilità} di questi ultimi.
\end{itemize}
\textbf{Secondari}
\begin{itemize}
\item\textbf{Terminale:}\\
  Il sistema riceverà comandi mendiante l'utilizzo di una \gloss{CLI}, pertanto il terminale rappresenta l'attore necessario ai primari per raggiugere lo scopo prefissato, ovvero
  l'inserimento dei comandi da inviare al sistema;
\item\textbf{Driver:}\\
  Nel caso di utilizzo del sistema da un programma esterno in linguaggio Scala, il driver rappresenta effettivamente l'attore necessario agli utenti (attori primari) per raggiugere
  lo scopo prefissato, ovvero l'inserimento dei comandi da inviare al sistema.
\end{itemize}

\subsubsection{Visibilità}

Rispetto al classico sistema di permessi di lettura e scrittura, \textbf{Actorbase}
prevede un meccanismo semplificato di \gloss{visibilità} secondo le seguenti regole:
\begin{itemize}
\item Ogni utente autenticato inizialmente ha \gloss{visibilità} solo sulle proprie \gloss{collezioni} e/o su \gloss{collezioni} di proprietà altrui su cui è in collaborazione;
\item L'utente amministratore ha \gloss{visibilità} su tutto il sistema \textbf{Actorbase}.
\end{itemize}
La \gloss{visibilità} garantisce privilegi sia in lettura che in scrittura. Di
conseguenza tutte le azioni descritte nei casi d'uso si applicano al livello di
\gloss{visibilità} proprio dell'attore che le effettua.

\subsubsection{Scalabilità orizzontale}

Il vantaggio più grande che il prodotto dovrà offrire sarà la possibilità di
\gloss{scale out} semplificata al massimo. Dovrà quindi essere possibile
aggiungere potenza di calcolo e capacità di carico mediante affiancamento di
macchine aggiuntive. I dati contenuti all'interno della base di dati originaria
dovranno essere redistribuiti e bilanciati tra i nuovi nodi inseriti. In questo
modo è idealmente possibile ottenere infinita capacità di carico semplicemente
aggiungendo risorse hardware qualora necessario.

\subsection{Visione generale}

L'immagine che segue rappresenta uno schema della visione d'insieme del funzionamento di Actorbase. Al suo interno è possibile identificare
l'utente che, mediante CLI e accesso alla rete, è in grado di interagire con la rete Actorbase. Essa è formata da un sistema di database distribuiti la cui distribuzione è assolutamente trasparente all'utente. Ogni elemento del sistema è in comunicazione con tutte le altre istanze con cui è in grado di interagire in maniera autonoma.
\begin{figure}[H]
  \begin{center}
    \includegraphics[width=0.7\textwidth,keepaspectratio]{UseCases/ActorbaseNetwork.png}
    \caption*{Actorbase: Visione generale semplificata rete Actorbase}
  \end{center}
\end{figure}
\subsection{Visione generale delle interazioni tramite \gloss{CLI}}
\begin{figure}[H]
  \begin{center}
    \includegraphics[width=0.7\textwidth,keepaspectratio]{UseCases/CLI-Generale.png}
    \caption*{Visione generale - Interazioni tramite \gloss{CLI}}
  \end{center}
\end{figure}

\subsection{UC1: Login}

\begin{figure}[H]
  \begin{center}
    \includegraphics[width=0.7\textwidth,keepaspectratio]{UseCases/UC1.png}
    \caption*{Caso d'uso 1: Login}
  \end{center}
\end{figure}
\textbf{Attori primari:} Utente non autenticato\\ \\
\textbf{Attori secondari:} Terminale\\ \\
\textbf{Scopo e descrizione:}
L’utente ha appena avviato la \gloss{CLI}, risulta essere non autenticato e può effettuare il login diventando utente
autenticato.\\ \\
\textbf{Precondizione:} Il server è in ascolto.\\ \\
\textbf{Scenario principale:}
\begin{itemize}
\item UC1.1 Inserimento username;
\item UC1.2 Inserimento password.
\end{itemize}
\textbf{Estensioni:}
\begin{itemize}
\item Nel caso in cui l'utente inserisca una coppia username-password non riconosciute dal sistema:
  \begin{itemize}
  \item UC9 Visualizzazione Errore credenziali errate.
  \end{itemize}
\end{itemize}
\textbf{Postcondizioni:} L'utente ha acceduto al sistema con successo e risulta essere autenticato.

\subsubsection{UC1.1: Inserimento username}

\textbf{Attori primari:} Utente non autenticato\\ \\
\textbf{Attori secondari:} Terminale\\ \\
\textbf{Scopo e descrizione:}
L'utente ha appena avviato la \gloss{CLI}, risulta essere non autenticato e può inserire il proprio username per effettuare il login.\\ \\
\textbf{Precondizioni:} Il server è in ascolto, l'utente intende effettuare il login.
\textbf{Scenario principale:}
\begin{itemize}
\item L'utente può inserire il proprio username.
\end{itemize}
\textbf{Postcondizioni:} L'utente ha inserito il proprio username.

\subsubsection{UC1.2: Inserimento password}

\textbf{Attori primari:} Utente non autenticato\\ \\
\textbf{Attori secondari:} Terminale\\ \\
\textbf{Scopo e descrizione:}
L'utente ha appena avviato la \gloss{CLI}, risulta essere non autenticato e può inserire la propria password per effettuare il login.\\ \\
\textbf{Precondizioni:} Il server è in ascolto, l'utente intende effettuare il login.
\textbf{Scenario principale:}
\begin{itemize}
\item L'utente può inserire la propria password.
\end{itemize}
\textbf{Postcondizioni:} L'utente ha inserito la propria password.

\subsection{UC2: Richiesta aiuto}

\textbf{Attori primari:} Utente non autenticato, Utente autenticato, Utente amministratore\\ \\
\textbf{Attore secondario:} Terminale\\ \\
\textbf{Scopo e descrizione:} L'utente intende richiedere aiuto sulla piattaforma o su un comando specifico.\\Nel caso di richiesta di aiuto generale diversi tipi di utente riceveranno la lista dei comandi da loro eseguibili\\ \\
\textbf{Precondizione:} L'utente intende richiedere supporto.\\ \\
\textbf{Scenario principale:}
\begin{itemize}
\item UC2.1 Inserimento comando per richiesta di aiuto generale;
\item UC2.2 Inserimento comando per richiesta di aiuto specifico.
\end{itemize}
\textbf{Postcondizioni:} Il sistema ha risposto producendo in output l'aiuto richiesto.

\subsection{UC2.1: Richiesta di aiuto generale}

\textbf{Attore primario:} Utente\\ \\
\textbf{Attore secondario:} Terminale\\ \\
\textbf{Scopo e descrizione:} L'utente intende richiedere aiuto generale sull'utilizzo del sistema \textbf{Actorbase}.\\ \\
\textbf{Precondizione:} L'utente intende richiedere supporto generale.\\ \\
\textbf{Scenario principale:}
\begin{itemize}
\item Inserimento comando per richiesta di aiuto.
\end{itemize}
\textbf{Postcondizioni:} Il sistema ha risposto visualizzando l'aiuto richiesto dall'utente in output.

\subsection{UC2.2: Richiesta di aiuto specifico}

\textbf{Attore primario:} Utente\\ \\
\textbf{Attore secondario:} Terminale\\ \\
\textbf{Scopo e descrizione:} L'utente intende richiedere aiuto su uno specifico comando del sistema \textbf{Actorbase}.\\ \\
\textbf{Precondizione:} L'utente intende richiedere supporto generale.\\ \\
\textbf{Scenario principale:}
\begin{itemize}
\item inserimento comando per richiesta di aiuto;
\item Inserimento nome comando.
\end{itemize}
\textbf{Postcondizioni:} Il sistema ha risposto visualizzando l'aiuto richiesto dall'utente in output.

\subsection{UC3: Operazioni sulle collezioni}

\begin{figure}[H]
  \begin{center}
    \includegraphics[width=0.7\textwidth,keepaspectratio]{UseCases/UC3.png}
    \caption*{Caso d'uso 3: Operazioni sulle \gloss{collezioni}}
  \end{center}
\end{figure}
\textbf{Attore primario:} Utente autenticato\\ \\
\textbf{Attore secondario:} Terminale\\ \\
\textbf{Scopo e descrizione:} L'utente è autenticato e desidera effettuare delle operazioni sulle \gloss{collezioni}. Le operazioni previste sono:
creazione di una nuova \gloss{collezione}, visualizzazione della lista delle \gloss{collezioni} esistenti, cancellazione di \gloss{collezioni} esistenti, modifica del nome di \gloss{collezioni} esistenti e
aggiunta o rimozione collaboratori a \gloss{collezione} esistente.\\ \\
\textbf{Precodizione:} L'utente intende effettuare operazioni su \gloss{collezioni}.\\ \\
\textbf{Scenario principale:}
\begin{itemize}
\item UC3.1 Creazione \gloss{collezione};
\item UC3.2 Visualizzazione della lista delle \gloss{collezioni};
\item UC3.3 Cancellazione \gloss{collezione};
\item UC3.4 Modifica nome \gloss{collezione};
\item UC3.5 Aggiunta \gloss{collaboratore};
\item UC3.6 Rimozione \gloss{collaboratore};
\item UC3.7 Esportazione su file.
\end{itemize}
\textbf{Estensioni:}
\begin{itemize}
\item Nel caso in cui nelle procedure di creazione collezione o modifica nome collezione, l'utente inserisca il nome di una collezione già censita all'iterno del sistema:
  \begin{itemize}
  \item UC3.7 Visualizzazione errore \gloss{collezione} già esistente.
  \end{itemize}
\item Nel caso in cui nelle procedure di modifica, cancellazione collezione, aggiunta collaboratore a collezione o rimozione collaboratore a collezione, l'utente inserisca il nome di una collezione non censita all'interno del sistema:
  \begin{itemize}
  \item UC3.8 Visualizzazione errore collezione inesistente.
  \end{itemize}
\item Nel caso in cui nelle procedure di aggiunta o rimozione collaboratore a collezione l'utente inserisca uno username non censito all'interno del sistema:
  \begin{itemize}
  \item UC3.9 Visualizzazione errore username inesistente.
  \end{itemize}
\end{itemize}
\textbf{Postcondizioni:} L'operazione richiesta è stata eseguita con successo.

\subsection{UC3.1: Creazione collezione}

\textbf{Attore primario:} Utente autenticato\\ \\
\textbf{Attore secondario:} Terminale\\ \\
\textbf{Scopo e descrizione:} L'utente richiede la creazione di una \gloss{collezione} all'interno del \gloss{database}, deve inserire il nome della nuova \gloss{collezione}.\\ \\
\textbf{Precodizione:} Il \gloss{database} è pronto a ricevere comandi e l'utente intende creare una \gloss{collezione}.\\ \\
\textbf{Scenario principale:}
\begin{itemize}
\item L'utente può inserire il nome \gloss{collezione}
\end{itemize}
\textbf{Postcondizioni:} La \gloss{collezione} è stata creata.

\subsection{UC3.2: Visualizzazione della lista delle collezioni}

\textbf{Attore primario:} Utente autenticato\\ \\
\textbf{Attore secondario:} Terminale\\ \\
\textbf{Scopo e descrizione:} L'utente intende elencare tutte le \gloss{collezioni} all'interno di \textbf{Actorbase} secondo la propria \gloss{visibilità}.\\ \\
\textbf{Precodizione:} Il \gloss{database} è pronto a ricevere comandi e l'utente intende elencare le \gloss{collezioni} al suo interno.\\ \\
\textbf{Scenario principale:}
\begin{itemize}
\item Inserimento nome collezione.
\end{itemize}
\textbf{Postcondizioni:} Il \gloss{database} restituisce la lista delle \gloss{collezioni} esistenti secondo la \gloss{visibilità} dell'utente.

\subsection{UC3.3: Modifica nome collezione}

\begin{figure}[H]
  \begin{center}
    \includegraphics[width=0.7\textwidth,keepaspectratio]{UseCases/UC3_3.png}
    \caption*{Caso d'uso 1: Login}
  \end{center}
\end{figure}
\textbf{Attore primario:} Utente autenticato\\ \\
\textbf{Attore secondario:} Terminale\\ \\
\textbf{Scopo e descrizione:} L’utente intende modificare il nome di una \gloss{collezione} presente nel \gloss{database}.\\ \\
\textbf{Precodizione:} Il server è in ascolto e l’utente intende modificare il nome di una \gloss{collezione}.\\ \\
\textbf{Scenario principale:}
\begin{itemize}
\item UC3.3.1 Inserimento nome \gloss{collezione};
\item UC3.3.2 Inserimento nuovo nome \gloss{collezione}.
\end{itemize}
\textbf{Estensioni:}
\begin{itemize}
\item Nel caso in cui l'utente inserisca come nuovo nome per la collezione da modificare, un nome di una collezione già esistente all'interno del sistema:
  \begin{itemize}
  \item UC3.7 Visualizzazione errore collezione già esistente.
  \end{itemize}
\item Nel caso in cui l'utente inserisca come nome della collezione da modificare, un nome di una collezione inesistente all'interno del sistema:
  \begin{itemize}
  \item UC3.8 Visualizzazione errore collezione inesistente.
  \end{itemize}
\end{itemize}
\textbf{Postcondizioni:} Il nome della \gloss{collezione} è stato modificato correttamente.

\subsubsection{UC3.3.1: Inserimento nome collezione}

\textbf{Attore primario:} Utente autenticato\\ \\
\textbf{Attore secondario:} Terminale\\ \\
\textbf{Scopo e descrizione:} L'utente intende modificare il nome di un \gloss{collezione} presente nel \gloss{database}, deve inserire il nome della collezione da modificare.\\ \\
\textbf{Precondzione:} Il server è in ascolto e l'utente intende modificare il nome di una \gloss{collezione}, deve inserire il nome della collezione da modificare.\\ \\
\textbf{Scenario principale:}
\begin{itemize}
\item L'utente può inserire il nome della collezione da modificare.
\end{itemize}
\textbf{Postcondizioni:} Il nome della \gloss{collezione} da modificare è stato inserito

\subsubsection{UC3.3.2: Inserimento nuovo nome collezione}

\textbf{Attore primario:} Utente autenticato\\ \\
\textbf{Attore secondario:} Terminale\\ \\
\textbf{Scopo e descrizione:} L'utente intende modificare il nome di un \gloss{collezione} presente nel \gloss{database}, deve inserire il nuovo nome della collezione da modificare.\\ \\
\textbf{Precondzione:} Il server è in ascolto e l'utente intende modificare il nome di una \gloss{collezione}, deve inserire il nuovo nome della collezione da modificare.\\ \\
\textbf{Scenario principale:}
\begin{itemize}
\item L'utente può inserire il nuovo nome della collezione da modificare.
\end{itemize}
\textbf{Postcondizioni:} Il nuovo nome della \gloss{collezione} da modificare è stato inserito

\subsection{UC3.4: Cancellazione collezione}

\textbf{Attore primario:} Utente autenticato\\ \\
\textbf{Attore secondario:} Terminale\\ \\
\textbf{Scopo e descrizione:} L’utente intende cancellare una \gloss{collezione} presente nel \gloss{database}.\\ \\
\textbf{Precodizione:} Il server è in ascolto e l’utente intende cancellare una \gloss{collezione}.\\ \\
\textbf{Scenario principale:}
\begin{itemize}
\item UC3.4.1 Inserimento nome \gloss{collezione};
\end{itemize}
\textbf{Estensioni:}
\begin{itemize}
\item Nel caso in cui l'utente inserisca il nome di una collezione da cancellare inesistente:
  \begin{itemize}
  \item UC3.8 Visualizzazione errore \gloss{collezione} inesistente;
  \end{itemize}
\end{itemize}
\textbf{Postcondizioni:} La \gloss{collezione} è stato cancellata correttamente.

\subsubsection{UC3.4.1: Inserimento nome collezione}

\textbf{Attore primario:} Utente autenticato\\ \\
\textbf{Attore secondario:} Terminale\\ \\
\textbf{Scopo e descrizione:} L'utente intende cancellare il nome di un \gloss{collezione} presente nel \gloss{database}, deve inserire il nome della collezione da cancellare.\\ \\
\textbf{Precondzione:} Il server è in ascolto e l'utente intende cancellare il nome di una \gloss{collezione}, deve inserire il nome della collezione da cancellare.\\ \\
\textbf{Scenario principale:}
\begin{itemize}
\item L'utente può inserire il nome della collezione da cancellare.
\end{itemize}
\textbf{Postcondizioni:} Il nome della \gloss{collezione} da cancellare è stato inserito

\subsection{UC3.5: Aggiunta di un collaboratore a collezione}

\begin{figure}[H]
  \begin{center}
    \includegraphics[width=0.7\textwidth,keepaspectratio]{UseCases/UC3_5.png}
    \caption*{Caso d'uso 1: Login}
  \end{center}
\end{figure}
\textbf{Attore primario:} Utente autenticato\\ \\
\textbf{Attore secondario:} Terminale\\ \\
\textbf{Scopo e descrizione:} L'utente intende aggiungere un \gloss{collaboratore} ad una \gloss{collezione} di sua proprietà. Deve inserire il nome della \gloss{collezione} e lo username dell'utente da aggiungere.\\ \\
\textbf{Precondizione:} Il \gloss{database} è pronto a ricevere comandi e l'utente intende aggiungere un \gloss{collaboratore} ad una \gloss{collezione} di sua proprietà.\\ \\
\textbf{Scenario principale:}
\begin{itemize}
\item UC3.5.1 Inserimento nome \gloss{collezione};
\item UC3.5.2 Inserimento username \gloss{collaboratore}.
\end{itemize}
\textbf{Postcondizione:} L'utente designato per la collaborazione è stato aggiunto tra i \gloss{collaboratori} della \gloss{collezione} scelta.

\subsection{UC3.6: Rimozione di un collaboratore a collezione}

\begin{figure}[H]
  \begin{center}
    \includegraphics[width=0.7\textwidth,keepaspectratio]{UseCases/UC3_6.png}
    \caption*{Caso d'uso 1: Login}
  \end{center}
\end{figure}
\textbf{Attore primario:} Utente autenticato\\ \\
\textbf{Scopo e descrizione:} L'utente intende rimuovere un \gloss{collaboratore} da una \gloss{collezione} di sua proprietà. Deve inserire il nome della \gloss{collezione} e lo username del \gloss{collaboratore} da rimuovere.\\ \\
\textbf{Precondizione:} Il \gloss{database} è pronto a ricevere comandi e l'utente intende rimuovere un \gloss{collaboratore} da una \gloss{collezione} di sua proprietà.\\ \\
\textbf{Scenario principale:}
\begin{itemize}
\item Inserimento nome \gloss{collezione};
\item Inserimento username \gloss{collaboratore}.
\end{itemize}
\textbf{Postcondizione:} L'utente designato per la rimozione dai \gloss{collaboratori} è stato rimosso dalla \gloss{collezione} scelta.

\subsection{UC3.7: Esportazione su file}

\subsection{UC3.7.1: Esportazione una o più collezione}

\subsection{UC3.7.2: Esportazione di tutto il database}

\subsection{UC4: Operazioni sugli item}

\begin{figure}[H]
  \begin{center}
    \includegraphics[width=0.7\textwidth,keepaspectratio]{UseCases/UC4.png}
    \caption*{Caso d'uso 4: Operazioni sugli \gloss{item}}
  \end{center}
\end{figure}
\textbf{Attore primario:} Utente autenticato\\ \\
\textbf{Scopo e descrizione:} L'utente intende effettuare operazioni sugli \gloss{item} di una \gloss{collezione} esistente. Le operazioni previste sono:
inserimento e cancellazione.\\ \\
\textbf{Precondizione:} Il \gloss{database} è pronto a ricevere comandi e l'utente intende effettuare operazioni di inserimento o cancellazione su una \gloss{collezione} esistente.\\ \\
\textbf{Scenario principale:}
\begin{itemize}
\item UC1.5.1 Inserimento nuovo \gloss{item};
\item UC1.5.2 Cancellazione \gloss{item}.
\end{itemize}
\textbf{Estensioni:}
\begin{itemize}
\item Nel caso in cui l'utente utilizzi un comando sconosciuto:
  \begin{itemize}
  \item viene visualizzato un messaggio di errore esplicativo (e rimando all'aiuto).
  \end{itemize}
\item Nel caso in cui l'utente utilizzi un comando conosciuto ma con parametri errati:
  \begin{itemize}
  \item viene visualizzato un messaggio di supporto che indica come usare correttamente il comando.
  \end{itemize}
\item Nel caso in cui la \gloss{collezione} su cui effettuare inserimento o cancellazione non sia presente:
  \begin{itemize}
  \item viene visualizzato un messaggio di errore esplicativo.
  \end{itemize}
\item Nel caso in cui l'\gloss{item} da rimuovere non sia presente:
  \begin{itemize}
  \item viene visualizzato un messaggio di errore esplicativo.
  \end{itemize}
\end{itemize}
\textbf{Postcondizione:} L'operazione richiesta è stata effettuata correttamente.

\subsection{UC4.1: Inserimento item}

\textbf{Attore primario:} Utente autenticato\\ \\
\textbf{Scopo e descrizione:} L'utente intende inserire un nuovo \gloss{item} all'interno di una \gloss{collezione} esistente. Deve selezionare la \gloss{collezione} e inserire i parametri di definizione dell'\gloss{item}. Questi sono la chiave e il valore associato.\\ \\
\textbf{Precondizione:} Il \gloss{database} è pronto a ricevere comandi e l'utente intende effettuare un inserimento di un nuovo \gloss{item}.\\ \\
\textbf{Scenario principale:}
\begin{itemize}
\item UC4.1.1 Inserimento item manuale;
\item UC4.1.2 Inserimento item da file.
\end{itemize}
\textbf{Postcondizione:} L'operazione di inserimento richiesta è stata effettuata correttamente.

\subsection{UC4.1.1: Inserimento item manuale}

\begin{figure}[H]
  \begin{center}
    \includegraphics[width=0.7\textwidth,keepaspectratio]{UseCases/UC4_1_1.png}
    \caption*{Caso d'uso 4: Operazioni sugli \gloss{item}}
  \end{center}
\end{figure}

\subsection{UC4.1.2: Inserimento item da file}

\begin{figure}[H]
  \begin{center}
    \includegraphics[width=0.7\textwidth,keepaspectratio]{UseCases/UC4_1_2.png}
    \caption*{Caso d'uso 4: Operazioni sugli \gloss{item}}
  \end{center}
\end{figure}

\subsubsection{UC4.1.2.1: Inserimento path file}

\subsection{UC4.1.1.1: Inserimento nome collezione}

\subsection{UC4.1.1.2: Inserimento chiave}

\subsection{UC4.1.1.3: Inserimento valore}

\subsection{UC4.1.1.4: Inserimento flag sovrascrittura}

\subsection{UC4.2: Cancellazione item}

\begin{figure}[H]
  \begin{center}
    \includegraphics[width=0.7\textwidth,keepaspectratio]{UseCases/UC4_2.png}
    \caption*{Caso d'uso 4: Operazioni sugli \gloss{item}}
  \end{center}
\end{figure}

\subsection{UC4.2.1: Inserimento nome collezione}

\subsection{UC4.2.2: Inserimento chiave item}

\subsection{UC5: Ricerca}

\subsection{UC5.1: Ricerca su una o più collezioni}

\subsection{UC5.2: Ricerca su intero database}

\subsection{UC6: Modifica password}

\begin{figure}[H]
  \begin{center}
    \includegraphics[width=0.7\textwidth,keepaspectratio]{UseCases/UC6.png}
    \caption*{Caso d'uso 4: Operazioni sugli \gloss{item}}
  \end{center}
\end{figure}

\subsection{UC6.1: Inserimento vecchia password}

\subsection{UC6.2: Inserimento nuova password}

\subsection{UC6.3: Conferma password}

\subsection{UC7: Logout}

\subsection{UC8: Gestione utenti}

\begin{figure}[H]
  \begin{center}
    \includegraphics[width=0.7\textwidth,keepaspectratio]{UseCases/UC8.png}
    \caption*{Caso d'uso 4: Operazioni sugli \gloss{item}}
  \end{center}
\end{figure}

\subsection{UC8.1: Aggiunta utente}

\begin{figure}[H]
  \begin{center}
    \includegraphics[width=0.7\textwidth,keepaspectratio]{UseCases/UC8_1.png}
    \caption*{Caso d'uso 4: Operazioni sugli \gloss{item}}
  \end{center}
\end{figure}

\subsection{UC8.1.1: Inserimento username}

\subsection{UC8.1.2: Conferma aggiunta utente}

\subsection{UC8.2: Rimozione utente}

\begin{figure}[H]
  \begin{center}
    \includegraphics[width=0.7\textwidth,keepaspectratio]{UseCases/UC8_2.png}
    \caption*{Caso d'uso 4: Operazioni sugli \gloss{item}}
  \end{center}
\end{figure}

\subsection{UC8.2.1: Inserimento username}

\subsection{UC8.2.2: Conferma rimozione utente}

\subsection{UC8.3: Reset password}

\begin{figure}[H]
  \begin{center}
    \includegraphics[width=0.7\textwidth,keepaspectratio]{UseCases/UC8_3.png}
    \caption*{Caso d'uso 4: Operazioni sugli \gloss{item}}
  \end{center}
\end{figure}

\subsection{UC8.3.1: Inserimento username}

\subsection{UC8.3.2: Conferma reset}

\subsubsection{UC9: Visualizzazione Errore credenziali errate}

\textbf{Attori primari:} Utente non autenticato\\ \\
\textbf{Attori secondari:} Terminale\\ \\
\textbf{Scopo e descrizione:}
L'utente ha appena avviato la \gloss{CLI}, risulta essere non autenticato e ha già inserito il proprio username e la propria password per l'autenticazione all'interno del sistema.\\ \\
\textbf{Precondizioni:} Il server è in ascolto, l'utente ha inserito le proprie credenziali (username e password) e intende effettuare il login. Il sistema non riconosce le credenziali.
\textbf{Scenario principale:}
\begin{itemize}
\item Viene visualizzato un messaggio di errore esplicativo.
\end{itemize}
\textbf{Postcondizioni:} Il login fallisce, l'utente risulta essere non autenticato.

\subsection{UC: Interazioni generali Driver}
\begin{figure}[H]
  \begin{center}
    \includegraphics[width=0.7\textwidth,keepaspectratio]{UseCases/interazioni-generali-driver.png}
    \caption*{Caso d'uso X: Interazioni generali driver}
  \end{center}
\end{figure}
\textbf{Attori primari:} Utente non autenticato, Utente autenticato, Utente amministratore\\ \\
\textbf{Attori secondari:} Driver\\ \\
\textbf{Scopo e descrizione:} \\ \\%TODO
\textbf{Precondizione:} Il server è in ascolto.\\ \\
%TODO finire sto usecase

\subsection{UCX: Login}

\begin{figure}[H]
  \begin{center}
    \includegraphics[width=0.7\textwidth,keepaspectratio]{UseCases/UC10.png}
    \caption*{Caso d'uso X: Login}
  \end{center}
\end{figure}
\textbf{Attori primari:} Utente non autenticato\\ \\
\textbf{Attori secondari:} Terminale\\ \\
\textbf{Scopo e descrizione:}
L'utente intende effettuare login al database da un programma \gloss{scala} esterno, risulta essere non autenticato e può effettuare il login diventando utente autenticato.\\ \\
\textbf{Precondizione:} Il server è in ascolto.\\ \\
\textbf{Scenario principale:}
\begin{itemize}
\item UCx.y Inserimento username;
\item UCx.y+1 Inserimento password.
\end{itemize}
\textbf{Estensione:}
\begin{itemize}
  \item Nel caso in cui l'utente inserisca una coppia username e password non corrette:
  \begin{itemize}
    \item UCxx - Lancio eccezione credenziali errate
  \end{itemize}
\end{itemize}
\textbf{Postcondizioni:} L'utente ha acceduto al sistema con successo e risulta essere autenticato.

\subsubsection{UClogin.1: Inserimento username}

\textbf{Attori primari:} Utente non autenticato\\ \\
\textbf{Attori secondari:} Driver\\ \\
\textbf{Scopo e descrizione:}
L'utente intende effettuare login al database da un programma \gloss{scala} esterno, risulta essere non autenticato e può inserire il proprio username per effettuare il login.\\ \\
\textbf{Precondizioni:} Il server è in ascolto, l'utente intende effettuare il login.
\textbf{Scenario principale:}
\begin{itemize}
\item L'utente può inserire il proprio username.
\end{itemize}
\textbf{Postcondizioni:} L'utente ha inserito il proprio username.

\subsubsection{UC10.2: Inserimento password}

\textbf{Attori primari:} Utente non autenticato\\ \\
\textbf{Attori secondari:} Terminale\\ \\
\textbf{Scopo e descrizione:}
L'utente intende effettuare login al database da un programma \gloss{scala} esterno, risulta essere non autenticato e può inserire la propria password per effettuare il login.\\ \\
\textbf{Precondizioni:} Il server è in ascolto, l'utente intende effettuare il login.
\textbf{Scenario principale:}
\begin{itemize}
\item L'utente può inserire la propria password.
\end{itemize}
\textbf{Postcondizioni:} L'utente ha inserito la propria password.

\subsubsection{UC10.3: Visualizzazione Errore credenziali errate}

\textbf{Attori primari:} Utente non autenticato\\ \\
\textbf{Attori secondari:} Terminale\\ \\
\textbf{Scopo e descrizione:}
L'utente intende effettuare login al database da un programma \gloss{scala} esterno, risulta essere non autenticato e ha già inserito il proprio username e la propria password per l'autenticazione all'interno del sistema.\\ \\
\textbf{Precondizioni:} Il server è in ascolto, l'utente ha inserito le proprie credenziali (username e password) e intende effettuare il login. Il sistema non riconosce le credenziali.
\textbf{Scenario principale:}
\begin{itemize}
\item Viene lanciata un eccezione per errore credenziali errate.
\end{itemize}
\textbf{Postcondizioni:} Il login fallisce, l'utente risulta essere non autenticato.

\subsection{UC11: Operazioni sulle collezioni}

\begin{figure}[H]
  \begin{center}
    \includegraphics[width=0.7\textwidth,keepaspectratio]{UseCases/UC11.png}
    \caption*{Caso d'uso 3: Operazioni sulle \gloss{collezioni}}
  \end{center}
\end{figure}
\textbf{Attore primario:} Utente autenticato\\ \\
\textbf{Attore secondario:} Driver\\ \\
\textbf{Scopo e descrizione:} L'utente è autenticato e desidera effettuare delle operazioni sulle \gloss{collezioni}. Le operazioni previste sono:
creazione di una nuova \gloss{collezione}, visualizzazione della lista delle \gloss{collezioni} esistenti, cancellazione di \gloss{collezioni} esistenti, modifica del nome di \gloss{collezioni} esistenti,
aggiunta o rimozione collaboratori a \gloss{collezione} esistente e esportazione su file di una o più \gloss{collezioni}.\\ \\
\textbf{Precodizione:} L'utente intende effettuare operazioni su \gloss{collezioni}.\\ \\
\textbf{Scenario principale:}
\begin{itemize}
\item UC13.1 Creazione \gloss{collezione};
\item UC13.2 Visualizzazione della lista delle \gloss{collezioni};
\item UC13.3 Cancellazione \gloss{collezione};
\item UC13.4 Modifica nome \gloss{collezione};
\item UC13.5 Aggiunta \gloss{collaboratore};
\item UC13.6 Rimozione \gloss{collaboratore};
\item UC13.7 Esportazione di una o più di \gloss{collezionie} su file.
\end{itemize}
\textbf{Estensioni:}
\begin{itemize}
  \item Nel caso in cui nella procedura di creazione collezione l'utente inserisca il nome di una collezione già censita all'iterno del sistema:
  \begin{itemize}
    \item UC13.7 Lancio eccezione errore \gloss{collezione} già esistente.
  \end{itemize}
  \item Nel caso in cui nella procedura di modifica nome collezione l'utente inserisca come nuovo nome il nome di una collezione già censita all'interno del sistema:
  \begin{itemize}
    \item UC13.7 Lancio eccezione errore \gloss{collezione} già esistente.
  \end{itemize}
  \item Nel caso in cui nella procedura di cancellazione collezione l'utente inserisca il nome di una collezione non censita all'interno del sistema:
  \begin{itemize}
    \item UC13.8 Lancio eccezione errore collezione inesistente.
  \end{itemize}
  \item Nel caso in cui nella procedura di modifica nome collezione l'utente insererisca nome di una collezione non censita all'interno del sistema:
  \begin{itemize}
    \item UC13.8 Lancio eccezione errore collezione inesistente.
  \end{itemize}
  \item Nel caso in cui nella procedura di aggiunta collaboratore ad una  collezione l'utente inserisca il nome di una collezione non censita all'interno del sistema:
  \begin{itemize}
    \item UC13.8 Lancio eccezione errore collezione inesistente.
  \end{itemize}
  \item Nel caso in cui nella procedura di eliminazione di un collaboratore da una collezione l'utente inserisca il nome di una collezione non censita all'interno del sistema:
  \begin{itemize}
    \item UC13.8 Lancio eccezione errore collezione inesistente.
  \end{itemize}
  \item Nel caso in cui nella procedura di aggiunta di un collaboratore da una collezione l'utente inserisca il nome di una collezione non censita all'interno del sistema:
  \begin{itemize}
    \item UC13.8 Lancio eccezione errore username inesistente.
  \end{itemize}
  \item Nel caso in cui nella procedura di eliminazione di un collaboratore da una collezione l'utente inserisca il nome di una collezione non censita all'interno del sistema:
  \begin{itemize}
    \item UC13.8 Lancio eccezione errore username inesistente.
  \end{itemize}
\end{itemize}
\textbf{Postcondizioni:} L'operazione richiesta è stata eseguita con successo.

\subsection{UC14.1: Creazione collezione}

\textbf{Attore primario:} Utente autenticato\\ \\
\textbf{Attore secondario:} Driver\\ \\
\textbf{Scopo e descrizione:} L'utente richiede la creazione di una \gloss{collezione} all'interno del \gloss{database}, deve inserire il nome della nuova \gloss{collezione}.\\ \\
\textbf{Precodizione:} Il \gloss{database} è pronto a ricevere comandi e l'utente intende creare una \gloss{collezione}.\\ \\
\textbf{Scenario principale:}
\begin{itemize}
\item Inserimento nome \gloss{collezione}
\end{itemize}
\textbf{Estensioni:}
\begin{itemize}
\item Nel caso in cui l'utente inserisca il nome di una collezione già censita all'interno del sistema:
  \begin{itemize}
  \item UCx Lancio eccezione errore collezione già esistente.
  \end{itemize}
\end{itemize}
\textbf{Postcondizioni:} La \gloss{collezione} è stata creata.

\subsection{UC14.2: Visualizzazione della lista delle collezioni}

\textbf{Attore primario:} Utente autenticato\\ \\
\textbf{Attore secondario:} Driver\\ \\
\textbf{Scopo e descrizione:} L'utente intende elencare tutte le \gloss{collezioni} all'interno di \textbf{Actorbase} secondo la propria \gloss{visibilità}.\\ \\
\textbf{Precodizione:} Il \gloss{database} è pronto a ricevere comandi e l'utente intende avere un elenco delle \gloss{collezioni} al suo interno.\\ \\
\textbf{Scenario principale:}
\begin{itemize}
\item L'utente inserisce il comando per la visualizzazione della lista delle collezioni.
\end{itemize}
\textbf{Postcondizioni:} Il \gloss{database} restituisce la lista delle \gloss{collezioni} esistenti secondo la \gloss{visibilità} dell'utente.

\subsection{UC14.3: Modifica nome collezione}
\begin{figure}[H]
  \begin{center}
    \includegraphics[width=0.7\textwidth,keepaspectratio]{UseCases/UC11_3.png}
    \caption*{Caso d'uso 1: Login}
  \end{center}
\end{figure}
\textbf{Attore primario:} Utente autenticato\\ \\
\textbf{Attore secondario:} Driver\\ \\
\textbf{Scopo e descrizione:} L’utente intende modificare il nome di una \gloss{collezione} presente nel \gloss{database}.\\ \\
\textbf{Precodizione:} Il server è in ascolto e l’utente intende modificare il nome di una \gloss{collezione}.\\ \\
\textbf{Scenario principale:}
\begin{itemize}
\item UC14.3.1 Inserimento nome \gloss{collezione};
\item UC14.3.3 Inserimento nuovo nome \gloss{collezione}.
\end{itemize}
\textbf{Estensioni:}
\begin{itemize}
\item Nel caso in cui l'utente inserisca il nome di una collezione da modificare inesistente:
  \begin{itemize}
  \item UCx Lancio eccezione errore \gloss{collezione} inesistente;
  \end{itemize}
\item Nel caso in cui l'utente inserisca il nuovo nome di una collezione corrispondente al nome di una collezione già censita all'interno del sistema:
  \begin{itemize}
  \item UCy Lancio eccezione errore \gloss{collezione} già esistente.
  \end{itemize}
\end{itemize}
\textbf{Postcondizioni:} Il nome della \gloss{collezione} è stato modificato correttamente.

\subsubsection{UC14.3.1: Inserimento nome collezione}

\textbf{Attore primario:} Utente autenticato\\ \\
\textbf{Attore secondario:} Driver\\ \\
\textbf{Scopo e descrizione:} L'utente intende modificare il nome di un \gloss{collezione} presente nel \gloss{database}, deve inserire il nome della collezione da modificare.\\ \\
\textbf{Precondzione:} Il server è in ascolto e l'utente intende modificare il nome di una \gloss{collezione}, ha inserito il comando di modifica nome collezione e ora deve inserire il nome della collezione da modificare.\\ \\
\textbf{Scenario principale:}
\begin{itemize}
\item L'utente può inserire il nome della collezione da modificare.
\end{itemize}
\textbf{Postcondizioni:} Il nome della \gloss{collezione} da modificare è stato inserito

\subsubsection{UC14.3.2: Inserimento nuovo nome collezione}

\textbf{Attore primario:} Utente autenticato\\ \\
\textbf{Attore secondario:} Driver\\ \\
\textbf{Scopo e descrizione:} L'utente intende modificare il nome di un \gloss{collezione} presente nel \gloss{database}, deve inserire il nome della collezione da modificare.\\ \\
\textbf{Precondzione:} Il server è in ascolto e l'utente intende modificare il nome di una \gloss{collezione}, ha inserito il comando di modifica nome collezione, ha inserito il nome della collezione da modificare e ora deve inserire il nuovo nome della collezione.\\ \\
\textbf{Scenario principale:}
\begin{itemize}
\item L'utente può inserire il nuovo nome della collezione da modificare.
\end{itemize}
\textbf{Postcondizione:} Il nome della \gloss{collezione} è stato modificato.

\subsection{UC14.4: Cancellazione collezione}

\textbf{Attore primario:} Utente autenticato\\ \\
\textbf{Attore secondario:} Driver\\ \\
\textbf{Scopo e descrizione:} L’utente intende cancellare una \gloss{collezione} presente nel \gloss{database}.\\ \\
\textbf{Precodizione:} Il server è in ascolto e l’utente ha inserito il comando per la cancellazione di una \gloss{collezione}.\\ \\
\textbf{Scenario principale:}
\begin{itemize}
\item UC14.4.1 Inserimento nome \gloss{collezione};
\end{itemize}
\textbf{Estensioni:}
\begin{itemize}
\item Nel caso in cui l'utente inserisca il nome di una collezione da cancellare inesistente:
  \begin{itemize}
  \item UC15.8 Visualizzazione errore \gloss{collezione} inesistente;
  \end{itemize}
\end{itemize}
\textbf{Postcondizioni:} La \gloss{collezione} è stato cancellata correttamente.

\subsubsection{UC14.4.1: Inserimento nome collezione}

\textbf{Attore primario:} Utente autenticato\\ \\
\textbf{Attore secondario:} Driver\\ \\
\textbf{Scopo e descrizione:} L’utente intende cancellare una \gloss{collezione} presente nel \gloss{database}.\\ \\
\textbf{Precodizione:} Il server è in ascolto e l’utente ha inserito il comando per cancellare una \gloss{collezione}, deve inserire il nome della collezione da cancellare.\\ \\
\textbf{Scenario principale:}
\begin{itemize}
\item L'utente può inserire il nome della \gloss{collezione};
\end{itemize}
\textbf{Postcondizioni:} Il nome della \gloss{collezione} è stato inserito correttamente.

\subsection{UC14.5: Aggiunta di un collaboratore a collezione}
\begin{figure}[H]
  \begin{center}
    \includegraphics[width=0.7\textwidth,keepaspectratio]{UseCases/UC11_5.png}
    \caption*{Caso d'uso 14.5: Aggiunta collaboratore a una collezione}
  \end{center}
\end{figure}
\textbf{Attore primario:} Utente autenticato\\ \\
\textbf{Attore secondario:} Driver\\ \\
\textbf{Scopo e descrizione:} L'utente intende aggiungere un \gloss{collaboratore} ad una \gloss{collezione} di sua proprietà. Deve inserire il nome della \gloss{collezione} e lo username dell'utente da aggiungere.\\ \\
\textbf{Precondizione:} Il \gloss{database} è pronto a ricevere comandi e l'utente ha inserito il comando per aggiungere un \gloss{collaboratore} ad una \gloss{collezione} di sua proprietà.\\ \\
\textbf{Scenario principale:}
\begin{itemize}
\item UC14.5.1 Inserimento nome \gloss{collezione};
\item UC14.5.3 Inserimento username \gloss{collaboratore}.
\end{itemize}
\textbf{Estensioni:}
\begin{itemize}
  \item Nel caso in cui l'utente inserisca il nome di una collezione inesistente:
  \begin{itemize}
    \item UCx Lancio eccezione errore \gloss{collezione} inesistente;
  \end{itemize}
  \item Nel caso in cui l'utente inserisca uno username inesistente:
  \begin{itemize}
    \item UCx Lancio eccezione errore \gloss{username} inesistente;
  \end{itemize}
\end{itemize}
\textbf{Postcondizione:} L'utente designato per la collaborazione è stato aggiunto tra i \gloss{collaboratori} della \gloss{collezione} scelta.

\subsubsection{UC14.5.1: Inserimento nome collezione}

\textbf{Attore primario:} Utente autenticato\\ \\
\textbf{Attore secondario:} Driver\\ \\
\textbf{Scopo e descrizione:} L’utente intende aggiungere un collaboratore ad una \gloss{collezione} presente nel \gloss{database}.\\ \\
\textbf{Precodizione:} Il server è in ascolto e l’utente ha inserito il comando per aggiungere un collaboratore ad una \gloss{collezione}, deve inserire il nome della collezione.\\ \\
\textbf{Scenario principale:}
\begin{itemize}
\item L'utente può inserire il nome della \gloss{collezione};
\end{itemize}
\textbf{Postcondizioni:} Il nome della \gloss{collezione} è stato inserito correttamente.

\subsubsection{UC14.5.2: Inserimento username}

\textbf{Attore primario:} Utente autenticato\\ \\
\textbf{Attore secondario:} Driver\\ \\
\textbf{Scopo e descrizione:} L’utente intende aggiungere un collaboratore ad una \gloss{collezione} presente nel \gloss{database}.\\ \\
\textbf{Precodizione:} Il server è in ascolto e l’utente ha inserito il comando per aggiungere un collaboratore ad una \gloss{collezione}, deve inserire l'username di quest'ultimo.\\ \\
\textbf{Scenario principale:}
\begin{itemize}
\item L'utente può inserire lo username del designato collaboratore;
\end{itemize}
\textbf{Postcondizioni:} l'username è stato inserito correttamente.

\subsection{UC14.6: Rimozione di un collaboratore a collezione}
\begin{figure}[H]
  \begin{center}
    \includegraphics[width=0.7\textwidth,keepaspectratio]{UseCases/UC14_6.png}
    \caption*{Caso d'uso 14.6: Rimozione di un collaboratore a collazione}
  \end{center}
\end{figure}
\textbf{Attore primario:} Utente autenticato\\ \\
\textbf{Attore secondario:} Driver\\ \\
\textbf{Scopo e descrizione:} L'utente intende rimuovere un \gloss{collaboratore} da una \gloss{collezione} di sua proprietà. Deve inserire il nome della \gloss{collezione} e lo username del \gloss{collaboratore} da rimuovere.\\ \\
\textbf{Precondizione:} Il \gloss{database} è pronto a ricevere comandi e l'utente ha inserito il comando per rimuovere un \gloss{collaboratore} da una \gloss{collezione} di sua proprietà.\\ \\
\textbf{Scenario principale:}
\begin{itemize}
\item UC14.6.1 Inserimento nome \gloss{collezione};
\item UC14.6.2 Inserimento username \gloss{collaboratore}.
\end{itemize}
\textbf{Postcondizione:} L'utente designato per la rimozione dai \gloss{collaboratori} è stato rimosso dalla \gloss{collezione} scelta.

\subsubsection{UC14.6.1: Inserimento nome collezione}

\textbf{Attore primario:} Utente autenticato\\ \\
\textbf{Attore secondario:} Driver\\ \\
\textbf{Scopo e descrizione:} L’utente intende rimuovere un collaboratore ad una \gloss{collezione} presente nel \gloss{database}.\\ \\
\textbf{Precodizione:} Il server è in ascolto e l’utente ha inserito il comando per rimuovere un collaboratore ad una \gloss{collezione}, deve inserire il nome della collezione.\\ \\
\textbf{Scenario principale:}
\begin{itemize}
\item L'utente può inserire il nome della \gloss{collezione};
\end{itemize}
\textbf{Postcondizioni:} Il nome della \gloss{collezione} è stato inserito correttamente.

\subsubsection{UC14.6.2: Inserimento username}

\textbf{Attore primario:} Utente autenticato\\ \\
\textbf{Attore secondario:} Driver\\ \\
\textbf{Scopo e descrizione:} L’utente intende rimuovere un collaboratore ad una \gloss{collezione} presente nel \gloss{database}.\\ \\
\textbf{Precodizione:} Il server è in ascolto e l’utente ha inserito il comando per rimuovere un collaboratore ad una \gloss{collezione}, deve inserire l'username di quest'ultimo.\\ \\
\textbf{Scenario principale:}
\begin{itemize}
\item L'utente può inserire lo username del collaboratore da rimuovere;
\end{itemize}
\textbf{Postcondizioni:} l'username è stato inserito correttamente.

\subsection{UC11.7: Esportazione su file}

\subsection{UC11.7.1: Esportazione una o più collezione}

\subsection{UC11.7.2: Esportazione di tutto il database}

\subsection{UC12: Operazioni sugli item}

\begin{figure}[H]
  \begin{center}
    \includegraphics[width=0.7\textwidth,keepaspectratio]{UseCases/UC12.png}
    \caption*{Caso d'uso 4: Operazioni sugli \gloss{item}}
  \end{center}
\end{figure}
\textbf{Attore primario:} Utente autenticato\\ \\
\textbf{Scopo e descrizione:} L'utente intende effettuare operazioni sugli \gloss{item} di una \gloss{collezione} esistente. Le operazioni previste sono:
inserimento e cancellazione.\\ \\
\textbf{Precondizione:} Il \gloss{database} è pronto a ricevere comandi e l'utente intende effettuare operazioni di inserimento o cancellazione su una \gloss{collezione} esistente.\\ \\
\textbf{Scenario principale:}
\begin{itemize}
\item UC1.5.1 Inserimento nuovo \gloss{item};
\item UC1.5.2 Cancellazione \gloss{item}.
\end{itemize}
\textbf{Estensioni:}
\begin{itemize}
\item Nel caso in cui l'utente utilizzi un comando sconosciuto:
  \begin{itemize}
  \item viene visualizzato un messaggio di errore esplicativo (e rimando all'aiuto).
  \end{itemize}
\item Nel caso in cui l'utente utilizzi un comando conosciuto ma con parametri errati:
  \begin{itemize}
  \item viene visualizzato un messaggio di supporto che indica come usare correttamente il comando.
  \end{itemize}
\item Nel caso in cui la \gloss{collezione} su cui effettuare inserimento o cancellazione non sia presente:
  \begin{itemize}
  \item viene visualizzato un messaggio di errore esplicativo.
  \end{itemize}
\item Nel caso in cui l'\gloss{item} da rimuovere non sia presente:
  \begin{itemize}
  \item viene visualizzato un messaggio di errore esplicativo.
  \end{itemize}
\end{itemize}
\textbf{Postcondizione:} L'operazione richiesta è stata effettuata correttamente.

\subsection{UC12.1: Inserimento item}

\textbf{Attore primario:} Utente autenticato\\ \\
\textbf{Scopo e descrizione:} L'utente intende inserire un nuovo \gloss{item} all'interno di una \gloss{collezione} esistente. Deve selezionare la \gloss{collezione} e inserire i parametri di definizione dell'\gloss{item}. Questi sono la chiave e il valore associato.\\ \\
\textbf{Precondizione:} Il \gloss{database} è pronto a ricevere comandi e l'utente intende effettuare un inserimento di un nuovo \gloss{item}.\\ \\
\textbf{Scenario principale:}
\begin{itemize}
\item UC4.1.1 Inserimento item manuale;
\item UC4.1.2 Inserimento item da file.
\end{itemize}
\textbf{Postcondizione:} L'operazione di inserimento richiesta è stata effettuata correttamente.

\subsection{UC12.1.1: Inserimento item manuale}

\begin{figure}[H]
  \begin{center}
    \includegraphics[width=0.7\textwidth,keepaspectratio]{UseCases/UC4_1_1.png}
    \caption*{Caso d'uso 4: Operazioni sugli \gloss{item}}
  \end{center}
\end{figure}

\subsection{UC12.1.2: Inserimento item da file}

\begin{figure}[H]
  \begin{center}
    \includegraphics[width=0.7\textwidth,keepaspectratio]{UseCases/UC4_1_2.png}
    \caption*{Caso d'uso 4: Operazioni sugli \gloss{item}}
  \end{center}
\end{figure}

\subsection{UC12.1.2.1: Inserimento path file}

\subsection{UC12.1.1.1: Inserimento nome collezione}

\subsection{UC12.1.1.2: Inserimento chiave}

\subsection{UC12.1.1.3: Inserimento valore}

\subsection{UC12.1.1.4: Inserimento flag sovrascrittura}

\subsection{UC12.2: Cancellazione item}

\begin{figure}[H]
  \begin{center}
    \includegraphics[width=0.7\textwidth,keepaspectratio]{UseCases/UC12_2.png}
    \caption*{Caso d'uso 4: Operazioni sugli \gloss{item}}
  \end{center}
\end{figure}

\subsection{UC12.2.1: Inserimento nome collezione}

\subsection{UC12.2.2: Inserimento chiave item}

\subsection{UC13: Ricerca}

\subsection{UC13.1: Ricerca su una o più collezioni}

\subsection{UC13.2: Ricerca su intero database}

\subsection{UC14: Modifica password}

\begin{figure}[H]
  \begin{center}
    \includegraphics[width=0.7\textwidth,keepaspectratio]{UseCases/UC14.png}
    \caption*{Caso d'uso 4: Operazioni sugli \gloss{item}}
  \end{center}
\end{figure}

\subsection{UC14.1: Inserimento vecchia password}

\subsection{UC14.2: Inserimento nuova password}

\subsection{UC14.3: Conferma password}

\subsection{UC15: Logout}

\subsection{UC16: Gestione utenti}

\begin{figure}[H]
  \begin{center}
    \includegraphics[width=0.7\textwidth,keepaspectratio]{UseCases/UC16.png}
    \caption*{Caso d'uso 4: Operazioni sugli \gloss{item}}
  \end{center}
\end{figure}

\subsection{UC16.1: Aggiunta utente}

\begin{figure}[H]
  \begin{center}
    \includegraphics[width=0.7\textwidth,keepaspectratio]{UseCases/UC16_1.png}
    \caption*{Caso d'uso 4: Operazioni sugli \gloss{item}}
  \end{center}
\end{figure}

\subsection{UC16.1.1: Inserimento username}

\subsection{UC16.1.2: Conferma aggiunta utente}

\subsection{UC16.2: Rimozione utente}

\begin{figure}[H]
  \begin{center}
    \includegraphics[width=0.7\textwidth,keepaspectratio]{UseCases/UC16_2.png}
    \caption*{Caso d'uso 4: Operazioni sugli \gloss{item}}
  \end{center}
\end{figure}

\subsection{UC16.2.1: Inserimento username}

\subsection{UC16.2.2: Conferma rimozione utente}

\subsection{UC16.3: Reset password}

\begin{figure}[H]
  \begin{center}
    \includegraphics[width=0.7\textwidth,keepaspectratio]{UseCases/UC16_3.png}
    \caption*{Caso d'uso 4: Operazioni sugli \gloss{item}}
  \end{center}
\end{figure}

\subsection{UC16.3.1: Inserimento username}

\subsection{UC16.3.2: Conferma reset}

\section{Requisiti}

\subsection{Requisiti funzionali}

La descrizione dei requisiti segue le regole descritte nelle \href{run:../Interni/NormeDiProgetto\_v1.0.0.pdf}{Norme di Progetto}.
\begin{longtable}[H]{|l|p{2cm}|p{6cm}|p{4cm}|}
  \hline
  \textbf{Requisito} & \textbf{Tipologia} & \textbf{Descrizione} & \textbf{Fonti}\\
  \hline
  OBF1 & \multiLineCell{Funzionale\\Obbligatorio} & Il sistema dovrà fornire un Domain Specific Language per comunicare con il \gloss{database} & \multiLineCell{Capitolato\\UC1\\}\\
  \hline
  DEF1.1 & \multiLineCell{Funzionale\\Desiderabile} & Il sistema dovrà fornire un comando di assistenza & \multiLineCell{UC1.3\\UC1.3.1\\UC1.3.2\\}\\
  \hline
  DEF1.1.1 & \multiLineCell{Funzionale\\Desiderabile} & Il sistema dovrà fornire un comando di assistenza generale & \multiLineCell{UC1.3.1\\}\\
  \hline
  DEF1.1.2 & \multiLineCell{Funzionale\\Desiderabile} & Il sistema dovrà fornire un comando di assistenza specifico per un comando desiderato & \multiLineCell{UC1.3.2\\}\\
  \hline
  OBF2 & \multiLineCell{Funzionale\\Obbligatorio} & Il sistema dovrà fornire un meccanismo di autenticazione & \multiLineCell{UC1\\UC1.1\\UC1.2\\VERBALE20160112\\}\\
  \hline
  OBF2.1 & \multiLineCell{Funzionale\\Obbligatorio} & Il sistema dovrà fornire un meccanismo di registrazione & \multiLineCell{UC1.1\\VERBALE20160119\\}\\
  \hline
  OBF2.2 & \multiLineCell{Funzionale\\Obbligatorio} & Il sistema dovrà fornire un meccanismo di login & \multiLineCell{UC1.2\\UC2\\VERBALE20160119\\}\\
  \hline
  OBF3 & \multiLineCell{Funzionale\\Obbligatorio} & Il sistema dovrà fornire un meccanismo di persistenza dei dati su disco & \multiLineCell{Capitolato\\VERBALE20160112\\}\\
  \hline
  OBF4 & \multiLineCell{Funzionale\\Obbligatorio} & Il sistema dovrà fornire la possibilità di eseguire una serie di operazioni sulle \gloss{collezioni} & \multiLineCell{Capitolato\\UC1.4\\UC2\\}\\
  \hline
  OBF4.1 & \multiLineCell{Funzionale\\Obbligatorio} & Il sistema dovrà fornire la possibilità di creare una nuova \gloss{collezione} & \multiLineCell{UC1.4.1\\VERBALE20160112\\}\\
  \hline
  OBF4.1.1 & \multiLineCell{Funzionale\\Opzionale} & Il sistema deve dare la possibilità di scegliere uno \gloss{schema} delle chiavi che definiscono la \gloss{collezione} da creare & \multiLineCell{UC1.4.1\\VERBALE20160112\\VERBALE20160119\\}\\
  \hline
  OBF4.2 & \multiLineCell{Funzionale\\Obbligatorio} & Il sistema dovrà fornire la possibilità di elencare le \gloss{collezioni} presenti all'interno & \multiLineCell{UC1.4.3\\UC2\\VERBALE20160112\\}\\
  \hline
  OBF4.3 & \multiLineCell{Funzionale\\Obbligatorio} & Il sistema dovrà fornire la possibilità di cancellare una o più \gloss{collezioni} & \multiLineCell{UC1.4.3\\UC2\\VERBALE20160112\\}\\
  \hline
  OBF4.4 & \multiLineCell{Funzionale\\Obbligatorio} & Il sistema dovrà fornire la possibilità di modificare il nome delle \gloss{collezioni} & \multiLineCell{UC1.4.4\\UC2\\VERBALE20160112\\}\\
  \hline
  OBF4.5 & \multiLineCell{Funzionale\\Obbligatorio} & Il sistema dovrà fornire la possibilità di aggiungere collaboratori ad una propria \gloss{collezione} & \multiLineCell{UC1.4.5\\UC2\\VERBALE20160112\\}\\
  \hline
  OBF4.6 & \multiLineCell{Funzionale\\Obbligatorio} & Il sistema dovrà fornire la possibilità di rimuovere un \gloss{collaboratore} da una propria \gloss{collezione} & \multiLineCell{UC1.4.6\\UC2\\VERBALE20160112\\}\\
  \hline
  OBF5 & \multiLineCell{Funzionale\\Obbligatorio} & Il sistema dovrà fornire la possibilità di eseguire una serie di operazioni sugli \gloss{item} & \multiLineCell{Capitolato\\UC1.5\\UC1.5.1\\UC1.5.1.1\\UC1.5.1.2\\UC1.5.2\\UC2\\VERBALE20160112\\}\\
  \hline
  OBF5.1 & \multiLineCell{Funzionale\\Obbligatorio} & Il sistema dovrà permettere l'inserimento di nuovi \gloss{item} & \multiLineCell{Capitolato\\UC1.5.1\\UC2\\VERBALE20160112\\}\\
  \hline
  OBF5.1.1 & \multiLineCell{Funzionale\\Obbligatorio} & Il sistema dovrà permettere la sovrascrittura di un \gloss{item} attraverso inserimento & \multiLineCell{Capitolato\\UC1.5.1.1\\UC2\\VERBALE20160112\\}\\
  \hline
  OBF5.1.2 & \multiLineCell{Funzionale\\Obbligatorio} & Il sistema dovrà permettere l'aggiunta di un nuovo \gloss{item} senza eseguire sovrascrittura & \multiLineCell{Capitolato\\UC1.5.1.2\\UC2\\VERBALE20160112\\}\\
  \hline
  OBF5.2 & \multiLineCell{Funzionale\\Obbligatorio} & Il sistema dovrà permettere la cancellazione di un \gloss{item} & \multiLineCell{Capitolato\\UC1.5.2\\VERBALE20160112\\}\\
  \hline
  DEF5.3 & \multiLineCell{Funzionale\\Desiderabile} & Il sistema dovrà permettere tutte le operazioni sugli \gloss{item} anche attraverso importazione di \gloss{file} & \multiLineCell{UC1.5\\UC1.5.1\\UC1.5.1.1\\UC1.5.1.2\\UC1.5.2\\VERBALE20160112\\}\\
  \hline
  OBF6 & \multiLineCell{Funzionale\\Obbligatorio} & Il sistema dovrà fornire un meccanismo di ricerca & \multiLineCell{UC1.6\\UC2\\VERBALE20160112\\}\\
  \hline
  OBF6.1 & \multiLineCell{Funzionale\\Obbligatorio} & Il sistema dovrà fornire un meccanismo di ricerca generale sull'intera base di dati & \multiLineCell{UC1.6.1\\UC2\\VERBALE20160112\\}\\
  \hline
  OBF6.2 & \multiLineCell{Funzionale\\Obbligatorio} & Il sistema dovrà fornire la possibilità di effettuare ricerche su una o più \gloss{collezioni} & \multiLineCell{UC1.6.2\\UC2\\VERBALE20160112\\}\\
  \hline
  OBF7 & \multiLineCell{Funzionale\\Obbligatorio} & Il sistema dovrà fornire un meccanismo per la gestione del proprio \gloss{account} & \multiLineCell{UC1.7\\UC1.7.1\\UC1.7.2\\VERBALE20160112\\}\\
  \hline
  DEF7.1 & \multiLineCell{Funzionale\\Desiderabile} & Il sistema dovrà fornire la possibilità di cambiare il proprio username & \multiLineCell{UC1.7\\UC1.7.1\\VERBALE20160112\\}\\
  \hline
  OBF7.2 & \multiLineCell{Funzionale\\Obbligatorio} & Il sistema dovrà fornire la possibilità di cambiare la propria password & \multiLineCell{UC1.7\\UC1.7.2\\VERBALE20160112\\}\\
  \hline
  OBF8 & \multiLineCell{Funzionale\\Obbligatorio} & Il sistema dovrà permettere di effettuare logout & \multiLineCell{UC1.8\\VERBALE20160112\\}\\
  \hline
  DEF9 & \multiLineCell{Funzionale\\Desiderabile} & Il sistema dovrà fornire diverse azioni possibili sugli \gloss{account} agli amministratori & \multiLineCell{UC1.9\\UC1.9.1\\UC1.9.2\\UC2\\}\\
  \hline
  DEF9.1 & \multiLineCell{Funzionale\\Desiderabile} & Il sistema dovrà permettere agli amministratori di aggiungere nuovi utenti & \multiLineCell{UC1.9\\UC1.9.1\\UC2\\}\\
  \hline
  DEF9.2 & \multiLineCell{Funzionale\\Desiderabile} & Il sistema dovrà permettere agli amministratori di eliminare utenti esistenti & \multiLineCell{UC1.9\\UC1.9.2\\UC2\\}\\
  \hline
  OBF10 & \multiLineCell{Funzionale\\Obbligatorio} & Il sistema dovrà fornire un meccanismo di connessione ad altre istanze esterne & \multiLineCell{Capitolato\\UC1\\VERBALE20160112\\VERBALE20160119\\}\\
  \hline
  OBF11 & \multiLineCell{Funzionale\\Obbligatorio} & Il sistema dovrà permettere la visualizzazione di un messaggio di errore & \multiLineCell{UC1.10\\UC1.10.1\\UC1.10.2\\VERBALE20160112\\}\\
  \hline
  OBF11.1 & \multiLineCell{Funzionale\\Obbligatorio} & Il sistema dovrà permettere la visualizzazione di un messaggio di errore in caso di inserimenti di un comando errato & \multiLineCell{UC1.10\\UC1.10.1\\VERBALE20160112\\}\\
  \hline
  OBF11.2 & \multiLineCell{Funzionale\\Obbligatorio} & Il sistema dovrà permettere la visualizzazione di un messaggio d'errore in caso di timeout & \multiLineCell{UC1.10.2\\VERBALE20160112\\}\\
  \hline
  OBF11.3 & \multiLineCell{Funzionale\\Obbligatorio} & Il sistema dovrà permettere la visualizzazione di un messaggio di errore qualora si tenti di eseguire una operazione non permessa o impossibile da soddisfare & \multiLineCell{UC1.11.3\\VERBALE20160112\\}\\
  \hline
  DEF12 & \multiLineCell{Funzionale\\Desiderabile} & Dovrà essere fornito un \gloss{driver} Scala per interfacciarsi ad Actorbase & \multiLineCell{Capitolato\\UC2\\VERBALE20160112\\}\\
  \hline
  DEF12.1 & \multiLineCell{Funzionale\\Desiderabile} & Il \gloss{driver} dovrà permettere di effettuare il login & \multiLineCell{UC2\\}\\
  \hline
  DEF12.10 & \multiLineCell{Funzionale\\Desiderabile} & Il \gloss{driver} dovrà permettere la ricerca all'interno di \gloss{collezioni} & \multiLineCell{UC2\\}\\
  \hline
  DEF12.11 & \multiLineCell{Funzionale\\Desiderabile} & Il \gloss{driver} dovrà permettere il logout dal \gloss{database} & \multiLineCell{UC2\\}\\
  \hline
  DEF12.2 & \multiLineCell{Funzionale\\Desiderabile} & Il \gloss{driver} dovrà permettere la creazione di \gloss{collezioni} & \multiLineCell{UC2\\}\\
  \hline
  DEF12.3 & \multiLineCell{Funzionale\\Desiderabile} & Il \gloss{driver} dovrà permettere l'elencazione delle \gloss{collezioni} & \multiLineCell{UC2\\}\\
  \hline
  DEF12.4 & \multiLineCell{Funzionale\\Desiderabile} & Il \gloss{driver} dovrà permettere la modifica del nome di una \gloss{collezione} & \multiLineCell{UC2\\}\\
  \hline
  DEF12.5 & \multiLineCell{Funzionale\\Desiderabile} & Il \gloss{driver} dovrà permettere l'aggiunta di un \gloss{collaboratore} al \gloss{collezione} & \multiLineCell{UC2\\}\\
  \hline
  DEF12.6 & \multiLineCell{Funzionale\\Desiderabile} & Il \gloss{driver} dovrà permetter la rimozione di un \gloss{collaboratore} dalle \gloss{collezioni} & \multiLineCell{UC2\\}\\
  \hline
  DEF12.7 & \multiLineCell{Funzionale\\Desiderabile} & Il \gloss{driver} dovrà permettere l'inserimento di \gloss{item} nelle \gloss{collezioni} & \multiLineCell{UC2\\}\\
  \hline
  DEF12.7.1 & \multiLineCell{Funzionale\\Desiderabile} & Il \gloss{driver} dovrà permettere l'inserimento \gloss{item} con sovrascrittura nelle \gloss{collezioni} & \multiLineCell{UC2\\}\\
  \hline
  DEF12.7.2 & \multiLineCell{Funzionale\\Desiderabile} & Il \gloss{driver} dovrà permettere l'inserimento senza sovrascrittura nelle \gloss{collezioni} & \multiLineCell{UC2\\}\\
  \hline
  DEF12.8 & \multiLineCell{Funzionale\\Desiderabile} & Il \gloss{driver} dovrà permettere la cancellazione di \gloss{item} dalle \gloss{collezioni} & \multiLineCell{UC2\\}\\
  \hline
  DEF12.9 & \multiLineCell{Funzionale\\Desiderabile} & Il \gloss{driver} dovrà permettere la ricerca all'interno del \gloss{database} & \multiLineCell{UC2\\}\\
  \hline
  OBF19 & \multiLineCell{Funzionale\\Obbligatorio} & Sarà implementato l'\gloss{attore} di tipo \gloss{storekeeper} & \multiLineCell{Capitolato\\}\\
  \hline
  OBF19.1 & \multiLineCell{Funzionale\\Obbligatorio} & Gli \gloss{attori} di tipo \gloss{storekeeper} dovranno contenere le diverse coppie chiave/valore & \multiLineCell{Capitolato\\}\\
  \hline
  OBF19.2 & \multiLineCell{Funzionale\\Obbligatorio} & Gli \gloss{storekeeper} dovranno soddisfare le richieste provenienti dai rispettivi \gloss{storefinder} & \multiLineCell{Capitolato\\}\\
  \hline
  OBF20 & \multiLineCell{Funzionale\\Obbligatorio} & Sarà implementato l'\gloss{attore} di tipo \gloss{storefinder} & \multiLineCell{Capitolato\\}\\
  \hline
  OBF20.1 & \multiLineCell{Funzionale\\Obbligatorio} & Gli \gloss{attori} \gloss{storefinder} possono ricevere richieste dall'esterno & \multiLineCell{Capitolato\\}\\
  \hline
  OBF20.2 & \multiLineCell{Funzionale\\Obbligatorio} & Gli \gloss{attori} di tipo \gloss{storefinder} dovranno instradare le richieste ai giusti \gloss{storekeeper} & \multiLineCell{Capitolato\\}\\
  \hline
  OBF21 & \multiLineCell{Funzionale\\Obbligatorio} & Sarà implementato l'\gloss{attore} di tipo \gloss{warehousemen} & \multiLineCell{Capitolato\\}\\
  \hline
  OBF21.1 & \multiLineCell{Funzionale\\Obbligatorio} & Il sistema dovrà permettere agli \gloss{attori} \gloss{warehousemen} di mandare messaggi agli \gloss{attori} di tipo \gloss{storekeeper} & \multiLineCell{Capitolato\\}\\
  \hline
  OBF21.2 & \multiLineCell{Funzionale\\Obbligatorio} & Il sistema dovrà permettere agli \gloss{attori} \gloss{warehousemen} di ricevere messaggi dagli \gloss{attori} di tipo \gloss{storekeeper} & \multiLineCell{Capitolato\\}\\
  \hline
  OBF21.3 & \multiLineCell{Funzionale\\Obbligatorio} & Il sistema dovrà permettere agli \gloss{attori} \gloss{warehousemen} di persistere i dati su disco & \multiLineCell{Capitolato\\}\\
  \hline
  OPF22 & \multiLineCell{Funzionale\\Opzionale} & Sarà implementato l'\gloss{attore} di tipo ninja & \multiLineCell{Capitolato\\}\\
  \hline
  OPF22.1 & \multiLineCell{Funzionale\\Opzionale} & L'\gloss{attore} di tipo ninja dovrà poter essere eletto a \gloss{storekeeper} se necessario & \multiLineCell{Capitolato\\}\\
  \hline
  OPF22.2 & \multiLineCell{Funzionale\\Opzionale} & L'\gloss{attore} di tipo ninja dovrà essere aggiornato insieme al corrispondente \gloss{storekeeper} & \multiLineCell{Capitolato\\}\\
  \hline
  OPF23 & \multiLineCell{Funzionale\\Opzionale} & Sarà implementato l'\gloss{attore} di tipo manager & \multiLineCell{Capitolato\\}\\
  \hline
  OPF23.1 & \multiLineCell{Funzionale\\Opzionale} & L'\gloss{attore} manager dovrà decidere il numero massimo di \gloss{item} contenuti in un'istanza dello \gloss{storekeeper} & \multiLineCell{Capitolato\\}\\
  \hline
  OPF23.1.1 & \multiLineCell{Funzionale\\Opzionale} & L'\gloss{attore} manager dovrà decidere il numero massimo di \gloss{item} contenuti in un istanza di \gloss{storekeeper} in maniera statica & \multiLineCell{Capitolato\\}\\
  \hline
  OPF23.1.2 & \multiLineCell{Funzionale\\Opzionale} & L'\gloss{attore} manager dovrà decidere il numero massimo di \gloss{item} contenuti in un istanza di \gloss{storekeeper} in maniera dinamica in base al carico & \multiLineCell{Capitolato\\}\\
  \hline
  OPF23.2 & \multiLineCell{Funzionale\\Opzionale} & L'\gloss{attore} manager dovrà poter inviare e ricevere messaggi agli \gloss{storekeeper} & \multiLineCell{Capitolato\\}\\
  \hline
  OBF24 & \multiLineCell{Funzionale\\Obbligatorio} & Sarà implementato l'\gloss{attore} di tipo main & \multiLineCell{Capitolato\\}\\
  \hline
  OBF26 & \multiLineCell{Funzionale\\Obbligatorio} & Il sistema dovrà distinguere diversi tipi di \gloss{account}. Utente non autenticato, utente autenticato e utente amministratore & \multiLineCell{UC1\\UC1.1\\UC1.2\\UC1.7\\UC1.7.1\\UC1.7.2\\UC1.8\\UC1.9\\UC1.9.1\\UC1.9.2\\VERBALE20160112\\}\\
  \hline
\end{longtable}

\subsection{Requisiti di vincolo}

\begin{longtable}[H]{|l|p{2cm}|p{6cm}|p{4cm}|}
  \hline
  \textbf{Requisito} & \textbf{Tipologia} & \textbf{Descrizione} & \textbf{Fonti}\\
  \hline
  OBV13 & \multiLineCell{Vincolo\\Obbligatorio} & Il sistema dovrà funzionare in ambienti Ubuntu 14.04 o superiore & \multiLineCell{VERBALE20160119\\}\\
  \hline
  OBV14 & \multiLineCell{Vincolo\\Obbligatorio} & Il sistema dovrà funzionare su \gloss{JVM} versione 8 & \multiLineCell{Capitolato\\}\\
  \hline
  DEV15 & \multiLineCell{Vincolo\\Desiderabile} & Il sistema dovrà funzionare in ambienti Windows 10 o superiori & \multiLineCell{VERBALE20160119\\}\\
  \hline
  DEV16 & \multiLineCell{Vincolo\\Desiderabile} & Il sistema dovrà funzionare su sistemi OSX el capitan o superiori & \multiLineCell{VERBALE20160119\\}\\
  \hline
  OBV25 & \multiLineCell{Vincolo\\Obbligatorio} & Il sistema dovrà utilizzare la libreria Akka & \multiLineCell{Capitolato\\}\\
  \hline
\end{longtable}

\subsection{Requisiti di qualità}

\begin{longtable}[H]{|l|p{2cm}|p{6cm}|p{4cm}|}
  \hline
  \textbf{Requisito} & \textbf{Tipologia} & \textbf{Descrizione} & \textbf{Fonti}\\
  \hline
  OBQ17 & \multiLineCell{Qualitativo\\Obbligatorio} & Dovrà essere fornito un manuale utente redatto in lingua inglese & \multiLineCell{Capitolato\\}\\
  \hline
  OBF18 & \multiLineCell{Qualitativo\\Obbligatorio} & Dovranno essere rispettate tutte le norme e le metriche sulla stesura del codice come riportato nelle \textit{Norme di Progetto} & \multiLineCell{INTERNO\\}\\
  \hline
  OBQ18.1 & \multiLineCell{Qualitativo\\Obbligatorio} & Dovrà essere prodotta documentazione in Scaladoc del prodotto & \multiLineCell{NORME\\}\\
  \hline
\end{longtable}

\section{Tracciamento Requisiti-Fonti}

\begin{longtable}[H]{|p{5.5cm}|p{5.5cm}|}
  \hline
  \textbf{Requisito} & \textbf{Fonti}\\
  \hline
  OBF1 & \multiLineCell[t]{Capitolato\\UC1\\}\\
  \hline
  DEF1.1 & \multiLineCell[t]{UC1.3\\UC1.3.1\\UC1.3.2\\}\\
  \hline
  DEF1.1.1 & \multiLineCell[t]{UC1.3.1\\}\\
  \hline
  DEF1.1.2 & \multiLineCell[t]{UC1.3.2\\}\\
  \hline
  OBF2 & \multiLineCell[t]{UC1\\UC1.1\\UC1.2\\VERBALE20160112\\}\\
  \hline
  OBF2.1 & \multiLineCell[t]{UC1.1\\VERBALE20160119\\}\\
  \hline
  OBF2.2 & \multiLineCell[t]{UC1.2\\UC2\\VERBALE20160119\\}\\
  \hline
  OBF3 & \multiLineCell[t]{Capitolato\\VERBALE20160112\\}\\
  \hline
  OBF4 & \multiLineCell[t]{Capitolato\\UC1.4\\UC2\\}\\
  \hline
  OBF4.1 & \multiLineCell[t]{UC1.4.1\\VERBALE20160112\\}\\
  \hline
  OBF4.1.1 & \multiLineCell[t]{UC1.4.1\\VERBALE20160112\\VERBALE20160119\\}\\
  \hline
  OBF4.2 & \multiLineCell[t]{UC1.4.3\\UC2\\VERBALE20160112\\}\\
  \hline
  OBF4.3 & \multiLineCell[t]{UC1.4.3\\UC2\\VERBALE20160112\\}\\
  \hline
  OBF4.4 & \multiLineCell[t]{UC1.4.4\\UC2\\VERBALE20160112\\}\\
  \hline
  OBF4.5 & \multiLineCell[t]{UC1.4.5\\UC2\\VERBALE20160112\\}\\
  \hline
  OBF4.6 & \multiLineCell[t]{UC1.4.6\\UC2\\VERBALE20160112\\}\\
  \hline
  OBF5 & \multiLineCell[t]{Capitolato\\UC1.5\\UC1.5.1\\UC1.5.1.1\\UC1.5.1.2\\UC1.5.2\\UC2\\VERBALE20160112\\}\\
  \hline
  OBF5.1 & \multiLineCell[t]{Capitolato\\UC1.5.1\\UC2\\VERBALE20160112\\}\\
  \hline
  OBF5.1.1 & \multiLineCell[t]{Capitolato\\UC1.5.1.1\\UC2\\VERBALE20160112\\}\\
  \hline
  OBF5.1.2 & \multiLineCell[t]{Capitolato\\UC1.5.1.2\\UC2\\VERBALE20160112\\}\\
  \hline
  OBF5.2 & \multiLineCell[t]{Capitolato\\UC1.5.2\\VERBALE20160112\\}\\
  \hline
  DEF5.3 & \multiLineCell[t]{UC1.5\\UC1.5.1\\UC1.5.1.1\\UC1.5.1.2\\UC1.5.2\\VERBALE20160112\\}\\
  \hline
  OBF6 & \multiLineCell[t]{UC1.6\\UC2\\VERBALE20160112\\}\\
  \hline
  OBF6.1 & \multiLineCell[t]{UC1.6.1\\UC2\\VERBALE20160112\\}\\
  \hline
  OBF6.2 & \multiLineCell[t]{UC1.6.2\\UC2\\VERBALE20160112\\}\\
  \hline
  OBF7 & \multiLineCell[t]{UC1.7\\UC1.7.1\\UC1.7.2\\VERBALE20160112\\}\\
  \hline
  DEF7.1 & \multiLineCell[t]{UC1.7\\UC1.7.1\\VERBALE20160112\\}\\
  \hline
  OBF7.2 & \multiLineCell[t]{UC1.7\\UC1.7.2\\VERBALE20160112\\}\\
  \hline
  OBF8 & \multiLineCell[t]{UC1.8\\VERBALE20160112\\}\\
  \hline
  DEF9 & \multiLineCell[t]{UC1.9\\UC1.9.1\\UC1.9.2\\UC2\\}\\
  \hline
  DEF9.1 & \multiLineCell[t]{UC1.9\\UC1.9.1\\UC2\\}\\
  \hline
  DEF9.2 & \multiLineCell[t]{UC1.9\\UC1.9.2\\UC2\\}\\
  \hline
  OBF10 & \multiLineCell[t]{Capitolato\\UC1\\VERBALE20160112\\VERBALE20160119\\}\\
  \hline
  OBF11 & \multiLineCell[t]{UC1.10\\UC1.10.1\\UC1.10.2\\VERBALE20160112\\}\\
  \hline
  OBF11.1 & \multiLineCell[t]{UC1.10\\UC1.10.1\\VERBALE20160112\\}\\
  \hline
  OBF11.2 & \multiLineCell[t]{UC1.10.2\\VERBALE20160112\\}\\
  \hline
  OBF11.3 & \multiLineCell[t]{UC1.11.3\\VERBALE20160112\\}\\
  \hline
  DEF12 & \multiLineCell[t]{Capitolato\\UC2\\VERBALE20160112\\}\\
  \hline
  DEF12.1 & \multiLineCell[t]{UC2\\}\\
  \hline
  DEF12.10 & \multiLineCell[t]{UC2\\}\\
  \hline
  DEF12.11 & \multiLineCell[t]{UC2\\}\\
  \hline
  DEF12.2 & \multiLineCell[t]{UC2\\}\\
  \hline
  DEF12.3 & \multiLineCell[t]{UC2\\}\\
  \hline
  DEF12.4 & \multiLineCell[t]{UC2\\}\\
  \hline
  DEF12.5 & \multiLineCell[t]{UC2\\}\\
  \hline
  DEF12.6 & \multiLineCell[t]{UC2\\}\\
  \hline
  DEF12.7 & \multiLineCell[t]{UC2\\}\\
  \hline
  DEF12.7.1 & \multiLineCell[t]{UC2\\}\\
  \hline
  DEF12.7.2 & \multiLineCell[t]{UC2\\}\\
  \hline
  DEF12.8 & \multiLineCell[t]{UC2\\}\\
  \hline
  DEF12.9 & \multiLineCell[t]{UC2\\}\\
  \hline
  OBV13 & \multiLineCell[t]{VERBALE20160119\\}\\
  \hline
  OBV14 & \multiLineCell[t]{Capitolato\\}\\
  \hline
  DEV15 & \multiLineCell[t]{VERBALE20160119\\}\\
  \hline
  DEV16 & \multiLineCell[t]{VERBALE20160119\\}\\
  \hline
  OBQ17 & \multiLineCell[t]{Capitolato\\}\\
  \hline
  OBQ18 & \multiLineCell[t]{INTERNO\\}\\
  \hline
  OBQ18.1 & \multiLineCell[t]{NORME\\}\\
  \hline
  OBF19 & \multiLineCell[t]{Capitolato\\}\\
  \hline
  OBF19.1 & \multiLineCell[t]{Capitolato\\}\\
  \hline
  OBF19.2 & \multiLineCell[t]{Capitolato\\}\\
  \hline
  OBF20 & \multiLineCell[t]{Capitolato\\}\\
  \hline
  OBF20.1 & \multiLineCell[t]{Capitolato\\}\\
  \hline
  OBF20.2 & \multiLineCell[t]{Capitolato\\}\\
  \hline
  OBF21 & \multiLineCell[t]{Capitolato\\}\\
  \hline
  OBF21.1 & \multiLineCell[t]{Capitolato\\}\\
  \hline
  OBF21.2 & \multiLineCell[t]{Capitolato\\}\\
  \hline
  OBF21.3 & \multiLineCell[t]{Capitolato\\}\\
  \hline
  OPF22 & \multiLineCell[t]{Capitolato\\}\\
  \hline
  OPF22.1 & \multiLineCell[t]{Capitolato\\}\\
  \hline
  OPF22.2 & \multiLineCell[t]{Capitolato\\}\\
  \hline
  OPF23 & \multiLineCell[t]{Capitolato\\}\\
  \hline
  OPF23.1 & \multiLineCell[t]{Capitolato\\}\\
  \hline
  OPF23.1.1 & \multiLineCell[t]{Capitolato\\}\\
  \hline
  OPF23.1.2 & \multiLineCell[t]{Capitolato\\}\\
  \hline
  OPF23.2 & \multiLineCell[t]{Capitolato\\}\\
  \hline
  OBF24 & \multiLineCell[t]{Capitolato\\}\\
  \hline
  OBV25 & \multiLineCell[t]{Capitolato\\}\\
  \hline
  OBF26 & \multiLineCell[t]{UC1\\UC1.1\\UC1.2\\UC1.7\\UC1.7.1\\UC1.7.2\\UC1.8\\UC1.9\\UC1.9.1\\UC1.9.2\\VERBALE20160112\\}\\
  \hline
\end{longtable}
\newpage

\section{Tracciamento Fonti-Requisiti}

\begin{longtable}[H]{|p{5.5cm}|p{5.5cm}|}
  \hline
  \textbf{Fonte} & \textbf{Requisiti}\\
  \hline
  Capitolato & \multiLineCell[t]{DEF12\\OBF1\\OBF10\\OBF19\\OBF19.1\\OBF19.2\\OBF20\\OBF20.1\\OBF20.2\\OBF21\\OBF21.1\\OBF21.2\\OBF21.3\\OBF24\\OBF3\\OBF4\\OBF5\\OBF5.1\\OBF5.1.1\\OBF5.1.2\\OBF5.2\\OBQ17\\OBV14\\OBV25\\OPF22\\OPF22.1\\OPF22.2\\OPF23\\OPF23.1\\OPF23.1.1\\OPF23.1.2\\OPF23.2\\}\\
  \hline
  INTERNO & \multiLineCell[t]{OBF18\\}\\
  \hline
  NORME & \multiLineCell[t]{OBQ18\\}\\
  \hline
  UC1 & \multiLineCell[t]{OBF1\\OBF10\\OBF12\\OBF2\\}\\
  \hline
  UC1.1 & \multiLineCell[t]{OBF12\\OBF2\\OBF2.1\\}\\
  \hline
  UC1.10 & \multiLineCell[t]{OBF11\\OBF11.1\\}\\
  \hline
  UC1.10.1 & \multiLineCell[t]{OBF11\\OBF11.1\\}\\
  \hline
  UC1.10.2 & \multiLineCell[t]{OBF11\\OBF11.2\\}\\
  \hline
  UC1.11.3 & \multiLineCell[t]{OBF11.3\\}\\
  \hline
  UC1.2 & \multiLineCell[t]{OBF12\\OBF2\\OBF2.2\\}\\
  \hline
  UC1.3 & \multiLineCell[t]{DEF1.1\\}\\
  \hline
  UC1.3.1 & \multiLineCell[t]{DEF1.1\\DEF1.1.1\\}\\
  \hline
  UC1.3.2 & \multiLineCell[t]{DEF1.1\\DEF1.1.2\\}\\
  \hline
  UC1.4 & \multiLineCell[t]{OBF4\\}\\
  \hline
  UC1.4.1 & \multiLineCell[t]{OBF4.1\\OBF4.1.1\\}\\
  \hline
  UC1.4.3 & \multiLineCell[t]{OBF4.2\\OBF4.3\\}\\
  \hline
  UC1.4.4 & \multiLineCell[t]{OBF4.4\\}\\
  \hline
  UC1.4.5 & \multiLineCell[t]{OBF4.5\\}\\
  \hline
  UC1.4.6 & \multiLineCell[t]{OBF4.6\\}\\
  \hline
  UC1.5 & \multiLineCell[t]{DEF5.3\\OBF5\\}\\
  \hline
  UC1.5.1 & \multiLineCell[t]{DEF5.3\\OBF5\\OBF5.1\\}\\
  \hline
  UC1.5.1.1 & \multiLineCell[t]{DEF5.3\\OBF5\\OBF5.1.1\\}\\
  \hline
  UC1.5.1.2 & \multiLineCell[t]{DEF5.3\\OBF5\\OBF5.1.2\\}\\
  \hline
  UC1.5.2 & \multiLineCell[t]{DEF5.3\\OBF5\\OBF5.2\\}\\
  \hline
  UC1.6 & \multiLineCell[t]{OBF6\\}\\
  \hline
  UC1.6.1 & \multiLineCell[t]{OBF6.1\\}\\
  \hline
  UC1.6.2 & \multiLineCell[t]{OBF6.2\\}\\
  \hline
  UC1.7 & \multiLineCell[t]{DEF7.1\\OBF12\\OBF7\\OBF7.2\\}\\
  \hline
  UC1.7.1 & \multiLineCell[t]{DEF7.1\\OBF12\\OBF7\\}\\
  \hline
  UC1.7.2 & \multiLineCell[t]{OBF12\\OBF7\\OBF7.2\\}\\
  \hline
  UC1.8 & \multiLineCell[t]{OBF12\\OBF8\\}\\
  \hline
  UC1.9 & \multiLineCell[t]{DEF9\\DEF9.1\\DEF9.2\\OBF12\\}\\
  \hline
  UC1.9.1 & \multiLineCell[t]{DEF9\\DEF9.1\\OBF12\\}\\
  \hline
  UC1.9.2 & \multiLineCell[t]{DEF9\\DEF9.2\\OBF12\\}\\
  \hline
  UC2 & \multiLineCell[t]{DEF12\\DEF12.1\\DEF12.10\\DEF12.11\\DEF12.2\\DEF12.3\\DEF12.4\\DEF12.5\\DEF12.6\\DEF12.7\\DEF12.7.1\\DEF12.7.2\\DEF12.8\\DEF12.9\\DEF9\\DEF9.1\\DEF9.2\\OBF2.2\\OBF4\\OBF4.2\\OBF4.3\\OBF4.4\\OBF4.5\\OBF4.6\\OBF5\\OBF5.1\\OBF5.1.1\\OBF5.1.2\\OBF6\\OBF6.1\\OBF6.2\\}\\
  \hline
  VERBALE20160112 & \multiLineCell[t]{DEF12\\DEF5.3\\DEF7.1\\OBF10\\OBF11\\OBF11.1\\OBF11.2\\OBF11.3\\OBF12\\OBF2\\OBF3\\OBF4.1\\OBF4.1.1\\OBF4.2\\OBF4.3\\OBF4.4\\OBF4.5\\OBF4.6\\OBF5\\OBF5.1\\OBF5.1.1\\OBF5.1.2\\OBF5.2\\OBF6\\OBF6.1\\OBF6.2\\OBF7\\OBF7.2\\OBF8\\}\\
  \hline
  VERBALE20160119 & \multiLineCell[t]{DEV15\\DEV16\\OBF10\\OBF2.1\\OBF2.2\\OBF4.1.1\\OBV13\\}\\
  \hline
\end{longtable}

\section{Riepilogo}

\begin{table}[H]
  \centering
  \caption*{Riepilogo requisiti}
  \begin{tabular}{|l|l|l|l|}
    \hline
    Categoria & Obbligatorio & Desiderabile & Opzionale\\
    \hline
    Funzionale & 40 & 22 & 9 \\
    \hline
    di Qualità & 3 & 0 & 0 \\
    \hline
    di Vincolo & 3 & 2 & 0 \\
    \hline
  \end{tabular}
\end{table}
\newpage
\appendix
\listoftables
\listoffigures
\end{document}
