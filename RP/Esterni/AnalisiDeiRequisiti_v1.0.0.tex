\documentclass{scalatekids-article}
\begin{document}
\lfoot{Analisi dei Requisiti 1.0.0}
\newgeometry{top=3.5cm}
\begin{titlepage}
  \begin{center}
    \begin{center}
      \includegraphics[width=10cm]{sklogo.png}
    \end{center}
    \vspace{1cm}
    \begin{Huge}
      \begin{center}
        \textbf{Analisi dei Requisiti}
      \end{center}
    \end{Huge}
    \vspace{11pt}
    \bgroup
    \def\arraystretch{1.3}
    \begin{tabular}{r|l}
      \multicolumn{2}{c}{\textbf{Informazioni sul documento}} \\
      \hline
      \setbox0=\hbox{0.0.1\unskip}\ifdim\wd0=0pt
      \\
      \else
      \textbf{Versione} & 1.0.0\\
      \fi
      \textbf{Redazione} & \multiLineCell[t]{Davide Trevisan\\Marco Boseggia\\Michael Munaro}\\
      \textbf{Verifica} & \multiLineCell[t]{Alberto De Agostini\\Giacomo Vanin}\\
      \textbf{Approvazione} & \multiLineCell[t]{Andrea Giacomo Baldan}\\
      \textbf{Uso} & Esterno\\
      \textbf{Lista di Distribuzione} & \multiLineCell[t]{ScalateKids\\Prof. Tullio Vardanega\\Prof. Riccardo Cardin}\\
    \end{tabular}
    \egroup
    \vspace{22pt}
  \end{center}
\end{titlepage}
\restoregeometry{}
\clearpage
\pagenumbering{Roman}
\setcounter{page}{1}
\begin{flushleft}
  \vspace{0cm}
  {\large\bfseries Diario delle modifiche \par}
\end{flushleft}
\vspace{0cm}
\begin{center}
  \begin{tabular}{| l | l | l | l | p{5cm} |}
    \hline
    Versione & Autore & Ruolo & Data & Descrizione \\
    \hline
    1.1.1 & Andrea Giacomo Baldan & Analista & 2016-03-04 & Modifica casi d'uso UC1, 1.1, 1.2, 1.3, 1.4 e sottocasi d'uso\\
    \hline
    1.1.0 & Marco Boseggia & Verificatore & 2016-02-23 & Verifica sezione 3.2\\
    \hline
    1.0.1 & Francesco Agostini & Analista & 2016-02-22 & Stesura sezione 3.2\\
    \hline
    1.0.0 & Andrea Giacomo Baldan & Responsabile & 2016-01-20 & Approvazione documento\\
    \hline
    0.4.0 & Giacomo Vanin & Verificatore & 2016-01-19 & Verifica sezione Tracciamento\\
    \hline
    0.3.1 & Marco Boseggia & Analista & 2016-01-18 & Stesura sezione Tracciamento\\
    \hline
    0.3.0 & Alberto De Agostini & Verificatore & 2016-01-16 & Verifica sezione Classificazione Requisiti\\
    \hline
    0.2.3 & Michael Munaro & Analista & 2016-01-15 & Integrazione sezione Classificazione Requisiti\\
    \hline
    0.2.1 & Marco Boseggia & Analista & 2016-01-14 & Stesura sezione Classificazione Requisiti\\
    \hline
    0.2.2 & Davide Trevisan & Analista & 2016-01-14 & Correzione sezione Casi d'uso\\
    \hline
    0.2.0 & Giacomo Vanin & Verificatore & 2016-01-12 & Verifica sezione Casi d'uso\\
    \hline
    0.1.1 & Michael Munaro & Analista & 2016-01-10 & Integrazione sezione Casi d'uso\\
    \hline
    0.1.0 & Alberto De Agostini & Verificatore & 2016-01-07 & Verifica sezioni stese in precedenza\\
    \hline
    0.0.4 & Davide Trevisan & Analista & 2016-01-05 & Integrazione sezione Casi d'uso\\
    \hline
    0.0.3 & Marco Boseggia & Analista & 2016-01-04 & Inizio stesura sezione Casi d'uso\\
    \hline
    0.0.2 & Michael Munaro & Analista & 2016-01-03 & Stesura sezione Descrizione Generale\\
    \hline
    0.0.1 & Andrea Giacomo Baldan & Amministratore & 2015-12-16 & Creazione scheletro del documento\\
    \hline
  \end{tabular}
\end{center}
\tableofcontents
\newpage
\pagenumbering{arabic}

\section{Sommario}

\subsection{Scopo del documento}

Il seguente documento ha lo scopo di presentare le funzionalità che il prodotto
\textbf{Actorbase} esporrà all'utilizzatore finale. Inoltre elenca e descrive i
requisiti derivanti dalle suddette funzionalità, emersi durante le riunioni
interne ed esterne con il \textit{Proponente}.
\prodPurpose{}\glossExpl{}

\subsection{Riferimenti}

\subsubsection{Normativi}

\begin{itemize}
\item\textbf{Capitolato d'appalto C1:} \textit{Actorbase: a NoSQL DB based on the Actor model}\\
  \url{http://www.math.unipd.it/~tullio/IS-1/2015/Progetto/C1.pdf}
\item\textbf{Verbale esterno:}\\
  \href{run:../RR/Interni/VerbaleEsterno20160112\_v1.0.0.pdf}{Verbale Esterno 20160112 v1.0.0}\\
  \href{run:../RR/Interni/VerbaleEsterno20160119\_v1.0.0.pdf}{Verbale Esterno 20160119 v1.0.0}
\item\textbf{Norme di Progetto:}\\
  \href{run:../Interni/NormeDiProgetto\_v1.0.0.pdf}{Norme di Progetto v1.0.0}
\end{itemize}

\subsubsection{Informativi}

\begin{itemize}
\item\textbf{Dispense fornite dall'insegnamento Ingegneria del Software mod. A:}\\
  \url{http://www.math.unipd.it/~tullio/IS-1/2015/Dispense/L06.pdf}
\item\textbf{Dispense fornite dall'insegnamento Ingegneria del Software mod. A:}\\
  \url{http://www.math.unipd.it/~tullio/IS-1/2015/Dispense/E02.pdf}
\item\textbf{CAP theorem:}\\
  \url{https://en.wikipedia.org/wiki/CAP_theorem}
\item\textbf{Reactive Manifesto:}\\
  \url{http://www.reactivemanifesto.org/}\\
  \url{https://en.wikipedia.org/wiki/Reactive_programming}
\item\textbf{Amazon DynamoDB:}\\
  \url{http://docs.aws.amazon.com/amazondynamodb/latest/developerguide/Introduction.html}
\item\textbf{Leader election:}\\
  \url{https://en.wikipedia.org/wiki/Leader_election}
\end{itemize}

\section{Descrizione generale}

\subsection{Prospettive del prodotto}

l'obiettivo del prodotto è fornire un \gloss{database} \gloss{NoSQL} di tipo
\gloss{key-value}, quindi senza schemi predefiniti e tabelle; che utilizzi il
modello ad \gloss{attori} per garantire un alto grado di \gloss{concorrenza} e
\gloss{scalabilità orizzontale} idealmente illimitata.

\subsection{Funzioni}

Il software fornirà un sistema di interazione con l'utente basata su \gloss{UI}
testuale direttamente da riga di comando. Questa permetterà di effettuare le operazioni
basilari che ogni \gloss{database} fornisce:
\begin{itemize}
\item Creazione di una o più \gloss{collezioni};
\item Cancellazione di una o più \gloss{collezioni};
\item Modifica del nome di una \gloss{collezione};
\item Ricerca all'interno del \gloss{database} o all'interno di una o più \gloss{collezioni};
\item Inserimento \gloss{item} con o senza sovrascrittura;
\item Cancellazione \gloss{item};
\item Connessione ad altre istanze \textbf{Actorbase}.
\end{itemize}
Il programma inoltre permetterà la richiesta di un aiuto esplicativo sull'uso
dei comandi.

\subsection{Caratteristiche utenza}

Il prodotto è orientato all'utilizzo da parte di clientela interessata a
sviluppo di applicazioni \gloss{reactive}, che trattino grandi moli di dati e
debbano fornire brevissimi tempi di risposta sacrificando dunque le funzionalità
relazionali tipiche dei tradizionali \gloss{database} \textit{SQL} in favore di
un sistema fortemente \gloss{concorrente} e senza l'\gloss{overhead} generato
dagli schemi \gloss{SQL}.

\subsection{Vincoli}

Per far funzionare \textbf{Actorbase} sarà necessario disporre di un computer con
installata la \textit{\gloss{JVM} versione 8}. I sistemi operativi supportati sono:
\begin{itemize}
\item Ubuntu 14.04 o superiore;
\item Windows 10;
\item Mac OSX El Capitan o superiore.
\end{itemize}
I requisiti minimi per Java 8 sono:
\begin{itemize}
\item RAM:\@128 MB;\@
\item Spazio su disco: 124MB;\@
\item Processore: minimo Pentium 2 a 266 MHz;
\end{itemize}
Non ci sono ulteriori richieste hardware o software.

\section{Casi d'uso}

Le aspettative di esperienza utente derivano dall'utilizzo da parte dei
componenti del gruppo di \gloss{Amazon DynamoDB}, un \gloss{database}
\gloss{key-value} sviluppato da \textit{Amazon}, utilizzato come modello di
riferimento per lo sviluppo. I casi d'uso seguono le norme di stesura elencate
nel documento \href{run:../Interni/NormeDiProgetto\_v1.0.0.pdf}{Norme di
  Progetto v1.0.0}.

\subsection{Identificazione attori}

L'analisi del capitolato e le riunioni con il \textit{Proponente} hanno fatto emergere
alcune considerazioni sull'obiettivo che \textbf{Actorbase} si pone di raggiungere.
Trattandosi di un'applicazione distribuita il prodotto finale sarà costituito
da due \gloss{macro} componenti: un lato server e un lato client fornito di interfaccia
comandi testuale per poter dialogare con la base di dati.\\ Suddivisione attori:\\
\textbf{Primari}
\begin{itemize}
\item\textbf{Utente non autenticato:}\\
  Rappresenta l'utente generico che avvia la \gloss{CLI} (Command Line Interface) di \textbf{Actorbase} ma non è ancora connesso al lato server del prodotto;
\item\textbf{Utente autenticato:}\\
  Rappresenta l'utente generico connesso al lato server del prodotto;
\item\textbf{Amministratore:}\\
  Rappresenta un superutente con privilegi di \gloss{visibilità} maggiori rispetto a utenti generici e poteri di modifica sulla \gloss{visibilità} di questi ultimi.
\item\textbf{Programma esterno su JVM:}\\
  Nel caso di utilizzo del sistema da un programma esterno su \gloss{JVM}, sarà quest'ultimo ad essere attore primario rispetto al \gloss{driver} da utilizzare per
  interfacciarsi con il sistema \textbf{Actorbase}. Il sistema \textbf{Actorbase} sarà considerato un attore secondario del \gloss{driver}, in quanto serve allo scopo
  per raggiungere l'obiettivo richiesto dall'attore primario, ovvero l'inserimento dei comandi al sistema.
\end{itemize}
\textbf{Secondari}
\begin{itemize}
\item\textbf{Actorbase:}\\
  Nel caso di utilizzo del sistema da un programma esterno in linguaggio Scala, il sistema \textbf{Actorbase} rappresenta effettivamente l'attore necessario agli attori primari per raggiugere
  lo scopo prefissato, ovvero l'inserimento dei comandi da inviare al sistema stesso mediante utilizzo del \gloss{driver}.
\end{itemize}

\subsubsection{Visibilità}
\label{sec:visibilità}

Rispetto al classico sistema di permessi di lettura e scrittura, \textbf{Actorbase}
prevede un meccanismo semplificato di \gloss{visibilità} secondo le seguenti regole:
\begin{itemize}
\item Ogni utente autenticato inizialmente ha \gloss{visibilità} solo sulle proprie \gloss{collezioni} e/o su \gloss{collezioni} di proprietà altrui su cui è in collaborazione;
\item L'utente amministratore ha \gloss{visibilità} su tutto il sistema \textbf{Actorbase}.
\end{itemize}
La \gloss{visibilità} garantisce privilegi sia in lettura che in scrittura. Di
conseguenza tutte le azioni descritte nei casi d'uso si applicano al livello di
\gloss{visibilità} proprio dell'attore che le effettua.

\subsubsection{Scalabilità orizzontale}

Il vantaggio più grande che il prodotto dovrà offrire sarà la possibilità di
\gloss{scale out} semplificata al massimo. Dovrà quindi essere possibile
aggiungere potenza di calcolo e capacità di carico mediante affiancamento di
macchine aggiuntive. I dati contenuti all'interno della base di dati originaria
dovranno essere redistribuiti e bilanciati tra i nuovi nodi inseriti. In questo
modo è idealmente possibile ottenere infinita capacità di carico semplicemente
aggiungendo risorse hardware qualora necessario.

\subsection{Visione generale}

L'immagine che segue rappresenta uno schema della visione d'insieme del
funzionamento di Actorbase. Al suo interno è possibile identificare l'utente
che, mediante CLI e accesso alla rete, è in grado di interagire con la rete
Actorbase. Essa è formata da un sistema di database distribuiti la cui
distribuzione è assolutamente trasparente all'utente. Ogni elemento del sistema
è in comunicazione con tutte le altre istanze con cui è in grado di interagire
in maniera autonoma.
\begin{figure}[H]
  \begin{center}
    \includegraphics[width=0.7\textwidth,keepaspectratio]{UseCases/ActorbaseNetwork.png}
    \caption*{Actorbase: Visione generale semplificata rete Actorbase}
  \end{center}
\end{figure}

\subsection{UC1: Interazioni tramite CLI}

\begin{figure}[H]
  \begin{center}
    \includegraphics[width=0.6\textwidth,keepaspectratio]{UseCases/UC1.png}
    \caption*{Caso d'uso 1: Interazioni tramite CLI}
  \end{center}
\end{figure}
\textbf{Attori primari:} utente non autenticato, utente autenticato, utente amministratore\\
\textbf{Scopo e descrizione:} L'attore ha appena avviato la \gloss{CLI}, le operazioni che può eseguire sono:
\begin{itemize}
\item Nel caso sia un \textbf{utente non autenticato}:
  \begin{itemize}
  \item Effettuare il login;
  \item Richiedere aiuto a livello generale o a livello di comando.
  \end{itemize}
\item Nel caso sia un \textbf{utente autenticato} o un \textbf{utente amministratore}:
  \begin{itemize}
  \item Richiedere aiuto a livello generale o a livello di comando;
  \item Effettuare operazioni sulle \gloss{collezioni};
  \item Effettuare operazioni sugli \gloss{item};
  \item Effettuare una ricerca su una o più \gloss{collezioni};
  \item Modificare la propria password;
  \item Effettuare il logout.
  \end{itemize}
\item Nel caso sia un \textbf{utente amministratore}:
  \begin{itemize}
  \item Effettuare operazioni di gestione utenti.
  \end{itemize}
\end{itemize}
\textbf{Precondizione:} il server è in ascolto\\
\textbf{Scenario principale utente non autenticato:}
\begin{itemize}
\item UC1.1\ -\ Login;
\item UC1.2\ -\ Richiesta aiuto.
  \begin{itemize}
  \item UC1.2.1\ -\ Aiuto generale;
  \item UC1.2.2\ -\ Aiuto comando.
  \end{itemize}
\end{itemize}
\textbf{Scenario principale utente autenticato o utente amministratore:}
\begin{itemize}
\item UC1.2\ -\ Richiesta aiuto.
  \begin{itemize}
  \item UC1.2.1\ -\ Aiuto generale;
  \item UC1.2.2\ -\ Aiuto comando.
  \end{itemize}
\item UC1.3\ -\ Operazioni sulle \gloss{collezioni};
\item UC1.4\ -\ Operazioni sugli \gloss{item};
\item UC1.5\ -\ Ricerca su una o più \gloss{collezioni};
\item UC1.6\ -\ Modifica password;
\item UC1.7\ -\ Logout.
\end{itemize}
\textbf{Scenario principale utente amministratore:}
\begin{itemize}
\item UC1.8\ -\ Gestione utenti.
\end{itemize}
\textbf{Estensioni:}
\begin{itemize}
\item Condizione: in UC1.1 l'attore inserisce una coppia username password non riconosciuta dal sistema:
  \begin{itemize}
  \item UC1.9\ -\ Visualizzazione errore credenziali errate:
  \end{itemize}
\item Condizione: in UC1.6 l'attore inserisce una password attuale non corretta:
  \begin{itemize}
  \item UC1.10\ -\ Visualizzazione errore password attuale:
  \end{itemize}
\item Condizione: in UC1.6 l'attore inserisce la password nuova che non rispetta i vincoli imposti dal sistema: %TODO scrivere questi vincoli da qualche parte
  \begin{itemize}
  \item UC1.11\ -\ Visualizzazione errore nuova password
  \end{itemize}
\item Condizione: in UC1.6 l'attore inserisce la nuova password per confermarne la modifica e risulta essere non corretta: %TODO vabene così?
  \begin{itemize}
  \item UC1.12\ -\ Visualizzazione errore conferma nuova password
  \end{itemize}
\item Condizione: in UC1.5 l'attore non inserisce la chiave da ricercare:
  \begin{itemize}
  \item UC1.13\ -\ Ricerca dell'intera \gloss{collezione} senza chiave
  \end{itemize}
\item Condizione: in UC1.5 l'attore inserisce una lista \gloss{collezioni} vuota:
  \begin{itemize}
  \item UC1.14\ -\ Ricerca sull'intero database.
  \end{itemize}
\item Condizione: in UC1.5 l'attore inserisce una lista \gloss{collezioni} vuota e non inserisce la chiave da ricercare:
  \begin{itemize}
  \item UC1.15\ -\ Visualizzazione dell'intero database.
  \end{itemize}
\end{itemize}
\textbf{Postcondizione:} Il sistema ha eseguito correttamente il comando inserito dall'attore.


\subsection{UC1.1: Login}

\begin{figure}[H]
  \begin{center}
    \includegraphics[width=0.4\textwidth,keepaspectratio]{UseCases/UC1_1.png}
    \caption*{Caso d'uso 1.1: Login}
  \end{center}
\end{figure}
\textbf{Attori primari:} Utente non autenticato\\
\textbf{Scopo e descrizione:}
L’attore ha appena avviato la \gloss{CLI}, risulta essere non autenticato e può effettuare il login diventando utente
autenticato.\\
\textbf{Precondizione:} Il server è in ascolto.\\
\textbf{Scenario principale:}
\begin{itemize}
\item UC1.1.1\ -\ Inserimento username;
\item UC1.1.2\ -\ Inserimento password.
\end{itemize}
\textbf{Postcondizioni:} L'attore ha acceduto al sistema con successo e risulta essere autenticato.

\subsubsection{UC1.1.1: Inserimento username}

\textbf{Attori primari:} Utente non autenticato\\
\textbf{Scopo e descrizione:}
L'attore ha appena avviato la \gloss{CLI}, risulta essere non autenticato e può inserire il proprio username per effettuare il login.\\
\textbf{Precondizioni:} Il server è in ascolto, l'attore ha inserito il comando per effettuare il login.
\textbf{Scenario principale:}
\begin{itemize}
\item L'attore inserisce il proprio username.
\end{itemize}
\textbf{Postcondizioni:} L'attore ha inserito il proprio username con successo.

\subsubsection{UC1.1.2: Inserimento password}

\textbf{Attori primari:} Utente non autenticato\\
\textbf{Scopo e descrizione:}
L'attore ha appena avviato la \gloss{CLI}, risulta essere non autenticato e può inserire la propria password per effettuare il login.\\
\textbf{Precondizioni:} Il server è in ascolto, l'attore ha inserito il comando per effettuare il login e il suo username.
\textbf{Scenario principale:}
\begin{itemize}
\item L'attore inserisce la propria password.
\end{itemize}
\textbf{Postcondizioni:} L'attore ha inserito la propria password con successo.

\subsection{UC1.2: Richiesta aiuto}

\textbf{Attori primari:} Utente non autenticato, Utente autenticato, Utente amministratore\\
\textbf{Scopo e descrizione:} L'attore intende richiedere aiuto sulla piattaforma o su un comando specifico.\\Nel caso di richiesta di aiuto generale, i diversi tipi di utente riceveranno la lista dei comandi da loro eseguibili\.\
\textbf{Precondizione:} L'attore intende richiedere supporto.\\
\textbf{Scenario principale:}
\begin{itemize}
\item UC1.2.1\ -\ Aiuto generale;
\item UC1.2.2\ -\ Aiuto specifico.
\end{itemize}
\textbf{Postcondizioni:} Il sistema ha risposto producendo in output l'aiuto richiesto.

\subsection{UC1.2.1: Aiuto generale}

\textbf{Attore primario:} Utente non autenticato, Utente autenticato, Utente amministratore\\
\textbf{Scopo e descrizione:} L'attore intende richiedere aiuto generale sull'utilizzo del sistema \textbf{Actorbase}.\\
\textbf{Precondizione:} Il server è in ascolto.\\
\textbf{Scenario principale:}
\begin{itemize}
\item L'attore inserisce il comando di aiuto generale. %TODO va tolto
\end{itemize}
\textbf{Postcondizioni:} Il sistema ha risposto visualizzando l'aiuto richiesto dall'attore in output.

\subsection{UC1.2.2: Aiuto specifico}

\textbf{Attore primario:} Utente non autenticato, Utente autenticato, Utente amministratore\\
\textbf{Scopo e descrizione:} L'attore intende richiedere aiuto su uno specifico comando del sistema \textbf{Actorbase}.\\
\textbf{Precondizione:} Il server è in ascolto.\\
\textbf{Scenario principale:}
\begin{itemize}
\item UC1.2.2.1\ -\ Inserimento nome comando.
\end{itemize}
\textbf{Postcondizioni:} Il sistema ha risposto visualizzando l'aiuto richiesto dall'attore in output.

\subsubsection{UC1.2.2.1: Inserimento nome comando}

\textbf{Attore primario:} Utente non autenticato, Utente autenticato, Utente amministratore\\
\textbf{Scopo e descrizione:} L'attore intende richiedere aiuto specifico sull'utilizzo di un preciso comando del sistema.\\
\textbf{Precondizione:} Il server è in ascolto e l'attore ha inserito il comando di aiuto specifico.\\
\textbf{Scenario principale:}
\begin{itemize}
\item L'attore inserisce il nome comando. %TODO va messo sto caso d'uso? sto sclerando e sono solo le 11.39
\end{itemize}
\textbf{Postcondizioni:} Il sistema ha risposto visualizzando l'aiuto richiesto dall'attore in output.

\subsection{UC1.3: Operazioni sulle collezioni}

\begin{figure}[H]
  \begin{center}
    \includegraphics[width=0.6\textwidth,keepaspectratio]{UseCases/UC1_3.png}
    \caption*{Caso d'uso 1.3: Operazioni sulle \gloss{collezioni}}
  \end{center}
\end{figure}
\textbf{Attore primario:} Utente autenticato, utente amministratore\\
\textbf{Scopo e descrizione:} L'attore desidera effettuare delle operazioni sulle \gloss{collezioni}. Le operazioni previste sono:
\begin{itemize}
\item creazione di una nuova \gloss{collezione};
\item visualizzazione della lista delle \gloss{collezioni} esistenti;
\item cancellazione di \gloss{collezioni} esistenti;
\item modifica del nome di \gloss{collezioni} esistenti;
\item aggiunta \gloss{collaboratore} a \gloss{collezione} esistente;
\item rimozione \gloss{collaboratore} a \gloss{collezione} esistente.
\end{itemize}
\textbf{Precodizione:} Il server è in acolto, l'attore intende effettuare operazioni su \gloss{collezioni}.\\
\textbf{Scenario principale:}
\begin{itemize}
\item UC1.3.1\ -\ Creazione \gloss{collezione};
\item UC1.3.2\ -\ Visualizzazione della lista delle \gloss{collezioni};
\item UC1.3.3\ -\ Cancellazione \gloss{collezione};
\item UC1.3.4\ -\ Modifica nome \gloss{collezione};
\item UC1.3.5\ -\ Aggiunta \gloss{collaboratore};
\item UC1.3.6\ -\ Rimozione \gloss{collaboratore};
\item UC1.3.7\ -\ Esportazione di una o più \gloss{collezioni} su file.
\end{itemize}
\textbf{Estensioni:}
\begin{itemize}
\item Condizione: in UC1.3.1 o UC1.3.4 l'attore inserisce il nome di una \gloss{collezione} già censita all'iterno del sistema:
  \begin{itemize}
  \item UC1.3.10\ -\ Visualizzazione errore \gloss{collezione} già esistente.
  \end{itemize}
\item Condizione: in UC1.3.3 o UC1.3.4 o UC1.3.5 o UC1.3.6 l'attore inserisce il nome di una \gloss{collezione} non censita all'interno del sistema:
  \begin{itemize}
  \item UC1.3.8\ -\ Visualizzazione errore \gloss{collezione} inesistente.
  \end{itemize}
\item Condizione: in UC1.3.5 o UC1.3.6 l'attore inserisce uno username non censito all'interno del sistema:
  \begin{itemize}
  \item UC1.3.9\ -\ Visualizzazione errore username non presente tra i collaboratori.
  \end{itemize}
\item Condizione: in UC1.3.7 l'attore inserisce una lista \gloss{collezioni} vuota:
  \begin{itemize}
  \item UC1.3.11\ -\ Esportazione di tutto il database su file.
  \end{itemize}
\end{itemize}
\textbf{Postcondizioni:} L'operazione richiesta è stata eseguita con successo.

\subsection{UC1.3.1: Creazione collezione}

\textbf{Attore primario:} Utente autenticato, utente amministratore\\
\textbf{Scopo e descrizione:} L'attore richiede la creazione di una \gloss{collezione} all'interno del \gloss{database}, deve inserire il nome della nuova \gloss{collezione}.\\
\textbf{Precodizione:} Il server è in ascolto e l'attore intende creare una \gloss{collezione}.\\
\textbf{Scenario principale:}
\begin{itemize}
\item UC1.3.1.1\ -\ Inserimento nome \gloss{collezione}. % TODO qua bisogna mettere il sotto caso d'uso utente inserisce nome collezione da creare? credo di si
\end{itemize}
\textbf{Postcondizioni:} La \gloss{collezione} è stata creata con successo.

\subsubsection{UC1.3.1.1: Inserimento nome collezione}

\textbf{Attore primario:} Utente autenticato, utente amministratore\\
\textbf{Scopo e descrizione:} L'attore richiede la creazione di una \gloss{collezione} all'interno del \gloss{database}, deve inserire il nome della nuova \gloss{collezione}.\\
\textbf{Precodizione:} Il server è in ascolto e l'attore intende creare una \gloss{collezione}.\\
\textbf{Scenario principale:}
\begin{itemize}
\item L'attore inserisce il nome \gloss{collezione}.
\end{itemize}
\textbf{Postcondizioni:} La \gloss{collezione} è stata creata con successo.

\subsection{UC1.3.2: Visualizzazione della lista delle collezioni}

\textbf{Attore primario:} Utente autenticato, utente amministratore\\
\textbf{Scopo e descrizione:} L'attore intende elencare tutte le \gloss{collezioni} all'interno di \textbf{Actorbase} secondo la propria \gloss{visibilità} (vedi~\ref{sec:visibilità}).\\
\textbf{Precodizione:} Il server è in ascolto e l'attore intende elencare le \gloss{collezioni} al suo interno.\\
\textbf{Postcondizioni:} Il sistema restituisce la lista delle \gloss{collezioni} esistenti secondo la \gloss{visibilità} dell'attore, per ogni \gloss{collezione} viene visualizzato il nome della \gloss{collezione} stessa.

\subsection{UC1.3.3: Modifica nome collezione}

\begin{figure}[H]
  \begin{center}
    \includegraphics[width=0.4\textwidth,keepaspectratio]{UseCases/UC1_3_3.png}
    \caption*{Caso d'uso 1.3.3: Login}
  \end{center}
\end{figure}
\textbf{Attore primario:} Utente autenticato, utente amministratore\\
\textbf{Scopo e descrizione:} L’attore intende modificare il nome di una \gloss{collezione} all'interno del sistema.\\
\textbf{Precodizione:} Il server è in ascolto e l’attore intende modificare il nome di una \gloss{collezione}.\\
\textbf{Scenario principale:}
\begin{itemize}
\item UC1.3.3.1\ -\ Inserimento nome \gloss{collezione};
\item UC1.3.3.2\ -\ Inserimento nuovo nome \gloss{collezione}.
\end{itemize}
\textbf{Postcondizioni:} Il nome della \gloss{collezione} è stato modificato correttamente.

\subsubsection{UC1.3.3.1: Inserimento nome collezione}

\textbf{Attore primario:} Utente autenticato, utente amministratore\\
\textbf{Scopo e descrizione:} L'attore intende modificare il nome di un \gloss{collezione} presente nel \gloss{database}, deve inserire il nome della \gloss{collezione} da modificare.\\
\textbf{Precondzione:} Il server è in ascolto e l'attore intende modificare il nome di una \gloss{collezione}, deve inserire il nome della \gloss{collezione} da modificare.\\
\textbf{Scenario principale:}
\begin{itemize}
\item L'attore inserisce il nome della \gloss{collezione} da modificare.
\end{itemize}
\textbf{Postcondizioni:} Il nome della \gloss{collezione} da modificare è stato inserito

\subsubsection{UC1.3.3.2: Inserimento nuovo nome collezione}

\textbf{Attore primario:} Utente autenticato, utente amministratore\\
\textbf{Scopo e descrizione:} L'attore intende modificare il nome di un \gloss{collezione} presente nel \gloss{database}, deve inserire il nuovo nome della \gloss{collezione} da modificare.\\
\textbf{Precondzione:} Il server è in ascolto e l'attore intende modificare il nome di una \gloss{collezione}, deve inserire il nuovo nome della \gloss{collezione} da modificare.\\
\textbf{Scenario principale:}
\begin{itemize}
\item L'attore inserisce il nuovo nome della \gloss{collezione} da modificare.
\end{itemize}
\textbf{Postcondizioni:} Il nuovo nome della \gloss{collezione} da modificare è stato inserito

\subsection{UC1.3.4: Cancellazione collezione}

\textbf{Attore primario:} Utente autenticato, utente amministratore\\
\textbf{Scopo e descrizione:} L’attore intende cancellare una \gloss{collezione} presente nel \gloss{database}.\\
\textbf{Precodizione:} Il server è in ascolto e l’attore intende cancellare una \gloss{collezione}.\\
\textbf{Scenario principale:}
\begin{itemize}
\item UC1.3.4.1\ -\ Inserimento nome \gloss{collezione}.
\end{itemize}
\textbf{Postcondizioni:} La \gloss{collezione} è stato cancellata correttamente.

\subsubsection{UC1.3.4.1: Inserimento nome collezione}

\textbf{Attore primario:} Utente autenticato, utente amministratore\\
\textbf{Scopo e descrizione:} L'attore intende cancellare il nome di un \gloss{collezione} presente nel \gloss{database}, deve inserire il nome della \gloss{collezione} da cancellare.\\
\textbf{Precondzione:} Il server è in ascolto e l'attore intende cancellare il nome di una \gloss{collezione}, deve inserire il nome della \gloss{collezione} da cancellare.\\
\textbf{Scenario principale:}
\begin{itemize}
\item L'attore inserisce il nome della \gloss{collezione} da cancellare.
\end{itemize}
\textbf{Postcondizioni:} Il nome della \gloss{collezione} da cancellare è stato inserito

\subsection{UC1.3.5: Aggiunta di un collaboratore a collezione}

\begin{figure}[H]
  \begin{center}
    \includegraphics[width=0.4\textwidth,keepaspectratio]{UseCases/UC1_3_5.png}
    \caption*{Caso d'uso 1.3.5: aggiunta di un \gloss{collaboratore} a \gloss{collezione}}
  \end{center}
\end{figure}
\textbf{Attore primario:} Utente autenticato, utente amministratore\\
\textbf{Scopo e descrizione:} L'attore intende aggiungere un \gloss{collaboratore} ad una \gloss{collezione} di sua proprietà. Deve inserire il nome della \gloss{collezione} e lo username dell'utente da aggiungere.\\
\textbf{Precondizione:} Il \gloss{database} è pronto a ricevere comandi e l'attore intende aggiungere un \gloss{collaboratore} ad una \gloss{collezione} di sua proprietà.\\
\textbf{Scenario principale:}
\begin{itemize}
\item UC1.3.5.1\ -\ Inserimento nome \gloss{collezione};
\item UC1.3.5.2\ -\ Inserimento username \gloss{collaboratore}.
\end{itemize}
\textbf{Postcondizioni:} L'utente designato per la collaborazione è stato aggiunto tra i \gloss{collaboratori} della \gloss{collezione} scelta.

\subsubsection{UC1.3.5.1: Inserimento nome collezione}

\textbf{Attore primario:} Utente autenticato, utente amministratore\\
\textbf{Scopo e descrizione:} L'attore intende aggiungere un \gloss{collaboratore} ad una \gloss{collezione} di sua proprietà; deve inserire il nome della \gloss{collezione} in questione.\\
\textbf{Precondizione:} Il server è in ascolto e l'attore ha inserito il comando per l'aggiunta di un \gloss{collaboratore} ad una \gloss{collezione}.
\textbf{Scenario principale:}
\begin{itemize}
\item L'attore inserisce il nome della \gloss{collezione}.
\end{itemize}
\textbf{Postcondizioni:} Il nome della \gloss{collezione} designata all'aggiunta di un \gloss{collaboratore} è stato inserito correttamente.

\subsubsection{UC1.3.5.2: Inserimento username collaboratore}

\textbf{Attore primario:} Utente autenticato, utente amministratore\\
\textbf{Scopo e descrizione:} L'attore intende aggiungere un \gloss{collaboratore} ad una \gloss{collezione} di sua proprietà; deve inserire lo username del \gloss{collaboratore} da aggiugere.\\
\textbf{Precondizione:} Il server è in ascolto, l'attore ha inserito il comando per l'aggiunta di un \gloss{collaboratore} ad una \gloss{collezione} e ha inserito il nome della \gloss{collezione}.\\
\textbf{Scenario principale:}
\begin{itemize}
\item L'attore inserisce il nome del \gloss{collaboratore}.
\end{itemize}
\textbf{Postcondizioni:} Il nome del \gloss{collaboratore} designato all'aggiunta ad una \gloss{collezione} è stato inserito correttamente.

\subsection{UC1.3.6: Rimozione di un collaboratore da collezione}
\begin{figure}[H]
  \begin{center}
    \includegraphics[width=0.4\textwidth,keepaspectratio]{UseCases/UC1_3_6.png}
    \caption*{Caso d'uso 1.3.6: rimozione di un \gloss{collaboratore} da una \gloss{collezione}}
  \end{center}
\end{figure}
\textbf{Attore primario:} Utente autenticato, utente amministratore\\
\textbf{Scopo e descrizione:} L'attore intende rimuovere un \gloss{collaboratore} da una \gloss{collezione} di sua proprietà. Deve inserire il nome della \gloss{collezione} e lo username del \gloss{collaboratore} da rimuovere.\\
\textbf{Precondizione:} Il \gloss{database} è pronto a ricevere comandi e l'attore intende rimuovere un \gloss{collaboratore} da una \gloss{collezione} di sua proprietà.\\
\textbf{Scenario principale:}
\begin{itemize}
\item UC1.3.6.1\ -\ Inserimento nome \gloss{collezione};
\item UC1.3.6.2\ -\ Inserimento username \gloss{collaboratore}.
\end{itemize}
\textbf{Postcondizione:} L'attore designato per la rimozione dai \gloss{collaboratori} è stato rimosso dalla \gloss{collezione} scelta.

\subsubsection{UC1.3.6.1: Inserimento nome collezione}

\textbf{Attore primario:} Utente autenticato, utente amministratore\\
\textbf{Scopo e descrizione:} L'attore intende aggiungere un \gloss{collaboratore} ad una \gloss{collezione} di sua proprietà; deve inserire il nome della \gloss{collezione} in questione.\\
\textbf{Precondizione:} Il server è in ascolto e l'attore ha inserito il comando per la rimozione di un \gloss{collaboratore} da una \gloss{collezione}.
\textbf{Scenario principale:}
\begin{itemize}
\item L'attore inserisce il nome della \gloss{collezione}.
\end{itemize}
\textbf{Postcondizioni:} Il nome della \gloss{collezione} designata all'aggiunta di un \gloss{collaboratore} è stato inserito correttamente.

\subsubsection{UC1.3.6.2: Inserimento username collaboratore}

\textbf{Attore primario:} Utente autenticato, utente amministratore\\
\textbf{Scopo e descrizione:} L'attore intende aggiungere un \gloss{collaboratore} ad una \gloss{collezione} di sua proprietà; deve inserire lo username del \gloss{collaboratore} da aggiugere.\\
\textbf{Precondizione:} Il server è in ascolto, l'attore ha inserito il comando per la rimozione di un \gloss{collaboratore} da una \gloss{collezione} e ha inserito il nome della \gloss{collezione}.\\
\textbf{Scenario principale:}
\begin{itemize}
\item L'attore inserisce il nome del \gloss{collaboratore}.
\end{itemize}
\textbf{Postcondizioni:} Il nome del \gloss{collaboratore} designato all'aggiunta ad una \gloss{collezione} è stato inserito correttamente.

\subsection{UC1.3.7: Esportazione su file}

\textbf{Attore primario:} Utente autenticato, utente amministratore\\
\textbf{Scopo e descrizione:} L'attore intende esportare su file il contenuto di una o più \gloss{collezioni}\\
\textbf{Precondizione:} Il server è in ascolto e l'attore intende esportare il contenuto di una o più \gloss{collezioni} su file.
\textbf{Scenario principale:}
\begin{itemize}
\item UC1.3.7.1\ -\ Inserimento lista di \gloss{collezioni} da esportare. %TODO bisogna emttere il sotto caso d'uso inserire lista \gloss{collezioni}?
\end{itemize}
\textbf{Postcondizione:} Il contenuto delle \gloss{collezioni} selezionate è stato esportato su file.

\subsubsection{UC1.3.7.1: Inserimento lista di collezioni da esportare}

\textbf{Attore primario:} Utente autenticato, utente amministratore\\
\textbf{Scopo e descrizione:} L'attore intende esportare su file il contenuto di una o più \gloss{collezioni}\\
\textbf{Precondizione:} Il server è in ascolto e l'attore intende esportare il contenuto di una o più \gloss{collezioni} su file.
\textbf{Scenario principale:}
\begin{itemize}
\item Inserimento lista di \gloss{collezioni} da esportare.
\end{itemize}
\textbf{Postcondizione:} Il contenuto delle \gloss{collezioni} selezionate è stato esportato su file.

\subsection{UC1.3.8: Visualizzazione errore collezione inesistente}

\textbf{Attori primari:} Utente autenticato, utente amministratore\\
\textbf{Scopo e descrizione:} L'attore ha inserito un comando che necessita del nome della \gloss{collezione} su cui effettuare operazioni, lo username inserito risulta essere inesistente all'interno del sistema.\\
\textbf{Precondizione:}
Il server è in ascolto l'attore ha inserito un comando tra:
\begin{itemize}
\item UC1.3.3\ -\ Modifica nome \gloss{collezione}
\item UC1.3.4\ -\ Cancellazione \gloss{collezione}
\item UC1.3.5\ -\ Aggiunta \gloss{collaboratore} a \gloss{collezione}
\item UC1.3.6\ -\ Rimozione \gloss{collaboratore} da \gloss{collezione}
\end{itemize}
e i parametri per la sua esecuzione, tra cui il nome della \gloss{collezione}, il quale risulta essere inesistene all'interno del sistema.\\
\textbf{Scenario principale:}
\begin{itemize}
\item Viene visualizzato un messaggio di errore esplicativo.
\end{itemize}
\textbf{Postcondizioni:} Il comando inserito non viene eseguito, viene visualizzato il messaggio d'errore esplicativo.

\subsection{UC1.3.9: Visualizzazione errore username inesistente}
% TODO: da rivedere
\textbf{Attori primari:} Utente autenticato, utente amministratore\\
\textbf{Scopo e descrizione:}
L'attore ha inserito il comando per l'aggiunta \gloss{collaboratore} a
\gloss{collezione} o il comando per la rimozione \gloss{collaboratore} a
\gloss{collezione}. Ha inserito il nome della \gloss{collezione} a cui
aggiungere o rimuovere il \gloss{collaboratore} e lo username del
\gloss{collaboratore} stesso, quest'ultimo risulta essere inesistente all'interno
del sistema.\\
\textbf{Precondizioni:} Il server è in ascolto, l'attore ha inserito il comando per l'aggiunta \gloss{collaboratore} a
\gloss{collezione} o il comando per la rimozione \gloss{collaboratore} a
\gloss{collezione}. Ha inserito il nome della \gloss{collezione} a cui
aggiungere o rimuovere il \gloss{collaboratore} e lo username del
\gloss{collaboratore} stesso, quest'ultimo risulta essere inesistente all'interno
del sistema.
\textbf{Scenario principale:}
\begin{itemize}
\item Viene visualizzato un messaggio di errore esplicativo.
\end{itemize}
\textbf{Postcondizioni:} Il comando inserito non viene eseguito, viene visualizzato il messaggio d'errore esplicativo.

\subsection{UC1.3.10: Visualizzazione errore collezione già esistente}

\textbf{Attori primari:} Utente autenticato, utente amministratore\\
\textbf{Scopo e descrizione:}
L'attore ha inserito il comando per la modifica password, ha inserito i parametri necessari (password attuale, nuova password e password di conferma) per modificare la propria password.\\
\textbf{Precondizioni:} Il server è in ascolto, l'attore ha inserito il comando per la modifica password, ha inserito i parametri necessari (password attuale, nuova password e password di conferma) e intende modificare la propria password. Il sistema non riconosce la password di conferma inserita.\\
\textbf{Scenario principale:}
\begin{itemize}
\item Viene visualizzato un messaggio di errore esplicativo.
\end{itemize}
\textbf{Postcondizioni:} La modifica password fallisce, la password dell'attore resta quella di prima.

\subsection{UC1.3.11: Visualizzazione errore collezione già esistente}

\textbf{Attori primari:} Utente autenticato, utente amministratore\\
\textbf{Scopo e descrizione:}
L'attore ha inserito il comando per la modifica password, ha inserito i parametri necessari (password attuale, nuova password e password di conferma) per modificare la propria password.\\
\textbf{Precondizioni:} Il server è in ascolto, l'attore ha inserito il comando per la modifica password, ha inserito i parametri necessari (password attuale, nuova password e password di conferma) e intende modificare la propria password. Il sistema non riconosce la password di conferma inserita.\\
\textbf{Scenario principale:}
\begin{itemize}
\item Viene visualizzato un messaggio di errore esplicativo.
\end{itemize}
\textbf{Postcondizioni:} La modifica password fallisce, la password dell'attore resta quella di prima.

\subsection{UC1.4: Operazioni sugli item}

\begin{figure}[H]
  \begin{center}
    \includegraphics[width=0.6\textwidth,keepaspectratio]{UseCases/UC1_4.png}
    \caption*{Caso d'uso 1.4: Operazioni sugli \gloss{item}}
  \end{center}
\end{figure}
\textbf{Attore primario:} Utente autenticato, utente amministratore\\
\textbf{Scopo e descrizione:} L'attore intende effettuare operazioni sugli \gloss{item} di una \gloss{collezione} esistente. Le operazioni previste sono:
inserimento e cancellazione.\\
\textbf{Precondizione:} Il server è in ascolto e l'attore intende effettuare operazioni di inserimento o cancellazione su una \gloss{collezione} esistente.\\
\textbf{Scenario principale:}
\begin{itemize}
\item UC1.4.1\ -\ Inserimento nuovo \gloss{item};
  \begin{itemize}
  \item UC1.4.1.1\ -\ Inserimento \gloss{item} manuale;
  \item UC1.4.1.2\ -\ Inserimento \gloss{item} da file.
  \end{itemize}
\item UC1.4.2\ -\ Cancellazione \gloss{item}.
\end{itemize}
\textbf{Estensioni:}
\begin{itemize}
\item Condizione: in UC1.4.2 la \gloss{collezione} su cui effettuare cancellazione non sia presente:
  \begin{itemize}
  \item UC1.4.5\ -\ Visualizzazione errore \gloss{collezione} inesistente.
  \end{itemize}
\item Condizione: in UC1.4.1.1 o UC1.4.1.2, in caso di flag di sovrascrittura non attivo, la chiave dell'\gloss{item} da inserire sia già presente:
  \begin{itemize}
  \item UC1.4.4\ -\ Visualizzazione errore chiave già esistente.
  \end{itemize}
\item Condizione: in UC1.4.1 la \gloss{collezione} su cui effettuare l'inserimento sia inesistente:
  \begin{itemize}
  \item UC1.4.6\ -\ Creazione nuova \gloss{collezione}.
  \end{itemize}
\item Condizione: in UC1.4.1.2  il file che contiene gli \gloss{item} da inserire non sia presente nel path inserito:
  \begin{itemize}
  \item UC1.4.3\ -\ Visualizzazione errore file inesistente.
  \end{itemize}
\item Condizione: in UC1.4.1 l'attore inserisce il nome \gloss{collezione} inesistente all'interno del sistema:
  \begin{itemize}
  \item UC1.4.7\ -\ Inserimento con creazione nuova \gloss{collezione}.
  \end{itemize}
\end{itemize}
\textbf{Postcondizione:} L'operazione richiesta è stata effettuata correttamente.

\subsection{UC1.4.1: Inserimento item}

\textbf{Attore primario:} Utente autenticato, utente amministratore\\
\textbf{Scopo e descrizione:} L'attore intende inserire un nuovo \gloss{item} all'interno di una \gloss{collezione} esistente. Deve selezionare la \gloss{collezione} e inserire i parametri di definizione dell'\gloss{item}. Questi sono la chiave, il valore associato e il \gloss{flag} di sovrascrittura. Può scegliere di inserire mediante due modalità distinte:
\begin{itemize}
\item Manuale;
\item Da file.
\end{itemize}
\textbf{Precondizione:} Il server è in ascolto e l'attore intende effettuare un inserimento di un nuovo \gloss{item}.\\
\textbf{Scenario principale:}
\begin{itemize}
\item UC1.4.1.1\ -\ Inserimento \gloss{item} manuale;
\item UC1.4.1.2\ -\ Inserimento \gloss{item} da file.
\end{itemize}
\textbf{Postcondizioni:} L'operazione di inserimento richiesta è stata effettuata correttamente.

\subsection{UC1.4.1.1: Inserimento item manuale}

\begin{figure}[H]
  \begin{center}
    \includegraphics[width=0.5\textwidth,keepaspectratio]{UseCases/UC1_4_1_1.png}
    \caption*{Caso d'uso 1.4.1.1: Inserimento \gloss{item} manuale}
  \end{center}
\end{figure}
\textbf{Attore primario:} Utente autenticato, utente amministratore\\
\textbf{Scopo e descrizione:} L'attore intende inserire manualmente un nuovo \gloss{item} all'interno di una \gloss{collezione} esistente. Deve selezionare la \gloss{collezione} e inserire i parametri di definizione dell'\gloss{item}. Questi sono la chiave, il valore associato e il \gloss{flag} di sovrascrittura.\\
\textbf{Precondizione:} Il server è in ascolto e l'attore intende inserire un nuovo \gloss{item} in modalità manuale.\\
\textbf{Scenario principale:}
\begin{itemize}
\item UC1.4.1.1.1\ -\ Inserimento nome \gloss{collezione};
\item UC1.4.1.1.2\ -\ Inserimento chiave;
\item UC1.4.1.1.3\ -\ Inserimento valore;
\item UC1.4.1.1.4\ -\ Inserimento flag sovrascrittura.
\end{itemize}
\textbf{Postcondizioni:} L'\gloss{item} è stato inserito con successo.

\subsubsection{UC1.4.1.1.1: Inserimento nome collezione}

\textbf{Attore primario:} Utente autenticato, utente amministratore\\
\textbf{Scopo e descrizione:} L'attore intende inserire manualmente un nuovo \gloss{item} all'interno di una \gloss{collezione} esistente. Deve inserire il nome della \gloss{collezione}.\\
\textbf{Precondizione:} Il server è in ascolto e l'attore ha inserito il comando per l'inserimento \gloss{item} manuale.\\
\textbf{Scenario principale:}
\begin{itemize}
\item L'attore inserisce il nome della \gloss{collezione} su cui effettuare l'inserimento.
\end{itemize}
\textbf{Postcondizione:} Il nome della \gloss{collezione} è stato inserito correttamente.

\subsubsection{UC1.4.1.1.2: Inserimento chiave}

\textbf{Attore primario:} Utente autenticato, utente amministratore\\
\textbf{Scopo e descrizione:} L'attore intende inserire manualmente un nuovo \gloss{item} all'interno di una \gloss{collezione} esistente. Deve inserire la chiave dell'\gloss{item}.\\
\textbf{Precondizione:} Il server è in ascolto, l'attore ha inserito il comando per l'inserimento \gloss{item} manuale e ha inserito il nome della \gloss{collezione}.\\
\textbf{Scenario principale:}
\begin{itemize}
\item L'attore inserisce la chiave associata all'\gloss{item}.
\end{itemize}
\textbf{Postcondizioni:} La chiave è stata inserita correttamente.

\subsubsection{UC1.4.1.1.3: Inserimento valore}

\textbf{Attore primario:} Utente autenticato, utente amministratore\\
\textbf{Scopo e descrizione:} L'attore intende inserire manualmente un nuovo \gloss{item} all'interno di una \gloss{collezione} esistente. Deve inserire il valore associato alla chiave dell'\gloss{item}.\\
\textbf{Precondizione:} Il server è in ascolto, l'attore ha inserito il comando per l'inserimento \gloss{item} manuale, ha inserito il nome della \gloss{collezione} e ha inserito la chiave.\\
\textbf{Scenario principale:}
\begin{itemize}
\item L'attore inserisce il valore associato alla chiave dell'\gloss{item}.
\end{itemize}
\textbf{Postcondizioni:} Il valore è stato inserito correttamente.

\subsubsection{UC1.4.1.1.4: Inserimento flag sovrascrittura}

\textbf{Attore primario:} Utente autenticato, utente amministratore\\
\textbf{Scopo e descrizione:} L'attore intende inserire manualmente un nuovo \gloss{item} all'interno di una \gloss{collezione} esistente. Deve inserire il \gloss{flag} di sovrascrittura \gloss{item}.\\
\textbf{Precondizione:} Il server è in ascolto, l'attore ha inserito il comando per l'inserimento \gloss{item} manuale, ha inserito il nome della \gloss{collezione}, ha inserito la chiave e il valore dell'\gloss{item}.\\
\textbf{Scenario principale:}
\begin{itemize}
\item L'attore inserisce il \gloss{flag} di sovrascrittura \gloss{item}. %TODO il flag può fare errore... lo inglobiamo negli errori di battitura?
\end{itemize}
\textbf{Postcondizioni:} Il \gloss{flag} è stato inserito correttamente.

\subsection{UC1.4.1.2: Inserimento item da file}

\textbf{Attore primario:} Utente autenticato, utente amministratore\\
\textbf{Scopo e descrizione:} L'attore intende inserire nuovi \gloss{item} da file. Deve inserire il \gloss{path} del file su filesystem.\\
\textbf{Precondizione:} Il server è in ascolto e l'attore intende inserire nuovi \gloss{item} da file.\\
\textbf{Scenario principale:}
\begin{itemize}
\item L'attore inserisce \gloss{path} del file.
\end{itemize}
\textbf{Postcondizioni:} Gli \gloss{item} scritti all'interno del file inserito sono stati inseriti con successo all'interno del sistema.%TODO vaben così? %Il \gloss{path} è stato inserito correttamente.

\subsubsection{UC1.4.1.2.1: Inserimento path file}

\textbf{Attore primario:} Utente autenticato, utente amministratore\\
\textbf{Scopo e descrizione:} L'attore intende inserire nuovi \gloss{item} da file. Deve inserire il \gloss{path} del file su filesystem.\\
\textbf{Precondizione:} Il server è in ascolto e l'attore ha inserito il comando per l'inserimento di \gloss{item} da file.\\
\textbf{Scenario principale:}
\begin{itemize}
\item L'attore inserisce il \gloss{path} del file.
\end{itemize}
\textbf{Postcondizioni:} Il \gloss{path} è stato inserito correttamente.

\subsection{UC1.4.2: Cancellazione item}

\begin{figure}[H]
  \begin{center}
    \includegraphics[width=0.4\textwidth,keepaspectratio]{UseCases/UC1_4_2.png}
    \caption*{Caso d'uso 1.4.2: Cancellazione \gloss{item}}
  \end{center}
\end{figure}
\textbf{Attore primario:} Utente autenticato, utente amministratore\\
\textbf{Scopo e descrizione:} L'attore intende cancellare un \gloss{item} da una \gloss{collezione}.\\
\textbf{Precondizioni:} Il server è in ascolto e l'attore intende cancellare un \gloss{item}.\\
\textbf{Scenario principale:}
\begin{itemize}
\item UC1.4.2.1\ -\ Inserimento nome \gloss{collezione};
\item UC1.4.2.2\ -\ Inserimento chiave \gloss{item}.
\end{itemize}
\textbf{Postcondizioni:} L'\gloss{item} è stato cancellato correttamente.

\subsection{UC1.4.2.1: Inserimento nome collezione}

\textbf{Attore primario:} Utente autenticato, utente amministratore\\
\textbf{Scopo e descrizione:} L'attore intende cancellare un \gloss{item} all'interno di una \gloss{collezione} esistente. Deve inserire il nome della \gloss{collezione}.\\
\textbf{Precondizione:} Il server è in ascolto e l'attore ha inserito il comando di cancellazione \gloss{item}.\\
\textbf{Scenario principale:}
\begin{itemize}
\item L'attore inserisce il nome della \gloss{collezione} su cui effettuare la cancellazione.
\end{itemize}
\textbf{Postcondizione:} Il nome della \gloss{collezione} è stato inserito correttamente.

\subsection{UC1.4.2.2: Inserimento chiave item}

\textbf{Attore primario:} Utente autenticato, utente amministratore\\
\textbf{Scopo e descrizione:} L'attore intende cancellare un \gloss{item} all'interno di una \gloss{collezione} esistente. Deve inserire la chiave dell'\gloss{item}.\\
\textbf{Precondizione:} Il server è in ascolto, l'attore ha inserito il comando di cancellazione \gloss{item} e ha inserito il nome della \gloss{collezione}.\\
\textbf{Scenario principale:}
\begin{itemize}
\item L'attore inserisce la chiave dell'\gloss{item} da eliminare.
\end{itemize}
\textbf{Postcondizioni:} La chiave è stata inserita correttamente.

\subsection{UC1.4.3: Visualizzazione errore file inesistente}

\textbf{Attore primario:} Utente autenticato, utente amministratore\\
\textbf{Scopo e descrizione:} L'attore, con l'intenzione di inserire \gloss{item} da file, ha inserito un \gloss{path} che porta ad un file inesistente nel filesystem.\\
\textbf{Precondizione:} Il server è in ascolto, l'attore ha inserito un \gloss{path} che punta ad un file inesistente nel filesystem della propria macchina.\\
\textbf{Scenario principale:}
\begin{itemize}
\item Viene visualizzato un messaggio di errore esplicativo.
\end{itemize}
\textbf{Postcondizioni:} L'inserimento per importazione file fallisce.

\subsection{UC1.4.4: Visualizzazione errore chiave già esistente}

\textbf{Attore primario:} Utente autenticato, utente amministratore\\
\textbf{Scopo e descrizione:} L'attore, con l'intenzione di inserire \gloss{item} in modalità manuale o per esportazione da file, ha inserito un \gloss{item} con \gloss{flag} di sovrascrittura
non attivo e una chiave di ricerca già presente all'interno della \gloss{collezione} designata all'inserimento.\\
\textbf{Precondizione:} Il server è in ascolto, l'attore, con l'intezione di inserire un item con \gloss{flag} di sovrascrittura non attivo, ha inserito una chiave di ricerca già presente
all'interno della \gloss{collezione} designata all'inserimento.\\
\textbf{Scenario principale:}
\begin{itemize}
\item Viene visualizzato un messaggio di errore esplicativo.
\end{itemize}
\textbf{Postcondizioni:} L'inserimento nuovo \gloss{item} fallisce.

\subsection{UC1.4.5: Visualizzazione errore collezione inesistente}

\textbf{Attore primario:} Utente autenticato, utente amministratore\\
\textbf{Scopo e descrizione:} L'attore, con l'intenzione di cancellare un \gloss{item} da una \gloss{collezione}, ha inserito il nome di una \gloss{collezione} inesistente
all'interno del sistema.\\
\textbf{Precondizione:} Il server è in ascolto, l'attore, con l'intezione di cancellare un \gloss{item} da una \gloss{collezione}, ha inserito il nome di una \gloss{collezione} inesistente
all'interno del sistema.\\
\textbf{Scenario principale:}
\begin{itemize}
\item Viene visualizzato un messaggio di errore esplicativo.
\end{itemize}
\textbf{Postcondizioni:} La cancellazione \gloss{item} fallisce.

\subsection{UC1.4.6: Visualizzazione errore file malformato}

\textbf{Attore primario:} Utente autenticato, utente amministratore\\
\textbf{Scopo e descrizione:} L'attore, con l'intenzione di inserire \gloss{item} da file, ha inserito un \gloss{path} che porta ad un file inesistente nel filesystem.\\
\textbf{Precondizione:} Il server è in ascolto, l'attore ha inserito un \gloss{path} che punta ad un file inesistente nel filesystem della propria macchina.\\
\textbf{Scenario principale:}
\begin{itemize}
\item Viene visualizzato un messaggio di errore esplicativo.
\end{itemize}
\textbf{Postcondizioni:} L'inserimento per importazione file fallisce.

\subsection{UC1.4.7: Inserimento con creazione nuova collezione}

\textbf{Attore primario:} Utente autenticato, utente amministratore\\
\textbf{Scopo e descrizione:} L'attore intende inserire un nuovo \gloss{item} all'interno di una \gloss{collezione} esistente. Deve selezionare la \gloss{collezione} e inserire i parametri di definizione dell'\gloss{item}. Questi sono la chiave, il valore associato e il \gloss{flag} di sovrascrittura. Può scegliere di inserire mediante due modalità distinte:
\begin{itemize}
\item Manuale;
\item Da file.
\end{itemize}
Il nome della \gloss{collezione} risulta essere inesistente all'interno del sistema.\\
\textbf{Precondizione:} Il server è in ascolto e l'attore intende effettuare un inserimento di un nuovo \gloss{item}, la \gloss{collezione} designata all'inserimento è inesistente all'interno del sistema.\\
\textbf{Scenario principale:}
\begin{itemize}
\item UC1.4.1.1\ -\ Inserimento \gloss{item} manuale;
\item UC1.4.1.2\ -\ Inserimento \gloss{item} da file.
\end{itemize}
\textbf{Postcondizioni:} L'operazione di inserimento richiesta è stata effettuata correttamente, la \gloss{collezione} designata all'inserimento è stata creata.

\subsection{UC1.5: Ricerca}

\begin{figure}[H]
  \begin{center}
    \includegraphics[width=0.4\textwidth,keepaspectratio]{UseCases/UC1_5.png}
    \caption*{Caso d'uso 1.5: Ricerca su una o più \gloss{collezioni}}
  \end{center}
\end{figure}
\textbf{Attore primario:} Utente autenticato, utente amministratore\\
\textbf{Scopo e descrizione:} L'attore intende effettuare una ricerca di \gloss{item} all'interno del sistema; può decidere di effettuare la ricerca su una lista di \gloss{collezioni} eventualmente vuota, in tal caso la
ricerca verrà estesa a tutto il sistema entro i limiti imposti dal grado di visibilità (vedi~\ref{sec:visibilità}) dell'utente.\\
In caso di inserimento di nome \gloss{collezione} inesistente, o in caso di risultati di ricerca nulli, il sistema riporterà una lista risultata vuota.\\ % TODO: stesso per chiavi?
\textbf{Precondizione:} Il server è in ascolto, l'attore intende effettuare una ricerca di \gloss{item} all'interno del sistema.\\
\textbf{Scenario principale:}
\begin{itemize}
\item UC1.5.1\ -\ Inserimento lista \gloss{collezioni};
\item UC1.5.2\ -\ Inserimento chiave di ricerca.
\end{itemize}
\textbf{Postcondizioni:} Il sistema ha riportato la lista dei risultati di ricerca.

\subsection{UC1.5.1: Inserimento lista collezioni}

\textbf{Attore primario:} Utente autenticato, utente amministratore\\
\textbf{Scopo e descrizione:} L'attore intende effettuare una ricerca di \gloss{item} all'interno del sistema; deve inserire la lista delle \gloss{collezioni} su cui effettuare tale ricerca. In caso di lista di ricerca vuota
la funzione di ricerca verrà estesa a tutto il sistema entro i limiti imposti dal grado di visibilità (vedi~\ref{sec:visibilità}) dell'utente.\\
In caso di inserimento di nome \gloss{collezione} inesistente, o in caso di risultati di ricerca nulli, il sistema riporterà una lista risultata vuota.\\ % TODO: stesso per chiavi?
\textbf{Precondizione:} Il server è in ascolto, l'attore intende effettuare una ricerca di \gloss{item} all'interno del  sistema.\\
\textbf{Scenario principale:}
\begin{itemize}
\item Inserimento lista \gloss{collezioni}.
\end{itemize}
\textbf{Postcondizioni:} La lista di \gloss{collezioni} su cui effettuare la ricerca è stata inserita correttamente.

\subsection{UC1.5.2: Inserimento chiave di ricerca}

\textbf{Attore primario:} Utente autenticato, utente amministratore\\
\textbf{Scopo e descrizione:} L'attore intende effettuare una ricerca di \gloss{item} all'interno del sistema; deve inserire la chiave di cui effettuare tale ricerca.\\
In caso di inserimento di nome \gloss{collezione} inesistente, o in caso di risultati di ricerca nulli, il sistema riporterà una lista risultata vuota.\\ % TODO: stesso per chiavi?
\textbf{Precondizione:} Il server è in ascolto, l'attore intende effettuare una ricerca di \gloss{item} all'interno del  sistema.\\
\textbf{Scenario principale:}
\begin{itemize}
\item Inserimento chiave di ricerca
\end{itemize}
\textbf{Postcondizioni:} La chiave da ricerca all'interno del sistema è stata inserita correttamente.

\subsection{UC1.6: Modifica password}
% TODO: dialogo
\begin{figure}[H]
  \begin{center}
    \includegraphics[width=0.4\textwidth,keepaspectratio]{UseCases/UC1_6.png}
    \caption*{Caso d'uso 1.6: Modifica password}
  \end{center}
\end{figure}
\textbf{Attori primari:} Utente autenticato, utente amministratore\\
\textbf{Scopo e descrizione:} L'attore intende modificare la propria password, deve inserire la propria password attuale, inserire poi la nuova password e confermarla.\\
\textbf{Precondizione:} Il server è in ascolto.
\textbf{Scenario principale:}
\begin{itemize}
\item UC1.6.1\ -\ Inserimento password attuale;
\item UC1.6.2\ -\ Inserimento nuova password;
\item UC1.6.3\ -\ Conferma password.
\end{itemize}
\textbf{Postcondizioni:} La password è stata sostituita con la nuova.

\subsubsection{UC1.6.1: Inserimento vecchia password}

\textbf{Attori primari:} Utente autenticato, utente amministratore\\
\textbf{Scopo e descrizione:} L'attore intende modificare la propria password, deve inserire la propria password attuale.\\
\textbf{Precondizione:} Il server è in ascolto e l'attore ha inserito il comando per la modifica password.
\textbf{Scenario principale:}
\begin{itemize}
\item L'attore inserisce la propria password attuale.
\end{itemize}
\textbf{Postcondizioni:} La password è stata inserita.

\subsubsection{UC1.6.2: Inserimento nuova password}

\textbf{Attori primari:} Utente autenticato, utente amministratore\\
\textbf{Scopo e descrizione:} L'attore intende modificare la propria password, deve inserire la nuova password.\\
\textbf{Precondizione:} Il server è in ascolto, l'attore ha inserito il comando per la modifica password e ha inserito la vecchia password.
\textbf{Scenario principale:}
\begin{itemize}
\item L'attore inserisce la nuova password.
\end{itemize}
\textbf{Postcondizioni:} La nuova password è stata inserita.

\subsubsection{UC1.6.3: Conferma password}

\textbf{Attori primari:} Utente autenticato, utente amministratore\\
\textbf{Scopo e descrizione:} L'attore intende modificare la propria password, deve confermare la modifica reinserendo la nuova password.\\
\textbf{Precondizione:} Il server è in ascolto, l'attore ha inserito il comando per la modifica password, ha inserito la vecchia password e ha inserito la nuova password.
\textbf{Scenario principale:}
\begin{itemize}
\item L'attore inserisce la password di conferma.
\end{itemize}
\textbf{Postcondizioni:} La password di conferma è stata inserita.

\subsubsection{UC1.6.4: Visualizzazione errore password attuale}

\textbf{Attori primari:} Utente autenticato, utente amministratore\\
\textbf{Scopo e descrizione:}
L'attore ha inserito il comando per la modifica password, ha inserito i parametri necessari (password attuale, nuova password e password di conferma) per modificare la propria password.\\
\textbf{Precondizioni:} Il server è in ascolto, l'attore ha inserito il comando per la modifica password, ha inserito i parametri necessari (password attuale, nuova password e password di conferma) e intende modificare la propria password. Il sistema non riconosce la password attuale inserita.\\
\textbf{Scenario principale:}
\begin{itemize}
\item Viene visualizzato un messaggio di errore esplicativo.
\end{itemize}
\textbf{Postcondizioni:} La modifica password fallisce, la password dell'attore rimane invariata.

\subsubsection{UC1.6.5: Visualizzazione errore nuova password}

\textbf{Attori primari:} Utente autenticato, utente amministratore\\
\textbf{Scopo e descrizione:}
L'attore ha inserito il comando per la modifica password, ha inserito i parametri necessari (password attuale, nuova password e password di conferma) per modificare la propria password.\\
\textbf{Precondizioni:} Il server è in ascolto, l'attore ha inserito il comando per la modifica password, ha inserito i parametri necessari (password attuale, nuova password e password di conferma) e intende modificare la propria password. Il sistema non riconosce la nuova password inserita.\\
\textbf{Scenario principale:}
\begin{itemize}
\item Viene visualizzato un messaggio di errore esplicativo.
\end{itemize}
\textbf{Postcondizioni:} La modifica password fallisce, la password dell'attore rimane invariata.

\subsubsection{UC1.6.6: Visualizzazione errore nuova password}

\textbf{Attori primari:} Utente autenticato, utente amministratore\\
\textbf{Scopo e descrizione:}
L'attore ha inserito il comando per la modifica password, ha inserito i parametri necessari (password attuale, nuova password e password di conferma) per modificare la propria password.\\
\textbf{Precondizioni:} Il server è in ascolto, l'attore ha inserito il comando per la modifica password, ha inserito i parametri necessari (password attuale, nuova password e password di conferma) e intende modificare la propria password. Il sistema non riconosce la password di conferma inserita.\\
\textbf{Scenario principale:}
\begin{itemize}
\item Viene visualizzato un messaggio di errore esplicativo.
\end{itemize}
\textbf{Postcondizioni:} La modifica password fallisce, la password dell'attore rimane invariata.

\subsection{UC1.7: Logout}

\textbf{Attori primari:} Utente autenticato, utente amministratore\\
\textbf{Scopo e descrizione:} L'attore intende effettuare il logout dal sistema\\
\textbf{Precondizione:} Il server è in ascolto e l'attore intende effettuare il logout dal sistema\\
\textbf{Scenario principale:}
\begin{itemize}
\item L'attore può inserire il comando di logout.
\end{itemize}
\textbf{Postcondizioni:} L'attore ha effettuato il logout, risulta essere un attore non autenticato.

\subsection{UC1.8: Gestione utenti}

\begin{figure}[H]
  \begin{center}
    \includegraphics[width=0.5\textwidth,keepaspectratio]{UseCases/UC1_8.png}
    \caption*{Caso d'uso 4: Operazioni sugli \gloss{item}}
  \end{center}
\end{figure}

\textbf{Attore primario:} Utente amministratore\\
\textbf{Scopo e descrizione:} L'attore intende effettuare operazioni di gestione sugli utenti del sistema; le operazioni possibili sono:
\begin{itemize}
\item Aggiunta di un nuovo utente;
\item Rimozione di un utente;
\item Effettuare un reset della password.
\end{itemize}
\textbf{Precondizione:} Il server è in ascolto e l'amministratore intende effettuare operazioni di gestione sugli utenti.\\
\textbf{Scenario principale:}
\begin{itemize}
\item UC1.8.1\ -\ Aggiunta utente;
\item UC1.8.2\ -\ Rimozione utente;
\item UC1.8.3\ -\ Reset password.
\end{itemize}
\textbf{Estensioni:}
\begin{itemize}
\item Condizione: in UC1.8.1 l'attore inserisce uno username già censito all'interno del sistema:
  \begin{itemize}
  \item UC1.8.4\ -\ Visualizzazione errore username già presente;
  \end{itemize}
\item Condizione: in UC1.8.2 o in UC1.8.3 l'attore inserisce uno username non esistente all'interno del sistema:
  \begin{itemize}
  \item UC1.8.5\ -\ Visualizzazione errore username non presente.
  \end{itemize}
\end{itemize}
\textbf{Postcondizioni:} L'operazione di gestione richiesta è andata a buon fine, gli eventuali cambiamenti sono stati resi persistenti all'interno del sistema.

\subsection{UC1.8.1: Aggiunta utente}

\begin{figure}[H]
  \begin{center}
    \includegraphics[width=0.4\textwidth,keepaspectratio]{UseCases/UC1_8_1.png}
    \caption*{Caso d'uso 4: Operazioni sugli \gloss{item}}
  \end{center}
\end{figure}
\textbf{Attore primario:} Utente amministratore\\
\textbf{Scopo e descrizione:} L'attore intende aggiungere un nuovo utente all'interno del sistema. Deve inserire lo username e confermare l'operazione\\
\textbf{Precondizione:} Il server è in ascolto e l'attore intende aggiungere un nuovo utente all'interno del sistema.\\
\textbf{Scenario principale:}
\begin{itemize}
\item UC1.8.1.1\ -\ Inserimento username;
\item UC1.8.1.2\ -\ Conferma aggiunta utente.
\end{itemize}
\textbf{Postcondizioni:} Il nuovo utente è stato inserito all'interno del sistema con successo.

\subsection{UC1.8.1.1: Inserimento username}

\textbf{Attore primario:} Utente amministratore\\
\textbf{Scopo e descrizione:} L'attore intende aggiungere un nuovo utente all'interno del sistema. Deve inserire lo username del nuovo utente.\\
\textbf{Precondizione:} Il server è in ascolto e l'attore intende aggiungere un nuovo utente all'interno del sistema.\\
\textbf{Scenario principale:}
\begin{itemize}
\item L'attore inserisce username nuovo utente.
\end{itemize}
\textbf{Postcondizioni:} Lo username del nuovo utente è stato inserito con successo.

\subsection{UC1.8.1.2: Conferma aggiunta utente}

\textbf{Attore primario:} Utente amministratore\\
\textbf{Scopo e descrizione:} L'attore intende aggiungere un nuovo utente all'interno del sistema. Deve confermare l'aggiunta del nuovo utente all'interno del sistema.\\
\textbf{Precondizione:} Il server è in ascolto e l'attore intende aggiungere un nuovo utente all'interno del sistema.\\
\textbf{Scenario principale:}
\begin{itemize}
\item L'attore conferma l'aggiunta di un nuovo utente.
\end{itemize}
\textbf{Postcondizioni:} L'aggiunta del nuovo utente è stata confermata con successo.

\subsection{UC1.8.2: Rimozione utente}

\begin{figure}[H]
  \begin{center}
    \includegraphics[width=0.4\textwidth,keepaspectratio]{UseCases/UC1_8_2.png}
    \caption*{Caso d'uso 4: Operazioni sugli \gloss{item}}
  \end{center}
\end{figure}
\textbf{Attore primario:} Utente amministratore\\
\textbf{Scopo e descrizione:} L'attore intende rimuovere un utente dal sistema. Deve inserire lo username e confermare l'operazione\\
\textbf{Precondizione:} Il server è in ascolto e l'attore intende rimuovere un utente dal sistema.\\
\textbf{Scenario principale:}
\begin{itemize}
\item UC1.8.2.1\ -\ Inserimento username;
\item UC1.8.2.2\ -\ Conferma rimozione utente.
\end{itemize}
\textbf{Postcondizioni:} Il nuovo utente è stato rimosso dal sistema con successo.

\subsubsection{UC1.8.2.1: Inserimento username}

\textbf{Attore primario:} Utente amministratore\\
\textbf{Scopo e descrizione:} L'attore intende rimuovere un utente dal sistema. Deve inserire lo username dell'utente da rimuovere.\\
\textbf{Precondizione:} Il server è in ascolto e l'attore intende rimuovere un utente dal sistema.\\
\textbf{Scenario principale:}
\begin{itemize}
\item L'attore inserisce username utente da rimuovere.
\end{itemize}
\textbf{Postcondizioni:} Lo username dell'utente da rimuovere è stato inserito con successo.

\subsubsection{UC1.8.2.2: Conferma rimozione utente}

\textbf{Attore primario:} Utente amministratore\\
\textbf{Scopo e descrizione:} L'attore intende rimuovere un utente dal sistema. Deve confermare la rimozione dell' utente dal sistema.\\
\textbf{Precondizione:} Il server è in ascolto e l'attore intende rimuovere un utente dal sistema.\\
\textbf{Scenario principale:}
\begin{itemize}
\item L'attore conferma la rimozione dell'utente.
\end{itemize}
\textbf{Postcondizioni:} La rimozione dell'utente è stata confermata con successo.

\subsection{UC1.8.3: Reset password}

\begin{figure}[H]
  \begin{center}
    \includegraphics[width=0.4\textwidth,keepaspectratio]{UseCases/UC1_8_3.png}
    \caption*{Caso d'uso 4: Operazioni sugli \gloss{item}}
  \end{center}
\end{figure}
\textbf{Attore primario:} Utente amministratore\\
\textbf{Scopo e descrizione:} L'attore intende effettuare il reset della password di un utente all'interno del sistema. Deve inserire lo username e confermare il reset, la password
verrà modificata al valore prescelto ``actorbase''.\\
\textbf{Precondizione:} Il server è in ascolto e l'attore intende effettuare il reset della password di un utente all'interno del sistema.\\
\textbf{Scenario principale:}
\begin{itemize}
\item UC1.8.3.1\ -\ Inserimento username;
\item UC1.8.3.2\ -\ Conferma reset password.
\end{itemize}
\textbf{Postcondizioni:} La password dell'utente designato è stata modificata in ``actorbase'' con successo.

\subsubsection{UC1.8.3.1: Inserimento username}

\textbf{Attore primario:} Utente amministratore\\
\textbf{Scopo e descrizione:} L'attore intende effettuare il reset della password di un utente all'interno del sistema. Deve inserire lo username e confermare il reset, la password
verrà modificata al valore prescelto ``actorbase''.\\
\textbf{Precondizione:} Il server è in ascolto e l'attore effettuare il reset della password di un utente all'interno del sistema.\\
\textbf{Scenario principale:}
\begin{itemize}
\item L'attore inserisce username utente designato al reset password.
\end{itemize}
\textbf{Postcondizioni:} Lo username dell'utente designato al reset password è stato inserito con successo.

\subsubsection{UC1.8.3.2: Conferma reset}

\textbf{Attore primario:} Utente amministratore\\
\textbf{Scopo e descrizione:} L'attore intende effettuare il reset della password di un utente del sistema. Deve confermare il reset della password dell'utente designato.\\
\textbf{Precondizione:} Il server è in ascolto e l'attore intende effettuare il reset della password di un utente all'interno del sistema.\\
\textbf{Scenario principale:}
\begin{itemize}
\item L'attore conferma il reset della password dell'utente designato.
\end{itemize}
\textbf{Postcondizioni:} Il reset della password dell'utente designato è stato confermato con successo.

\subsection{UC1.8.4: Visualizzazione errore username già presente}

\textbf{Attore primario:} Utente amministratore\\
\textbf{Scopo e descrizione:} L'attore intende aggiungere un nuovo utente al sistema, deve inserire lo username del nuovo utente e confermarne l'aggiunta al sistema.
Lo username inserito risulta essere già censito all'interno del sistema.
\textbf{Precondizione:} Il server è in ascolto e l'attore intende aggiungere un nuovo utente al sistema. Lo username del nuovo utente da inserire risulta essere già
censito all'interno del sistema.
\textbf{Scenario principale:}
\begin{itemize}
\item Viene visualizzato un messaggio di errore esplicativo.
\end{itemize}
\textbf{Postcondizioni:} L'aggiunta nuovo utente al sistema fallisce.

\subsection{UC1.8.5: Visualizzazione errore username inesistente}

\textbf{Attore primario:} Utente amministratore\\
\textbf{Scopo e descrizione:} L'attore intende rimuovere un utente dal sistema o, in alternativa, effettuare un reset della password di un utente. Deve inserire lo username dell'utente designato
alla rimozione dal sistema o al reset della password e confermare l'operazione.
Lo username inserito risulta essere inesistente all'interno del sistema.
\textbf{Precondizione:} Il server è in ascolto e l'attore intende rimuovere un utente dal sistema, o in alternativa, effettuare un reset della password di un utente.
Lo username dell' utente su cui effettuare l'operazione risulta essere inesistente all'interno del sistema.
\textbf{Scenario principale:}
\begin{itemize}
\item Viene visualizzato un messaggio di errore esplicativo.
\end{itemize}
\textbf{Postcondizioni:} L'operazione richiesta fallisce.

\subsection{UC1.9: Visualizzazione errore credenziali errate}

\textbf{Attori primari:} Utente non autenticato\\
\textbf{Scopo e descrizione:}
L'attore ha appena avviato la \gloss{CLI}, risulta essere non autenticato e ha già inserito il proprio username e la propria password per l'autenticazione all'interno del sistema.\\
\textbf{Precondizioni:} Il server è in ascolto, l'attore ha inserito le proprie credenziali (username e password) e intende effettuare il login. Il sistema non riconosce le credenziali.\\
\textbf{Scenario principale:}
\begin{itemize}
\item Viene visualizzato un messaggio di errore esplicativo.
\end{itemize}
\textbf{Postcondizioni:} Il login fallisce, l'attore risulta essere non autenticato.

\subsection{UC1.10: Visualizzazione errore password attuale}

\textbf{Attori primari:} Utente autenticato, utente amministratore\\
\textbf{Scopo e descrizione:} L'attore intende modificare la propria password, ha già inserito la password attuale e la nuova password con conferma.\\
\textbf{Precondizioni:} Il server è in ascolto, l'attore ha inserito la propria password attuale, la nuova password e la conferma. Il sistema non riconosce la password attuale pre modifica.\\
\textbf{Scenario principale:}
\begin{itemize}
\item Viene visualizzato un messaggio di errore esplicativo.
\end{itemize}
\textbf{Postcondizioni:} La modifica della password fallisce, la password rimane invariata.

\subsection{UC1.11: Visualizzazione errore nuova password}

\textbf{Attori primari:} Utente autenticato, utente amministratore\\
\textbf{Scopo e descrizione:} L'attore intende modificare la propria password, ha già inserito la password attuale e la nuova password con conferma.\\
\textbf{Precondizioni:} Il server è in ascolto, l'attore ha inserito la propria password attuale, la nuova password e la conferma. Il sistema non accetta la nuova password in quanto non conforme ai
vincoli imposti. I vincoli imposti dal sistema sono:
\begin{itemize}
\item Lunghezza password minima di 8 caratteri.
\end{itemize}
\textbf{Scenario principale:}
\begin{itemize}
\item Viene visualizzato un messaggio di errore esplicativo.
\end{itemize}
\textbf{Postcondizioni:} La modifica della password fallisce, la password rimane invariata.

\subsection{UC1.12: Visualizzazione errore conferma nuova password}

\textbf{Attori primari:} Utente autenticato, utente amministratore\\
\textbf{Scopo e descrizione:} L'attore intende modificare la propria password, ha già inserito la password attuale e la nuova password con conferma.\\
\textbf{Precondizioni:} Il server è in ascolto, l'attore ha inserito la propria password attuale, la nuova password e la conferma. Il sistema non accetta la conferma nuova password in quanto non conforme ai
vincoli imposti. I vincoli imposti sulla conferma password dal sistema sono:
\begin{itemize}
\item Lunghezza password minima di 8 caratteri;
\item La password da confermare deve corripondere alla password inserita in UC1.6.2.
\end{itemize}
\textbf{Scenario principale:}
\begin{itemize}
\item Viene visualizzato un messaggio di errore esplicativo.
\end{itemize}
\textbf{Postcondizioni:} La modifica della password fallisce, la password rimane invariata.

\subsection{UC1.13: Visualizzazione dell'intera collezione}

\textbf{Attori primari:} Utente autenticato, utente amministratore\\
\textbf{Scopo e descrizione:} L'attore intende effettuare una ricerca su una lista di \gloss{collezioni} senza specificare chiavi di ricerca.\\
\textbf{Precondizioni:} Il server è in ascolto, l'attore ha inserito una lista non vuota di \gloss{collezioni} e nessuna chiave di ricerca.\\
\textbf{Scenario principale:}
\begin{itemize}
\item Vengono visualizzate le coppie chiave-valore dell'intera \gloss{collezione}.
\end{itemize}
\textbf{Postcodizioni:} Viene visualizzata l'intera \gloss{collezione}.

\subsection{UC1.14: Ricerca sull'intero database}

\textbf{Attori primari:} Utente autenticato, utente amministratore\\
\textbf{Scopo e descrizione:} L'attore intende effettuare una ricerca su una lista di \gloss{collezioni} vuota. Deve specificare una chiave di ricerca.\\
\textbf{Precondizioni:} Il server è in ascolto, l'attore ha inserito una lista vuota di \gloss{collezioni} ed una chiave di ricerca.\\
\textbf{Scenario principale:}
\begin{itemize}
\item Vengono ricercate e visualizzate le coppie chiave-valore sull'intero \gloss{database}.
\end{itemize}
\textbf{Postcodizioni:} La ricerca avviene sull'intero \gloss{database} con successo.

\subsection{UC1.15: Visualizzazione dell'intero database}

\textbf{Attori primari:} Utente autenticato, utente amministratore\\
\textbf{Scopo e descrizione:} L'attore intende effettuare una ricerca su una lista di \gloss{collezioni} vuota senza specificare chiavi di ricerca.\\
\textbf{Precondizioni:} Il server è in ascolto, l'attore ha inserito una lista vuota di \gloss{collezioni} e nessuna chiave di ricerca.\\
\textbf{Scenario principale:}
\begin{itemize}
\item Vengono ricercate e visualizzate le coppie chiave-valore sull'intero \gloss{database} comprensivo di tutte le \gloss{collezioni} secondo la visibilità dell'attore (vedi~\ref{sec:visibilità}).
\end{itemize}
\textbf{Postcodizioni:} Viene visualizzato l'intero \gloss{database} comprensivo di tutte le \gloss{collezioni} con successo.

%%%%%%%%%%%%%%%%%%%%%%%%%%%%%%%%%%%%%%%%%%%%%%%%%%%
% 	  QUI INIZIANO I CASI D'USO DEL DRIVER        %
%%%%%%%%%%%%%%%%%%%%%%%%%%%%%%%%%%%%%%%%%%%%%%%%%%%

\subsection{UC2: Interazioni tramite Driver}
\begin{figure}[H]
  \begin{center}
    \includegraphics[width=0.9\textwidth,keepaspectratio]{UseCases/UC2.png}
    \caption*{Caso d'uso 2: Interazioni tramite driver}
  \end{center}
\end{figure}
\textbf{Attori primari:} Programma esterno su \gloss{JVM}\\
\textbf{Attore secondario:} Actorbase\\
\textbf{Scopo e descrizione:} L'attore intende effettuare operazioni al database da un programma \gloss{Scala} esterno, le operazioni che può eseguire sono:
\begin{itemize} % TODO togliere la gerarchia utenti
\item Nel caso in cui l'attore abbia privilegi pari ad un \textbf{utente non autenticato} può:
  \begin{itemize}
  \item Effettuare il login
  \end{itemize}
\item Nel caso in cui l'attore abbia privilegi pari ad un \textbf{utente autenticato} o un \textbf{utente amministratore} può:
  \begin{itemize}
  \item Effettuare operazioni sulle \gloss{collezioni}
  \item Effettuare operazioni sugli \gloss{item}
  \item Effettuare una ricerca su una o più \gloss{collezioni}
  \item Modificare la propria password
  \item Effettuare il logout
  \end{itemize}
\item Nel caso in cui l'attore abbia privilegi pari ad un \textbf{utente amministratore} può:
  \begin{itemize}
  \item Effettuare operazioni di gestione utente
  \end{itemize}
\end{itemize}
\textbf{Precondizione:} il server è in ascolto\\
\textbf{Scenario principale utente non autenticato:}
\begin{itemize}
\item UC2.1\ -\ Login
\end{itemize}
\textbf{Scenario principale utente autenticato o utente amministratore:}
\begin{itemize}
\item UC2.2\ -\ Operazioni sulle collezioni
\item UC2.3\ -\ Operazioni sugli item
\item UC2.4\ -\ Ricerca su una o più collezioni
\item UC2.5\ -\ Modifica password
\item UC2.6\ -\ Logout
\end{itemize}
\textbf{Scenario principale utente amministratore:}
\begin{itemize}
\item UC2.7\ -\ Gestione utenti
\end{itemize}
\textbf{Estensioni:}
\begin{itemize}
\item Condizione: in UC2.1 l'attore inserisce una coppia username password non riconosciuta dal sistema:
  \begin{itemize}
  \item UC2.8\ -\ Lancio eccezione credenziali errate
  \end{itemize}
\item Condizione: in UC2.5 l'attore inserisce una password attuale non corretta:
  \begin{itemize}
  \item UC2.9\ -\ Lancio eccezione password attuale
  \end{itemize}
\item Condizione: in UC2.5 l'attore inserisce la password nuova che non rispetta i vincoli imposti dal sistema: %TODO scrivere questi vincoli da qualche parte
  \begin{itemize}
  \item UC2.10\ -\ Lancio eccezione nuova password
  \end{itemize}
\item Condizione: in UC2.4 l'attore inserisce una lista di collezioni vuota:
  \begin{itemize}
  \item UC2.11\ -\ Ricerca sull'intero database
  \end{itemize}
\item Condizione: in UC2.4 l'attore non inserisce una chiave:
  \begin{itemize}
  \item UC2.12\ -\ Visualizzazione della intera collezione
  \end{itemize}
\item Condizione: in UC2.4 l'attore inserisce una lista di collezioni vuota e non inserisce una chiave:
  \begin{itemize}
  \item UC2.13\ -\ Visualizzazione dell'intero database
  \end{itemize}
\end{itemize}
\textbf{Postcondizione:} Il sistema ha eseguito correttamente il comando inserito dall'utente.

\subsection{UC2.1: Login}

\begin{figure}[H]
  \begin{center}
    \includegraphics[width=0.5\textwidth,keepaspectratio]{UseCases/UC2_1.png}
    \caption*{Caso d'uso 2.1: Login}
  \end{center}
\end{figure}
\textbf{Attori primari:} Programma esterno su \gloss{JVM}\\
\textbf{Attori secondari:} Actorbase\\
\textbf{Scopo e descrizione:}
L'attore intende effettuare login al database da un programma \gloss{Scala} esterno, risulta essere non autenticato e può effettuare il login diventando utente autenticato.\\
\textbf{Precondizione:} Il server è in ascolto.\\
\textbf{Scenario principale:}
\begin{itemize}
\item UC2.1.1 Inserimento username;
\item UC2.1.2 Inserimento password.
\end{itemize}
\textbf{Postcondizioni:} L'attore ha acceduto al sistema con successo e risulta essere autenticato.

\subsubsection{UC2.1.1: Inserimento username}

\textbf{Attori primari:} Programma esterno su \gloss{JVM}\\
\textbf{Attori secondari:} Actorbase\\
\textbf{Scopo e descrizione:}
L'attore intende effettuare login al database da un programma \gloss{Scala} esterno, risulta essere non autenticato e può inserire il proprio username per effettuare il login.\\
\textbf{Precondizioni:} Il server è in ascolto, l'attore ha inserito il comando di login.\\
\textbf{Scenario principale:}
\begin{itemize}
\item L'attore può inserire il proprio username.
\end{itemize}
\textbf{Postcondizioni:} L'attore ha inserito il proprio username.

\subsubsection{UC2.1.2: Inserimento password}

\textbf{Attori primari:} Programma esterno su \gloss{JVM}\\
\textbf{Attori secondari:} Actorbase\\
\textbf{Scopo e descrizione:}
L'attore intende effettuare login al database da un programma \gloss{Scala} esterno, risulta essere non autenticato e può inserire la propria password per effettuare il login.\\
\textbf{Precondizioni:} Il server è in ascolto, l'attore ha inserito il comando di login e il proprio username.\\
\textbf{Scenario principale:}
\begin{itemize}
\item L'attore può inserire la propria password.
\end{itemize}
\textbf{Postcondizioni:} L'attore ha inserito la propria password.

\subsection{UC2.2: Operazioni sulle collezioni}

\begin{figure}[H]
  \begin{center}
    \includegraphics[width=1.0\textwidth,keepaspectratio]{UseCases/UC2_2.png}
    \caption*{Caso d'uso 2.2: Operazioni sulle \gloss{collezioni}}
  \end{center}
\end{figure}
\textbf{Attori primari:} Programma esterno su \gloss{JVM}\\
\textbf{Attori secondari:} Actorbase\\
\textbf{Scopo e descrizione:} L'attore è autenticato e desidera effettuare delle operazioni sulle \gloss{collezioni} da un programma \gloss{Scala} esterno. L'attore può effettuare le seguenti operazioni:
\begin{itemize}
\item Creazione una nuova \gloss{collezione};
\item Visualizzazione della lista delle \gloss{collezioni} esistenti;
\item Cancellazione di \gloss{collezioni} esistenti;
\item Modifica del nome di \gloss{collezioni} esistenti;
\item Aggiunta di collaboratori a \gloss{collezioni} esistenti;
\item Rimozione di collaboratori da \gloss{collezioni} esistenti;
\item Esportazione su file di una o più \gloss{collezioni}.
\end{itemize}
\textbf{Precondizione:} L'attore intende effettuare operazioni su \gloss{collezioni}.\\
\textbf{Scenario principale:}
\begin{itemize}
\item UC2.2.1 Creazione \gloss{collezione};
\item UC2.2.2 Visualizzazione della lista delle \gloss{collezioni};
\item UC2.2.3 Modifica nome \gloss{collezione};
\item UC2.2.4 Cancellazione \gloss{collezione};
\item UC2.2.5 Aggiunta \gloss{collaboratore};
\item UC2.2.6 Rimozione \gloss{collaboratore};
\item UC2.2.7 Esportazione di una o più \gloss{collezioni} su file.
\end{itemize}
\textbf{Estensioni:}
\begin{itemize}
\item Condizione: in UC2.2.1 l'attore inserisce il nome di una \gloss{collezione} già censita all'interno del sistema:
  \begin{itemize}
  \item UC2.2.8 Lancio eccezione \gloss{collezione} già esistente.
  \end{itemize}
\item Condizione: in UC2.2.3  l'attore inserisca come nuovo nome il nome di una \gloss{collezione} già censita all'interno del sistema:
  \begin{itemize}
  \item UC2.2.8 Lancio eccezione \gloss{collezione} già esistente.
  \end{itemize}
\item Condizione: in UC2.2.4 l'attore inserisca il nome di una \gloss{collezione} non censita all'interno del sistema:
  \begin{itemize}
  \item UC2.2.9 Lancio eccezione \gloss{collezione} inesistente.
  \end{itemize}
\item Condizione: in UC2.2.3 l'attore inserisca nome di una \gloss{collezione} non censita all'interno del sistema:
  \begin{itemize}
  \item UC2.2.9 Lancio eccezione \gloss{collezione} inesistente.
  \end{itemize}
\item Condizione: in UC2.2.5 l'attore inserisca il nome di una collezione non censita all'interno del sistema:
  \begin{itemize}
  \item UC2.2.9 Lancio eccezione collezione inesistente.
  \end{itemize}
\item Condizione: in UC2.2.6 l'attore inserisca il nome di una collezione non censita all'interno del sistema:
  \begin{itemize}
  \item UC2.2.9 Lancio eccezione collezione inesistente.
  \end{itemize}
\item Condizione: in UC2.2.5 l'attore inserisca il nome di una collezione non censita all'interno del sistema:
  \begin{itemize}
  \item UC2.2.10 Lancio eccezione username inesistente.
  \end{itemize}
\item Condizione: in UC2.2.6 l'attore inserisca il nome di una collezione non censita all'interno del sistema:
  \begin{itemize}
  \item UC2.2.10 Lancio eccezione username inesistente.
  \end{itemize}
\item Condizione: in UC2.2.7 l'attore inseriesce una lista di collezioni vuota:
  \begin{itemize}
  \item UC2.2.11 Esportazione di tutto il database su file.
  \end{itemize}
\end{itemize}
\textbf{Postcondizioni:} L'operazione richiesta è stata eseguita con successo.

\subsection{UC2.2.1: Creazione collezione}

\textbf{Attori primari:} Programma esterno su \gloss{JVM}\\
\textbf{Attori secondari:} Actorbase\\
\textbf{Scopo e descrizione:} L'attore richiede la creazione di una \gloss{collezione} all'interno del \gloss{database}, deve inserire il nome della nuova \gloss{collezione}.\\
\textbf{Precondizione:} Il server è pronto a ricevere comandi e l'attore intende creare una \gloss{collezione}.\\
\textbf{Scenario principale:}
\begin{itemize}
\item UC2.2.1.1 Inserimento nome \gloss{collezione}
\end{itemize}
\textbf{Postcondizioni:} La \gloss{collezione} è stata creata.

\subsubsection{UC2.2.1.1: Inserimento nome collezione}

\textbf{Attore primario:} Programma esterno su \gloss{JVM}\\
\textbf{Attore secondario:} Actorbase\\
\textbf{Scopo e descrizione:} L'attore richiede la creazione di una \gloss{collezione} all'interno del \gloss{database}, ha inserito il comando di creazione collezione e deve inserire il nome della collezione da aggiungere.\\
\textbf{Precondizione:} Il server è pronto a ricevere comandi e l'attore ha inserito il comando di creazione \gloss{collezione}.\\
\textbf{Scenario principale:}
\begin{itemize}
\item L'attore può inserire il nome della \gloss{collezione}.
\end{itemize}
\textbf{Postcondizioni:} Il nome della collezione è stato inserito correttamente.

\subsection{UC2.2.2: Visualizzazione della lista delle collezioni}

\textbf{Attore primario:} Programma esterno su \gloss{JVM}\\
\textbf{Attore secondario:} Actorbase\\
\textbf{Scopo e descrizione:} L'attore intende elencare tutte le \gloss{collezioni} all'interno di \textbf{Actorbase} secondo la propria \gloss{visibilità}.\\
\textbf{Precondizione:} Il \gloss{database} è pronto a ricevere comandi e l'attore intende avere un elenco delle \gloss{collezioni} al suo interno.\\
\textbf{Scenario principale:}
\begin{itemize}
\item L'attore può inserire il comando per la visualizzazione della lista delle collezioni.
\end{itemize}
\textbf{Postcondizioni:} Il \gloss{database} restituisce la lista delle \gloss{collezioni} esistenti secondo la \gloss{visibilità} dell'attore, per ogni collezione viene visualizzato il nome della collezione stessa.

\subsection{UC2.2.3: Modifica nome collezione}

\begin{figure}[H]
  \begin{center}
    \includegraphics[width=0.7\textwidth,keepaspectratio]{UseCases/UC2_2_3.png}
    \caption*{Caso d'uso 2.2.3: Modifica nome collezione}
  \end{center}
\end{figure}
\textbf{Attore primario:} Programma esterno su \gloss{JVM}\\
\textbf{Attore secondario:} Actorbase\\
\textbf{Scopo e descrizione:} L'attore intende modificare il nome di una \gloss{collezione} presente nel \gloss{database}.\\
\textbf{Precondizione:} Il server è in ascolto e l’attore intende modificare il nome di una \gloss{collezione}.\\
\textbf{Scenario principale:}
\begin{itemize}
\item UC2.2.3.1 Inserimento nome \gloss{collezione};
\item UC2.2.3.2 Inserimento nuovo nome \gloss{collezione}.
\end{itemize}
\textbf{Postcondizioni:} Il nome della \gloss{collezione} è stato modificato correttamente.

\subsubsection{UC2.2.3.1: Inserimento nome collezione}

\textbf{Attore primario:} Programma esterno su \gloss{JVM}\\
\textbf{Attore secondario:} Actorbase\\
\textbf{Scopo e descrizione:} L'attore intende modificare il nome di una \gloss{collezione} presente nel \gloss{database}, deve inserire il nome della collezione da modificare.\\
\textbf{Precondizione:} Il server è in ascolto e l'attore intende modificare il nome di una \gloss{collezione}, ha inserito il comando di modifica nome collezione.\\
\textbf{Scenario principale:}
\begin{itemize}
\item L'attore può inserire il nome della collezione da modificare.
\end{itemize}
\textbf{Postcondizioni:} Il nome della \gloss{collezione} da modificare è stato inserito.

\subsubsection{UC2.2.3.2: Inserimento nuovo nome collezione}

\textbf{Attore primario:} Programma esterno su \gloss{JVM}\\
\textbf{Attore secondario:} Actorbase\\
\textbf{Scopo e descrizione:} L'attore intende modificare il nome di un \gloss{collezione} presente nel \gloss{database}, deve inserire il nome della collezione da modificare.\\
\textbf{Precondizione:} Il server è in ascolto e l'attore intende modificare il nome di una \gloss{collezione}, ha inserito il comando di modifica nome collezione e il nome della collezione da modificare.\\
\textbf{Scenario principale:}
\begin{itemize}
\item L'attore può inserire il nuovo nome della collezione da modificare.
\end{itemize}
\textbf{Postcondizione:} Il nome della \gloss{collezione} è stato modificato.

\subsection{UC2.2.4: Cancellazione collezione}

\textbf{Attore primario:} Programma esterno su \gloss{JVM}\\
\textbf{Attore secondario:} Actorbase\\
\textbf{Scopo e descrizione:} L'attore intende cancellare una \gloss{collezione} presente nel \gloss{database}.\\
\textbf{Precodizione:} Il server è in ascolto e l'attore intende cancellare una collezione.\\
\textbf{Scenario principale:}
\begin{itemize}
\item UC2.2.4.1 Inserimento nome \gloss{collezione};
\end{itemize}
\textbf{Postcondizioni:} La \gloss{collezione} è stato cancellata correttamente.

\subsubsection{UC2.2.4.1: Inserimento nome collezione}

\textbf{Attore primario:} Programma esterno su \gloss{JVM}\\
\textbf{Attore secondario:} Actorbase\\
\textbf{Scopo e descrizione:} L'attore intende cancellare una \gloss{collezione} presente nel \gloss{database}.\\
\textbf{Precondizione:} Il server è in ascolto e l'attore ha inserito il comando per cancellare una \gloss{collezione}.\\
\textbf{Scenario principale:}
\begin{itemize}
\item L'attore può inserire il nome della \gloss{collezione}.
\end{itemize}
\textbf{Postcondizioni:} Il nome della \gloss{collezione} è stato inserito correttamente.

\subsection{UC2.2.5: Aggiunta di un collaboratore a collezione}

\begin{figure}[H]
  \begin{center}
    \includegraphics[width=0.7\textwidth,keepaspectratio]{UseCases/UC2_2_5.png}
    \caption*{Caso d'uso 2.2.5: Aggiunta collaboratore a una collezione}
  \end{center}
\end{figure}
\textbf{Attore primario:} Programma esterno su \gloss{JVM}\\
\textbf{Attore secondario:} Actorbase\\
\textbf{Scopo e descrizione:} L'attore intende aggiungere un \gloss{collaboratore} ad una \gloss{collezione} di sua proprietà. Deve inserire il nome della \gloss{collezione} e lo username dell'utente da aggiungere.\\
\textbf{Precondizione:} Il server è in ascolto e l'attore intende aggiungere un \gloss{collaboratore} ad una \gloss{collezione} di sua proprietà.\\
\textbf{Scenario principale:}
\begin{itemize}
\item UC2.2.5.1 Inserimento nome \gloss{collezione};
\item UC2.2.5.2 Inserimento username \gloss{collaboratore}.
\end{itemize}
\textbf{Postcondizione:} L'attore designato per la collaborazione è stato aggiunto tra i \gloss{collaboratori} della \gloss{collezione} scelta.

\subsubsection{UC2.2.5.1: Inserimento nome collezione}

\textbf{Attore primario:} Programma esterno su \gloss{JVM}\\
\textbf{Attore secondario:} Actorbase\\
\textbf{Scopo e descrizione:} L’attore intende aggiungere un collaboratore ad una \gloss{collezione} presente nel \gloss{database}.\\
\textbf{Precondizione:} Il server è in ascolto e l’attore ha inserito il comando per aggiungere un collaboratore ad una \gloss{collezione}.\\
\textbf{Scenario principale:}
\begin{itemize}
\item L'attore può inserire il nome della \gloss{collezione};
\end{itemize}
\textbf{Postcondizioni:} Il nome della \gloss{collezione} è stato inserito correttamente.

\subsubsection{UC2.2.5.2: Inserimento username}

\textbf{Attore primario:} Programma esterno su \gloss{JVM}\\
\textbf{Attore secondario:} Actorbase\\
\textbf{Scopo e descrizione:} L’attore intende aggiungere un collaboratore ad una \gloss{collezione} presente nel \gloss{database}.\\
\textbf{Precondizione:} Il server è in ascolto e l’attore ha inserito il comando per aggiungere un collaboratore ad una \gloss{collezione} e il nome di quest'ultima.\\
\textbf{Scenario principale:}
\begin{itemize}
\item L'attore può inserire lo username del collaboratore;
\end{itemize}
\textbf{Postcondizioni:} Lo username è stato inserito correttamente.

\subsection{UC2.2.6: Rimozione di un collaboratore da collezione}

\begin{figure}[H]
  \begin{center}
    \includegraphics[width=0.7\textwidth,keepaspectratio]{UseCases/UC2_2_6.png}
    \caption*{Caso d'uso 2.2.6: Rimozione di un collaboratore da collezione}
  \end{center}
\end{figure}
\textbf{Attore primario:} Programma esterno su \gloss{JVM}\\
\textbf{Attore secondario:} Actorbase\\
\textbf{Scopo e descrizione:} L'attore intende rimuovere un \gloss{collaboratore} da una \gloss{collezione} di sua proprietà. Deve inserire il nome della \gloss{collezione} e lo username del \gloss{collaboratore} da rimuovere.\\
\textbf{Precondizione:} Il server è in ascolto e l'attore intende rimuovere un \gloss{collaboratore} da una \gloss{collezione} di sua proprietà.\\
\textbf{Scenario principale:}
\begin{itemize}
\item UC2.2.6.1 Inserimento nome \gloss{collezione};
\item UC2.2.6.2 Inserimento username \gloss{collaboratore}.
\end{itemize}
\textbf{Postcondizione:} L'utente designato per la rimozione dai \gloss{collaboratori} è stato rimosso dalla \gloss{collezione} scelta.

\subsubsection{UC2.2.6.1: Inserimento nome collezione}

\textbf{Attore primario:} Programma esterno su \gloss{JVM}\\
\textbf{Attore secondario:} Actorbase\\
\textbf{Scopo e descrizione:} L’attore intende rimuovere un collaboratore da una \gloss{collezione} presente nel \gloss{database}.\\
\textbf{Precondizione:} Il server è in ascolto e l’attore ha inserito il comando per rimuovere un collaboratore ad una \gloss{collezione}.\\
\textbf{Scenario principale:}
\begin{itemize}
\item L'utente può inserire il nome della \gloss{collezione};
\end{itemize}
\textbf{Postcondizioni:} Il nome della \gloss{collezione} è stato inserito correttamente.

\subsubsection{UC2.2.6.2: Inserimento username}

\textbf{Attore primario:} Programma esterno su \gloss{JVM}\\
\textbf{Attore secondario:} Actorbase\\
\textbf{Scopo e descrizione:} L’attore intende rimuovere un collaboratore da una \gloss{collezione} presente nel \gloss{database}.\\
\textbf{Precondizione:} Il server è in ascolto e l’attore ha inserito il comando per rimuovere un collaboratore ad una \gloss{collezione} e il nome di quest'ultima.\\
\textbf{Scenario principale:}
\begin{itemize}
\item L'attore può inserire lo username del collaboratore da rimuovere;
\end{itemize}
\textbf{Postcondizioni:} Lo username è stato inserito correttamente.

\subsection{UC2.2.7: Esportazione di una o più collezioni su file}

\textbf{Attore primario:} Programma esterno su \gloss{JVM}\\
\textbf{Attore secondario:} Actorbase\\
\textbf{Scopo e descrizione:} L'attore intende esportare su file il contenuto di una o più collezioni da un programma \gloss{Scala} esterno.\\
\textbf{Precondizione:} Il server è in ascolto e l'attore intende esportare il contenuto di una o più collezioni su file.\\
\textbf{Scenario principale:}
\begin{itemize}
\item UC2.2.7.1 Inserimento lista collezioni da esportare.
\end{itemize}
\textbf{Postcondizione:} Il contenuto delle collezioni selezionate è stato esportato su file.

\subsubsection{UC2.2.7.1: Inserimento lista collezioni}

\textbf{Attore primario:} Programma esterno su \gloss{JVM}\\
\textbf{Attore secondario:} Actorbase\\
\textbf{Scopo e descrizione:} L'attore intende esportare su file il contenuto di una o più collezioni da un programma \gloss{Scala} esterno.\\
\textbf{Precondizione:} Il server è in ascolto e l'attore ha inserito il domando per l'esportazione di una o più collezioni su file.\\
\textbf{Scenario principale:}
\begin{itemize}
\item L'attore può inserire la lista collezioni da esportare.
\end{itemize}
\textbf{Postcondizione:} La lista delle collezioni da esportare è stato inserito correttamente

\subsection{UC2.2.8: Lancio eccezione collezione già esistente}

\textbf{Attore primario:} Programma esterno su \gloss{JVM}\\
\textbf{Attore secondario:} Actorbase\\
\textbf{Scopo e descrizione:} L'attore intende creare una nuova collezione o modificare il nome di una già esistente.\\
\textbf{Precondizione:} Il server è in ascolto, l'attore ha inserito il comando di creazione di una collezione o di modifica nome collezione e ha inserito il nome della collezione.\\
\textbf{Scenaio principale:}
\begin{itemize}
\item Viene lanciata una eccezione collezione già esistente.
\end{itemize}
\textbf{Postcondizione:} L'operazione che ha sollevato l'eccezione fallisce.

\subsection{UC2.2.9: Lancio eccezione collezione inesistente}

\textbf{Attore primario:} Programma esterno su \gloss{JVM}\\
\textbf{Attore secondario:} Actorbase\\
\textbf{Scopo e descrizione:} L'attore intende effettuare una delle seguenti operazioni:
\begin{itemize}
\item Modifica nome collezione;
\item Cancellazione collezione;
\item Aggiunta collaboratore a collezione;
\item Rimozione collaboratore da collezione.
\end{itemize}
ma durante l'inserimento dei parametri inserisce un nome collezione non censito all'interno del sistema.\\
\textbf{Precondizione:} Il server è in ascolto, l'attore ha inserito uno dei comandi sopra elencati e ha inserito il nome della collezione .\\
\textbf{Scenaio principale:}
\begin{itemize}
\item Viene lanciata una eccezione collezione inesistente.
\end{itemize}
\textbf{Postcondizione:} L'operazione che ha sollevato l'eccezione fallisce.

\subsection{UC2.2.10: Lancio eccezione username inesistente}

\textbf{Attore primario:} Programma esterno su \gloss{JVM}\\
\textbf{Attore secondario:} Actorbase\\
\textbf{Scopo e descrizione:} L'attore intende aggiungere o rimuovere un collaboratore a una collezione.\\
\textbf{Precondizione:} Il server è in ascolto, l'attore ha inserito il comando di aggiunta o di rimozione collaboratore, ha inserito il nome della collezione e l'username del collaboratore designato.\\
\textbf{Scenaio principale:}
\begin{itemize}
\item Viene lanciata una eccezione username inesistente
\end{itemize}
\textbf{Postcondizione:} L'operazione che ha sollevato l'eccezione fallisce.

\subsection{UC2.2.11: Esportazione di tutto il database su file}

\textbf{Attore primario:} Programma esterno su \gloss{JVM}\\
\textbf{Attore secondario:} Actorbase\\
\textbf{Scopo e descrizione:} L'attore intende esportare su file il contenuto di tutto il database (secondo la sua visibilità) da un programma \gloss{Scala} esterno.\\ % TODO linkare visibilità
\textbf{Precondizione:} Il server è in ascolto e l'attore ha inserito il comando di esportazione di una o più collezioni su file e ha inserito una lista di collezioni vuote.\\
\textbf{Postcondizione:} Il contenuto del database è stato esportato su file.

\subsection{UC2.3: Operazioni sugli item}

\begin{figure}[H]
  \begin{center}
    \includegraphics[width=1.0\textwidth,keepaspectratio]{UseCases/UC2_3.png}
    \caption*{Caso d'uso 2.3: Operazioni sugli \gloss{item}}
  \end{center}
\end{figure}
\textbf{Attore primario:} Programma esterno su \gloss{JVM}\\
\textbf{Attore secondario:} Actorbase\\
\textbf{Scopo e descrizione:} L'attore intende effettuare operazioni sugli \gloss{item} di una \gloss{collezione} esistente da un programma \gloss{Scala} esterno. Le operazioni previste sono:
inserimento e cancellazione.\\
\textbf{Precondizione:} Il \gloss{database} è pronto a ricevere comandi e l'attore intende effettuare operazioni di inserimento o cancellazione su una \gloss{collezione}.\\
\textbf{Scenario principale:}
\begin{itemize}
\item UC2.3.1 Inserimento nuovo \gloss{item};
  \begin{itemize}
  \item UC2.3.1.1 Inserimento \gloss{item} manuale;
  \item UC2.3.1.2 Inserimento \gloss{item} item da file.
  \end{itemize}
\item UC2.3.2 Cancellazione \gloss{item}.
\end{itemize}
\textbf{Estensioni:}
\begin{itemize}
\item Condizione: in UC2.3.2 l'attore inserisce un nome di una la \gloss{collezione} su cui effettuare cancellazione e non è presente nel sistema:
  \begin{itemize}
  \item UC2.3.5 Lancio eccezione collezione inesistente.
  \end{itemize}
\item Condizione: in UC2.3.1 l'attore inserisce la chiave dell'item da inserire ed è già presente nella collezione:
  \begin{itemize}
  \item UC2.3.4 Lancio eccezione chiave già esistente.
  \end{itemize}
\item Condizione: in UC2.3.1 l'attore inserisce un nome collezione non presente nel sistema:
  \begin{itemize}
  \item UC2.3.7 Inserimento con creazione nuova \gloss{collezione}.
  \end{itemize}
\item Condizione: in UC2.3.1.2 l'attore iscerisce un path relativo ad un file non esistente:
  \begin{itemize}
  \item UC2.3.3 Lancio eccezione file inesistente.
  \end{itemize}
\item Condizione: in UC2.3.1.2 l'attore inserisce un path relativo ad un file malformato: %TODO spiegare malformato
  \begin{itemize}
  \item UC2.3.6 Lancio eccezione file malformato
  \end{itemize}
\end{itemize}
\textbf{Postcondizione:} L'operazione richiesta è stata effettuata correttamente.

\subsection{UC2.3.1: Inserimento item}

\textbf{Attore primario:} Programma esterno su \gloss{JVM}\\
\textbf{Attore secondario:} Actorbase\\
\textbf{Scopo e descrizione:} L'attore intende inserire un nuovo \gloss{item} all'interno di una \gloss{collezione} esistente. Deve selezionare la \gloss{collezione} e inserire i parametri di definizione dell'\gloss{item}. Questi sono la chiave, il valore associato e il \gloss{flag} di sovrascrittura. Può scegliere di effettuare l'inserimento mediante due modalità distinte:
\begin{itemize}
\item Manuale;
\item Da file.
\end{itemize}
\textbf{Precondizione:} Il \gloss{database} è pronto a ricevere comandi e l'attore intende effettuare un inserimento di un nuovo \gloss{item}.\\
\textbf{Scenario principale:}
\begin{itemize}
\item UC2.3.1.1 Inserimento item manuale;
\item UC2.3.1.2 Inserimento item da file.
\end{itemize}
\textbf{Estensioni:}
\begin{itemize}
\item Condizione: in UC2.3.1.1  l'attore inserisce la chiave dell'item da inserire ed è già presente nella collezione:
  \begin{itemize}
  \item UC2.3.4 Lancio eccezione chiave già esistente.
  \end{itemize}
\item Condizione: in UC2.3.1.1 l'attore inserisce un nome collezione non presente nel sistema:
  \begin{itemize}
  \item UC2.2.6 Creazione nuova collezione. %TODO no
  \end{itemize}
\item Condizione: in UC2.3.1.2  l'attore inserisce la chiave dell'item da inserire ed è già presente nella collezione:
  \begin{itemize}
  \item UC2.3.4 Lancio eccezione chiave già esistente.
  \end{itemize}
\item Condizione: in UC2.3.1.2 l'attore inserisce un nome collezione non presente nel sistema:
  \begin{itemize}
  \item UC2.3.7 Inserimento con creazione nuova collezione.
  \end{itemize}
\item Condizione: in UC2.3.1.2 l'attore iscerisce un path relativo ad un file non esistente:
  \begin{itemize}
  \item UC2.3.3 Lancio eccezione file inesistente.
  \end{itemize}
\item Condizione: in UC2.3.1.2 l'attore inserisce un path relativo ad un file malformato: %TODO spiegare malformato
  \begin{itemize}
  \item UC2.3.6 Lancio eccezione file malformato
  \end{itemize}
\end{itemize}
\textbf{Postcondizioni:} L'operazione di inserimento richiesta è stata effettuata correttamente.

\subsection{UC2.3.1.1: Inserimento item manuale}

\begin{figure}[H]
  \begin{center}
    \includegraphics[width=0.7\textwidth,keepaspectratio]{UseCases/UC2_3_1_1.png}
    \caption*{Caso d'uso 2.3.1.1: Inserimento item manuale}
  \end{center}
\end{figure}
\textbf{Attore primario:} Programma esterno su \gloss{JVM}\\
\textbf{Attore secondario:} Actorbase\\
\textbf{Scopo e descrizione:} L'attore intende inserire manualmente un nuovo \gloss{item} all'interno di una \gloss{collezione} esistente. Deve selezionare la \gloss{collezione} e inserire i parametri di definizione dell'\gloss{item}. Questi sono la chiave, il valore associato e il \gloss{flag} di sovrascrittura.\\
\textbf{Precondizione:} Il server è in ascolto e l'attore intende inserire un nuovo item in modalità manuale.\\
\textbf{Scenario principale:}
\begin{itemize}
\item UC2.3.1.1.1 Inserimento nome collezione;
\item UC2.3.1.1.2 Inserimento chiave;
\item UC2.3.1.1.3 Inserimento valore;
\item UC2.3.1.1.4 Inserimento flag sovrascrittura.
\end{itemize}
\textbf{Postcondizioni:} L'item è stato inserito con successo.

\subsubsection{UC2.3.1.1.1: Inserimento nome collezione}

\textbf{Attore primario:} Programma esterno su \gloss{JVM}\\
\textbf{Attore secondario:} Actorbase\\
\textbf{Scopo e descrizione:} L'attore intende inserire manualmente un nuovo \gloss{item} all'interno di una \gloss{collezione} esistente. Deve inserire il nome della collezione.\\
\textbf{Precondizione:} Il server è in ascolto e l'attore ha inserito il comando per l'inserimento item manuale.\\
\textbf{Scenario principale:}
\begin{itemize}
\item L'attore può inserire il nome della collezione su cui effettuare l'inserimento.
\end{itemize}
\textbf{Postcondizione:} Il nome della collezione è stato inserito correttamente.

\subsubsection{UC2.3.1.1.2: Inserimento chiave}

\textbf{Attore primario:} Programma esterno su \gloss{JVM}\\
\textbf{Attore secondario:} Actorbase\\
\textbf{Scopo e descrizione:} L'attore intende inserire manualmente un nuovo \gloss{item} all'interno di una \gloss{collezione} esistente. Deve inserire la chiave dell'item.\\
\textbf{Precondizione:} Il server è in ascolto, l'attore ha inserito il comando per l'inserimento item manuale e ha inserito il nome della collezione.\\
\textbf{Scenario principale:}
\begin{itemize}
\item L'attore può inserire la chiave associata all'item.
\end{itemize}
\textbf{Postcondizioni:} La chiave è stata inserita correttamente.

\subsubsection{UC2.3.1.1.3: Inserimento valore}

\textbf{Attore primario:} Programma esterno su \gloss{JVM}\\
\textbf{Attore secondario:} Actorbase\\
\textbf{Scopo e descrizione:} L'attore intende inserire manualmente un nuovo \gloss{item} all'interno di una \gloss{collezione} esistente. Deve inserire il valore associato alla chiave dell'item.\\
\textbf{Precondizione:} Il server è in ascolto, l'attore ha inserito il comando per l'inserimento item manuale, il nome della collezione e la chiave da inserire.\\
\textbf{Scenario principale:}
\begin{itemize}
\item L'attore può inserire il valore associato alla chiave dell'item.
\end{itemize}
\textbf{Postcondizioni:} Il valore è stato inserito correttamente.

\subsubsection{UC2.3.1.1.4: Inserimento flag sovrascrittura}

\textbf{Attore primario:} Programma esterno su \gloss{JVM}\\
\textbf{Attore secondario:} Actorbase\\
\textbf{Scopo e descrizione:} L'attore intende inserire manualmente un nuovo \gloss{item} all'interno di una \gloss{collezione} esistente. Deve inserire il \gloss{flag} di sovrascrittura item.\\
\textbf{Precondizione:} Il server è in ascolto, l'attore ha inserito il comando per l'inserimento item manuale, il nome della collezione, la chiave e il valore dell'\gloss{item}.\\
\textbf{Scenario principale:}
\begin{itemize}
\item L'attore può inserire il \gloss{flag} di sovrascrittura item.
\end{itemize}
\textbf{Postcondizioni:} Il \gloss{flag} è stato inserito correttamente.

\subsection{UC2.3.1.2: Inserimento item da file}

\textbf{Attore primario:} Programma esterno su \gloss{JVM}\\
\textbf{Attore secondario:} Actorbase\\
\textbf{Scopo e descrizione:} L'attore intende inserire nuovi \gloss{item} da file. Deve inserire il \gloss{path} del file su filesystem.\\
\textbf{Precondizione:} Il server è in ascolto e l'attore intende inserire nuovi \gloss{item} da file.\\
\textbf{Scenario principale:}
\begin{itemize}
\item L'attore può inserire il \gloss{path} del file.
\end{itemize}
\textbf{Estensioni:}
\begin{itemize}
\item Nel caso in cui il file non esista nel filesystem:
  \begin{itemize}
  \item UC2.3.3 Lancio eccezione file inesistente.
  \end{itemize}
\item Nel caso in cui il file sia malformato:
  \begin{itemize}
  \item UC2.3.6 Lancio eccezione file malformato.
  \end{itemize}
\end{itemize}
\textbf{Postcondizioni:} Gli item scritti all'interno del file inserito sono stati inseriti con successo all'interno del sistema.%TODO vaben così?

\subsubsection{UC2.3.1.2.1: Inserimento path file} %TODO da tenere?

\textbf{Attore primario:} Programma esterno su \gloss{JVM}\\
\textbf{Attore secondario:} Actorbase\\
\textbf{Scopo e descrizione:} L'attore intende inserire nuovi \gloss{item} da file. Deve inserire il \gloss{path} del file.\\
\textbf{Precondizione:} Il server è in ascolto e l'attore ha inserito il comando per l'inserimento di \gloss{item} da file.\\
\textbf{Scenario principale:}
\begin{itemize}
\item L'attore può inserire il \gloss{path} del file.
\end{itemize}
\textbf{Postcondizioni:} Il \gloss{path} è stato inserito correttamente.

\subsection{UC2.3.2: Cancellazione item}

\begin{figure}[H]
  \begin{center}
    \includegraphics[width=0.5\textwidth,keepaspectratio]{UseCases/UC2_3_2.png}
    \caption*{Caso d'uso 2.3.2: Cancellazione \gloss{item}}
  \end{center}
\end{figure}
\textbf{Attore primario:} Programma esterno su \gloss{JVM}\\
\textbf{Attore secondario:} Actorbase\\
\textbf{Scopo e descrizione:} L'attore intende cancellare un item da una collezione.\\
\textbf{Precondizioni:} Il server è in ascolto e l'attore intende cancellare un item.\\
\textbf{Scenario principale:}
\begin{itemize}
\item UC2.3.2.1 Inserimento nome \gloss{collezione};
\item UC2.3.2.2 Inserimento chiave \gloss{item}.
\end{itemize}
\textbf{Postcondizioni:} L'\gloss{item} è stato cancellato correttamente.

\subsection{UC2.3.2.1: Inserimento nome collezione}

\textbf{Attore primario:} Programma esterno su \gloss{JVM}\\
\textbf{Attore secondario:} Actorbase\\
\textbf{Scopo e descrizione:} L'attore intende cancellare un \gloss{item} all'interno di una \gloss{collezione} esistente. Deve inserire il nome della collezione.\\
\textbf{Precondizione:} Il server è in ascolto e l'attore ha inserito il comando di cancellazione item.\\
\textbf{Scenario principale:}
\begin{itemize}
\item L'attore può inserire il nome della collezione su cui effettuare la cancellazione.
\end{itemize}
\textbf{Postcondizioni:} Il nome della collezione è stato inserito correttamente.

\subsection{UC2.3.2.2: Inserimento chiave item}

\textbf{Attore primario:} Programma esterno su \gloss{JVM}\\
\textbf{Attore secondario:} Actorbase\\
\textbf{Scopo e descrizione:} L'attore intende cancellare un \gloss{item} all'interno di una \gloss{collezione} esistente. Deve inserire la chiave dell'item.\\
\textbf{Precondizione:} Il server è in ascolto, l'attore ha inserito il comando di cancellazione item e ha inserito il nome della collezione.\\
\textbf{Scenario principale:}
\begin{itemize}
\item L'attore può inserire la chiave dell'item da eliminare.
\end{itemize}
\textbf{Postcondizioni:} La chiave è stata inserita correttamente.

\subsection{UC2.3.3: Lancio eccezione file inesistente}

\textbf{Attore primario:} Programma esterno su \gloss{JVM}\\
\textbf{Attore secondario:} Actorbase\\
\textbf{Scopo e descrizione:}
L'attore intende aggiungere item da file.\\
\textbf{Precondizioni:} Il server è in ascolto, l'attore ha inserito il comando di inserimento item da file e il path di un file non presente nel filesystem.\\
\textbf{Scenario principale:}
\begin{itemize}
\item Viene lanciata un'eccezione file inesistente.
\end{itemize}
\textbf{Postcondizioni:} L'aggiunta item fallisce.

\subsection{UC2.3.4: Lancio eccezione chiave già esistente}

\textbf{Attore primario:} Programma esterno su \gloss{JVM}\\
\textbf{Attore secondario:} Actorbase\\
\textbf{Scopo e descrizione:}
L'attore intende aggiungere item a una collezione.\\
\textbf{Precondizioni:} Il server è in ascolto, l'attore ha inserito il comando di inserimento item e i parametri dell'item da inserire tra cui il flag di sovrascrittura inattivo.\\
\textbf{Scenario principale:}
\begin{itemize}
\item Viene lanciata un'eccezione chiave già esistente.
\end{itemize}
\textbf{Postcondizioni:} L'aggiunta item fallisce, l'item non viene aggiunto alla collezione.

\subsection{UC2.3.5: Lancio eccezione collezione inesistente}

\textbf{Attore primario:} Programma esterno su \gloss{JVM}\\
\textbf{Attore secondario:} Actorbase\\
\textbf{Scopo e descrizione:}
L'attore intende cancellare un item da una collezione.\\
\textbf{Precondizioni:} Il server è in ascolto, l'attore ha inserito il comando di cancellazione item e un nome collezione non già presente all'interno del sistema.\\
\textbf{Scenario principale:}
\begin{itemize}
\item Viene lanciata un'eccezione collezione inesistente.
\end{itemize}
\textbf{Postcondizioni:} La cancellazione item fallisce.

\subsection{UC2.3.6: Lancio eccezione file malformato}

\textbf{Attore primario:} Programma esterno su \gloss{JVM}\\
\textbf{Attore secondario:} Actorbase\\
\textbf{Scopo e descrizione:}
L'attore intende aggiungere item da file.\\
\textbf{Precondizioni:} Il server è in ascolto, l'attore ha inserito il comando di inserimento item da file e il path di un file non formato correttamente.\\ %TODO linkare malformato
\textbf{Scenario principale:}
\begin{itemize}
\item Viene lanciata un'eccezione file malformato.
\end{itemize}
\textbf{Postcondizioni:} L'aggiunta item fallisce.

\subsection{UC2.3.7: Inserimento con creazione nuova collezione}

\textbf{Attore primario:} Programma esterno su \gloss{JVM}\\
\textbf{Attore secondario:} Actorbase\\
\textbf{Scopo e descrizione:} L'attore intende inserire un nuovo \gloss{item} all'interno di una \gloss{collezione} esistente. Deve selezionare la \gloss{collezione} e inserire i parametri di definizione dell'\gloss{item}. Questi sono la chiave, il valore associato e il \gloss{flag} di sovrascrittura. Può scegliere di inserire mediante due modalità distinte:
\begin{itemize}
\item Manuale;
\item Da file.
\end{itemize}
Il nome della \gloss{collezione} risulta essere inesistente all'interno del sistema.\\
\textbf{Precondizione:} Il server è in ascolto e l'attore intende effettuare un inserimento di un nuovo \gloss{item}, la \gloss{collezione} designata all'inserimento è inesistente all'interno del sistema.\\
\textbf{Scenario principale:}
\begin{itemize}
\item UC2.3.1.1\ -\ Inserimento \gloss{item} manuale;
\item UC2.3.1.2\ -\ Inserimento \gloss{item} da file.
\end{itemize}
\textbf{Postcondizioni:} L'operazione di inserimento richiesta è stata effettuata correttamente, la \gloss{collezione} designata all'inserimento è stata creata.

\subsection{UC2.4: Ricerca}

\begin{figure}[H]
  \begin{center}
    \includegraphics[width=0.6\textwidth,keepaspectratio]{UseCases/UC2_4.png}
    \caption*{Caso d'uso 2.4: Ricerca \gloss{item}}
  \end{center}
\end{figure}
\textbf{Attore primario:} Programma esterno su \gloss{JVM}\\
\textbf{Attore secondario:} Actorbase\\
\textbf{Scopo e descrizione:} L'attore intende effettuare una ricerca di \gloss{item} all'interno del sistema da un programma Scala esterno.\\
In caso di inserimento di nome \gloss{collezione} inesistente, o in caso di risultati di ricerca nulli, il sistema riporterà una lista risultata vuota.\\ % TODO: stesso per chiavi?
\textbf{Precondizione:} Il server è in ascolto, l'attore intende effettuare una ricerca di \gloss{item} all'interno del sistema.\\
\textbf{Scenario principale:}
\begin{itemize}
\item UC2.4.1 Inserimento lista collezioni;
\item UC2.4.2 Inserimento chiave di ricerca.
\end{itemize}
\textbf{Postcondizioni:} Il sistema ha riportato la lista dei risultati di ricerca.

\subsubsection{UC2.4.1: Inserimento lista collezioni}

\textbf{Attore primario:} Programma esterno su \gloss{JVM}\\
\textbf{Attore secondario:} Actorbase\\
\textbf{Scopo e descrizione:} L'attore intende effettuare una ricerca di \gloss{item} all'interno del sistema; deve inserire la lista delle collezioni su cui effettuare tale ricerca.\\
\textbf{Precondizione:} Il server è in ascolto, l'attore ha inserito il comando ricerca.\\
\textbf{Scenario principale:}
\begin{itemize}
\item L'attore può inserire la lista delle \gloss{collezioni}.
\end{itemize}
\textbf{Postcondizioni:} La lista di collezioni su cui effettuare la ricerca è stata inserita correttamente.

\subsubsection{UC2.4.2: Inserimento chiave di ricerca}

\textbf{Attore primario:} Programma esterno su \gloss{JVM}\\
\textbf{Attore secondario:} Actorbase\\
\textbf{Scopo e descrizione:} L'attore intende effettuare una ricerca di \gloss{item} all'interno del sistema; deve inserire la chiave di cui effettuare tale ricerca.\\
\textbf{Precondizione:} Il server è in ascolto, l'attore ha inserito il comando ricerca e la lista di collezioni su cui effettuare la ricerca.\\
\textbf{Scenario principale:}
\begin{itemize}
\item L'utente può inserire la chiave di ricerca
\end{itemize}
\textbf{Postcondizioni:} La chiave da ricerca all'interno del sistema è stata inserita correttamente.

\subsection{UC2.5: Modifica password}

% TODO: dialogo
\begin{figure}[H]
  \begin{center}
    \includegraphics[width=0.7\textwidth,keepaspectratio]{UseCases/UC2_5.png}
    \caption*{Caso d'uso 2.5: Modifica password}
  \end{center}
\end{figure}
\textbf{Attore primario:} Programma esterno su \gloss{JVM}\\
\textbf{Attore secondario:} Actorbase\\
\textbf{Scopo e descrizione:} L'attore intende modificare la propria password da un programma \gloss{Scala} esterno, deve inserire la propria password attuale e la nuova password.\\
\textbf{Precondizione:} Il server è in ascolto.\\
\textbf{Scenario principale:}
\begin{itemize}
\item UC2.5.1 Inserimento password attuale;
\item UC2.5.2 Inserimento nuova password;
\end{itemize}
\textbf{Postcondizioni:} La password è stata sostituita con la nuova.

\subsubsection{UC2.5.1: Inserimento password attuale}

\textbf{Attore primario:} Programma esterno su \gloss{JVM}\\
\textbf{Attore secondario:} Actorbase\\
\textbf{Scopo e descrizione:} L'attore intende modificare la propria password, deve inserire la propria password attuale.\\
\textbf{Precondizione:} Il server è in ascolto e l'attore ha inserito il comando per la modifica password.\\
\textbf{Scenario principale:}
\begin{itemize}
\item L'attore può inserire la propria password attuale.
\end{itemize}
\textbf{Postcondizioni:} La password è stata inserita.

\subsubsection{UC2.5.2: Inserimento nuova password}

\textbf{Attore primario:} Programma esterno su \gloss{JVM}\\
\textbf{Attore secondario:} Actorbase\\
\textbf{Scopo e descrizione:} L'attore intende modificare la propria password, deve inserire la nuova password.\\
\textbf{Precondizione:} Il server è in ascolto, l'attore ha inserito il comando per la modifica password e ha inserito la vecchia password.
\textbf{Scenario principale:}
\begin{itemize}
\item L'attore può inserire la nuova password.
\end{itemize}
\textbf{Postcondizioni:} La nuova password è stata inserita.

\subsection{UC2.6: Logout}

\textbf{Attore primario:} Programma esterno su \gloss{JVM}\\
\textbf{Attore secondario:} Actorbase\\
\textbf{Scopo e descrizione:} L'attore intende effettuare il logout dal sistema\\
\textbf{Precondizione:} Il server è in ascolto e l'attore intende effettuare il logout dal sistema\\
\textbf{Scenario principale:}
\begin{itemize}
\item L'attore può inserire il comando di logout.
\end{itemize}
\textbf{Postcondizioni:} L'attore ha effettuato il logout.

\subsection{UC2.7: Gestione utenti}

\begin{figure}[H]
  \begin{center}
    \includegraphics[width=0.9\textwidth,keepaspectratio]{UseCases/UC2_7.png}
    \caption*{Caso d'uso 2.7: Gestione utenti}
  \end{center}
\end{figure}
\textbf{Attore primario:} Programma esterno su \gloss{JVM}\\
\textbf{Attore secondario:} Actorbase\\
\textbf{Scopo e descrizione:} L'attore intende effettuare operazioni di gestione sugli utenti del sistema; le operazioni possibili sono:
\begin{itemize}
\item Aggiunta di un nuovo utente;
\item Rimozione di un utente;
\item Effettuare un reset della password.
\end{itemize}
\textbf{Precondizione:} Il server è in ascolto, l'attore ha privilegi amministrativi e intende effettuare operazioni di gestione sugli utenti.\\
\textbf{Scenario principale:}
\begin{itemize}
\item UC2.7.1\ -\ Aggiunta utente;
\item UC2.7.2\ -\ Rimozione utente;
\item UC2.7.3\ -\ Reset password.
\end{itemize}
\textbf{Estensioni:}
\begin{itemize}
\item Condizione: in UC2.7.1 l'attore inserisce uno username già censito all'interno del sistema:
  \begin{itemize}
  \item UC2.7.4\ -\ Lancio eccezione username già presente;
  \end{itemize}
\item Condizione: in UC2.7.2 o in UC2.7.3 l'attore inserisce uno username non esistente all'interno del sistema:
  \begin{itemize}
  \item UC2.7.5\ -\ Lancio eccezione username non presente.
  \end{itemize}
\end{itemize}
\textbf{Postcondizioni:} L'operazione di gestione richiesta è andata a buon fine, gli eventuali cambiamenti sono stati resi persistenti all'interno del sistema.

\subsection{UC2.7.1: Aggiunta utente}

\textbf{Attore primario:} Programma esterno su \gloss{JVM}\\
\textbf{Attore secondario:} Actorbase\\
\textbf{Scopo e descrizione:} L'attore intende aggiungere un nuovo utente all'interno del sistema. Deve inserire lo username e confermare l'operazione\\
\textbf{Precondizione:} Il server è in ascolto, l'attore ha privilegi amministrativi e intende aggiungere un nuovo utente all'interno del sistema.\\
\textbf{Scenario principale:}
\begin{itemize}
\item UC2.7.1.1\ -\ Inserimento username;
\end{itemize}
\textbf{Postcondizioni:} Il nuovo utente è stato inserito all'interno del sistema con successo.

\subsubsection{UC2.7.1.1: Inserimento username}

\textbf{Attore primario:} Programma esterno su \gloss{JVM}\\
\textbf{Attore secondario:} Actorbase\\
\textbf{Scopo e descrizione:} L'attore intende aggiungere un nuovo utente all'interno del sistema. Deve inserire lo username del nuovo utente.\\
\textbf{Precondizione:} Il server è in ascolto, l'attore ha privilegi amministrativi e ha inserito il comando di aggiunta utente.\\
\textbf{Scenario principale:}
\begin{itemize}
\item L'attore può inserire lo username del nuovo utente.
\end{itemize}
\textbf{Postcondizioni:} Lo username del nuovo utente è stato inserito con successo.

\subsection{UC2.7.2: Rimozione utente}

\textbf{Attore primario:} Programma esterno su \gloss{JVM}\\
\textbf{Attore secondario:} Actorbase\\
\textbf{Scopo e descrizione:} L'attore intende rimuovere un utente dal sistema. Deve inserire lo username di quest'ultimo.\\
\textbf{Precondizione:} Il server è in ascolto, l'attore ha privilegi amministrativi e intende rimuovere un utente dal sistema.\\
\textbf{Scenario principale:}
\begin{itemize}
\item UC2.7.2.1\ -\ Inserimento username.
\end{itemize}
\textbf{Postcondizioni:} L'utente è stato rimosso dal sistema con successo.

\subsubsection{UC2.7.2.1: Inserimento username}

\textbf{Attore primario:} Programma esterno su \gloss{JVM}\\
\textbf{Attore secondario:} Actorbase\\
\textbf{Scopo e descrizione:} L'attore intende rimuovere un utente dal sistema. Deve inserire lo username dell'utente da rimuovere.\\
\textbf{Precondizione:} Il server è in ascolto, l'attore ha privilegi amministrativi ed ha inserito il comando per la rimozione utente.\\
\textbf{Scenario principale:}
\begin{itemize}
\item L'attore può inserire lo username dell'utente da rimuovere.
\end{itemize}
\textbf{Postcondizioni:} Lo username dell'utente da rimuovere è stato inserito con successo.

\subsection{UC2.7.3: Reset password}

\textbf{Attore primario:} Programma esterno su \gloss{JVM}\\
\textbf{Attore secondario:} Actorbase\\
\textbf{Scopo e descrizione:} L'attore intende effettuare il reset della password di un utente all'interno del sistema. Deve inserire lo username, la password verrà modificata al valore prescelto ``actorbase''.\\
\textbf{Precondizione:} Il server è in ascolto. L'attore ha privilegi amministrativi e intende effettuare il reset della password di un utente all'interno del sistema.\\
\textbf{Scenario principale:}
\begin{itemize}
\item UC2.7.3.1\ -\ Inserimento username.
\end{itemize}
\textbf{Postcondizioni:} La password dell'utente designato è stata modificata in ``actorbase'' con successo.

\subsubsection{UC2.7.3.1: Inserimento username}

\textbf{Attore primario:} Programma esterno su \gloss{JVM}\\
\textbf{Attore secondario:} Actorbase\\
\textbf{Scopo e descrizione:} L'attore intende effettuare il reset della password di un utente all'interno del sistema. Deve inserire lo username e confermare il reset, la password
verrà modificata al valore prescelto ``actorbase''.\\
\textbf{Precondizione:} Il server è in ascolto, l'attore ha privilegi amministrativi e ha inserito il comando per il reset password.\\
\textbf{Scenario principale:}
\begin{itemize}
\item L'attore può inserire lo username dell'utente designato al reset password.
\end{itemize}
\textbf{Postcondizioni:} Lo username dell'utente designato al reset password è stato inserito con successo.

\subsection{UC2.7.4: Lancio eccezione username già presente}

\textbf{Attore primario:} Programma esterno su \gloss{JVM}\\
\textbf{Attore secondario:} Actorbase\\
\textbf{Scopo e descrizione:}
L'attore intende aggiungere un utente al sistema.\\
\textbf{Precondizioni:} Il server è in ascolto, l'attore ha privilegi amministrativi, ha inserito il comando di aggiunta utente e uno username già presente all'interno del sistema.\\
\textbf{Scenario principale:}
\begin{itemize}
\item Viene lanciata un'eccezione username già presente.
\end{itemize}
\textbf{Postcondizioni:} L'aggiunta utente fallisce, l'utente non viene aggiunto al sistema.

\subsection{UC2.7.5: Lancio eccezione username non presente}

\textbf{Attore primario:} Programma esterno su \gloss{JVM}\\
\textbf{Attore secondario:} Actorbase\\
\textbf{Scopo e descrizione:}
L'attore intende rimuovere un utente o effettuare il reset password di un utente.\\
\textbf{Precondizioni:} Il server è in ascolto, l'attore ha privilegi amministrativi, ha inserito il comando di rimozione utente o di reset password e uno username non presente all'interno del sistema.\\
\textbf{Scenario principale:}
\begin{itemize}
\item Viene lanciata un'eccezione username non presente.
\end{itemize}
\textbf{Postcondizioni:} Il comando inserito fallisce.

\subsection{UC2.8: Lancio eccezione credenziali errate}

\textbf{Attore primario:} Programma esterno su \gloss{JVM}\\
\textbf{Attore secondario:} Actorbase\\
\textbf{Scopo e descrizione:}
L'attore intende effettuare login al server, risulta essere non autenticato, ha inserito il comando di login, il proprio username e la propria password per effettuare l'autenticazione all'interno del sistema.\\
\textbf{Precondizioni:} Il server è in ascolto, l'attore ha inserito il comando di login e le proprie credenziali (username e password). Il sistema non riconosce le credenziali.\\
\textbf{Scenario principale:}cd RP
\begin{itemize}
\item Viene lanciata un'eccezione per credenziali errate.
\end{itemize}
\textbf{Postcondizioni:} Il login fallisce, l'utente risulta essere non autenticato.

\subsection{UC2.9: Lancio eccezione password attuale}

\textbf{Attore primario:} Programma esterno su \gloss{JVM}\\
\textbf{Attore secondario:} Actorbase\\
\textbf{Scopo e descrizione:}
L'attore intende effettuare una modifica alla propria password, ha inserito il comando di modifica password, la password attuale e la nuova password.\\
\textbf{Precondizioni:} Il server è in ascolto, l'attore ha inserito il comando di modifica password, la password attuale e la nuova password. Il sistema non riconosce la password attuale inserita.\\
\textbf{Scenario principale:}
\begin{itemize}
\item Viene lanciata un'eccezione password attuale.
\end{itemize}
\textbf{Postcondizioni:} La modifica password fallisce.

\subsection{UC2.10: Lancio eccezione nuova password}

\textbf{Attore primario:} Programma esterno su \gloss{JVM}\\
\textbf{Attore secondario:} Actorbase\\
\textbf{Scopo e descrizione:}
L'attore intende effettuare una modifica alla propria password, ha inserito il comando di modifica password, la password attuale e la nuova password.\\
\textbf{Precondizioni:} Il server è in ascolto, l'attore ha inserito il comando di modifica password, la password attuale e la nuova password. La nuova password inserita non rispetta i vincoli del sistema.\\ %TODO linkarli
\textbf{Scenario principale:}
\begin{itemize}
\item Viene lanciata un'eccezione nuova password.
\end{itemize}
\textbf{Postcondizioni:} La modifica password fallisce.

\subsection{UC2.11: Ricerca sull'intero database}

\textbf{Attore primario:} Programma esterno su \gloss{JVM}\\
\textbf{Attore secondario:} Actorbase\\
\textbf{Scopo e descrizione:}
L'attore intende effettuare una ricerca sull'intero database di item con una cerca chiave.\\
\textbf{Precondizioni:} Il server è in ascolto, l'attore ha inserito il comando di ricerca, una lista di collezioni vuota e la chiave di ricerca.\\
\textbf{Scenario principale:}
\begin{itemize}
\item Vengono visualizzati tutti gli item relativi alla chiave inserita dell'intero database.
\end{itemize}
\textbf{Postcondizioni:} Vengono ritornati tutti gli item relativi alla chiave inserita.

\subsection{UC2.12: Visualizzazione dell'intera collezione}

\textbf{Attore primario:} Programma esterno su \gloss{JVM}\\
\textbf{Attore secondario:} Actorbase\\
\textbf{Scopo e descrizione:} L'attore intende effettuare una ricerca su una lista di \gloss{collezioni} senza specificare chiavi di ricerca.\\
\textbf{Precondizioni:} Il server è in ascolto, l'attore ha inserito il comando di ricerca, una lista non vuota di \gloss{collezioni} e nessuna chiave di ricerca.\\
\textbf{Scenario principale:}
\begin{itemize}
\item Vengono visualizzate le coppie chiave-valore dell'intera \gloss{collezione}.
\end{itemize}
\textbf{Postcodizioni:} Viene visualizzata l'intera \gloss{collezione}.

\subsection{UC2.13: Visualizzazione dell'intero database}

\textbf{Attore primario:} Programma esterno su \gloss{JVM}\\
\textbf{Attore secondario:} Actorbase\\
\textbf{Scopo e descrizione:} L'attore intende effettuare una visualizzazione  dell'intero database.\\
\textbf{Precondizioni:} Il server è in ascolto, l'attore ha inserito il comando di ricerca, una lista vuota di \gloss{collezioni} e nessuna chiave di ricerca.\\
\textbf{Scenario principale:}
\begin{itemize}
\item Vengono visualizzate le coppie chiave-valore dell'intero database.
\end{itemize}
\textbf{Postcodizioni:} Viene visualizzato l'intero database.


\section{Requisiti}

\subsection{Requisiti funzionali}

La descrizione dei requisiti segue le regole descritte nelle \href{run:../Interni/NormeDiProgetto\_v1.0.0.pdf}{Norme di Progetto}.
\begin{longtable}[H]{|l|p{2cm}|p{6cm}|p{4cm}|}
  \hline
  \textbf{Requisito} & \textbf{Tipologia} & \textbf{Descrizione} & \textbf{Fonti}\\
  \hline
  OBF1 & \multiLineCell{Funzionale\\Obbligatorio} & Il sistema dovrà fornire un Domain Specific Language per comunicare con il \gloss{database} & \multiLineCell{Capitolato\\UC1\\}\\
  \hline
  DEF1.1 & \multiLineCell{Funzionale\\Desiderabile} & Il sistema dovrà fornire un comando di assistenza & \multiLineCell{UC1.3\\UC1.3.1\\UC1.3.2\\}\\
  \hline
  DEF1.1.1 & \multiLineCell{Funzionale\\Desiderabile} & Il sistema dovrà fornire un comando di assistenza generale & \multiLineCell{UC1.3.1\\}\\
  \hline
  DEF1.1.2 & \multiLineCell{Funzionale\\Desiderabile} & Il sistema dovrà fornire un comando di assistenza specifico per un comando desiderato & \multiLineCell{UC1.3.2\\}\\
  \hline
  OBF2 & \multiLineCell{Funzionale\\Obbligatorio} & Il sistema dovrà fornire un meccanismo di autenticazione & \multiLineCell{UC1\\UC1.1\\UC1.2\\VERBALE20160112\\}\\
  \hline
  OBF2.1 & \multiLineCell{Funzionale\\Obbligatorio} & Il sistema dovrà fornire un meccanismo di registrazione & \multiLineCell{UC1.1\\VERBALE20160119\\}\\
  \hline
  OBF2.2 & \multiLineCell{Funzionale\\Obbligatorio} & Il sistema dovrà fornire un meccanismo di login & \multiLineCell{UC1.2\\UC2\\VERBALE20160119\\}\\
  \hline
  OBF3 & \multiLineCell{Funzionale\\Obbligatorio} & Il sistema dovrà fornire un meccanismo di persistenza dei dati su disco & \multiLineCell{Capitolato\\VERBALE20160112\\}\\
  \hline
  OBF4 & \multiLineCell{Funzionale\\Obbligatorio} & Il sistema dovrà fornire la possibilità di eseguire una serie di operazioni sulle \gloss{collezioni} & \multiLineCell{Capitolato\\UC1.4\\UC2\\}\\
  \hline
  OBF4.1 & \multiLineCell{Funzionale\\Obbligatorio} & Il sistema dovrà fornire la possibilità di creare una nuova \gloss{collezione} & \multiLineCell{UC1.4.1\\VERBALE20160112\\}\\
  \hline
  OBF4.1.1 & \multiLineCell{Funzionale\\Opzionale} & Il sistema deve dare la possibilità di scegliere uno \gloss{schema} delle chiavi che definiscono la \gloss{collezione} da creare & \multiLineCell{UC1.4.1\\VERBALE20160112\\VERBALE20160119\\}\\
  \hline
  OBF4.2 & \multiLineCell{Funzionale\\Obbligatorio} & Il sistema dovrà fornire la possibilità di elencare le \gloss{collezioni} presenti all'interno & \multiLineCell{UC1.4.3\\UC2\\VERBALE20160112\\}\\
  \hline
  OBF4.3 & \multiLineCell{Funzionale\\Obbligatorio} & Il sistema dovrà fornire la possibilità di cancellare una o più \gloss{collezioni} & \multiLineCell{UC1.4.3\\UC2\\VERBALE20160112\\}\\
  \hline
  OBF4.4 & \multiLineCell{Funzionale\\Obbligatorio} & Il sistema dovrà fornire la possibilità di modificare il nome delle \gloss{collezioni} & \multiLineCell{UC1.4.4\\UC2\\VERBALE20160112\\}\\
  \hline
  OBF4.5 & \multiLineCell{Funzionale\\Obbligatorio} & Il sistema dovrà fornire la possibilità di aggiungere collaboratori ad una propria \gloss{collezione} & \multiLineCell{UC1.4.5\\UC2\\VERBALE20160112\\}\\
  \hline
  OBF4.6 & \multiLineCell{Funzionale\\Obbligatorio} & Il sistema dovrà fornire la possibilità di rimuovere un \gloss{collaboratore} da una propria \gloss{collezione} & \multiLineCell{UC1.4.6\\UC2\\VERBALE20160112\\}\\
  \hline
  OBF5 & \multiLineCell{Funzionale\\Obbligatorio} & Il sistema dovrà fornire la possibilità di eseguire una serie di operazioni sugli \gloss{item} & \multiLineCell{Capitolato\\UC1.5\\UC1.5.1\\UC1.5.1.1\\UC1.5.1.2\\UC1.5.2\\UC2\\VERBALE20160112\\}\\
  \hline
  OBF5.1 & \multiLineCell{Funzionale\\Obbligatorio} & Il sistema dovrà permettere l'inserimento di nuovi \gloss{item} & \multiLineCell{Capitolato\\UC1.5.1\\UC2\\VERBALE20160112\\}\\
  \hline
  OBF5.1.1 & \multiLineCell{Funzionale\\Obbligatorio} & Il sistema dovrà permettere la sovrascrittura di un \gloss{item} attraverso inserimento & \multiLineCell{Capitolato\\UC1.5.1.1\\UC2\\VERBALE20160112\\}\\
  \hline
  OBF5.1.2 & \multiLineCell{Funzionale\\Obbligatorio} & Il sistema dovrà permettere l'aggiunta di un nuovo \gloss{item} senza eseguire sovrascrittura & \multiLineCell{Capitolato\\UC1.5.1.2\\UC2\\VERBALE20160112\\}\\
  \hline
  OBF5.2 & \multiLineCell{Funzionale\\Obbligatorio} & Il sistema dovrà permettere la cancellazione di un \gloss{item} & \multiLineCell{Capitolato\\UC1.5.2\\VERBALE20160112\\}\\
  \hline
  DEF5.3 & \multiLineCell{Funzionale\\Desiderabile} & Il sistema dovrà permettere tutte le operazioni sugli \gloss{item} anche attraverso importazione di \gloss{file} & \multiLineCell{UC1.5\\UC1.5.1\\UC1.5.1.1\\UC1.5.1.2\\UC1.5.2\\VERBALE20160112\\}\\
  \hline
  OBF6 & \multiLineCell{Funzionale\\Obbligatorio} & Il sistema dovrà fornire un meccanismo di ricerca & \multiLineCell{UC1.6\\UC2\\VERBALE20160112\\}\\
  \hline
  OBF6.1 & \multiLineCell{Funzionale\\Obbligatorio} & Il sistema dovrà fornire un meccanismo di ricerca generale sull'intera base di dati & \multiLineCell{UC1.6.1\\UC2\\VERBALE20160112\\}\\
  \hline
  OBF6.2 & \multiLineCell{Funzionale\\Obbligatorio} & Il sistema dovrà fornire la possibilità di effettuare ricerche su una o più \gloss{collezioni} & \multiLineCell{UC1.6.2\\UC2\\VERBALE20160112\\}\\
  \hline
  OBF7 & \multiLineCell{Funzionale\\Obbligatorio} & Il sistema dovrà fornire un meccanismo per la gestione del proprio \gloss{account} & \multiLineCell{UC1.7\\UC1.7.1\\UC1.7.2\\VERBALE20160112\\}\\
  \hline
  DEF7.1 & \multiLineCell{Funzionale\\Desiderabile} & Il sistema dovrà fornire la possibilità di cambiare il proprio username & \multiLineCell{UC1.7\\UC1.7.1\\VERBALE20160112\\}\\
  \hline
  OBF7.2 & \multiLineCell{Funzionale\\Obbligatorio} & Il sistema dovrà fornire la possibilità di cambiare la propria password & \multiLineCell{UC1.7\\UC1.7.2\\VERBALE20160112\\}\\
  \hline
  OBF8 & \multiLineCell{Funzionale\\Obbligatorio} & Il sistema dovrà permettere di effettuare logout & \multiLineCell{UC1.8\\VERBALE20160112\\}\\
  \hline
  DEF9 & \multiLineCell{Funzionale\\Desiderabile} & Il sistema dovrà fornire diverse azioni possibili sugli \gloss{account} agli amministratori & \multiLineCell{UC1.9\\UC1.9.1\\UC1.9.2\\UC2\\}\\
  \hline
  DEF9.1 & \multiLineCell{Funzionale\\Desiderabile} & Il sistema dovrà permettere agli amministratori di aggiungere nuovi utenti & \multiLineCell{UC1.9\\UC1.9.1\\UC2\\}\\
  \hline
  DEF9.2 & \multiLineCell{Funzionale\\Desiderabile} & Il sistema dovrà permettere agli amministratori di eliminare utenti esistenti & \multiLineCell{UC1.9\\UC1.9.2\\UC2\\}\\
  \hline
  OBF10 & \multiLineCell{Funzionale\\Obbligatorio} & Il sistema dovrà fornire un meccanismo di connessione ad altre istanze esterne & \multiLineCell{Capitolato\\UC1\\VERBALE20160112\\VERBALE20160119\\}\\
  \hline
  OBF11 & \multiLineCell{Funzionale\\Obbligatorio} & Il sistema dovrà permettere la visualizzazione di un messaggio di errore & \multiLineCell{UC1.10\\UC1.10.1\\UC1.10.2\\VERBALE20160112\\}\\
  \hline
  OBF11.1 & \multiLineCell{Funzionale\\Obbligatorio} & Il sistema dovrà permettere la visualizzazione di un messaggio di errore in caso di inserimenti di un comando errato & \multiLineCell{UC1.10\\UC1.10.1\\VERBALE20160112\\}\\
  \hline
  OBF11.2 & \multiLineCell{Funzionale\\Obbligatorio} & Il sistema dovrà permettere la visualizzazione di un messaggio d'errore in caso di timeout & \multiLineCell{UC1.10.2\\VERBALE20160112\\}\\
  \hline
  OBF11.3 & \multiLineCell{Funzionale\\Obbligatorio} & Il sistema dovrà permettere la visualizzazione di un messaggio di errore qualora si tenti di eseguire una operazione non permessa o impossibile da soddisfare & \multiLineCell{UC1.11.3\\VERBALE20160112\\}\\
  \hline
  DEF12 & \multiLineCell{Funzionale\\Desiderabile} & Dovrà essere fornito un \gloss{driver} Scala per interfacciarsi ad Actorbase & \multiLineCell{Capitolato\\UC2\\VERBALE20160112\\}\\
  \hline
  DEF12.1 & \multiLineCell{Funzionale\\Desiderabile} & Il \gloss{driver} dovrà permettere di effettuare il login & \multiLineCell{UC2\\}\\
  \hline
  DEF12.10 & \multiLineCell{Funzionale\\Desiderabile} & Il \gloss{driver} dovrà permettere la ricerca all'interno di \gloss{collezioni} & \multiLineCell{UC2\\}\\
  \hline
  DEF12.11 & \multiLineCell{Funzionale\\Desiderabile} & Il \gloss{driver} dovrà permettere il logout dal \gloss{database} & \multiLineCell{UC2\\}\\
  \hline
  DEF12.2 & \multiLineCell{Funzionale\\Desiderabile} & Il \gloss{driver} dovrà permettere la creazione di \gloss{collezioni} & \multiLineCell{UC2\\}\\
  \hline
  DEF12.3 & \multiLineCell{Funzionale\\Desiderabile} & Il \gloss{driver} dovrà permettere l'elencazione delle \gloss{collezioni} & \multiLineCell{UC2\\}\\
  \hline
  DEF12.4 & \multiLineCell{Funzionale\\Desiderabile} & Il \gloss{driver} dovrà permettere la modifica del nome di una \gloss{collezione} & \multiLineCell{UC2\\}\\
  \hline
  DEF12.5 & \multiLineCell{Funzionale\\Desiderabile} & Il \gloss{driver} dovrà permettere l'aggiunta di un \gloss{collaboratore} al \gloss{collezione} & \multiLineCell{UC2\\}\\
  \hline
  DEF12.6 & \multiLineCell{Funzionale\\Desiderabile} & Il \gloss{driver} dovrà permetter la rimozione di un \gloss{collaboratore} dalle \gloss{collezioni} & \multiLineCell{UC2\\}\\
  \hline
  DEF12.7 & \multiLineCell{Funzionale\\Desiderabile} & Il \gloss{driver} dovrà permettere l'inserimento di \gloss{item} nelle \gloss{collezioni} & \multiLineCell{UC2\\}\\
  \hline
  DEF12.7.1 & \multiLineCell{Funzionale\\Desiderabile} & Il \gloss{driver} dovrà permettere l'inserimento \gloss{item} con sovrascrittura nelle \gloss{collezioni} & \multiLineCell{UC2\\}\\
  \hline
  DEF12.7.2 & \multiLineCell{Funzionale\\Desiderabile} & Il \gloss{driver} dovrà permettere l'inserimento senza sovrascrittura nelle \gloss{collezioni} & \multiLineCell{UC2\\}\\
  \hline
  DEF12.8 & \multiLineCell{Funzionale\\Desiderabile} & Il \gloss{driver} dovrà permettere la cancellazione di \gloss{item} dalle \gloss{collezioni} & \multiLineCell{UC2\\}\\
  \hline
  DEF12.9 & \multiLineCell{Funzionale\\Desiderabile} & Il \gloss{driver} dovrà permettere la ricerca all'interno del \gloss{database} & \multiLineCell{UC2\\}\\
  \hline
  OBF19 & \multiLineCell{Funzionale\\Obbligatorio} & Sarà implementato l'\gloss{attore} di tipo \gloss{storekeeper} & \multiLineCell{Capitolato\\}\\
  \hline
  OBF19.1 & \multiLineCell{Funzionale\\Obbligatorio} & Gli \gloss{attori} di tipo \gloss{storekeeper} dovranno contenere le diverse coppie chiave/valore & \multiLineCell{Capitolato\\}\\
  \hline
  OBF19.2 & \multiLineCell{Funzionale\\Obbligatorio} & Gli \gloss{storekeeper} dovranno soddisfare le richieste provenienti dai rispettivi \gloss{storefinder} & \multiLineCell{Capitolato\\}\\
  \hline
  OBF20 & \multiLineCell{Funzionale\\Obbligatorio} & Sarà implementato l'\gloss{attore} di tipo \gloss{storefinder} & \multiLineCell{Capitolato\\}\\
  \hline
  OBF20.1 & \multiLineCell{Funzionale\\Obbligatorio} & Gli \gloss{attori} \gloss{storefinder} possono ricevere richieste dall'esterno & \multiLineCell{Capitolato\\}\\
  \hline
  OBF20.2 & \multiLineCell{Funzionale\\Obbligatorio} & Gli \gloss{attori} di tipo \gloss{storefinder} dovranno instradare le richieste ai giusti \gloss{storekeeper} & \multiLineCell{Capitolato\\}\\
  \hline
  OBF21 & \multiLineCell{Funzionale\\Obbligatorio} & Sarà implementato l'\gloss{attore} di tipo \gloss{warehousemen} & \multiLineCell{Capitolato\\}\\
  \hline
  OBF21.1 & \multiLineCell{Funzionale\\Obbligatorio} & Il sistema dovrà permettere agli \gloss{attori} \gloss{warehousemen} di mandare messaggi agli \gloss{attori} di tipo \gloss{storekeeper} & \multiLineCell{Capitolato\\}\\
  \hline
  OBF21.2 & \multiLineCell{Funzionale\\Obbligatorio} & Il sistema dovrà permettere agli \gloss{attori} \gloss{warehousemen} di ricevere messaggi dagli \gloss{attori} di tipo \gloss{storekeeper} & \multiLineCell{Capitolato\\}\\
  \hline
  OBF21.3 & \multiLineCell{Funzionale\\Obbligatorio} & Il sistema dovrà permettere agli \gloss{attori} \gloss{warehousemen} di persistere i dati su disco & \multiLineCell{Capitolato\\}\\
  \hline
  OPF22 & \multiLineCell{Funzionale\\Opzionale} & Sarà implementato l'\gloss{attore} di tipo ninja & \multiLineCell{Capitolato\\}\\
  \hline
  OPF22.1 & \multiLineCell{Funzionale\\Opzionale} & L'\gloss{attore} di tipo ninja dovrà poter essere eletto a \gloss{storekeeper} se necessario & \multiLineCell{Capitolato\\}\\
  \hline
  OPF22.2 & \multiLineCell{Funzionale\\Opzionale} & L'\gloss{attore} di tipo ninja dovrà essere aggiornato insieme al corrispondente \gloss{storekeeper} & \multiLineCell{Capitolato\\}\\
  \hline
  OPF23 & \multiLineCell{Funzionale\\Opzionale} & Sarà implementato l'\gloss{attore} di tipo manager & \multiLineCell{Capitolato\\}\\
  \hline
  OPF23.1 & \multiLineCell{Funzionale\\Opzionale} & L'\gloss{attore} manager dovrà decidere il numero massimo di \gloss{item} contenuti in un'istanza dello \gloss{storekeeper} & \multiLineCell{Capitolato\\}\\
  \hline
  OPF23.1.1 & \multiLineCell{Funzionale\\Opzionale} & L'\gloss{attore} manager dovrà decidere il numero massimo di \gloss{item} contenuti in un istanza di \gloss{storekeeper} in maniera statica & \multiLineCell{Capitolato\\}\\
  \hline
  OPF23.1.2 & \multiLineCell{Funzionale\\Opzionale} & L'\gloss{attore} manager dovrà decidere il numero massimo di \gloss{item} contenuti in un istanza di \gloss{storekeeper} in maniera dinamica in base al carico & \multiLineCell{Capitolato\\}\\
  \hline
  OPF23.2 & \multiLineCell{Funzionale\\Opzionale} & L'\gloss{attore} manager dovrà poter inviare e ricevere messaggi agli \gloss{storekeeper} & \multiLineCell{Capitolato\\}\\
  \hline
  OBF24 & \multiLineCell{Funzionale\\Obbligatorio} & Sarà implementato l'\gloss{attore} di tipo main & \multiLineCell{Capitolato\\}\\
  \hline
  OBF26 & \multiLineCell{Funzionale\\Obbligatorio} & Il sistema dovrà distinguere diversi tipi di \gloss{account}. Utente non autenticato, utente autenticato e utente amministratore & \multiLineCell{UC1\\UC1.1\\UC1.2\\UC1.7\\UC1.7.1\\UC1.7.2\\UC1.8\\UC1.9\\UC1.9.1\\UC1.9.2\\VERBALE20160112\\}\\
  \hline
\end{longtable}

\subsection{Requisiti di vincolo}

\begin{longtable}[H]{|l|p{2cm}|p{6cm}|p{4cm}|}
  \hline
  \textbf{Requisito} & \textbf{Tipologia} & \textbf{Descrizione} & \textbf{Fonti}\\
  \hline
  OBV13 & \multiLineCell{Vincolo\\Obbligatorio} & Il sistema dovrà funzionare in ambienti Ubuntu 14.04 o superiore & \multiLineCell{VERBALE20160119\\}\\
  \hline
  OBV14 & \multiLineCell{Vincolo\\Obbligatorio} & Il sistema dovrà funzionare su \gloss{JVM} versione 8 & \multiLineCell{Capitolato\\}\\
  \hline
  DEV15 & \multiLineCell{Vincolo\\Desiderabile} & Il sistema dovrà funzionare in ambienti Windows 10 o superiori & \multiLineCell{VERBALE20160119\\}\\
  \hline
  DEV16 & \multiLineCell{Vincolo\\Desiderabile} & Il sistema dovrà funzionare su sistemi OSX el capitan o superiori & \multiLineCell{VERBALE20160119\\}\\
  \hline
  OBV25 & \multiLineCell{Vincolo\\Obbligatorio} & Il sistema dovrà utilizzare la libreria Akka & \multiLineCell{Capitolato\\}\\
  \hline
\end{longtable}

\subsection{Requisiti di qualità}

\begin{longtable}[H]{|l|p{2cm}|p{6cm}|p{4cm}|}
  \hline
  \textbf{Requisito} & \textbf{Tipologia} & \textbf{Descrizione} & \textbf{Fonti}\\
  \hline
  OBQ17 & \multiLineCell{Qualitativo\\Obbligatorio} & Dovrà essere fornito un manuale utente redatto in lingua inglese & \multiLineCell{Capitolato\\}\\
  \hline
  OBF18 & \multiLineCell{Qualitativo\\Obbligatorio} & Dovranno essere rispettate tutte le norme e le metriche sulla stesura del codice come riportato nelle \textit{Norme di Progetto} & \multiLineCell{INTERNO\\}\\
  \hline
  OBQ18.1 & \multiLineCell{Qualitativo\\Obbligatorio} & Dovrà essere prodotta documentazione in Scaladoc del prodotto & \multiLineCell{NORME\\}\\
  \hline
\end{longtable}

\section{Tracciamento Requisiti-Fonti}

\begin{longtable}[H]{|p{5.5cm}|p{5.5cm}|}
  \hline
  \textbf{Requisito} & \textbf{Fonti}\\
  \hline
  OBF1 & \multiLineCell[t]{Capitolato\\UC1\\}\\
  \hline
  DEF1.1 & \multiLineCell[t]{UC1.3\\UC1.3.1\\UC1.3.2\\}\\
  \hline
  DEF1.1.1 & \multiLineCell[t]{UC1.3.1\\}\\
  \hline
  DEF1.1.2 & \multiLineCell[t]{UC1.3.2\\}\\
  \hline
  OBF2 & \multiLineCell[t]{UC1\\UC1.1\\UC1.2\\VERBALE20160112\\}\\
  \hline
  OBF2.1 & \multiLineCell[t]{UC1.1\\VERBALE20160119\\}\\
  \hline
  OBF2.2 & \multiLineCell[t]{UC1.2\\UC2\\VERBALE20160119\\}\\
  \hline
  OBF3 & \multiLineCell[t]{Capitolato\\VERBALE20160112\\}\\
  \hline
  OBF4 & \multiLineCell[t]{Capitolato\\UC1.4\\UC2\\}\\
  \hline
  OBF4.1 & \multiLineCell[t]{UC1.4.1\\VERBALE20160112\\}\\
  \hline
  OBF4.1.1 & \multiLineCell[t]{UC1.4.1\\VERBALE20160112\\VERBALE20160119\\}\\
  \hline
  OBF4.2 & \multiLineCell[t]{UC1.4.3\\UC2\\VERBALE20160112\\}\\
  \hline
  OBF4.3 & \multiLineCell[t]{UC1.4.3\\UC2\\VERBALE20160112\\}\\
  \hline
  OBF4.4 & \multiLineCell[t]{UC1.4.4\\UC2\\VERBALE20160112\\}\\
  \hline
  OBF4.5 & \multiLineCell[t]{UC1.4.5\\UC2\\VERBALE20160112\\}\\
  \hline
  OBF4.6 & \multiLineCell[t]{UC1.4.6\\UC2\\VERBALE20160112\\}\\
  \hline
  OBF5 & \multiLineCell[t]{Capitolato\\UC1.5\\UC1.5.1\\UC1.5.1.1\\UC1.5.1.2\\UC1.5.2\\UC2\\VERBALE20160112\\}\\
  \hline
  OBF5.1 & \multiLineCell[t]{Capitolato\\UC1.5.1\\UC2\\VERBALE20160112\\}\\
  \hline
  OBF5.1.1 & \multiLineCell[t]{Capitolato\\UC1.5.1.1\\UC2\\VERBALE20160112\\}\\
  \hline
  OBF5.1.2 & \multiLineCell[t]{Capitolato\\UC1.5.1.2\\UC2\\VERBALE20160112\\}\\
  \hline
  OBF5.2 & \multiLineCell[t]{Capitolato\\UC1.5.2\\VERBALE20160112\\}\\
  \hline
  DEF5.3 & \multiLineCell[t]{UC1.5\\UC1.5.1\\UC1.5.1.1\\UC1.5.1.2\\UC1.5.2\\VERBALE20160112\\}\\
  \hline
  OBF6 & \multiLineCell[t]{UC1.6\\UC2\\VERBALE20160112\\}\\
  \hline
  OBF6.1 & \multiLineCell[t]{UC1.6.1\\UC2\\VERBALE20160112\\}\\
  \hline
  OBF6.2 & \multiLineCell[t]{UC1.6.2\\UC2\\VERBALE20160112\\}\\
  \hline
  OBF7 & \multiLineCell[t]{UC1.7\\UC1.7.1\\UC1.7.2\\VERBALE20160112\\}\\
  \hline
  DEF7.1 & \multiLineCell[t]{UC1.7\\UC1.7.1\\VERBALE20160112\\}\\
  \hline
  OBF7.2 & \multiLineCell[t]{UC1.7\\UC1.7.2\\VERBALE20160112\\}\\
  \hline
  OBF8 & \multiLineCell[t]{UC1.8\\VERBALE20160112\\}\\
  \hline
  DEF9 & \multiLineCell[t]{UC1.9\\UC1.9.1\\UC1.9.2\\UC2\\}\\
  \hline
  DEF9.1 & \multiLineCell[t]{UC1.9\\UC1.9.1\\UC2\\}\\
  \hline
  DEF9.2 & \multiLineCell[t]{UC1.9\\UC1.9.2\\UC2\\}\\
  \hline
  OBF10 & \multiLineCell[t]{Capitolato\\UC1\\VERBALE20160112\\VERBALE20160119\\}\\
  \hline
  OBF11 & \multiLineCell[t]{UC1.10\\UC1.10.1\\UC1.10.2\\VERBALE20160112\\}\\
  \hline
  OBF11.1 & \multiLineCell[t]{UC1.10\\UC1.10.1\\VERBALE20160112\\}\\
  \hline
  OBF11.2 & \multiLineCell[t]{UC1.10.2\\VERBALE20160112\\}\\
  \hline
  OBF11.3 & \multiLineCell[t]{UC1.11.3\\VERBALE20160112\\}\\
  \hline
  DEF12 & \multiLineCell[t]{Capitolato\\UC2\\VERBALE20160112\\}\\
  \hline
  DEF12.1 & \multiLineCell[t]{UC2\\}\\
  \hline
  DEF12.10 & \multiLineCell[t]{UC2\\}\\
  \hline
  DEF12.11 & \multiLineCell[t]{UC2\\}\\
  \hline
  DEF12.2 & \multiLineCell[t]{UC2\\}\\
  \hline
  DEF12.3 & \multiLineCell[t]{UC2\\}\\
  \hline
  DEF12.4 & \multiLineCell[t]{UC2\\}\\
  \hline
  DEF12.5 & \multiLineCell[t]{UC2\\}\\
  \hline
  DEF12.6 & \multiLineCell[t]{UC2\\}\\
  \hline
  DEF12.7 & \multiLineCell[t]{UC2\\}\\
  \hline
  DEF12.7.1 & \multiLineCell[t]{UC2\\}\\
  \hline
  DEF12.7.2 & \multiLineCell[t]{UC2\\}\\
  \hline
  DEF12.8 & \multiLineCell[t]{UC2\\}\\
  \hline
  DEF12.9 & \multiLineCell[t]{UC2\\}\\
  \hline
  OBV13 & \multiLineCell[t]{VERBALE20160119\\}\\
  \hline
  OBV14 & \multiLineCell[t]{Capitolato\\}\\
  \hline
  DEV15 & \multiLineCell[t]{VERBALE20160119\\}\\
  \hline
  DEV16 & \multiLineCell[t]{VERBALE20160119\\}\\
  \hline
  OBQ17 & \multiLineCell[t]{Capitolato\\}\\
  \hline
  OBQ18 & \multiLineCell[t]{INTERNO\\}\\
  \hline
  OBQ18.1 & \multiLineCell[t]{NORME\\}\\
  \hline
  OBF19 & \multiLineCell[t]{Capitolato\\}\\
  \hline
  OBF19.1 & \multiLineCell[t]{Capitolato\\}\\
  \hline
  OBF19.2 & \multiLineCell[t]{Capitolato\\}\\
  \hline
  OBF20 & \multiLineCell[t]{Capitolato\\}\\
  \hline
  OBF20.1 & \multiLineCell[t]{Capitolato\\}\\
  \hline
  OBF20.2 & \multiLineCell[t]{Capitolato\\}\\
  \hline
  OBF21 & \multiLineCell[t]{Capitolato\\}\\
  \hline
  OBF21.1 & \multiLineCell[t]{Capitolato\\}\\
  \hline
  OBF21.2 & \multiLineCell[t]{Capitolato\\}\\
  \hline
  OBF21.3 & \multiLineCell[t]{Capitolato\\}\\
  \hline
  OPF22 & \multiLineCell[t]{Capitolato\\}\\
  \hline
  OPF22.1 & \multiLineCell[t]{Capitolato\\}\\
  \hline
  OPF22.2 & \multiLineCell[t]{Capitolato\\}\\
  \hline
  OPF23 & \multiLineCell[t]{Capitolato\\}\\
  \hline
  OPF23.1 & \multiLineCell[t]{Capitolato\\}\\
  \hline
  OPF23.1.1 & \multiLineCell[t]{Capitolato\\}\\
  \hline
  OPF23.1.2 & \multiLineCell[t]{Capitolato\\}\\
  \hline
  OPF23.2 & \multiLineCell[t]{Capitolato\\}\\
  \hline
  OBF24 & \multiLineCell[t]{Capitolato\\}\\
  \hline
  OBV25 & \multiLineCell[t]{Capitolato\\}\\
  \hline
  OBF26 & \multiLineCell[t]{UC1\\UC1.1\\UC1.2\\UC1.7\\UC1.7.1\\UC1.7.2\\UC1.8\\UC1.9\\UC1.9.1\\UC1.9.2\\VERBALE20160112\\}\\
  \hline
\end{longtable}
\newpage

\section{Tracciamento Fonti-Requisiti}

\begin{longtable}[H]{|p{5.5cm}|p{5.5cm}|}
  \hline
  \textbf{Fonte} & \textbf{Requisiti}\\
  \hline
  Capitolato & \multiLineCell[t]{DEF12\\OBF1\\OBF10\\OBF19\\OBF19.1\\OBF19.2\\OBF20\\OBF20.1\\OBF20.2\\OBF21\\OBF21.1\\OBF21.2\\OBF21.3\\OBF24\\OBF3\\OBF4\\OBF5\\OBF5.1\\OBF5.1.1\\OBF5.1.2\\OBF5.2\\OBQ17\\OBV14\\OBV25\\OPF22\\OPF22.1\\OPF22.2\\OPF23\\OPF23.1\\OPF23.1.1\\OPF23.1.2\\OPF23.2\\}\\
  \hline
  INTERNO & \multiLineCell[t]{OBF18\\}\\
  \hline
  NORME & \multiLineCell[t]{OBQ18\\}\\
  \hline
  UC1 & \multiLineCell[t]{OBF1\\OBF10\\OBF12\\OBF2\\}\\
  \hline
  UC1.1 & \multiLineCell[t]{OBF12\\OBF2\\OBF2.1\\}\\
  \hline
  UC1.10 & \multiLineCell[t]{OBF11\\OBF11.1\\}\\
  \hline
  UC1.10.1 & \multiLineCell[t]{OBF11\\OBF11.1\\}\\
  \hline
  UC1.10.2 & \multiLineCell[t]{OBF11\\OBF11.2\\}\\
  \hline
  UC1.11.3 & \multiLineCell[t]{OBF11.3\\}\\
  \hline
  UC1.2 & \multiLineCell[t]{OBF12\\OBF2\\OBF2.2\\}\\
  \hline
  UC1.3 & \multiLineCell[t]{DEF1.1\\}\\
  \hline
  UC1.3.1 & \multiLineCell[t]{DEF1.1\\DEF1.1.1\\}\\
  \hline
  UC1.3.2 & \multiLineCell[t]{DEF1.1\\DEF1.1.2\\}\\
  \hline
  UC1.4 & \multiLineCell[t]{OBF4\\}\\
  \hline
  UC1.4.1 & \multiLineCell[t]{OBF4.1\\OBF4.1.1\\}\\
  \hline
  UC1.4.3 & \multiLineCell[t]{OBF4.2\\OBF4.3\\}\\
  \hline
  UC1.4.4 & \multiLineCell[t]{OBF4.4\\}\\
  \hline
  UC1.4.5 & \multiLineCell[t]{OBF4.5\\}\\
  \hline
  UC1.4.6 & \multiLineCell[t]{OBF4.6\\}\\
  \hline
  UC1.5 & \multiLineCell[t]{DEF5.3\\OBF5\\}\\
  \hline
  UC1.5.1 & \multiLineCell[t]{DEF5.3\\OBF5\\OBF5.1\\}\\
  \hline
  UC1.5.1.1 & \multiLineCell[t]{DEF5.3\\OBF5\\OBF5.1.1\\}\\
  \hline
  UC1.5.1.2 & \multiLineCell[t]{DEF5.3\\OBF5\\OBF5.1.2\\}\\
  \hline
  UC1.5.2 & \multiLineCell[t]{DEF5.3\\OBF5\\OBF5.2\\}\\
  \hline
  UC1.6 & \multiLineCell[t]{OBF6\\}\\
  \hline
  UC1.6.1 & \multiLineCell[t]{OBF6.1\\}\\
  \hline
  UC1.6.2 & \multiLineCell[t]{OBF6.2\\}\\
  \hline
  UC1.7 & \multiLineCell[t]{DEF7.1\\OBF12\\OBF7\\OBF7.2\\}\\
  \hline
  UC1.7.1 & \multiLineCell[t]{DEF7.1\\OBF12\\OBF7\\}\\
  \hline
  UC1.7.2 & \multiLineCell[t]{OBF12\\OBF7\\OBF7.2\\}\\
  \hline
  UC1.8 & \multiLineCell[t]{OBF12\\OBF8\\}\\
  \hline
  UC1.9 & \multiLineCell[t]{DEF9\\DEF9.1\\DEF9.2\\OBF12\\}\\
  \hline
  UC1.9.1 & \multiLineCell[t]{DEF9\\DEF9.1\\OBF12\\}\\
  \hline
  UC1.9.2 & \multiLineCell[t]{DEF9\\DEF9.2\\OBF12\\}\\
  \hline
  UC2 & \multiLineCell[t]{DEF12\\DEF12.1\\DEF12.10\\DEF12.11\\DEF12.2\\DEF12.3\\DEF12.4\\DEF12.5\\DEF12.6\\DEF12.7\\DEF12.7.1\\DEF12.7.2\\DEF12.8\\DEF12.9\\DEF9\\DEF9.1\\DEF9.2\\OBF2.2\\OBF4\\OBF4.2\\OBF4.3\\OBF4.4\\OBF4.5\\OBF4.6\\OBF5\\OBF5.1\\OBF5.1.1\\OBF5.1.2\\OBF6\\OBF6.1\\OBF6.2\\}\\
  \hline
  VERBALE20160112 & \multiLineCell[t]{DEF12\\DEF5.3\\DEF7.1\\OBF10\\OBF11\\OBF11.1\\OBF11.2\\OBF11.3\\OBF12\\OBF2\\OBF3\\OBF4.1\\OBF4.1.1\\OBF4.2\\OBF4.3\\OBF4.4\\OBF4.5\\OBF4.6\\OBF5\\OBF5.1\\OBF5.1.1\\OBF5.1.2\\OBF5.2\\OBF6\\OBF6.1\\OBF6.2\\OBF7\\OBF7.2\\OBF8\\}\\
  \hline
  VERBALE20160119 & \multiLineCell[t]{DEV15\\DEV16\\OBF10\\OBF2.1\\OBF2.2\\OBF4.1.1\\OBV13\\}\\
  \hline
\end{longtable}

\section{Riepilogo}

\begin{table}[H]
  \centering
  \caption*{Riepilogo requisiti}
  \begin{tabular}{|l|l|l|l|}
    \hline
    Categoria & Obbligatorio & Desiderabile & Opzionale\\
    \hline
    Funzionale & 40 & 22 & 9 \\
    \hline
    di Qualità & 3 & 0 & 0 \\
    \hline
    di Vincolo & 3 & 2 & 0 \\
    \hline
  \end{tabular}
\end{table}
\newpage
\appendix
\listoftables
\listoffigures
\end{document}
