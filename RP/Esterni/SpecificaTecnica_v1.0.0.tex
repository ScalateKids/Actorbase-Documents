\documentclass{scalatekids-article}
\begin{document}
\lfoot{Specifica Tecnica 1.0.0}
\newgeometry{top=3.5cm}
\begin{titlepage}
  \begin{center}
    \begin{center}
      \includegraphics[width=10cm]{sklogo.png}
    \end{center}
    \vspace{1cm}
    \begin{Huge}
      \begin{center}
        \textbf{Specifica Tecnica}
      \end{center}
    \end{Huge}
    \vspace{11pt}
    \bgroup
    \def\arraystretch{1.3}
    \begin{tabular}{r|l}
      \multicolumn{2}{c}{\textbf{Informazioni sul documento}} \\
      \hline
      \setbox0=\hbox{0.0.1\unskip}\ifdim\wd0=0pt
      \\
      \else
      \textbf{Versione} & 1.0.0\\
      \fi
      \textbf{Redazione} & \multiLineCell[t]{}\\
      \textbf{Verifica} & \multiLineCell[t]{}\\
      \textbf{Approvazione} & \multiLineCell[t]{}\\
      \textbf{Uso} & Esterno\\
      \textbf{Lista di Distribuzione} & \multiLineCell[t]{ScalateKids\\Prof. Tullio Vardanega\\Prof. Riccardo Cardin}\\
    \end{tabular}
    \egroup
    \vspace{22pt}
  \end{center}
\end{titlepage}
\restoregeometry
\clearpage
\pagenumbering{Roman}
\setcounter{page}{1}
\begin{flushleft}
  \vspace{0cm}
  {\large\bfseries Diario delle modifiche \par}
\end{flushleft}
\vspace{0cm}
\begin{center}
  \begin{tabular}{| l | l | l | l | p{5cm} |}
    \hline
    Versione & Autore & Ruolo & Data & Descrizione \\
    \hline
    0.0.2 & Marco Boseggia & Progettista & 2016-02-26 & Stesura sezioni 1 e 2\\
    \hline
    0.0.1 & Andrea Giacomo Baldan & Amministratore & 2016-02-26 & Creazione scheletro del documento\\
    \hline
  \end{tabular}
\end{center}
\tableofcontents
\newpage
\pagenumbering{arabic}
\section{Sommario}
\subsection{Scopo del documento}
Il seguente documento ha lo scopo di descrivere la progettazione ad alto livello che il gruppo \textit{ScalateKids} ha scelto per il progetto \textbf{ActorBase}.\\
Sarà descritta l'\gloss{architettura} generale del progetto, in particolare la scelta dei \gloss{design pattern} e delle componenti che andranno a comporre il software.
\prodPurpose
\glossExpl
\subsection{Riferimenti}
\subsubsection{Normativi}
\begin{itemize}
\item\textbf{Capitolato d'appalto C1:} \textit{Actorbase: a NoSQL DB based on the Actor model;}\\
  \url{http://www.math.unipd.it/~tullio/IS-1/2015/Progetto/C1.pdf}
\item\textbf{Norme di Progetto:} \href{run:../Interni/NormeDiProgetto\_v1.0.0.pdf}{Norme di Progetto v1.0.0.}
\end{itemize}
\subsubsection{Informativi}
\begin{itemize}
\item\textbf{Piano di Progetto:} \href{run:./PianoDiProgetto\_v1.0.0.pdf}{Piano di Progetto v1.0.0}
\item\textbf{Dispense fornite dall'insegnamento Ingegneria del Software mod. A:}\\
  \url{http://www.math.unipd.it/~tullio/IS-1/2015/}
\item\textbf{Documentazione di \gloss{Akka} su \gloss{Scala}:}
  \url{http://doc.akka.io/docs/akka/2.4.2/scala.html}
\end{itemize}
\newpage
\section{Tecnologie utilizzate}
In questa sezione saranno descritte le tecnologie scelte per lo sviluppo del progetto \textbf{ActorBase} e le motivazioni che ci hanno spinto a sceglierle.
\subsection{Akka}
La scelta della libreria \gloss{Akka} è stata dettata dal capitolato. Tuttavia l'avremmo scelta comunque per i seguenti motivi:
\begin{itemize}
\item\textbf{Implementazione del \gloss{modello ad attori}:} \gloss{Akka} rende semplice la creazione e l'utilizzo degli \gloss{attori}. Grazie a ciò viene naturale la distribuibilità del progetto e l'asincronia tra i processi;
\item\textbf{Tolleranza agli errori:} \gloss{Akka} implementa un sistema di gerarchia di supervisori. Questa gerarchia consente ai supervisori di un \gloss{attore}, nel caso in cui quest'ultimo lanci un'eccezione, di mandarlo in \gloss{crash} e farlo poi ripartire. In questo modo il sistema è resiliente agli errori;
\item\textbf{Persistenza:} Ogni messaggio ricevuto da ciascun \gloss{attore} può essere salvato in maniera tale che al riavvio esso possa riprendere da dov'era prima del suo spegnimento.
\end{itemize}
\subsection{Scala}
Il capitolato richiede l'utilizzo di un linguaggio di programmazione a scelta tra \gloss{Scala} e \gloss{Java}. Abbiamo scelto \gloss{Scala} per i seguenti motivi:
\begin{itemize}
\item\textbf{\gloss{Programmazione funzionale}:} Questa tecnica di programmazione evita che ci siano degli effetti collaterali tra funzioni, rendendole quindi \gloss{thread-safe};
\item\textbf{Implementazione di \gloss{Akka}:} \gloss{Akka} risulta più facilmente implementabile in \gloss{Scala} che in \gloss{Java}. Questo perchè la \gloss{programmazione funzionale} con i suoi paradigmi si integra meglio con il \gloss{modello ad attori};%%TODO completare lista
\end{itemize}

\section{Descrizione architettura}

\subsection{Metodo e formalismo di specifica}

L'approccio alla progettazione è stato meet-in-the-middle, la prima suddivisione ad alto livello è stata tra sistema come struttura ad attori e
componenti esterne per utilizzare la base di dati (operazioni CRUD)

\subsection{Architettura generale}

Macroscopicamente il sistema si suddivide in due componenti principali, assimilabili ad un paradigma Client-Server, il sistema ad attori
dove risiedono la struttura della base di dati ed un server TCP pronto a ricevere comandi dall'esterno rappresentano la componente server,
la Command Line Interface (CLI) rappresenta la componente client mediante la quale è possibile inviare comandi al sistema e ricevere l'output
prodotto da essi, essa farà uso della componente \gloss{driver} per comunicare con il sistema lato server mediante un protocollo di comunicazione
appositamente creato per lo scopo.

\begin{figure}[H]
  \begin{center}
    \includegraphics[width=0.2\textwidth,keepaspectratio]{RP/RP-ClientServer.png}
    \caption*{Diagramma architettura generale}
  \end{center}
\end{figure}

\section{Componenti}

Il sistema è suddiviso in tre macro-componenti.

\subsection{Server}

\subsection{Command Line Interface}

\subsection{Driver}

\section{Diagrammi attività}

\section{Design Pattern}

\subsection{Stime di fattibilità e di bisogno di risorse}

\section{Tracciamento}

\end{document}