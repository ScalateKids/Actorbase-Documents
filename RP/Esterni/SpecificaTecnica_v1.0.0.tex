\documentclass{scalatekids-article}
\usepackage[italian]{babel}
\begin{document}
\lfoot{Specifica Tecnica 1.0.0}
\newgeometry{top=3.5cm}
\begin{titlepage}
  \begin{center}
    \begin{center}
      \includegraphics[width=10cm]{sklogo.png}
    \end{center}
    \vspace{1cm}
    \begin{Huge}
      \begin{center}
        \textbf{Specifica Tecnica}
      \end{center}
    \end{Huge}
    \vspace{11pt}
    \bgroup
    \def\arraystretch{1.3}
    \begin{tabular}{r|l}
      \multicolumn{2}{c}{\textbf{Informazioni sul documento}} \\
      \hline
      \setbox0=\hbox{0.0.1\unskip}\ifdim\wd0=0pt
      \\
      \else
      \textbf{Versione} & 1.0.0\\
      \fi
      \textbf{Redazione} & \multiLineCell[t]{}\\
      \textbf{Verifica} & \multiLineCell[t]{}\\
      \textbf{Approvazione} & \multiLineCell[t]{}\\
      \textbf{Uso} & Esterno\\
      \textbf{Lista di Distribuzione} & \multiLineCell[t]{ScalateKids\\Prof. Tullio Vardanega\\Prof. Riccardo Cardin}\\
    \end{tabular}
    \egroup
    \vspace{22pt}
  \end{center}
\end{titlepage}
\restoregeometry
\clearpage
\pagenumbering{Roman}
\setcounter{page}{1}
\begin{flushleft}
  \vspace{0cm}
  {\large\bfseries Diario delle modifiche \par}
\end{flushleft}
\vspace{0cm}
\begin{center}
  \begin{tabular}{| l | l | l | l | p{5cm} |}
    \hline
    Versione & Autore & Ruolo & Data & Descrizione \\
    \hline
    0.4.0 & Michael Munaro & Verificatore & 2016-03-25 & Verifica sezione 5\\
    \hline
    0.3.1 & Giacomo Vanin & Progettista & 2016-03-24 & Stesura sezione 5\\
    \hline
    0.3.0 & Alberto De Agostini & Verificatore & 2016-03-22 & Verifica sezioni da 4.6 a 4.13\\
    \hline
    0.2.2 & Marco Boseggia & Progettista & 2016-03-22 & Stesura sezioni 4.10 4.11 4.12 4.13\\
    \hline
    0.2.1 & Marco Boseggia & Progettista & 2016-03-21 & Stesura sezioni 4.6 4.7 4.8 4.9\\
    \hline
    0.2.0 & Michael Munaro & Verificatore & 2016-03-21 & Verifica sottosezioni 3.1 3.2 4.1 4.2 4.3 4.4 4.5\\
    \hline
    0.1.2 & Andrea Giacomo Baldan & Progettista & 2016-03-17 & Stesura sottosezioni 4.1 4.2 4.3 4.4 4.5\\
    \hline
    0.1.1 & Andrea Giacomo Baldan & Progettista & 2016-03-17 & Stesura sottosezioni 3.1 3.2\\
    \hline
    0.1.0 & Michael Munaro & Verificatore & 2016-03-16 & Verifica sezioni 1 e 2\\
    \hline
    0.0.2 & Marco Boseggia & Progettista & 2016-02-26 & Stesura sezioni 1 e 2\\
    \hline
    0.0.1 & Andrea Giacomo Baldan & Amministratore & 2016-02-26 & Creazione scheletro del documento\\
    \hline
  \end{tabular}
\end{center}
\tableofcontents
\newpage
\pagenumbering{arabic}
\section{Sommario}
\subsection{Scopo del documento}
Il seguente documento ha lo scopo di descrivere la progettazione ad alto livello che il gruppo \textit{ScalateKids} ha scelto per il progetto \textbf{ActorBase}.\\
Sarà descritta l'\gloss{architettura} generale del progetto, in particolare la scelta dei \gloss{design pattern} e delle componenti che andranno a comporre il software.
\prodPurpose
\glossExpl
\subsection{Riferimenti}
\subsubsection{Normativi}
\begin{itemize}
\item\textbf{Capitolato d'appalto C1:} \textit{Actorbase: a NoSQL DB based on the Actor model;}\\
  \url{http://www.math.unipd.it/~tullio/IS-1/2015/Progetto/C1.pdf}
\item\textbf{Norme di Progetto:} \href{run:../Interni/NormeDiProgetto\_v1.0.0.pdf}{Norme di Progetto v1.0.0.}
\end{itemize}
\subsubsection{Informativi}
\begin{itemize}
\item\textbf{Piano di Progetto:} \href{run:./PianoDiProgetto\_v1.0.0.pdf}{Piano di Progetto v1.0.0}
\item\textbf{Dispense fornite dall'insegnamento Ingegneria del Software mod. A:}\\
  \url{http://www.math.unipd.it/~tullio/IS-1/2015/}
\item\textbf{Documentazione di \gloss{Akka} su \gloss{Scala}:}
  \url{http://doc.akka.io/docs/akka/2.4.2/scala.html}
\end{itemize}
\newpage
\section{Tecnologie utilizzate}
In questa sezione saranno descritte le tecnologie scelte per lo sviluppo del progetto \textbf{ActorBase} e le motivazioni che ci hanno spinto a sceglierle.
\subsection{Akka}
La scelta della libreria \gloss{Akka} è stata dettata dal capitolato. Tuttavia l'avremmo scelta comunque per i seguenti motivi:
\begin{itemize}
\item\textbf{Implementazione del \gloss{modello ad attori}:} \gloss{Akka} rende semplice la creazione e l'utilizzo degli \gloss{attori}. Grazie a ciò viene naturale la distribuibilità del progetto e l'asincronia tra i processi;
\item\textbf{Tolleranza agli errori:} \gloss{Akka} implementa un sistema di gerarchia di supervisori. Questa gerarchia consente ai supervisori di un \gloss{attore}, nel caso in cui quest'ultimo lanci un'eccezione, di mandarlo in \gloss{crash} e farlo poi ripartire. In questo modo il sistema è resiliente agli errori;
\item\textbf{Persistenza:} Ogni messaggio ricevuto da ciascun \gloss{attore} può essere salvato in maniera tale che al riavvio esso possa riprendere da dov'era prima del suo spegnimento.
\end{itemize}
\subsection{Scala}
Il capitolato richiede l'utilizzo di un linguaggio di programmazione a scelta tra \gloss{Scala} e \gloss{Java}. Abbiamo scelto \gloss{Scala} per i seguenti motivi:
\begin{itemize}
\item\textbf{\gloss{Programmazione funzionale}:} Questa tecnica di programmazione evita che ci siano degli effetti collaterali tra funzioni, rendendole quindi \gloss{thread-safe};
\item\textbf{Implementazione di \gloss{Akka}:} \gloss{Akka} risulta più facilmente implementabile in \gloss{Scala} che in \gloss{Java}. Questo perchè la \gloss{programmazione funzionale} con i suoi paradigmi si integra meglio con il \gloss{modello ad attori};%%TODO completare lista
\end{itemize}

\section{Descrizione architettura}

\subsection{Metodo e formalismo di specifica}

L'approccio alla progettazione è stato meet-in-the-middle, la prima suddivisione ad alto livello è stata tra sistema come struttura ad attori e
componenti esterne per utilizzare la base di dati (operazioni CRUD)

\subsection{Architettura generale}

Macroscopicamente il sistema si suddivide in due componenti principali, assimilabili ad un paradigma Client-Server, il sistema ad attori
dove risiedono la struttura della base di dati ed un server TCP pronto a ricevere comandi dall'esterno rappresentano la componente server,
la Command Line Interface (CLI) rappresenta la componente client mediante la quale è possibile inviare comandi al sistema e ricevere l'output
prodotto da essi, essa farà uso della componente \gloss{driver} per comunicare con il sistema lato server mediante un protocollo di comunicazione
appositamente creato.

\begin{figure}[H]
  \begin{center}
    \includegraphics[width=0.2\textwidth,keepaspectratio]{RP/RP-ClientServer.png}
    \caption{Diagramma architettura generale}
  \end{center}
\end{figure}

% TODO:
% \begin{figure}
%   \centering

%   \caption{Architettura generale - vista package}
%   \label{fig:architetturaGeneralePackage}
% \end{figure}

L'intero sistema è rappresentabile mediante una variante del design pattern
\gloss{MVC}, le tre componenti principali si prestano naturalmente
all'assegnazione delle tre parti che formano il design pattern, con la
differenza rispetto ad un approccio più classico, che l'aggiornamento della
parte \gloss{view} avviene sempre e comunque passando per la parte controller,
ciò è dovuto alla natura client-server del prodotto, che richiede un protocollo
di comunicazione ben preciso tra le due parti. Il sistema formato dai package
dedicati agli attori \gloss{akka} e dal server \gloss{TCP} per la comunicazione
con l'esterno sono assimilabili alla parte \gloss{model}, rappresenta la
\gloss{business logic} di \textbf{Actorbase}. La \gloss{CLI} è assimilabile alla
parte \gloss{view}, unico punto di interazione con l'utente, si occupa di
processare i comandi in input e restituire l'output opportunamente formattato,
la componente \gloss{driver} fa da ponte di comunicazione tra \gloss{CLI} e
server, occupandosi della ``traduzione'' dei comandi input secondo il protocollo
di comunicazione scelto, rappresenta la parte \gloss{controller}.

\section{Componenti}

\subsection{actorbase}

% TODO:
% \begin{figure}
%   \centering

%   \caption{Componente Actorbase}
%   \label{fig:componenteActorbase}
% \end{figure}

\subsubsection{Descrizione}

La componente principale rappresentata dal package globale \textbf{actorbase},
contiene le parti tipiche del design pattern \gloss{MVC}.

\subsubsection{Package contenuti}

\begin{itemize}
\item actorbase::model;
\item actorbase::controller;
\item actorbase::view.
\end{itemize}

Rappresentano le tre macro-componenti del sistema.

\subsection{actorbase::model}

\subsubsection{Descrizione}

\gloss{Package} per la componente \gloss{Model} dell'architettura \gloss{MVC}, rappresenta anche la componente system del prodotto.

\subsubsection{Package contenuti}

\begin{itemize}
\item system.
\end{itemize}

\subsection{actorbase::model::system}

\subsubsection{Descrizione}

\gloss{Package} per la componente system del prodotto.

\subsubsection{Package contenuti}

\begin{itemize}
\item server;
\item actors.
\end{itemize}

\subsection{actorbase::model::system::actors}

\subsubsection{Descrizione}

\gloss{Package} contenente gli \gloss{attori} che agiranno sul \gloss{database}.

\subsubsection{Interazioni con altre componenti}

\begin{itemize}
\item actorbase::model::system::server.
\end{itemize}

\subsubsection{Classi}

\paragraph{actorbase::model::system::actors::Warehouseman}

\subparagraph{Descrizione}

Classe che rappresenta un \gloss{attore} di tipo \gloss{Warehouseman}.

\subparagraph{Utilizzo}

Viene utilizzata per effettuare il salvataggio su filesystem del \gloss{database}.

\subparagraph{Classi ereditate}

\begin{itemize}
\item actorbase::model::system::actors::Actor.
\end{itemize}

\subparagraph{Relazioni con altre classi}

\begin{itemize}
\item actorbase::model::system::actors::Persistence.
\end{itemize}

\paragraph{actorbase::model::system::actors::Persistence}

\subparagraph{Descrizione}

Interfaccia che consente il salvataggio su filesystem.

\subparagraph{Utilizzo}

Viene utilizzata per offrire un'interfaccia per il salvataggio su filesystem.

\subparagraph{Relazioni con altre classi}

\begin{itemize}
\item actorbase::model::system::actors::Warehouseman.
\end{itemize}

\paragraph{actorbase::model::system::actors::BSONPersistence}

\subparagraph{Descrizione}

Classe che consente di salvare su filesystem in formato \gloss{BSON}.

\subparagraph{Utilizzo}

Viene utilizzata per effettuare il salvataggio su filesystem in formato \gloss{BSON} del \gloss{database}.

\subparagraph{Classi ereditate}

\begin{itemize}
\item actorbase::model::system::actors::Persistence.
\end{itemize}

\paragraph{actorbase::model::system::actors::Storefinder}

\subparagraph{Descrizione}

Classe che rappresenta un \gloss{attore} di tipo \gloss{Storefinder}.

\subparagraph{Utilizzo}

Viene utilizzata per tenere traccia dei contenuti degli attori di tipo \gloss{Storekeeper}.

\subparagraph{Classi ereditate}

\begin{itemize}
\item actorbase::model::system::actors::Actor.
\end{itemize}

\paragraph{actorbase::model::system::actors::MainActor}

\subparagraph{Descrizione}

Classe che rappresenta un \gloss{attore} di tipo \gloss{MainActor}.

\subparagraph{Utilizzo}

Viene utilizzata per tenere traccia dei contenuti degli attori di tipo \gloss{Storefinder}.

\subparagraph{Classi ereditate}

\begin{itemize}
\item actorbase::model::system::actors::Storefinder.
\end{itemize}

\paragraph{actorbase::model::system::actors::Storekeeper}

\subparagraph{Descrizione}

Classe che rappresenta un \gloss{attore} di tipo \gloss{Storekeeper}.

\subparagraph{Utilizzo}

Viene utilizzata per contenere gli elementi all'interno del \gloss{database}.

\subparagraph{Classi ereditate}

\begin{itemize}
\item actorbase::model::system::actors::Actor.
\end{itemize}

\paragraph{actorbase::model::system::actors::Manager}

\subparagraph{Descrizione}

Classe che rappresenta un \gloss{attore} di tipo \gloss{Manager}.

\subparagraph{Utilizzo}

Viene utilizzata per effettuare procedure di \gloss{load balancing}.

\subparagraph{Classi ereditate}

\begin{itemize}
\item actorbase::model::system::actors::Actor.
\end{itemize}

\paragraph{actorbase::model::system::actors::Ninja}

\subparagraph{Descrizione}

Classe che rappresenta un \gloss{attore} di tipo \gloss{Ninja}.

\subparagraph{Utilizzo}

Viene utilizzata per consentire di effettuare un \gloss{backup} dei contenuti degli attori di tipo \gloss{Storekeeper}.

\subparagraph{Classi ereditate}

\begin{itemize}
\item actorbase::model::system::actors::Actor.
\end{itemize}

\subsection{actorbase::controller}

\subsubsection{Descrizione}

\gloss{Package} per la componente \gloss{Controller} dell'architettura \gloss{MVC}.

\subsubsection{Package contenuti}

\begin{itemize}
\item driver.
\end{itemize}

\subsection{actorbase::controller::driver}

\subsubsection{Descrizione}

\gloss{Package} per la componente \gloss{driver} del prodotto.

\subsubsection{Package contenuti}

\begin{itemize}
\item client;
\item controls;
\item data.
\end{itemize}

\subsection{actorbase::controller::driver::client}

\subsubsection{Descrizione}

\gloss{Package} per la componente client del \gloss{driver} del \gloss{controller}.

\subsubsection{Interazioni con altre componenti}

\begin{itemize}
\item actorbase::model::system::server;
\item actorbase::view::cli.%TODO giusto?
\end{itemize}

\subsubsection{Classi}

\paragraph{actorbase::controller::driver::client::ActorbaseClient}

\subparagraph{Descrizione}

Classe che rappresenta un'istanza del \gloss{driver}.

\subparagraph{Relazioni con altre classi}

\begin{itemize}
\item actorbase::view::cli::commands::Command;
\item actorbase::controller::driver::client::Connection;
\item actorbase::controller::driver::controls::Control.
\end{itemize}

\paragraph{actorbase::controller::driver::client::Connection}

\subparagraph{Descrizione}

Classe per la connessione tra il \gloss{driver} e il \gloss{server}

\subparagraph{Relazioni con altre classi}
\begin{itemize}
\item actorbase::controller::driver::client::ActorbaseClient;
\item actorbase::model::system::server::TCPServer.
\end{itemize}

\subsection{actorbase::controller::driver::controls}

\subsubsection{Descrizione}

Componente del \gloss{controller} per l'esecuzione del comando e l'interazione con il \gloss{server}.

\subsubsection{Interazioni con altre componenti}

\begin{itemize}
\item actorbase::model::system::server.
\end{itemize}

\subsubsection{Classi}

\paragraph{actorbase::controller::driver::controls::Control}

\subparagraph{Descrizione}

Interfaccia generica che rappresenta un comando.

\subparagraph{Utilizzo}

Viene utilizzata per offrire un'interfaccia comune a tutti i comandi.

\subparagraph{Relazioni con altre classi}

\begin{itemize}
\item actorbase::controller::driver::client::ActorbaseClient;
\item actorbase::model::system::server::TCPServer.
\end{itemize}

\paragraph{actorbase::controller::driver::controls::FindControl}

\subparagraph{Descrizione}

Classe per il comando di ricerca.

\subparagraph{Utilizzo}

Viene utilizzata per mandare al \gloss{server} il comando di ricerca con i parametri immessi dall'utente.

\subparagraph{Classi ereditate}

\begin{itemize}
\item actorbase::controller::driver::controls::Control.
\end{itemize}

\paragraph{actorbase::controller::driver::controls::LogoutControl}

\subparagraph{Descrizione}

Classe per il comando di logout.

\subparagraph{Utilizzo}

Viene utilizzata per mandare al \gloss{server} il comando di logout.

\subparagraph{Classi ereditate}

\begin{itemize}
\item actorbase::controller::driver::controls::Control.
\end{itemize}

\paragraph{actorbase::controller::driver::controls::ResetPasswordControl}

\subparagraph{Descrizione}

Classe per il comando di reset della password.

\subparagraph{Utilizzo}

Viene utilizzata per mandare al \gloss{server} il comando di reset della password dell'utente specificato in input.

\subparagraph{Classi ereditate}

\begin{itemize}
\item actorbase::controller::driver::controls::Control.
\end{itemize}

\paragraph{actorbase::controller::driver::controls::ListControl}

\subparagraph{Descrizione}

Classe per il comando di visualizzazione dell'elenco dei nomi di tutte le collezioni presenti nel \gloss{database}.

\subparagraph{Utilizzo}

Viene utilizzata per mandare al \gloss{server} il comando di visualizzazione dell'elenco dei nomi di tutte le collezioni presenti nel \gloss{database}.

\subparagraph{Classi ereditate}

\begin{itemize}
\item actorbase::controller::driver::controls::Control.
\end{itemize}

\paragraph{actorbase::controller::driver::controls::AddContributorControl}

\subparagraph{Descrizione}

Classe per il comando di aggiunta collaboratore a una collezione.

\subparagraph{Utilizzo}

Viene utilizzata per mandare al \gloss{server} il comando di aggiunta di un collaboratore a una collezione con il nome della \gloss{collezione} e lo username specificati in input.

\subparagraph{Classi ereditate}

\begin{itemize}
\item actorbase::controller::driver::controls::Control.
\end{itemize}

\paragraph{actorbase::controller::driver::controls::LoginControl}

\subparagraph{Descrizione}

Classe per il comando di login.

\subparagraph{Utilizzo}

Viene utilizzata per mandare al \gloss{server} il comando di login con le credenziali immesse dall'utente.

\subparagraph{Classi ereditate}

\begin{itemize}
\item actorbase::controller::driver::controls::Control.
\end{itemize}

\paragraph{actorbase::controller::driver::controls::DeleteControl}

\subparagraph{Descrizione}

Classe per il comando di rimozione una \gloss{collezione}.

\subparagraph{Utilizzo}

Viene utilizzata per mandare al \gloss{server} il comando di rimozione di una collezione con i parametri specificati in input.

\subparagraph{Classi ereditate}

\begin{itemize}
\item actorbase::controller::driver::controls::Control.
\end{itemize}

\paragraph{actorbase::controller::driver::controls::RemoveContributorControl}

\subparagraph{Descrizione}

Classe per il comando di rimozione di un collaboratore da una collezione.

\subparagraph{Utilizzo}

Viene utilizzata per mandare al \gloss{server} il comando di rimozione di un collaboratore da una collezione con con il nome della \gloss{collezione} e lo username specificati in input.

\subparagraph{Classi ereditate}

\begin{itemize}
\item actorbase::controller::driver::controls::Control.
\end{itemize}

\paragraph{actorbase::controller::driver::controls::RenameControl}

\subparagraph{Descrizione}

Classe per il comando rinominazione di un item.

\subparagraph{Utilizzo}

Viene utilizzata per mandare al \gloss{server} il comando di rinominazione di un item con i parametri specificati in input.

\subparagraph{Classi ereditate}

\begin{itemize}
\item actorbase::controller::driver::controls::Control.
\end{itemize}

\paragraph{actorbase::controller::driver::controls::AddUserControl}

\subparagraph{Descrizione}

Classe per il comando di aggiunta di un utente.

\subparagraph{Utilizzo}

Viene utilizzata per mandare al \gloss{server} il comando di aggiunta di un utente con i parametri specificati in input.

\subparagraph{Classi ereditate}

\begin{itemize}
\item actorbase::controller::driver::controls::Control.
\end{itemize}

\paragraph{actorbase::controller::driver::controls::RemoveControl}

\subparagraph{Descrizione}

Classe per il comando di rimozione di un \gloss{item}.

\subparagraph{Utilizzo}

Viene utilizzata per mandare al \gloss{server} il comando di rimozione di un \gloss{item} con i parametri immessi dall'utente.

\subparagraph{Classi ereditate}

\begin{itemize}
\item actorbase::controller::driver::controls::Control.
\end{itemize}

\paragraph{actorbase::controller::driver::controls::RemoveUserControl}

\subparagraph{Descrizione}

Classe per il comando di rimozione di un utente.

\subparagraph{Utilizzo}

Viene utilizzata per mandare al \gloss{server} il comando di rimozione di un utente con i parametri immessi dall'utente.

\subparagraph{Classi ereditate}

\begin{itemize}
\item actorbase::controller::driver::controls::Control.
\end{itemize}

\paragraph{actorbase::controller::driver::controls::ImportControl}

\subparagraph{Descrizione}

Classe per il comando di importazione da un file \gloss{JSON}.

\subparagraph{Utilizzo}

Viene utilizzata per mandare al \gloss{server} il comando di importazione da un file \gloss{JSON} con i parametri immessi dall'utente.

\subparagraph{Classi ereditate}

\begin{itemize}
\item actorbase::controller::driver::controls::Control.
\end{itemize}

\paragraph{actorbase::controller::driver::controls::InsertControl}

\subparagraph{Descrizione}

Classe per il comando di inserimento di un \gloss{item} in una \gloss{collezione}.

\subparagraph{Utilizzo}

Viene utilizzata per mandare al \gloss{server} il comando di inserimento di un \gloss{item} in una \gloss{collezione} con i parametri immessi dall'utente.

\subparagraph{Classi ereditate}

\begin{itemize}
\item actorbase::controller::driver::controls::Control.
\end{itemize}

\paragraph{actorbase::controller::driver::controls::CreateControl}

\subparagraph{Descrizione}

Classe per il comando di creazione di una nuova \gloss{collezione}.

\subparagraph{Utilizzo}

Viene utilizzata per mandare al \gloss{server} il comando di creazione di una nuova \gloss{collezione} con i parametri immessi dall'utente.

\subparagraph{Classi ereditate}

\begin{itemize}
\item actorbase::controller::driver::controls::Control.
\end{itemize}

\paragraph{actorbase::controller::driver::controls::ExportControl}

\subparagraph{Descrizione}

Classe per il comando di esportazione in file \gloss{JSON}.

\subparagraph{Utilizzo}

Viene utilizzata per mandare al \gloss{server} il comando di esportazione in file \gloss{JSON} con i parametri immessi dall'utente.

\subparagraph{Classi ereditate}

\begin{itemize}
\item actorbase::controller::driver::controls::Control.
\end{itemize}

\subsection{actorbase::controller::driver::data}

\subsubsection{Descrizione}

\gloss{Package} per effettuare il parsing dei risultati ritornati dal \gloss{server}.

\subsubsection{Interazioni con altre componenti}
\begin{itemize}
\item actorbase::view::cli::commands;
\item actorbase::controller::driver::client;
\item actorbase::controller::driver::controls.
\end{itemize}

\subsubsection{Classi}

\paragraph{actorbase::controller::driver::data::DataStructure}

\subparagraph{Descrizione}

Interfaccia che consente al \gloss{driver} di effettuare il parsing dei file \gloss{JSON}.

\subparagraph{Relazioni con altre classi}

\begin{itemize}
\item actorbase::view::cli::commands::Command;
\item actorbase::controller::driver::client::Connection;
\item actorbase::controller::driver::controls::Control.
\end{itemize}

\paragraph{actorbase::controller::driver::data::Collection}

\subparagraph{Descrizione}

Classe che consente al \gloss{driver} di effettuare il parsing dei file \gloss{JSON} differenziandone le \gloss{collezioni}.

\subparagraph{Classi ereditate}

\begin{itemize}
\item actorbase::controller::driver::data::DataStructure.
\end{itemize}

\subparagraph{Relazioni con altre classi}

\begin{itemize}
\item actorbase::controller::driver::data::Item.
\item actorbase::controller::driver::data::CollectionCursor.
\end{itemize}

\paragraph{actorbase::controller::driver::data::Item}

\subparagraph{Descrizione}

Classe che consente al \gloss{driver} di effettuare il parsing dei file \gloss{JSON} differenziandone gli \gloss{item}.

\subparagraph{Classi ereditate}

\begin{itemize}
\item actorbase::controller::driver::data::DataStructure.
\end{itemize}

\subparagraph{Relazioni con altre classi}

\begin{itemize}
\item actorbase::controller::driver::data::ItemCursor.
\end{itemize}

\paragraph{actorbase::controller::driver::data::Cursor}

\subparagraph{Descrizione}

Interfaccia che offre un cursore per lo scorrimento di una lista.

\paragraph{actorbase::controller::driver::data::ItemCursor}

\subparagraph{Descrizione}

Classe che offre al \gloss{driver} un cursore per lo scorrimento di una lista di \gloss{item}.

\subparagraph{Classi ereditate}

\begin{itemize}
\item actorbase::controller::driver::data::Cursor.
\end{itemize}

\paragraph{actorbase::controller::driver::data::CollectionCursor}

\subparagraph{Descrizione}

Classe che offre al \gloss{driver} un cursore per lo scorrimento di una lista di \gloss{collezioni}.

\subparagraph{Classi ereditate}

\begin{itemize}
\item actorbase::controller::driver::data::Cursor.
\end{itemize}

\subsection{actorbase::view}

\subsubsection{Descrizione}

\gloss{Package} per la componente \gloss{View} dell'architettura \gloss{MVC}.

\subsubsection{Package contenuti}

\begin{itemize}
\item cli.
\end{itemize}

\subsection{actorbase::view::cli}

\subsubsection{Descrizione}

\gloss{Package} per la componente \gloss{cli} del prodotto.

\subsubsection{Package contenuti}

\begin{itemize}
\item actorbaseCli;
\item commands.
\end{itemize}

\subsection{actorbase::view::cli::actorbaseCli}

\subsubsection{Descrizione}

Componente parte della \gloss{View} per la gestione dei comandi e dell'output prodotto verso l'utente.

\subsubsection{Relazioni con altre classi}

\begin{itemize}
\item actorbase::view::cli::commands;
\item SpringShell.
\end{itemize}

\subsubsection{Classi}

\paragraph{actorbase::view::cli::actorbaseCli::PromptProvider}

\subparagraph{Descrizione}

Classe per la creazione del \gloss{prompt} dei comandi.

\subparagraph{Utilizzo}

Viene utilizzata per la creazione di un \gloss{prompt} per l'inserimento dei comandi e la visualizzazione dell'output.

\subparagraph{Classi ereditate}

\begin{itemize}
\item DefaultPromptProvider;
\item SpringShell.
\end{itemize}

\paragraph{actorbase::view::cli::actorbaseCli::HistoryFileNameProvider}

\subparagraph{Descrizione}

Classe per la gestione dello storico dei comandi \gloss{CLI}.

\subparagraph{Utilizzo}

Viene utilizzata per la creazione e gestione del sistema di storico dei comandi \gloss{CLI}.

\subparagraph{Classi ereditate}

\begin{itemize}
\item DefaultFilenameProvider;
\item SpringShell.
\end{itemize}

\paragraph{actorbase::view::cli::actorbaseCli::BannerProvider}

\subparagraph{Descrizione}

Classe per la creazione del \gloss{banner} d'ingresso alla \gloss{CLI}.

\subparagraph{Utilizzo}

Viene utilizzata per la creazione di un \gloss{banner} di presentazione all'ingresso della \gloss{CLI}.

\subparagraph{Classi ereditate}

\begin{itemize}
\item DefaultBannerProvider;
\item SpringShell.
\end{itemize}

\paragraph{actorbase::view::cli::actorbaseCli::CLICommands}

\subparagraph{Descrizione}

Classe per la gestione dei comandi della \gloss{CLI}.

\subparagraph{Utilizzo}

Viene utilizzata per la gestione mediante metodi appositi dei comandi della \gloss{CLI}, è responsabile della creazione e gestione dell'help sia generale che specifico per comando.

\subparagraph{Classi ereditate}

\begin{itemize}
\item CommandMarker.
\end{itemize}

\subparagraph{Relazioni con altre classi}

\begin{itemize}
\item actorbase::view::cli::commands::Command.
\end{itemize}

\paragraph{actorbase::view::cli::actorbaseCli::ResultViewer}

\subparagraph{Descrizione}

Classe per la gestione e la formattazione in output dei risultati ottenuti dall'esecuzione del comando ricevuto in input.

\subparagraph{Utilizzo}

Viene utilizzata per la gestione e la formattazione in output dei risultati ottenuti dal server mediante l'inserimento dei comandi.

\subsection{actorbase::view::cli::commands}

\subsubsection{Descrizione}

Componente parte della \gloss{View} per l'esecuzione del comando e l'interazione con il \gloss{Controller}.

\subsubsection{Interazioni con altre componenti}

\begin{itemize}
\item actorbase::view::cli::actorbaseCli;
\item actorbase::controller::driver::client.
\end{itemize}

\subsubsection{Classi}

\paragraph{actorbase::view::cli::commands::Command}

\subparagraph{Descrizione}

Interfaccia generica che rappresenta un comando.

\subparagraph{Utilizzo}

Viene utilizzata per offrire un'interfaccia comune a tutti i comandi.

\subparagraph{Relazioni con altre classi}

\begin{itemize}
\item actorbase::view::cli::actorbaseCli::CLICommands;
\item actorbase::view::cli::actorbaseCli::ResultViewer;
\item actorbase::controller::driver::client::ActorbaseClient.
\end{itemize}

\paragraph{actorbase::view::cli::commands::FindCommand}

\subparagraph{Descrizione}

Classe per il comando di ricerca.

\subparagraph{Utilizzo}

Viene utilizzata per mandare al \gloss{controller} il comando di ricerca con i parametri immessi dall'utente.

\subparagraph{Classi ereditate}

\begin{itemize}
\item actorbase::view::cli::commands::Command.
\end{itemize}

\paragraph{actorbase::view::cli::commands::LogoutCommand}

\subparagraph{Descrizione}

Classe per il comando di logout.

\subparagraph{Utilizzo}

Viene utilizzata per mandare al \gloss{controller} il comando di logout.

\subparagraph{Classi ereditate}

\begin{itemize}
\item actorbase::view::cli::commands::Command.
\end{itemize}

\paragraph{actorbase::view::cli::commands::ResetPasswordCommand}

\subparagraph{Descrizione}

Classe per il comando di reset della password.

\subparagraph{Utilizzo}

Viene utilizzata per mandare al \gloss{controller} il comando di reset della password dell'utente specificato in input.

\subparagraph{Classi ereditate}

\begin{itemize}
\item actorbase::view::cli::commands::Command.
\end{itemize}

\paragraph{actorbase::view::cli::commands::ListCommand}

\subparagraph{Descrizione}

Classe per il comando di visualizzazione dell'elenco dei nomi di tutte le collezioni presenti nel \gloss{database}.

\subparagraph{Utilizzo}

Viene utilizzata per mandare al \gloss{controller} il comando di visualizzazione dell'elenco dei nomi di tutte le collezioni presenti nel \gloss{database}.

\subparagraph{Classi ereditate}

\begin{itemize}
\item actorbase::view::cli::commands::Command.
\end{itemize}

\paragraph{actorbase::view::cli::commands::AddContributorCommand}

\subparagraph{Descrizione}

Classe per il comando di aggiunta collaboratore a una collezione.

\subparagraph{Utilizzo}

Viene utilizzata per mandare al \gloss{controller} il comando di aggiunta di un collaboratore a una collezione con il nome della \gloss{collezione} e lo username specificati in input.

\subparagraph{Classi ereditate}

\begin{itemize}
\item actorbase::view::cli::commands::Command.
\end{itemize}

\paragraph{actorbase::view::cli::commands::LoginCommand}

\subparagraph{Descrizione}

Classe per il comando di login.

\subparagraph{Utilizzo}

Viene utilizzata per mandare al \gloss{controller} il comando di login con le credenziali immesse dall'utente.

\subparagraph{Classi ereditate}

\begin{itemize}
\item actorbase::view::cli::commands::Command.
\end{itemize}

\paragraph{actorbase::view::cli::commands::DeleteCommand}

\subparagraph{Descrizione}

Classe per il comando di rimozione una \gloss{collezione}.

\subparagraph{Utilizzo}

Viene utilizzata per mandare al \gloss{controller} il comando di rimozione di una collezione con i parametri specificati in input.

\subparagraph{Classi ereditate}

\begin{itemize}
\item actorbase::view::cli::commands::Command.
\end{itemize}

\paragraph{actorbase::view::cli::commands::RemoveContributorCommand}

\subparagraph{Descrizione}

Classe per il comando di rimozione di un collaboratore da una collezione.

\subparagraph{Utilizzo}

Viene utilizzata per mandare al \gloss{controller} il comando di rimozione di un collaboratore da una collezione con con il nome della \gloss{collezione} e lo username specificati in input.

\subparagraph{Classi ereditate}

\begin{itemize}
\item actorbase::view::cli::commands::Command.
\end{itemize}

\paragraph{actorbase::view::cli::commands::RenameCommand}

\subparagraph{Descrizione}

Classe per il comando rinominazione di un item.

\subparagraph{Utilizzo}

Viene utilizzata per mandare al \gloss{controller} il comando di rinominazione di un item con i parametri specificati in input.

\subparagraph{Classi ereditate}

\begin{itemize}
\item actorbase::view::cli::commands::Command.
\end{itemize}

\paragraph{actorbase::view::cli::commands::AddUserCommand}

\subparagraph{Descrizione}

Classe per il comando di aggiunta di un utente.

\subparagraph{Utilizzo}

Viene utilizzata per mandare al \gloss{controller} il comando di aggiunta di un utente con i parametri specificati in input.

\subparagraph{Classi ereditate}

\begin{itemize}
\item actorbase::view::cli::commands::Command.
\end{itemize}

\paragraph{actorbase::view::cli::commands::RemoveCommand}

\subparagraph{Descrizione}

Classe per il comando di rimozione di un \gloss{item}.

\subparagraph{Utilizzo}

Viene utilizzata per mandare al \gloss{controller} il comando di rimozione di un \gloss{item} con i parametri immessi dall'utente.

\subparagraph{Classi ereditate}

\begin{itemize}
\item actorbase::view::cli::commands::Command.
\end{itemize}

\paragraph{actorbase::view::cli::commands::RemoveUserCommand}

\subparagraph{Descrizione}

Classe per il comando di rimozione di un utente.

\subparagraph{Utilizzo}

Viene utilizzata per mandare al \gloss{controller} il comando di rimozione di un utente con i parametri immessi dall'utente.

\subparagraph{Classi ereditate}

\begin{itemize}
\item actorbase::view::cli::commands::Command.
\end{itemize}

\paragraph{actorbase::view::cli::commands::ImportCommand}

\subparagraph{Descrizione}

Classe per il comando di importazione da un file \gloss{JSON}.

\subparagraph{Utilizzo}

Viene utilizzata per mandare al \gloss{controller} il comando di importazione da un file \gloss{JSON} con i parametri immessi dall'utente.

\subparagraph{Classi ereditate}

\begin{itemize}
\item actorbase::view::cli::commands::Command.
\end{itemize}

\paragraph{actorbase::view::cli::commands::InsertCommand}

\subparagraph{Descrizione}

Classe per il comando di inserimento di un \gloss{item} in una \gloss{collezione}.

\subparagraph{Utilizzo}

Viene utilizzata per mandare al \gloss{controller} il comando di inserimento di un \gloss{item} in una \gloss{collezione} con i parametri immessi dall'utente.

\subparagraph{Classi ereditate}

\begin{itemize}
\item actorbase::view::cli::commands::Command.
\end{itemize}

\paragraph{actorbase::view::cli::commands::CreateCommand}

\subparagraph{Descrizione}

Classe per il comando di creazione di una nuova \gloss{collezione}.

\subparagraph{Utilizzo}

Viene utilizzata per mandare al \gloss{controller} il comando di creazione di una nuova \gloss{collezione} con i parametri immessi dall'utente.

\subparagraph{Classi ereditate}

\begin{itemize}
\item actorbase::view::cli::commands::Command.
\end{itemize}

\paragraph{actorbase::view::cli::commands::ExportCommand}

\subparagraph{Descrizione}

Classe per il comando di esportazione in file \gloss{JSON}.

\subparagraph{Utilizzo}

Viene utilizzata per mandare al \gloss{controller} il comando di esportazione in file \gloss{JSON} con i parametri immessi dall'utente.

\subparagraph{Classi ereditate}

\begin{itemize}
\item actorbase::view::cli::commands::Command.
\end{itemize}

\section{Diagrammi di attività}
In questa sezione, vengono illustrati i diagrammi di attività che descrivono l'interazione dell'utente con Actorbase tramite l'utilizzo della \gloss{CLI} ad esso associata.
 É stato disegnato un diagramma ad alto livello in cui sono rappresentate le operazioni possibili sul database. Esse vengono poi illustrate tramite dei sotto-diagrammi per poterne comprendere meglio il funzionamento e facilitarne la lettura.

\subsection{Visione generale}
L'utente, dopo aver aperto la \gloss{CLI}, ha la possibilità di autenticarsi al database oppure chiedere dell'aiuto. Una volta autenticato, esso può operare su collezioni e/o item, interrogare il database, modificare la propria password, effettuare il logout oppure, se si tratta di un amministratore, gestire gli utenti del database.

\begin{figure}[H]
	\begin{center}
		\includegraphics[width=0.8\textwidth, keepaspectratio]{img/diagrammiAttivita/visioneGenerale.jpeg}
		\caption{Diagrammi attività - Visione generale}
	\end{center}
\end{figure}

\subsection{Operazioni su collezioni e/o item}
\subsubsection{Creazione collezione}
Questo tipo di operazione permette di inserire una collezione all'interno del database.
Come si vede dal grafico che segue, l'utente dovrà inserire il comando per la creazione della collezione, il nome della collezione stessa e i suoi parametri.
Una volta premuto il tasto invio, l'operazione andrà a buon termine se l'utente ha scritto correttamente il comando altrimenti verrà visualizzato un messaggio di errore.

\begin{figure}[H]
	\begin{center}
		\includegraphics[width=0.3\textwidth, keepaspectratio]{img/diagrammiAttivita/creazioneCollezione.jpeg}
		\caption{Diagrammi attività - Creazione collezione}
	\end{center}
\end{figure}

\subsubsection{Cancellazione collezione}
Questo tipo di operazione permette di cancellare una collezione dal database.
Come si vede dal grafico che segue, l'utente dovrà inserire il comando per la cancellazione di una collezione e il nome della collezione stessa.
Una volta premuto il tasto invio, l'operazione andrà a buon termine se l'utente ha scritto correttamente il comando altrimenti verrà visualizzato un messaggio di errore.

\begin{figure}[H]
	\begin{center}
		\includegraphics[width=0.3\textwidth, keepaspectratio]{img/diagrammiAttivita/cancCollezione.jpeg}
		\caption{Diagrammi attività - Cancellazione collezione}
	\end{center}
\end{figure}

\subsubsection{Visualizza collezioni}
Questo tipo di operazione permette di visualizzare le collezioni presenti all'interno dal database.
Come si vede dal grafico che segue, l'utente dovrà inserire il comando per la visualizzazione delle collezioni.
Una volta premuto il tasto invio, l'operazione andrà a buon termine (ricevendo la lista delle collezioni) se l'utente ha scritto correttamente il comando altrimenti verrà visualizzato un messaggio di errore.

\begin{figure}[H]
	\begin{center}
		\includegraphics[width=0.3\textwidth, keepaspectratio]{img/diagrammiAttivita/visCollezione.jpeg}
		\caption{Diagrammi attività - Visualizzazione collezioni}
	\end{center}
\end{figure}

\subsubsection{Modifica nome collezione}
Questo tipo di operazione permette di modificare il nome di una collezione presente nel database.
Come si vede dal grafico che segue, l'utente dovrà inserire il comando per la rinominazione della collezione, il nome della collezione stessa e il nuovo nome per essa.
Una volta premuto il tasto invio, l'operazione andrà a buon termine se l'utente ha scritto correttamente il comando altrimenti verrà visualizzato un messaggio di errore.

\begin{figure}[H]
	\begin{center}
		\includegraphics[width=0.3\textwidth, keepaspectratio]{img/diagrammiAttivita/modNomeCollezione.jpeg}
		\caption{Diagrammi attività - Modifica nome collezione}
	\end{center}
\end{figure}

\subsubsection{Inserimento item}
Questo tipo di operazione permette di inserire un item all'interno di una collezione del database.
Come si vede dal grafico che segue, l'utente dovrà inserire il comando per l'inserimento item, il valore e parametri dell'item stesso e il nome della collezione dove inserire l'item.
Una volta premuto il tasto invio, l'operazione andrà a buon termine (ricevendo la lista delle collezioni) se l'utente ha scritto correttamente il comando altrimenti verrà visualizzato un messaggio di errore.

\begin{figure}[H]
	\begin{center}
		\includegraphics[width=0.3\textwidth, keepaspectratio]{img/diagrammiAttivita/inserimentoItem.jpeg}
		\caption{Diagrammi attività - Inserimento item}
	\end{center}
\end{figure}

\subsubsection{Rimozione item}
Questo tipo di operazione permette di rimuovere un item dall'interno di una \gloss{collezione} del database.
Come si vede dal grafico che segue, l'utente dovrà inserire il comando per l'eliminazone di un item, il nome della collezione da dove rimuovere l'item e il nome dell'item stesso.
Una volta premuto il tasto invio, l'operazione andrà a buon termine (ricevendo la lista delle collezioni) se l'utente ha scritto correttamente il comando altrimenti verrà visualizzato un messaggio di errore.

\begin{figure}[H]
	\begin{center}
		\includegraphics[width=0.3\textwidth, keepaspectratio]{img/diagrammiAttivita/rimozioneItem.jpeg}
		\caption{Diagrammi attività - Rimozione item}
	\end{center}
\end{figure}

\subsubsection{Aggiunta collaboratore}
Questo tipo di operazione permette di aggiungere un collaboratore ad una collezione presente nel database.
Come si vede dal grafico che segue, l'utente dovrà inserire il comando per l'aggiunta di un collaboratore ad una collezione, lo username del collaboratore e il nome della collezione alla quale aggiungerlo.
Una volta premuto il tasto invio, l'operazione andrà a buon termine se l'utente ha scritto correttamente il comando altrimenti verrà visualizzato un messaggio di errore.

\begin{figure}[H]
	\begin{center}
		\includegraphics[width=0.3\textwidth, keepaspectratio]{img/diagrammiAttivita/aggCollaboratore.jpeg}
		\caption{Diagrammi attività - Aggiunta collaboratore}
	\end{center}
\end{figure}

\subsubsection{Rimozione collaboratore}
Questo tipo di operazione permette di rimuovere un collaboratore da una collezione presente nel database.
Come si vede dal grafico che segue, l'utente dovrà inserire il comando per la rimozione di un collaboratore da una collezione, lo username del collaboratore e il nome della collezione dalla quale rimuoverlo.
Una volta premuto il tasto invio, l'operazione andrà a buon termine se l'utente ha scritto correttamente il comando altrimenti verrà visualizzato un messaggio di errore.

\begin{figure}[H]
	\begin{center}
		\includegraphics[width=0.3\textwidth, keepaspectratio]{img/diagrammiAttivita/rimozioneCollaboratore.jpeg}
		\caption{Diagrammi attività - Rimozione collaboratore}
	\end{center}
\end{figure}

\subsubsection{Import}
Questo tipo di operazione permette di importare nel database collezioni o item tramite file \gloss{JSON}.
Come si vede dal grafico che segue, l'utente dovrà inserire il comando per l'importazione e il path che porta al file desiderato.
Una volta premuto il tasto invio, l'operazione andrà a buon termine se l'utente ha scritto correttamente il comando altrimenti verrà visualizzato un messaggio di errore.

\begin{figure}[H]
	\begin{center}
		\includegraphics[width=0.3\textwidth, keepaspectratio]{img/diagrammiAttivita/import.jpeg}
		\caption{Diagrammi attività - Import}
	\end{center}
\end{figure}

\subsection{Interrogazione del database}
Questo tipo di operazione permette di interrogare il database tramite delle query.
Come si vede dal grafico che segue, l'utente dovrà inserire il comando per l'interrogazione del database e i parametri per la ricerca.
Una volta premuto il tasto invio, l'operazione andrà a buon termine se l'utente ha scritto correttamente il comando, ricevendo il risultato della query, altrimenti verrà visualizzato un messaggio di errore.

\begin{figure}[H]
	\begin{center}
		\includegraphics[width=0.3\textwidth, keepaspectratio]{img/diagrammiAttivita/query.jpeg}
		\caption{Diagrammi attività - Interrogazione del database}
	\end{center}
\end{figure}

\subsection{Modifica password}
Questo tipo di operazione permette di modificare la propria password.
Come si vede dal grafico che segue, l'utente dovrà inserire il comando per la modifica della password, la vecchia password, la nuova password e confermare quest'ultima.
Una volta premuto il tasto invio, l'operazione andrà a buon termine se l'utente ha scritto correttamente il comando altrimenti verrà visualizzato un messaggio di errore.

\begin{figure}[H]
	\begin{center}
		\includegraphics[width=0.3\textwidth, keepaspectratio]{img/diagrammiAttivita/modificaPsw.jpeg}
		\caption{Diagrammi attività - Modifica password}
	\end{center}
\end{figure}

\subsection{Gestione utenti}
La gestione utenti, possibile solo ad un utente amministratore, prevede la possibilità di aggiungere o rimuovere un utente dal database e la possibilità di resettare la password di un utente.
Per aggiungere o rimuovere un utente, l'amministratore dovrà inserire il rispettivo comando, lo username dell'utente da aggiungere/rimuovere dal database e una conferma.
Per resettare la password di un utente, l'amministratore dovrà scrivere il comando per il reset della password, lo username dell'utente interessato e una conferma.
Una volta premuto il tasto invio, le operazione andranno a buon termine se l'utente ha scritto correttamente il comando altrimenti verrà visualizzato un messaggio di errore.

\begin{figure}[H]
	\begin{center}
		\includegraphics[width=0.3\textwidth, keepaspectratio]{img/diagrammiAttivita/gestioneUtenti.jpeg}
		\caption{Diagrammi attività - Gestione utenti}
	\end{center}
\end{figure}

\section{Design Pattern}

\subsection{Stime di fattibilità e di bisogno di risorse}

\section{Tracciamento}

\subsection{Tracciamento componenti - requisiti}

\subsection{Tracciamento requisiti - componenti}

\appendix
\section{Descrizione design pattern}
\listoftables
\listoffigures
\end{document}