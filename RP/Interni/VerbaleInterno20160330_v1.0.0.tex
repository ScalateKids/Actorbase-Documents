\documentclass{scalatekids-article}
\def\changemargin#1#2{\list{}{\rightmargin#2\leftmargin#1}\item[]}
\let\endchangemargin=\endlist
\begin{document}
\lfoot{Verbale esterno 2016-03-30 1.0.0}
\newgeometry{top=3.5cm}
\begin{titlepage}
  \begin{center}
    \begin{center}
      \includegraphics[width=10cm]{sklogo.png}
    \end{center}
    \vspace{1cm}
    \begin{Huge}
      \begin{center}
        \textbf{Verbale Esterno 2016-03-30}
      \end{center}
    \end{Huge}
    \vspace{11pt}
    \bgroup
    \def\arraystretch{1.3}
    \begin{tabular}{r|l}
      \multicolumn{2}{c}{\textbf{Informazioni sul documento}} \\
      \hline
      \setbox0=\hbox{0.0.1\unskip}\ifdim\wd0=0pt
      \\
      \else
      \textbf{Versione} & 1.0.0\\
      \fi
      \textbf{Redazione} & \multiLineCell[t]{Marco Boseggia}\\
      \textbf{Verifica} & \multiLineCell[t]{Michael Munaro}\\
      \textbf{Approvazione} & \multiLineCell[t]{Giacomo Vanin}\\
      \textbf{Uso} & Interno\\
      \textbf{Lista di Distribuzione} & \multiLineCell[t]{ScalateKids\\Prof. Tullio Vardanega\\Prof. Riccardo Cardin}\\
    \end{tabular}
    \egroup
    \vspace{22pt}
  \end{center}
\end{titlepage}
\restoregeometry
\clearpage
\pagenumbering{arabic}
\setcounter{page}{1}
\begin{flushleft}
  \vspace{0cm}
         {\large\bfseries Diario delle modifiche \par}
\end{flushleft}
\vspace{0cm}
\begin{center}
  \begin{tabular}{| l | l | l | l | l |}
    \hline
    Versione & Autore & Ruolo & Data & Descrizione \\
    \hline
    1.0.0 & Giacomo Vanin & Responsabile & 2016-03-30 & Validazione documento\\
    \hline
    0.1.1 & Michael Munaro & Verificatore & 2016-03-30 & Verifica documento\\
    \hline
    0.1.0 & Marco Boseggia & Progettista & 2016-03-30 & Prima stesura documento\\
    \hline
    0.0.1 & Marco Boseggia & Progettista & 2016-03-30 & Creazione scheletro del documento\\
    \hline
  \end{tabular}
\end{center}
\tableofcontents
\newpage
\section{Informazioni sulla riunione}
\begin{itemize}
\item \textbf{Data}: 2016-03-30
\item \textbf{Luogo}: Torre Archimede
\item \textbf{Orario d'inizio}: 13:15
\item \textbf{Durata}: 30'
\item \textbf{Partecipanti interni}: \textit{ScalateKids}
  \begin{itemize}
  \item Andrea Giacomo Baldan 
  \item Alberto De Agostini
  \item Marco Boseggia
  \item Michael Munaro
  \item Francesco Agostini
  \item Giacomo Vanin
  \item Davide Trevisan
  \end{itemize}
\end{itemize}

\section{Domande e risposte}
In questo capitolo sono elencate le domande fatte dal gruppo \textit{ScalateKids} in grassetto e le risposte del proponente \textit{Riccardo Cardin} in corsivo.

\subsection{Abbiamo trovato un componente (spring shell) che potremmo riutilizzare per la realizzazione della nostra \gloss{CLI}. Cosa ne pensa a riguardo?}
\begin{quote}
  \textit{L'idea è buona poiche permetterebbe un facile dialogo tra \gloss{CLI} e \gloss{driver} in quanto entrambi utilizzano \gloss{JVM}. Tuttavia il suo utilizzo porta ad avere molte dipendenze dovute allo \gloss{spring framework} nel progetto, quindi cercate di valutare bene gli aspetti positivi e negativi di tale scelta e vedete se è il caso di intraprendere questa strada.}
\end{quote}
\subsection{Abbiamo pensato ad una possibile divisione delle componenti ad alto livello secondo il pattern \gloss{MVC}, in particolare la \gloss{CLI} sarebbe la nostra \gloss{View}, il \gloss{Driver} sarebbe il nostro \gloss{Controller} e il package sistema sarebbe il \gloss{Model}, secondo lei è una scelta corretta?\\}
\begin{quote}
  \textit{Non credo sia una buona idea, le componenti che dite sono componenti separate tra loro, non cercherei di forzare un collegamento tra loro dal punto di vista architetturale.}
\end{quote}
\subsection{Avendo più nodi del \gloss{cluster} avevamo pensato o di scegliere dei nodi designati per salvare i dati, oppure far si che ogni nodo salvi un pezzo di database e all'accensione il ripopolamento avverrebbe tramite gli stessi nodi\\}
\begin{quote}
  \textit{Credo sia opportuno ipotizzare che al riavvio del database distribuito tra diversi nodi tutti questi debbano essere presenti nuovamente altrimenti saranno necessarie operazioni da sistemista. Quindi opterei per la seconda scelta.}
\end{quote}
\subsection{Se potessimo usare i \gloss{Persistence Actor} potremmo far spegnere nodi inutilizzati da molto tempo con previo snapshot su disco, questo snapshot poi sarà utilizzato per \"riaccendere\" l'attore se richiesto\\}
\begin{quote}
  \textit{State attenti con una feature come questa. Se ad esempio decideste di mettere in sleep attori dopo 5 secondi di inattività, qualora la vostra applicazione avesse dei momenti di poco carico potrebbe andare in sleep una gran parte del database se non tutto, andando a rallentare le prossime operazioni. In più ogni attore che va in sleep andrà a scrivere un file su disco e se non prevedete un modo intelligente di cancellarli potrebbero accumularsi in fretta.}
\end{quote}
\subsection{Abbiamo deciso di avere un numero di attori main configurabile per ogni nodo; ogni volta che un main decide di creare uno storefinder per la creazione di una nuova \gloss{collection} questo deve informare gli altri attori main per far si che ognuno abbia sempre una tabella aggiornata oppure ogni main ha il suo pezzetto di mappa e se la richiesta che riceve non può essere processata da lui allora manda il messaggio agli altri attori main\\}
\begin{quote}
  \textit{Le due scelte hanno i suoi pro e i suoi contro, la prima è più dispendiosa in fatto di memoria mentre la seconda è un pò più lenta. Secondo me entrambe le scelte vanno bene, cercate di pensare a cosa conviene a voi e cercate di far si che una scelta non escluda l'altra.}
\end{quote}
\subsection{Come possiamo rappresentare le relazioni tra attori in \gloss{UML}?\\}
\begin{quote}
  \textit{Nei diagrammi delle classi e dei package dovete solo pensare alle dipendenze e non ai messaggi. I messaggi andranno rappresentati semmai nei diagrammi di sequenza.}
\end{quote}


\section{Decisioni prese}
\begin{itemize}
  \item Cercare un alternativa a Spring Shell;
  \item Non suddividere con pattern MVC le varie componenti ;
  \item Dividere la persistenza dei dati tra i vari nodi del cluster;
  \item Valutare bene se mandare in sleep gli attori che non fanno operazioni per un certo tempo;
  \item Valutare se conviene tenere aggiornate le mappe di tutti gli storefinder su tutti i main oppure ogni main tiene un pezzo di mappa e instrada le richieste agli altri quando non può processare la richiesta che riceve;
  \item I messaggi tra attori andranno rappresentati solo nei diagrammi di sequenza
\end{itemize}

\end{document}