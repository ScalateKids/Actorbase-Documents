\documentclass{scalatekids-article}
\begin{document}
\lhead{ScalateKids}
\rhead{Studio di Fattibilità \thepage}
\lfoot{Studio di Fattibilità 0.0.1}
\begin{titlepage}
  \centering
  \includegraphics[width=0.45\textwidth]{logo.png}\par\vspace{1cm}
  \vspace{1.5cm}
         {\Huge\bfseries Studio di Fattibilità \par}
         \begin{center}
           \vspace{1.0cm}
                  {\large\bfseries Informazioni sul documento \par}
         \end{center}
         \vspace{-1cm}
         \begin{center}
           \line(1,0){300}
         \end{center}
         \vspace{0cm}
         \begin{tabular}[c]{l|l}
           \textbf{Versione} & 0.0.1\\
           \textbf{Redazione} & Redattore1\\ & Redattore2\\ & Redattore3..\\
           \textbf{Verifica} & Verificatore1\\ & Verificatore2..\\
           \textbf{Responsabile} & Responsabile\\
           \textbf{Uso} & Interno\\
           \textbf{Lista di distribuzione} & ScalateKids
         \end{tabular}
\end{titlepage}
\clearpage
\setcounter{page}{1}
\begin{flushleft}
  \vspace{0cm}
         {\large\bfseries Diario delle modifiche \par}
\end{flushleft}
\vspace{0cm}
\begin{center}
  \begin{tabular}{|l | l | l | l | l |}
    \hline
    Versione & Autore & Ruolo & Data & Descrizione \\
    \hline
    0.0.1 & Andrea Giacomo Baldan & & 2015-12-21 & Creazione scheletro del documento\\
    \hline
  \end{tabular}
\end{center}
\tableofcontents
\section{Sommario}
\subsection{Descrizione}
Questo documento mira ad esporre le considerazioni fatte nelle prime riunioni
del gruppo per analizzare e decidere quale capitolato scegliere e le motivazioni
derivanti che ci hanno spinto nella nostra scelta.

\subsection{Scopo del Prodotto}
Lo scopo del progetto è implementare un database \gloss{NoSQL} distribuito (di
tipo \gloss{key-value}) basato sul modello ad \gloss{attori}. Si richiede inoltre
la definizione di un \gloss{domain specific language} (DSL) da utilizzare da riga
di comando per poter interagire con il database.
\subsection{Glossario}
Abbiamo implementato un Glossario per spiegare e evitare ambiguità di
linguaggio dei documenti. Nello specifico sono stati inseriti nel glossario:
acronimi, termini tecnici, abbreviazioni e vocaboli che possono richiedere
ulteriori chiarimenti.

\section{Analisi capitolato scelto}

Riferimento: \url{http://www.math.unipd.it/~tullio/IS-1/2015/Progetto/C1.pdf}\\

\subsection{Sommario e descrizione capitolato}
Si è scelto il capitolato proposto dal professor Riccardo Cardin; il capitolato
prevede la creazione di un database NoSQL distribuito di tipo key-value basato
sul modello ad attori e la definizione di un linguaggio specifico per poter
interagire con tale database da riga di commando.

\subsection{Studio del capitolato}
Questo progetto racchiude più domini di conoscenza tra i quali certamente
consideriamo più importanti il modello ad \gloss{attori}, i database di tipo
\gloss{key-value} e il concetto di distribuzione/concorrenza.

\subsection{Riscontro possibile nel mondo reale}
Da diversi anni a questa parte per modelli di distribuzione molto grandi i
database a modello \gloss{Relazionale} sono stati "scavalcati" da altri tipi di
database NoSQL tra cui i database \gloss{key-value} poichè possiedono diverse
proprietà atte ad implementare diverse strutture applicative. Uno dei più
importanti database di questo tipo al momento è sicuramente \gloss{Amazon Dynamo}.
Troviamo quindi molto attuale e stimolante la realizzazione di un progetto come
questo.

\subsection{Tecnologie}
\begin{itemize}
\item Scala e più in particolare la libreria \gloss{AKKA} per l'implementazione del modello ad attori
\end{itemize}

\subsection{Possibili difficoltà o problemi}
Nessun membro del gruppo ha mai studiato o lavorato con le tencnologie
richieste, in particolare \gloss{Scala} e la libreria \gloss{Akka}. Quindi pensiamo non sarà
facile acquisire la padronanza di queste tecnologie e tanto meno capire il
funzionamento e concetto del modello ad attori. Nessuno di noi, inoltre, ha mai
creato un dominio specifico per l'interazione con un proprio prodotto (altra
difficoltà che incontreremo).

\subsection{Valutazione}
Abbiamo scelto questo capitolato principalmente per i seguenti motivi:
\begin{itemize}
\item Interesse del gruppo per le tecnologie richieste e per la modernità del prodotto che si andrà a creare.
\item Interesse per il concetto del modello ad attori.
\item Le esperienze e le conoscenze di tecnologie che si avranno acquisito a fine progetto riteniamo potranno essere utili a fini di curriculum e nel mondo lavorativo.
\end{itemize}
D'altra parte il gruppo ha pensato ai seguenti riscontri negativi:
\begin{itemize}
\item Conoscenza abbastanza scarsa del gruppo nell'utilizzo di tali tecnologie e degli argomenti in generale.
\item Quantità di lavoro difficilmente prevedibile proprio per la mancanta conoscenza dell'ambito descritto.
\end{itemize}

\section{Analisi altri capitolati}
\subsection{C2 CLIPS}
Riferimento: \url{http://www.math.unipd.it/~tullio/IS-1/2015/Progetto/C2.pdf}\\
Abbiamo trovato la spiegazione del capitolato un pò vaga e lo scopo prefissato
dal fornitore sembra essere molto esplorativo in quanto viene richiesto di
ricercare nuove idee di utilizzo per i dispositivi \gloss{Beacon}. Inoltre l'uso di
questi ultimi necessita un grande dispendio di risorse per i dispositivi mobili
dell'utente finale poichè devrà essere installata e aperta l'applicazione ed
attivata la localizzazione gps e la rete dati mobile. Troviamo, dunque, un pò
immatura tale tecnologia.
\subsection{C3 Internet of Things}
Riferimento: \url{http://www.math.unipd.it/~tullio/IS-1/2015/Progetto/C3.pdf}\\
Nonostante le tecnologie proposte fossero di interesse per il gruppo abbiamo
deciso di non optare per questo capitolato per via della sua ambiguità, la
grande quantità di tecnologie da imparare e per la mole di lavoro, secondo noi,
troppo onerosa.
\subsection{C4 MaaS}
Riferimento: \url{http://www.math.unipd.it/~tullio/IS-1/2015/Progetto/C4.pdf}\\ Il
gruppo ha trovato questo capitolato poco stimolante dato che l'obbietivo finale
del progetto richiede la trasformazione da programma a servizio di un software
già esistente.
\subsection{C5 Quizzipedia}
Riferimento: \url{http://www.math.unipd.it/~tullio/IS-1/2015/Progetto/C5.pdf}\\
Abbiamo trovato anche questo capitolato poco stimolante e le tecnologie
richieste sono quasi tutte già apprese da altri esami del corso di laurea.
Inoltre abbiamo trovato molti servizi web per la creazione di quiz online simili
a quello richiesto perciò non riteniamo questo essere un prodotto interessante
per il mercato. L'unica nota positiva e "sfidante" è il requisito opzionale,
ovvero la possibilità di gestire domande aperte; questo da solo non giustifica
la scelta del capitolato.
Qualche esempio:
\begin{itemize}
\item\url{https://www.onlinequizcreator.com/it/}
\item\url{www.proprofs.com/quiz-school/create-a-quiz.php}
\item\url{http://www.poll-maker.com/QuizMaker}
\end{itemize}
\subsection{C6 SiVoDim}
Riferimento: \url{http://www.math.unipd.it/~tullio/IS-1/2015/Progetto/C6.pdf}\\
Questo capitolato era la seconda opzione di scelta emersa dalla riunione del gruppo.
Lati positivi:
\begin{itemize}
\item Interesse nelle tecnologie mobile.
\item Spiegazione e obbiettivi del capitolato molto chiari.
\end{itemize}
Lati negativi:
\begin{itemize}
\item Scarsa conoscenza delle tecnologie mobile.
\item Grande differenza tra requisiti obbligatori e requisiti opzionali/desiderabili.
\end{itemize}
\end{document}
