\documentclass{scalatekids-article}
\def\changemargin#1#2{\list{}{\rightmargin#2\leftmargin#1}\item[]}
\let\endchangemargin=\endlist
\begin{document}
\lfoot{Verbale esterno 2016-01-12 1.0.0}
\newgeometry{top=3.5cm}
\begin{titlepage}
  \begin{center}
    \begin{center}
      \includegraphics[width=10cm]{sklogo.png}
    \end{center}
    \vspace{1cm}
    \begin{Huge}
      \begin{center}
        \textbf{Verbale Esterno 2016-01-12}
      \end{center}
    \end{Huge}
    \vspace{11pt}
    \bgroup
    \def\arraystretch{1.3}
    \begin{tabular}{r|l}
      \multicolumn{2}{c}{\textbf{Informazioni sul documento}} \\
      \hline
      \setbox0=\hbox{0.0.1\unskip}\ifdim\wd0=0pt
      \\
      \else
      \textbf{Versione} & 1.0.0\\
      \fi
      \textbf{Redazione} & \multiLineCell[t]{Alberto De Agostini}\\
      \textbf{Verifica} & \multiLineCell[t]{Marco Boseggia}\\
      \textbf{Approvazione} & \multiLineCell[t]{Andrea Giacomo Baldan}\\
      \textbf{Uso} & Esterno\\
      \textbf{Lista di Distribuzione} & \multiLineCell[t]{ScalateKids\\Prof. Tullio Vardanega\\Prof. Riccardo Cardin}\\
    \end{tabular}
    \egroup
    \vspace{22pt}
  \end{center}
\end{titlepage}
\restoregeometry
\clearpage
\pagenumbering{arabic}
\setcounter{page}{1}
\begin{flushleft}
  \vspace{0cm}
         {\large\bfseries Diario delle modifiche \par}
\end{flushleft}
\vspace{0cm}
\begin{center}
  \begin{tabular}{| l | l | l | l | l |}
    \hline
    Versione & Autore & Ruolo & Data & Descrizione \\
    \hline
    1.0.0 & Andrea Giacomo Baldan & Responsabile & 2016-01-13 & Validazione documento\\
    \hline
    0.1.1 & Marco Boseggia & Verificatore & 2016-01-12 & Verifica documento\\
    \hline
    0.1.0 & Alberto De Agostini & Analista & 2016-01-12 & Prima stesura documento\\
    \hline
    0.0.1 & Alberto De Agostini & Analista & 2016-01-12 & Creazione scheletro del documento\\
    \hline
  \end{tabular}
\end{center}
\tableofcontents
\newpage
\section{Informazioni sulla riunione}
\begin{itemize}
\item \textbf{Data}: 2015-01-12
\item \textbf{Luogo}: Torre Archimede aula 1C150
\item \textbf{Orario d'inizio}: 13:15
\item \textbf{Durata}: 25'
\item \textbf{Partecipanti interni}: \textit{ScalateKids}
  \begin{itemize}
  \item Andrea Giacomo Baldan
  \item Alberto De Agostini
  \item Francesco Agostini
  \item Giacomo Vanin
  \item Marco Boseggia
  \item Michael Munaro
  \end{itemize}
\item \textbf{Partecipanti esterni}:
  \begin{itemize}
  \item Riccardo Cardin
  \end{itemize}
\end{itemize}
\section{Domande e risposte}
In questo capitolo sono elencate le domande fatte dal gruppo \textit{ScalateKids} in grassetto e le risposte del proponente \textit{Riccardo Cardin} in corsivo.
\textbf{\\ \\Nel capitolato non c'è una chiara divisione tra requisiti desiderabili e requisiti opzionali, c'è semplicemente un elenco di requisiti opzionali. Vi è una preferenza da parte sua tra questi?}
\begin{quote}
  \textit{Avendoli lasciati tutti opzionali nel capitolato, rimando a voi la
    scelta su quali pensate siano più importanti, cercando di implementare prima
    gli opzionali che al meglio completano quelli obbligatori.\\ \\}
\end{quote}
\textbf{Dovremo istanziare un terminale isolato? (come ad esempio una \gloss{shell} MySql)}
\begin{quote}
  \textit{No, la \gloss{CLI} da creare sarà composta da un gruppo di semplici
    comandi da usare direttamente da \gloss{terminale}, il quale non sarà altro
    che (uno dei) client che si connetteranno al DB al fine di effettuare
    operazioni. Ciò naturalmente non implica che il prodotto sia da sviluppare
    esclusivamente per ambienti linux, \gloss{terminale} è inteso come
    \gloss{ui} comandi generica(es \gloss{bash}).\\ \\}
\end{quote}
\textbf{Nel capitolato, come requisiti obbligatori, sono presenti solo i comandi per inserimento, cancellazione e modifica dati. Un database generalmente consente più operazioni di queste, alcune delle quali riteniamo essere essenziali (es. \gloss{query}). Sono queste da considerarsi requisiti opzionali?}
\begin{quote}
  \textit{Sicuramente dev'essere possibile interrogare il database, gli altri comandi basilari sono dei requisiti obbligatori impliciti. La lista delle funzionalità tuttavia è una cosa che voi dovete decidere e scrivere nei vostri documenti.\\ \\}
\end{quote}
\textbf{Nel capitolato c'è scritto che gli \gloss{attori} di tipo \textit{Manager} sono responsabili della gestione del numero massimo di coppie chiave / valore contenute in un’istanza di un \gloss{attore storekeeper}. Con che criterio deve essere scelto questo numero massimo?}
\begin{quote}
  \textit{Può essere scelto in diversi modi; il più facile di tutti è senz'altro scegliere un numero prefissato da impostare, altrimenti si potrebbe scegliere di cambiare questo numero dinamicamente in base al carico. Tuttavia questo non è un compito semplice, quindi anche in questo caso la decisione spetta a voi.\\ \\}
\end{quote}
\textbf{Abbiamo deciso di utilizzare il linguaggio \textit{Scala} per sviluppare il prodotto. Le ultime versioni di questo linguaggio non sono compatibili tra loro, quale versione ci consiglia di utilizzare? (2.10 o 2.11)}
\begin{quote}
  \textit{Suggerirei di utilizzare le ultime versioni, soprattutto dal punto di
    vista di compatibilità, utilizzare la \gloss{major} più recente vi darebbe qualche
    garanzia in più nella compatibilità con le librerie chiave del progetto
    (\gloss{akka}).La \gloss{JVM} di riferimento dovrebbe essere dunque la
    versione 8.\\ \\}
\end{quote}
\textbf{Cosa dovrà arrivare all'utente finale visto come utilizzatore? Le librerie \gloss{akka} dovrebbero figurare come dipendenze da installare o dovrebbero essere fornite assieme al prodotto?}
\begin{quote}
  \textit{Dovrete consegnare un eseguibile comprensivo di librerie
    \gloss{bundled} per semplificare l'installazione all'utente. Tuttavia è da
    tenere presente che il prodotto dovrà essere al 100\% \gloss{scalabile
    orizzontalmente}, il suo ambito di utilizzo quindi dovrà spaziare dal singolo
    privato, a grandi reti di server, all'occorenza.}
\end{quote}
\textbf{L'attore \gloss{warehousemen} dovrà occuparsi della persistenza su disco dei dati inseriti nelle mappe. Quale approccio consiglia?}
\begin{quote}
  \textit{Sicuramente dovrà essere fornita un interfaccia da parte dell'attore
    che si occupa della persistenza, come effettivamente venga implementato il
    salvataggio dati su disco è lasciato a voi, l'unico requisito importante è
    l'estendibilità(e questo si applica a tutto il sistema), deve essere
    possibile estendere l'attore che si occupa di persistenza senza difficoltà,
    nel caso in cui un utente decida di implementare il proprio
    \gloss{warehousemen} e sostituirlo alla versione ``base''}
\end{quote}
\end{document}
