\section{Documenti}
	// TODO: bisogna aggiungere la parte in cui diciamo che usiamo una virtualbox per facilitare tutto

\subsection{Struttura dei documenti}
Abbiamo standardizzato la scrittura di tutti i documenti attraverso un \summ{template} \LaTeX\xspace creato da noi e presente su \textit{github} nella \summ{repository} \textit{ActorBase - Documents}.\\
L'uso del template ci permette di creare una serie di documenti stilisticamente uniformi tra loro e ci facilita la modifica delle parti in comune tra essi.\\
Il template regola:
\begin{itemize}
\item Formattazione dei documenti (font);
\item Formattazione delle pagine (header e footer pagine);
\item Raggruppa tutti i pacchetti utilizzati;
\item Aggiunge comandi personalizzati;
\item Collegamenti tra indice e categorie o sezioni
\end{itemize}
	// TODO: rivedere cosa regola il template

Ogni documento presenta in ordine :
\begin{itemize}
\item Diario delle modifiche: è una tabella in cui sono scritte le diverse modifiche fatte dai membri sul documento;
\item Indice: indice con collegamenti alle categorie e alle sezioni del documento;
\item Introduzione: contiene alcune sottosezioni:
	\begin{itemize}
	\item Scopo del documento;
	\item Glossario;
	\item Riferimenti inerenti al documento in questione;
	\end{itemize}
\item Resto del documento: contenuto specifico;
\end{itemize}
	\\TODO: riferimenti li mettiamo?


Abbiamo creato uno script \textit{perl} per la creazione automatica dello scheletro di un generico documento. 

\subsection{Struttura Pagina}
	//TODO: controllare che questa struttura delle pagina sia effettivamente così
L'intestazione di ogni pagina contiene:
\begin{itemize}
\item Logo gruppo;
\item Nome gruppo;
\item Sezione corrente del documento;
\end{itemize}
Il piè pagina invece contiene:
\begin{itemize}
\item Nome doocumento;
\item Pagina corrente rispetto al numero di pagine totali;
\end{itemize}


\subsection{Norme tipografiche}
Per uniformare il più possibile la stesura di tutti i documenti abbiamo scelto delle regole che tutto il gruppo deve seguire.

\subsubsection{Punteggiatura}
\begin{itemize}
\item \textbf{Punteggiatura}: tutti i segni di punteggiatura devono essere seguiti da uno spazio e non avere spazi precedenti al segno stesso;
\item \textbf{Maiuscole}: le lettere maiuscole devono essere usate dove previsto dalla lingua italiana.
Saranno inoltre scritti con l'iniziale maiuscolta i ruoli, le persone inerenti al progetto e i documenti noti (es. \textit{Committente} \textit{Analisi dei Requisiti});
\end{itemize}

\subsubsection{Stile testo}:
\begin{itemize}
\item \textbf{Corsivo}: il corsivo dev'essere utilizzato per:
	\begin{itemize}
	\item Abbreviazioni;
	\item Citazioni;
	\item Documenti (es \textit{Analisi dei Requisiti});
	\item Porre un enfasi maggiore alla parola e/o frase;
	\end{itemize}
\item \textbf{Grassetto}: il grassetto dev'essere usato per:
	\begin{itemize}
	\item Evidenziare passaggi o concetti importanti;
	\item Sezioni e sottosezioni dei documenti;
	\end{itemize}
\item \textbf{Maiuscolo}: alcuni acronimi saranno scritti interamente in maiuscolo (es. \textit{SQL});
\item \textbf{Monospace}: i font monospace saranno usati per riportare parti di codice;
\end{itemize}

\subsection{Formati di riferimento}
\begin{itemize}
\item \textbf{Riferimenti}:
	\begin{itemize} 
	\item Percorsi web: per gli indirizzi web e per gli inidirizzi e-mail deve essere usato il comando \LaTeX\xspace 
 	\begin{center}
 	\textbackslash{url{percorso}}
 	\end{center}
	\item Link ad altri PDF: //TODO: dobbiamo decidere come fare sta cosa
	\item Link a sezioni interne al documento: per link a sezioni interne al documento corrente deve essere utilizzato il documento \LaTeX\xspace
	\begin{center} 
	\textbackslash{ref{riferimento a sezione}}
	\end{center}
	\end{itemize}
\item \textbf{Date}: Le date devono essere espresse seguento lo standard \textit{\summ{ISO}} 8601:2004:
	AAAA-MM-GG
	AAAA: rappresenta l'anno (es. 2015);
	MM:	rappresenta il mese (es. 12);
	GG: rappresenta il giorno (es. 21);

\item \textbf{Abbreviazioni}: per semplicità possono essere abbreviati i nomi dei seguenti documenti:
	AR: \textit{Analisi dei Requisiti}
	RR: \textit{Revisione dei Requisiti}
	GL: \textit{Glossario}
	NP: \textit{Norme di Progetto}
	RP: \textit{Revisione di Progettazione}
	PQ: \textit{Piano di Qualifica}
	RQ: \textit{Revisione di Qualifica}
	PP: \textit{Piano di Progetto}
	SF: \textit{Studio di Fattibilità}
	RA: \textit{Revisione di Accettazione}

\item \textbf{Nomi ricorrenti}:
	\begin{itemize}
	\item Ruoli: come già scritto in precedenza i ruoli di progetto devono essere scritti con la prima lettera di ogni parola maiuscola ed in corsivo, escludendo le proposizioni (es. \textit{Responsabile di Progetto});
	\item Nomi dei files: i nomi dei files relativi a documenti devono seguire la notazione \textit{\summ{CamelCase}}, seguiti da \textit{_v} e dalla versione del file (es. NormeDiProgetto_v1.0.1.pdf);
	\item Nome del progetto: sarà indicato come \textbf{Actorbase}
	\item Nome del committente: il committente sarà indicato come \textit{prof. Tullio Vardanega}
	\item Nome del proponente: il proponente sarà indicato come \textit{prof. Riccardo Cardin}
	\item Nome del gruppo: il gruppo sarà indicato come \textit{ScalateKids}
	\end{itemize}
\end{itemize}

\subsection{Tabelle}
//TODO: mettiamo tabelle? se si decidere come uniformarle

\subsection{Immagini}
Utilizziamo immagini con formato \textit{JPG} o \textit{PNG}, questi ultimi rendono immediata l'inculsione delle suddette immagini nei documenti.

\subsection{Integrazione termini di lingua straniera}
Per non incorrere in diverse modalità di integrazione per l'uso di vari termini di lingua straniera (es. file, repository ecc... ) abbiamo deciso di non pluralizzare i termini non italiani, esempi:\\
abbiamo utilizzato delle \textit{repository} per... (non \textit{repositories}).
Questa scelta è derivata da una ricerca che ha fornito diverse fonti riguardanti l'uso di questa norma nell'uso della lingua italiana, il sottostante sito web raccoglie diverse fonti trovate:\\
\begin{center}
\url{http://www.darsch.it/?pg=sphere&postid=954}
\end{center}

\subsection{Versionamento documenti}
La documentazione prodotta deve avere un numero di versione avente la seguente struttura:\\
\begin{center}
vX.Y.Z
\end{center}
in cui:
\begin{itemize}
\item \textbf{X}: aumenta con il numero di uscite formali del documento;
\item \textbf{Y}: rappresenta il numero di modifiche sostanziali effettuate sul documento. Se pari a 0 indica che il documento è ancora in fase di stesura in presente revisione;
\item \textbf{Z}: rappresenta il numero di modifiche minori effettuate sul documento.  
\end{itemize}

\subsection{Ciclo di vita dei documenti}
Tutti i documenti nascono nello stato di \textit{in lavorazione}.\\
Dopo l'avvenuta di tute le modifiche necessarie per arrivare ad una versione completa del documento, il suddetto si verrà a trovare nello stato \textit{da verificare}.\\
Alla fine della fase di verifica del documento, se approvato il documento andrà nello stato di \textit{approvato}.
I tre stati sono descritti brevemente come segue:
\begin{itemize}
\item \textbf{In lavorazione}: il documento nasce in questo stato e ci rimane fino a che non è avvenuta una sua completa stesura;
\item \textbf{Da verificare}: il documento resta in questo stato finchè i \textif{Verificatori} assegnati ad esso non effettueranno un controllo atto a trovare e a correggere errori di ogni tipo;
\item \textbf{Approvato}: dopo la fase di verifica il \textit{Responsabile} effettuerà una fase di approvazione, se questa fase sarà superata il documento sarà approvato e entrerà in una versione ufficiale.
\end{itemize}