\section{COMUNICAZIONI}
\subsection{Comunicazioni esterne}
Per le comunicazioni esterne è stata creata una apposita casella di poste elettronica: 
\begin{center}
scalatekids@gmail.com 	
\end{center}
Questa casella e-mail viene gestita dal \textit{Responsabile di Progetto} che è colui che si occupa di comunicare con il committente del progetto o con qualsiasi altra persona esterna al gruppo di lavoro. Sarà sempre quest'ultimo ad informare inoltrando le email di interesse ai vari membri del gruppo.

\subsection{Comunicazioni interne}
Per le comunicazioni interne è stata creata una \summ{mailing list}:
\begin{center}
TODO: INSERIRE LA MAILING LIST 
\end{center}
Tale indirizzo verrà utilizzato da tutti i membri del gruppo e solamente per lo scambio interno di informazioni inerenti al progetto.

Vengono usati altri strumenti più ufficiosi per lo scambio di informazioni:
\begin{itemize}
\item applicazioni di \summ{instant messaging} quali \textit{Whatsapp} e il suo servizio \textit{Whatsapp Web};
\item applicazioni \summ{VoIP} per effettuare audio conferenze con un qualsiasi numero di componenti, quale \textit{TeamSpeak3};
\item servizi online per la gestione di una \summ{whiteboard} condivisa, quale \textit{Twiddla};
\end{itemize}

Abbiamo deciso che qualora membri del gruppo si scambino informazioni attraverso questi strumenti sopra elencati dovranno, se presenti informazioni di interesse per il progetto, mandare un sunto della suddetta conversazione alla \summ{mailing list} per tenere informati tutti i membri del gruppo sullo stato di avanzamento.\\
Solo in caso di necessità derivate dalla mancanza di infrastrutture (es. collegamento a internet) si potrà comunicare attraverso SMS o chiamate telefoniche, anche in questo caso si dovrà verbalizzare quanto detto attraverso la \summ{mailing list}.

/*
	TODO: don't panic spiegano anche la struttura che devono avere le mail per uniformare tutte le comunicazioni del gruppo (interne ed esterne), da decidere come vogliamo proseguire.
*/
