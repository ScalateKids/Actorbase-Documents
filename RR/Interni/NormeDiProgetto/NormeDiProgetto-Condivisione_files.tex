\section{Condivisione Files}
Il gruppo usa diversi software per la condivisione di files in base alle necessità.

\subsection{Google Drive}
\textit{Google Drive} è usato per lo scambio di files e documenti all'interno del gruppo, principalmente per files che non necessitano di versionamento.
Questo spazio del gruppo è suddiviso in diverse cartelle per una migliore gestione e per una facilitazione per la ricerca di vari files.
Il \textit{Responsabile di Progetto} ha il potere amministrativo dell'account in questione, quindi sarà quest'ultima persona ad occuparsi di garantire i giusti diritti ai membri del gruppo.

\subsection{Repository - Github}
Per tutti i files che necessitano un sistema di \sum{versionamento} viene usato il servizio offerto da \textit{Github}, l'indirizzo web del gruppo è:
\begin{center}
https://github.com/scalatekids
\end{center}

In tale indirizzo sono poi presenti diverse \sum{repositories} per suddividire i vari files, sono presenti le \sum{repository}:
\begin{itemize}
  \item \textbf{Actorbase:} conterrà il codice sorgente del prodotto e le versioni definitive della documentazione.
  \item \textbf{ActorBase - Documents}: conterrà i file sorgenti usati per la creazione della documentazione.
\end{itemize}
	// TODO: man mano che aggiungiamo repositories credo dovremmo aggiungerle qui con annessa spiegazione.

\subsection{Redmine}
Il gruppo dispone di una sezione wiki su \textit{redmine} contenente una serie di link uscenti a tutorials e a guide utili a tutto il gruppo.
// TODO: da mettere sta cosa? scrivere qualcosa in più?
