\section{Ambiente di lavoro}
\subsection{Apparato di collaborazione}
Per coordinare il lavoro tra i componenti del gruppo, trovandosi spesso a dover
operare in remoto, l'azienda ha scelto di utilizzare quanto più possibile gli
strumenti di condivisione e versionamento erogati come servizi web,
centralizzando gli strumenti organizzativi su server privato.
\subsubsection{Servizi web}
I servizi web utilizzati sono:
\begin{itemize}
\item\textbf{Google Drive:} è usato per lo scambio di file e documenti all'interno del gruppo, principalmente per file che non necessitano di versionamento.\\
Questo spazio del gruppo è suddiviso in diverse cartelle per una migliore gestione e per facilitare la ricerca all'interno di esso.\\
Il \textit{Responsabile di Progetto} ha il potere amministrativo dell'account in questione, quindi sarà quest'ultima persona ad occuparsi di garantire i giusti diritti ai membri del gruppo.
\item\textbf{Github:} è usato per tutti i files che necessitano un sistema di \gloss{versionamento} viene usato il servizio offerto da \textit{Github}, l'indirizzo web del gruppo è:\\
\begin{center}
https://github.com/scalatekids
\end{center}
\item\textbf{Twiddla:} è un sistema che offre una lavagna online condivisa, questo servizio è utilizzato dai membri del gruppo per discutere e lavorare assieme //TODO: non so cosa scrivere su twiddla
\item\textbf{Gantter for Google Drive:} è uno strumento gratuito per la creazione di \gloss{diagrammi di Gantt}, è stato usato per la creazione del Gantt predittivo
\end{itemize}
\subsubsection{Server dedicato}
Sono state installate alcune applicazioni per il coordinamento del gruppo su
server dedicato, gestito dall'\textit{Amministratore} su direttive del
\textit{Responsabile}.\\ Il server è raggiungibile da interfaccia web previo
login, i componenti possono usufruire delle seguenti applicazioni:

\paragraph{Redmine}
Redmine è un'applicazione web atta alla gestione di progetti.\\
Offre diverse funzionalità che il gruppo dovrà utilizzare, quali:
\begin{itemize}
\item\textbf{Sezione wiki:} una pagina web in cui il gruppo andrà a condividere una serie di link di interesse riguardanti ogni tecnologia o aspetto inerente al progetto.
\item\textbf{Sistema di segnalazioni:} un sistema di segnalazioni gestite dall'\texit{Responsabile}, ogni segnalazione rappresenta un task assegnato ad uno o più membri del gruppo, il suo conseguimento servirà per il soddisfacimento di un requisito.
\item\textbf{Gantt:} \textit{Redmine} costruisce automaticamente un \gloss{diagramma di Gantt} con le segnalazioni create dall'\textit{Responsabile}
\end{itemize}
//TODO: potremmo aggiungere dei diagrammi UML2.0 per descrivere come usare redmine...

\paragraph{PHPMyAdmin}
PHPMyAdmin è uno strumento scritto in linguaggio \textit{PHP} che offre una facile amministrazione di \gloss{\textit{database}}.\\
Questo strumento verrà usato per la creazione di un back-end per la gestione dei requisiti.	//TODO: scrivere qualcosa di più a riguardo?
//TODO: da integrare se aggiungiamo il front-end

\subsubsection{Versionamento}
Lo strumento scelto è \gloss{git} principalmente per la grande diffusione negli
ambienti di sviluppo e per l'integrazione offerta dal servizio web
\textit{Github}, il quale è inoltre offre un esaustiva documentazione
sull'utilizzo del programma in questione. Infine \gloss{git} si è rivelato essere
il più conosciuto tra i membri del gruppo, l'alternativa \gloss{SVN}, nonostante
i simili principi di funzionamento, costituiva una scelta più difficoltosa, in
quanto nessun membro ha mai avuto esperienze di utilizzo.
Sono stati creati due \gloss{repository} all'indirizzo \url{https://github.com/actorbase}:
\begin{itemize}
\item Actorbase.\gloss{git}: Contiene i sorgenti del software.
\item Actorbase-Documents.\gloss{git}: Contiene i sorgenti \LaTeX\xspace e i PDF generati.
\end{itemize}

\subsubsection{Ambiente di lavoro individuale}
Per il lavoro individuale, è stato pensato ad un modo di uniformare quanto più
possibile l'ambiente di sviluppo tra i componenti; il \textit{Responsabile} e
l'\textit{Amministratore} hanno stabilito l'\gloss{IDE} di base ed è stata creato
un ambiente virtuale comune utilizzando il software \textbf{Virtualbox}. Tutti i
componenti del gruppo utilizzeranno gli strumenti all'interno della
\gloss{virtual machine} in modo da garantire omogeneità, e riproducibilità di
eventuali bug.\\ L'\textit{IDE} scelto è \textbf{IntelliJIdea}, considerato il
più appropriato e compatibile con le tecnologie centrali del progetto, gli
editor \textbf{vim}, \textbf{emacs} o \textbf{Sublime Text 2} potranno essere
utilizzati dai componenti del gruppo, purchè si tratti di modifiche rapide o
stesura di script di utilità generale.\\
Per la stesura dei documenti, l'\gloss{IDE} di riferimento è \textbf{TeXstudio}.


\\TODO: manca il software che utilizziamo per fare gli uml, e dirgli che versione di uml usiamo come standard
