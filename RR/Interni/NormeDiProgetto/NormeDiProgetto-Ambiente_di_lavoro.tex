\section{Ambiente di lavoro}
\subsection{Apparato di collaborazione}
Per coordinare il lavoro tra i componenti del gruppo, trovandosi spesso a dover
operare in remoto, l'azienda ha scelto di utilizzare quanto più possibile gli
strumenti di condivisione e versionamento erogati come servizi web,
centralizzando gli strumenti organizzativi su server privato.
\subsubsection{Servizi web}
I servizi web utilizzati sono:
\begin{itemize}
\item\textbf{Google Drive}
\item\textbf{Github}
\item\textbf{Twiddla}
\item\textbf{Gantter for Google Drive}
\end{itemize}
\subsubsection{Server dedicato}
Sono state installate alcune applicazioni per il coordinamento del gruppo su
server dedicato, gestito dall'\textit{Amministratore} su direttive del
\textit{Responsabile}.\\ Il server è raggiungibile da interfaccia web previo
login, i componenti possono usufruire delle seguenti applicazioni:
\begin{itemize}
\item\textbf{Redmine}
\item\textbf{PHPMyAdmin}
\end{itemize}
\subsubsection{Versionamento}
Lo strumento scelto è \summ{git} principalmente per la grande diffusione negli
ambienti di sviluppo e per l'integrazione offerta dal servizio web
\textit{Github}, il quale è inoltre offre un esaustiva documentazione
sull'utilizzo del programma in questione. Infine \summ{git} si è rivelato essere
il più conosciuto tra i membri del gruppo, l'alternativa \summ{SVN}, nonostante
i simili principi di funzionamento, costituiva una scelta più difficoltosa, in
quanto nessun membro ha mai avuto esperienze di utilizzo.
Sono stati creati due \summ{repository} all'indirizzo \url{https://github.com/actorbase}:
\begin{itemize}
\item Actorbase.\summ{git}: Contiene i sorgenti del software.
\item Actorbase-Documents.\summ{git}: Contiene i sorgenti \LaTeX\xspace e i PDF generati.
\end{itemize}
\subsubsection{Ambiente di lavoro individuale}
Per il lavoro individuale, è stato pensato ad un modo di uniformare quanto più
possibile l'ambiente di sviluppo tra i componenti; il \textit{Responsabile} e
l'\textit{Amministratore} hanno stabilito l'\summ{IDE} di base ed è stata creato
un ambiente virtuale comune utilizzando il software \textbf{Virtualbox}. Tutti i
componenti del gruppo utilizzeranno gli strumenti all'interno della
\summ{virtual machine} in modo da garantire omogeneità, e riproducibilità di
eventuali bug.\\ L'\textit{IDE} scelto è \textbf{IntelliJIdea}, considerato il
più appropriato e compatibile con le tecnologie centrali del progetto, gli
editor \textbf{vim}, \textbf{emacs} o \textbf{Sublime Text 2} potranno essere
utilizzati dai componenti del gruppo, purchè si tratti di modifiche rapide o
stesura di script di utilità generale.\\
Per la stesura dei documenti, l'\summ{IDE} di riferimento è \textbf{TeXstudio}.


\\TODO: manca il software che utilizziamo per fare gli uml, e dirgli che versione di uml usiamo come standard