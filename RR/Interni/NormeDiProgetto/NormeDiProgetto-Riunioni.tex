\section{Riunioni}

\subsection{Riunioni Interne}
Tutti i componenti del gruppo possono avanzare una richiesta per una riunione interna seguendo le regole per la richiesta (vedi \ref{Regole per la richiesta}).\\
Sarà il \textit{Responsabile di Progetto} a decidere se indirre o meno questa riunione, controllando la disponibilità dei membri ( //TODO: come la controlliamo?) e in caso avvertendo tutti attraverso una mail ufficiale con oggetto \textit{Riunione Interna <giorno>}.\\
E' richiesta la partecipazione di almeno 4 membri del gruppo.

In casi particolari, come per riunioni che trattano specifici ambiti non di interesse di tutti i membri o per l'avvicinarsi a importanti date, si può indirre una riunione di gruppo con meno compomenti presenti, sempre tramite l'approvazione del \textit{Responsabile} del gruppo.

\subsection{Riunioni esterne}
Anche le riunioni esterne (incontri col committente) possono essere proposte da qualunque membro del gruppo, spetta sempre al \textit{Responsabile di Progetto} la decisione finale.\\
E' necessaria la presenza di almeno il 50\% + 1 dei componenti.

\subsection{Regole per la richiesta}
Le richieste vanno effettuate tramite mail al responsabile e devono avere come oggetto:
\begin{center}
Richiesta riunione <tipo riunione>
\end{center}
tipo riunione può essere interna o esterna.
Il corpo della mail dovrà avere una struttura del tipo:
\begin{center}
Motivazione: <motivo riunione>
Data proposta: <data>
Luogo proposto: <luogo>
\end{center}

\subsection{Esiti Riunioni}
Ad ogni riunione verrà dato il compito a uno dei presenti di verbalizzare un riassunto sugli argomenti trattati e i chiarimenti esermi durante tale riunione.\\
Questa persona avrà poi l'obbligo di condividere questo verbale con tutti i componenti sul \textit{Drive} del gruppo e di notificare l'avvenuto caricamento tramite \textit{Mailing List}
