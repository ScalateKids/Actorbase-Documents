\documentclass{scalatekids-article}
\usepackage[official]{eurosym}
\begin{document}
\lfoot{Piano Di Progetto 1.0.0}
\newgeometry{top=3.5cm}
\begin{titlepage}
  \begin{center}
    \begin{center}
      \includegraphics[width=10cm]{sklogo.png}
    \end{center}
    \vspace{1cm}
    \begin{Huge}
      \begin{center}
        \textbf{Piano Di Progetto}
      \end{center}
    \end{Huge}
    \vspace{11pt}
    \bgroup
    \def\arraystretch{1.3}
    \begin{tabular}{r|l}
      \multicolumn{2}{c}{\textbf{Informazioni sul documento}} \\
      \hline
      \setbox0=\hbox{0.0.1\unskip}\ifdim\wd0=0pt
      \\
      \else
      \textbf{Versione} & 1.0.0\\
      \fi
      \textbf{Redazione} & \multiLineCell[t]{Alberto De Agostini\\Andrea Giacomo Baldan}\\
      \textbf{Verifica} & \multiLineCell[t]{Davide Trevisan\\Francesco Agostini}\\
      \textbf{Approvazione} & \multiLineCell[t]{Alberto De Agostini}\\
      \textbf{Uso} & Esterno\\
      \textbf{Lista di Distribuzione} & \multiLineCell[t]{ScalateKids\\Prof. Tullio Vardanega\\Prof. Riccardo Cardin}\\
    \end{tabular}
    \egroup
    \vspace{22pt}
  \end{center}
\end{titlepage}
\restoregeometry
\clearpage
\pagenumbering{arabic}
\setcounter{page}{1}
\begin{flushleft}
  \vspace{0cm}
         {\large\bfseries Diario delle modifiche}
\end{flushleft}
\vspace{0cm}
\begin{center}
  \begin{tabular}{| l | l | l | l | p{5cm} |}
    \hline
    Versione & Autore & Ruolo & Data & Descrizione \\
    \hline
    1.0.0 & Andrea Giacomo Baldan & Responsabile & 2016-01-21 & Approvazione documento\\
    \hline
    0.5.0 & Francesco Agostini & Verificatore & 2016-01-21 & Verifica capitolo Consuntivo\\
    \hline
    0.4.1 & Andrea Giacomo Baldan & Responsabile & 2016-01-21 & Stesura capitolo Consuntivo\\
    \hline
    0.4.0 & Francesco Agostini & Verificatore & 2016-01-10 & Verifica capitolo Prospetto economico\\
    \hline
    0.3.1 & Andrea Giacomo Baldan & Responsabile & 2016-01-09 & Stesura capitolo Prospetto economico\\
    \hline
    0.3.0 & Davide Trevisan & Verificatore & 2016-01-07 & Verifica capitoli Scelta ciclo di vita\\
    \hline
    0.2.1 & Alberto De Agostini & Responsabile & 2016-01-05 & Stesura capitolo Pianificazione Preventiva\\
    \hline
    0.2.0 & Davide Trevisan & Verificatore & 2016-01-05 & Verifica capitolo Analisi dei Rischi\\
    \hline
    0.1.2 & Andrea Giacomo Baldan & Responsabile & 2016-01-05 & Stesura capitolo Analisi dei Rischi\\
    \hline
    0.1.1 & Alberto De Agostini & Responsabile & 2016-01-03 & Stesura capitolo Scelta ciclo di vita\\
    \hline
    0.1.0 & Francesco Agostini & Verificatore & 2016-01-03 & Verifica capitolo Organigramma\\
    \hline
    0.0.2 & Andrea Giacomo Baldan & Responsabile & 2015-12-29 & Stesura capitolo Organigramma\\
    \hline
    0.0.1 & Andrea Giacomo Baldan & Amministratore & 2015-12-16 & Creazione scheletro del documento\\
    \hline
  \end{tabular}
\end{center}
\tableofcontents
\newpage
\pagenumbering{arabic}
\section{Sommario}
\subsection{Scopo del documento}
Il seguente documento ha lo scopo di specificare il piano con cui il gruppo \textit{ScalateKids} lavorerà sul progetto \textbf{Actorbase}.
Nel documento verranno specificati:
\begin{itemize}
\item {L'\gloss{organigramma} del gruppo;}
\item {L'analisi preventiva dell'utilizzo delle risorse;}
\item {L'utilizzo delle risorse durante lo svolgersi del progetto;}
\item {L'analisi dei fattori di rischio.}
\end{itemize}
\prodPurpose
\glossExpl
\subsection{Riferimenti}
\subsubsection{Normativi}
\begin{itemize}
\item \textbf{Norme di Progetto:} \href{run:../Interni/NormeDiProgetto\_v1.0.0.pdf}{Norme Di Progetto v1.0.0.}
\item \textbf{Capitolato d'appalto:} Actorbase \url{http://www.math.unipd.it/~tullio/IS-1/2015/Progetto/C2.pdf}.
\end{itemize}
\subsubsection{Informativi}
\begin{itemize}
\item \textbf{Diagrammi di Gantt:} \url{https://it.wikipedia.org/wiki/Diagramma_di_Gantt}.
\end{itemize}
\section{Scelta del modello di ciclo di vita}
Per lo sviluppo del progetto \textbf{Actorbase} il gruppo ha scelto di applicare
ai processi il modello incrementale. Questa scelta è dovuta alle seguenti
proprietà offerte da questo modello:
\begin{itemize}
\item {La scomposizione del sistema in sottosistemi di dimensione minore;}
\item {I cicli di incrementi pianificati;}
\item {Il rilascio di più \gloss{prototipi} durante lo sviluppo.}
\end{itemize}
La scomposizione in sottosistemi comporta la possibilità di concentrare le
risorse su un numero limitato di attività semplificando la gestione delle
risorse e del tempo. Inoltre è possibile scegliere le attività su cui
concentrarsi in modo da soddisfare per primi i requisiti più critici. In questo
modo i componenti che soddisfano i requisiti principali vengono testati più
volte e raffinati maggiormente rispetto ai requisiti opzionali. Questa
suddivisione aiuta anche la fase di verifica e test, in quanto rende possibile
il test su componenti piccole, rendendo i test più precisi. Le attività di
analisi e progettazione dell'architettura ad alto livello vengono svolti una
volta sola, dopo aver fissato i requisiti principali è possibile definire
l'architettura del sistema. Questo modello consente di avere un prototipo con
delle funzionalità di primaria importanza durante la fase di sviluppo, così da
poter avere un confronto con il Committente in corso d'opera. A partire da
questa base ogni incremento aggiunge nuove funzionalità al prodotto garantendo
quindi una convergenza entro tempi e costi preventivati.
\begin{figure}[H]
  \begin{center}
    \includegraphics[width=1.0\textwidth, keepaspectratio]{CicloIncrementale.png}
    \caption{Ciclo di vita incrementale}
  \end{center}
\end{figure}
\section{Scadenze}
Le attività di pianificazione del progetto saranno basate sulle scadenze qui riportate:
\begin{itemize}
\item {Revisione dei Requisiti (RR): 2016-02-16;}
\item {Revisione di Progetto (RP): 2016-04-18;}
\item {Revisione di Qualifica (RQ): 2016-05-23;}
\item {Revisione di Accettazione (RA): 2016-06-17.}
\end{itemize}
\section{Analisi dei rischi}
L'analisi dei rischi ha costituito una fase critica della pianificazione, in
quanto l'azienda, alla sua prima esperienza, ha dovuto pensare agli ipotetici
scenari negativi che possono formarsi col procedere della fase di sviluppo.
\subsection{Identificazione del rischio}
Per l'identificazione dei rischi il primo passo è stato un'attività di
\gloss{Brainstorming} suddiviso per categorie di pertinenza, in modo da esplorare
in maniera quanto più generale e ampia possibile i vari aspetti di ogni
possibile scenario, seguito da una fase di analisi di ogni rischio emerso e
l'impatto che questo avrebbe portato al raggiungimento degli obiettivi preposti
per la conclusione del progetto. Le categorie prese in considerazione sono:
\begin{itemize}
\item\textbf{Logistica}
\item\textbf{Competenza tecnica}
\item\textbf{Infrastruttura}
\item\textbf{Organizzativi}
\end{itemize}
\subsubsection{Parametri di quantificazione dei rischi}
Per ogni possibile rischio previsto l'analisi ha fornito i seguenti parametri:
\begin{itemize}
\item\textbf{Categoria}: Indica la categoria di appartenenza in cui ricade il
  rischio preso in considerazione;
\item\textbf{Probabilità}: Probabilità statistica che lo scenario indesiderato
  si presenti, può assumere i seguenti valori:
  \begin{itemize}
  \item\textbf{Basso}
  \item\textbf{Medio}
  \item\textbf{Alto}
  \item\textbf{Molto alto}
  \end{itemize}
\item\textbf{Impatto}: Grado di pericolosità dello scenario indesiderato, può
  assumere i seguenti valori:
  \begin{itemize}
  \item\textbf{Debole}
  \item\textbf{Medio}
  \item\textbf{Forte}
  \item\textbf{Molto Forte}
  \end{itemize}
\item\textbf{Descrizione}: Una descrizione del caso preso in considerazione;
\item\textbf{Contromisure di mitigazione}: Provvedimenti da attuare in
  previsione del rischio, e/o di mitigazione in caso di bisogno.
\end{itemize}
\begin{table}[H]
  \centering
  \scriptsize
  \caption{Tabella Analisi dei Rischi}
  \begin{tabular}{|m{1cm}|m{3cm}|m{3cm}|m{3cm}|m{3cm}|}
    \hline
    & \multicolumn{4}{|c|}{\textbf{Probabilità}}\\
    \hline
    \bf Impatto & \bf Bassa & \bf Media & \bf Alta & \textbf{Molto alta} \\
    \hline
    \bf Molto Forte & \cellcolor{red!50} & \cellcolor{red!50} & \cellcolor{red!50} &\cellcolor{red!50} \\
    \hline
    \bf Forte & \cellcolor{yellow!50}Guasti hardware & \cellcolor{yellow!50}Stime attività & \cellcolor{red!50}Analisi requisiti &\cellcolor{red!50}\\[8pt]
    \hline
    \bf Medio & \cellcolor{green!50} & \cellcolor{yellow!50} &\cellcolor{yellow!50}Tecnologia di impiego &\cellcolor{red!50}Inesperienza \\[8pt]
    \hline
    \bf Debole & \cellcolor{green!50}Dissensi tra componenti & \cellcolor{green!50} &\cellcolor{yellow!50}Ambiente di lavoro non omogeneo &\cellcolor{yellow!50}Impegni personali \\
    \hline
  \end{tabular} \\
\end{table}
\begin{table}[H]
  \centering
  \caption{Legenda colorazione rischi}
  \begin{tabular}{|c|c|l|}
    \hline \bf Colore & \bf Probabilità & \bf Legenda \\
    \hline \cellcolor{red! 50} & p > 75\% & Rischio non accettabile - riduzione obbligatoria \\
    \hline \cellcolor{yellow! 50} & 25\% < p < 75\% & Rischio accettabile. Considerare una riduzione. \\
    \hline \cellcolor{green! 50} & p < 25\% & Accettabile. \\
    \hline
  \end{tabular}
\end{table}
\paragraph{Impegni personali}
\textbf{Categoria}: Logistica\\
\textbf{Probabilità}: Alta\\
\textbf{Impatto}: Debole\\
\textbf{Descrizione}: Il gruppo può avere problemi a riunirsi fisicamente con la maggior parte dei membri presenti.
\textbf{Contromisure di mitigazione}: Il \textit{Responsabile di Progetto} ha stabilito una frequenza di incontri fissa,
l'azienda ha inoltre creato alcuni canali di comunicazione remota in modo da poter lavorare quanto più possibile in
coordinazione reciproca.
\paragraph{Inesperienza}
\textbf{Categoria}: Competenza tecnica\\
\textbf{Probabilità}: Alta\\
\textbf{Impatto}: Forte\\
\textbf{Descrizione}: Il gruppo può rimanere spiazzato dal metodo di lavoro da seguire, l'inesperienza nell'attuazione
di competenze di pianificazione e analisi può portare a rallentamenti nel perseguimento degli obiettivi.
\textbf{Contromisure di mitigazione}: I componenti del gruppo si impegneranno a studiare e applicarsi il più possibile
nella pratica delle competenze richieste, ed eventualmente, il \textit{Responsabile} potrà riassegnare alcuni ruoli provvisoriamente
basandosi sui punti deboli e forti dei componenti con maggiore difficoltà a portare a termine la propria \gloss{attività}.
\paragraph{Tecnologia di impiego}
\textbf{Categoria}: Competenza tecnica\\
\textbf{Probabilità}: Alta\\
\textbf{Impatto}: Forte\\
\textbf{Descrizione}: Le tecnologie impiegate nello sviluppo del progetto poggiano su principi noti in buona misura a tutti
i componenti del gruppo, tuttavia l'assenza di un grado di specializzazione tangibile può generare lacune anche gravi nell'utilizzo
degli strumenti in questione.
\textbf{Contromisure di mitigazione}: Ciascun componente del gruppo avrà il compito di documentarsi costantemente e applicarsi
nell'uso delle tecnologie del progetto.
\paragraph{Ambiente di lavoro omogeneo}
\textbf{Categoria}: Infrastruttura\\
\textbf{Probabilità}: Alta\\
\textbf{Impatto}: Debole\\
\textbf{Descrizione}: La quantità di strumenti da utilizzare e la loro grande versatilità, oltre a renderli potenti, genera anche
la necessità di un ambiente di lavoro quanto più omogeneo possibile, in modo da semplificare lo sviluppo e avere un comportamento
da parte delle macchine il più comune possibile, in modo da garantire eventualmente anche la riproducibilità di \gloss{bug} che possono
insorgere.
\textbf{Contromisure di mitigazione}: L'azienda (\textit{Responsabile}?) ha deciso di risolvere il problema utilizzando una macchina virtuale
preimpostata per l'utilizzo delle tecnologie inerenti al progetto per l'attività di sviluppo.
\paragraph{Guasti hardware}
\textbf{Categoria}: Infrastruttura\\
\textbf{Probabilità}: Bassa\\
\textbf{Impatto}: Forte\\
\textbf{Descrizione}: Buona parte del lavoro poggia su server privato gestito dall'\textit{Amministratore} su direttive dal \textit{Responsabile},
questo garantisce un buon grado di controllo, ma allo stesso tempo espone a maggiori rischi legati alla natura "personalizzata" dell'ambiente di lavoro.
\textbf{Contromisure di mitigazione}: Verranno eseguiti dei backup automatici schedulati con periodicità fissa.
\paragraph{Analisi dei Requisiti}
\textbf{Categoria}: Competenza tecnica\\
\textbf{Probabilità}: Alta\\
\textbf{Impatto}: Forte\\
\textbf{Descrizione}: La forte inesperienza del gruppo può portare ad una analisi dei requisiti superficiale o addirittura errata. Questo avrebbe un impatto molto negativo sul progetto.\\Una stesura di requisiti superficiale potrebbe inoltre portare ad una accettazione di requisiti opzionali maggiore di ciò che il gruppo riuscirà a produrre.
\paragraph{Stime di attività}
\textbf{Categoria}: Competenza tecnica\\
\textbf{Probabilità}: Media\\
\textbf{Impatto}: Forte\\
\textbf{Descrizione}: Una cattiva stima delle attività per inesperienza dei membri del gruppo potrebbe portare a forti ritardi rispetto al piano stabilito.
\paragraph{Dissensi tra componenti}
\textbf{Categoria}: Organizzativa\\
\textbf{Probabilità}: Debole\\
\textbf{Impatto}: Debole\\
\textbf{Descrizione}: Un dissenso tra componenti potrebbe portare a diversi problemi nella realizzazione del prodotto. Tuttavia questo rischio sembra poter avvenire con scarsa probabilità poiché il gruppo sembra unito e pronto a collaborare secondo regole prestabilite.
\section{Pianificazione preventiva}
\subsection{Suddivisione attività}
\label{sub:fasi}
Per semplificare la pianificazione del prospetto orario, sono state identificate
quattro attività principali che rappresentano l'intero ciclo di sviluppo del prodotto.\\
Esse sono:
\begin{itemize}
\item\textbf{Analisi (AN):} rappresenta la prima attività del progetto, dalla
  pubblicazione dei capitolati d'appalto all'inizio dell' attività di progettazione.\\
  Figure maggiormente coinvolte: \textit{Responsabile, Amministratore, Analista, Verificatore};
\item\textbf{Analisi di Dettaglio (AD) e Progettazione Architetturale (PA):} segue l' attività di Analisi, e
  racchiude il lasso temporale dedicato alla correzione degli errori rilevati in
  sede di Revisione di Requisiti e alla progettazione dell'architettura sulla
  base dei requisiti emersi. Comprende anche lo studio di \gloss{design pattern}
  congrui alla realizzazione del prodotto.\\ Figure maggiormente coinvolte:
  \textit{Responsabile, Amministratore, Progettista, Verificatore};
\item\textbf{Progettazione di Dettaglio (PD) e Codifica(C):} comprende la correzione degli errori rilevati in sede di Revisione di Progettazione e la stesura del \gloss{codice} secondo le direttive emerse durante l'attività di progettazione.\\Figure maggiormente coinvolte:
  \textit{Responsabile, Amministratore, Programmatore, Verificatore};
\item\textbf{Verifica e Validazione (VV):} in questa fase si andranno a correggere eventuali errori emersi in Revisione di Qualifica e si finirà il prodotto in tutti i suoi componenti per poi fare il collaudo finale col \textit{Proponente}.
\end{itemize}
Tutte le fasi descritte sono comprensive di numerose attività di \textbf{Verifica e Validazione}, in quanto necessaria durante tutto l'arco di sviluppo del prodotto.\\Ogni attività prevede l'impiego di alcuni ruoli in misura maggiore rispetto ad altri, nella distribuzione di questi sarà garantita un equa ripartizione del
carico di lavoro individuale.\\
Come riportato nelle \href{run:../Interni/NormeDiProgetto\_v1.0.0.pdf}{Norme Di Progetto v1.0.0} ogni componente del gruppo potrà ricoprire più ruoli contemporaneamente, purché sia garantita l'assenza di conflitti d'interesse tra le attività svolte.\\ \\
\textbf{Diagrammi di Gantt}\\
Sono riportati nei prossimi capitoli i \gloss{diagrammi di Gantt} relativi alle fasi elencate e un \gloss{diagramma di Gantt} preventivo di tutto l'arco temporale previsto per la realizzazione del prodotto.\\
Nei \gloss{diagrammi di Gantt} verranno riportati diversi elementi importanti:
\begin{itemize}
\item \textbf{Attività composta:} riportata con barra nera nei diagrammi indica una marco-attività composta in più attività minori riportate al di sotto di essa;
\item \textbf{Attività non critica:} riportata con barra blu nei diagrammi indica una attività che può essere svolta parallelamente a altre attività. Un ritardo su un'attività di questo tipo non causerebbe ritardi a cascata su altre attività;
\item \textbf{Attività critica:} riportata con una barra arancio nei diagrammi indica una attività che ha un forte impatto temporale nello svolgersi del progetto, pertanto un ritardo in un'attività di questo tipo causerebbe un ritardo a cascata nel resto del progetto impattando negativamente il piano temporale ed economico;
\item \textbf{Milestone:} riportata con un rombo nero nei diagrammi indica la data attesa per la conclusione delle attività. Coincide con la consegna del prodotto e documenti della successiva revisione.
\end{itemize}
\subsubsection{Tabelle di distribuzione dei ruoli}
Per semplificare la rappresentazione della distribuzione ore/ruoli durante il
progetto sono state raggruppate le ore con investimento e le ore effettive
rendicontate.\\ \\
\textbf{Legenda:}
\begin{itemize}
\item\textbf{Res:} \textit{Responsabile di Progetto};
\item\textbf{An:} \textit{Analista};
\item\textbf{Amm:} \textit{Amministratore};
\item\textbf{Pr:} \textit{Progettista};
\item\textbf{Pt:} \textit{Programmatore};
\item\textbf{Ve:} \textit{Verificatore}
\item\textbf{Inv:} Ore con investimento;
\item\textbf{Ren:} Ore con rendicontazione.
\end{itemize}
\subsubsection{Analisi}
Nell'attività di Analisi, i componenti ricopriranno i ruoli di progetto secondo la distribuzione seguente:
\begin{center}
  \scriptsize
  \begin{tabular}{| c | p{0.35cm} p{0.35cm} | p{0.35cm} p{0.35cm} | p{0.35cm} p{0.35cm} | p{0.35cm} p{0.35cm} | p{0.35cm} p{0.35cm} | p{0.35cm} p{0.35cm} | p{0.35cm} p{0.35cm} |}
    \hline
    \textbf{Nome} & \multicolumn{2}{|c|}{\textbf{Res}} & \multicolumn{2}{|c|}{\textbf{An}} & \multicolumn{2}{|c|}{\textbf{Amm}} & \multicolumn{2}{|c|}{\textbf{Pr}} & \multicolumn{2}{|c|}{\textbf{Pt}} & \multicolumn{2}{|c|}{\textbf{Ve}} & \multicolumn{2}{|c|}{\textbf{Tot}}\\
    \hline
    & \textbf{Inv} & \textbf{Ren} & \textbf{Inv} & \textbf{Ren} & \textbf{Inv} & \textbf{Ren} & \textbf{Inv} & \textbf{Ren} & \textbf{Inv} & \textbf{Ren} & \textbf{Inv} & \textbf{Ren} & \textbf{Inv} & \textbf{Ren}\\
    \hline
    Andrea Baldan & 12 & 10 & & & 10 & 8 & & & & & 12 & 9 & 34 & 27\\
    Alberto de Agostini & 12 & 9 & 12 & 10 & & & & & & & 18 & 15 & 42 & 34\\
    Michael Munaro & & & 14 & 12 & & & & & & & 16 & 14 & 30 & 26\\
    Giacomo Vanin & & & 14 & 12 & & & & & & & 16 & 15 & 30 & 27\\
    Marco Boseggia & & & 14 & 12 & & & & & & & 16 & 15 & 30 & 27\\
    Francesco Agostini & & & 5 & 4 & 10 & 8 & & & & & 14 & 12 & 29 & 24\\
    Davide Trevisan & & & 14 & 12 & & & & & & & 16 & 13 & 30 & 25\\
    \hline
  \end{tabular}
\end{center}
I seguenti grafici illustrano le ore investite per Persona suddivise tra ore investite e ore rendicontate, per l'attività di Analisi.
\begin{figure}[H]
  \begin{subfigure}[H]{0.47\textwidth}
    \includegraphics[width=1.2\textwidth,keepaspectratio]{AnalisiConInvestimento.png}
    \caption{Ore con investimento, attività di Analisi}
  \end{subfigure}
  \qquad
  \begin{subfigure}[H]{0.47\textwidth}
    \includegraphics[width=1.2\textwidth,keepaspectratio]{AnalisiConRendicontazione.png}
    \caption{Ore con rendicontazione, attività di Analisi}
  \end{subfigure}
\end{figure}

\newpage
\paragraph{Pianificazione temporale}
\textbf{Diagramma di Gantt}
\begin{figure}[H]
  \includegraphics[width=1.0\textwidth, keepaspectratio]{gantt/1-Revisione-Requisiti-gantt-preventivo.png}
  \caption{Gantt predittivo dell'attività di analisi}
\end{figure}

\newpage
\textbf{Distribuzione oraria}
\scriptsize
\begin{center}
  \begin{tabular}{| c | c | c | c |}
    \hline
    \textbf{Id} & \textbf{Nome} & \textbf{Ruolo} & \textbf{Ore}\\
    \hline
    \textbf{AN1} & \textbf{Norme di Progetto} & &\\
    \hline
    \textit{AN1.1} & \textit{Processi organizzativi} & &\\
    \hline
    AN1.1.1 & Ruoli di progetto & Amministratore 1 & 1\\
    \hline
    AN1.1.2 & Comunicazioni & Amministratore 1 & 2\\
    \hline
    AN1.1.3 & Norme di Codifica & Amministratore 2 & 3\\
    \hline
    \textit{AN1.2} & \textit{Processi di Sviluppo} & &\\
    \hline
    AN1.2.1 & Riunioni & Amministratore 2 & 2\\
    \hline
    AN1.2.2 & Requisiti di Progetto & Amministratore 1 & 3\\
    \hline
    \textit{AN1.3} & \textit{Processi di Supporto} & &\\
    \hline
    AN1.3.1 & Documenti & Amministratore 1 & 4\\
    \hline
    AN1.3.2 & Glossario & Amministratore 1 & 2\\
    \hline
    AN1.3.3 & Ambiente di Lavoro & Amministratore 2 & 3\\
    \hline
    AN1.4 & Attività di Verifica & \multiLineCell[t]{Verificatore 1\\Verificatore 3} & \multiLineCell[t]{8\\8}\\
    \hline
    \textbf{AN2} & \textbf{Studio di Fattibilità} & &\\
    \hline
    AN2.1 & Analisi e valutazione degli aspetti positivi dei capitolati & Analista 1 & 5\\
    \hline
    AN2.2 & Analisi e valutazione degli aspetti negativi dei capitolati & Analista 2 & 5\\
    \hline
    AN2.3 & Analisi e valutazione delle scadenze temporali & Analista 3 & 5\\
    \hline
    AN2.4 & Attività di verifica & \multiLineCell[t]{Verificatore 2\\Verificatore 4} & \multiLineCell[t]{4\\5}\\
    \hline
    \textbf{AN4} & \textbf{Analisi dei Requisiti} & &\\
    \hline
    AN4.1 & Descrizione Generale & \multiLineCell[t]{Analista 3\\Analista 2} & \multiLineCell[t]{3\\3}\\
    \hline
    AN4.2 & Diagrammi Casi d'uso & \multiLineCell[t]{Analista 4\\Analista 1\\Analista 3} & \multiLineCell[t]{5\\6\\5}\\
    \hline
    AN4.3 & Classificazione Requisiti & \multiLineCell[t]{Analista 3\\Analista 2} & \multiLineCell[t]{3\\4}\\
    \hline
    AN4.4 & Tracciamento & \multiLineCell[t]{Analista 1\\Analista 2\\Analista 3} & \multiLineCell[t]{3\\3\\4}\\
    \hline
    AN4.5 & Attività di Verifica & \multiLineCell[t]{Verificatore 1\\Verificatore 2\\Verificatore 3} & \multiLineCell[t]{9\\10\\9}\\
    \hline
    \textbf{AN3} & \textbf{Piano di Progetto} & &\\
    \hline
    AN3.1 & Organigramma & Responsabile & 2\\
    \hline
    AN3.2 & Scelta ciclo di vita & Responsabile & 4\\
    \hline
    AN3.3 & Analisi dei Rischi & Responsabile & 5\\
    \hline
    AN3.4 & Pianificazione Preventiva & Responsabile & 5\\
    \hline
    AN3.5 & Prospetto Economico & Responsabile & 2\\
    \hline
    AN3.6 & Consuntivo & Responsabile & 3\\
    \hline
    AN3.7 & Attività di Verifica & \multiLineCell[t]{Verificatore 4\\Verificatore 5} & \multiLineCell[t]{10\\10}\\
    \hline
    \textbf{AN5} & \textbf{Piano di Qualifica} & &\\
    \hline
    AN5.1 & Visione generale della strategia di verifica & \multiLineCell[t]{Responsabile\\Analista 1\\Verificatore 1} & \multiLineCell[t]{3\\5\\4}\\
    \hline
    AN5.2 & Gestione amministrativa della revisione & \multiLineCell[t]{Analista 1\\Analista 2} & \multiLineCell[t]{2\\3}\\
    \hline
    AN5.3 & Pianificazione dei Test & \multiLineCell[t]{Analista 2\\Analista 3} & \multiLineCell[t]{2\\3}\\
    \hline
    AN5.4 & Resoconto Attività di Verifica & Verificatore 4 & 8\\
    \hline
    AN5.5 & Standard di Qualità & Analista 4 & 5\\
    \hline
    AN5.6 & Attività di Verifica & \multiLineCell[t]{Verificatore 1\\Verificatore 3\\Verificatore 5} & \multiLineCell[t]{4\\5\\9}\\
    \hline
    \textbf{AN6} & \textbf{Glossario} & &\\
    \hline
    AN6.1 & Stesura & &\\
    \hline
    AN6.2 & Verifica & Verificatore 1 & 7
  \end{tabular}
\end{center}
\normalsize

\newpage
\subsubsection{Progettazione}
Durante l'attività di Progettazione, i componenti ricopriranno i ruoli di progetto secondo la distribuzione seguente:
\begin{center}
  \scriptsize
  \begin{tabular}{| c | p{0.35cm}  p{0.35cm} | p{0.35cm}  p{0.35cm} | p{0.35cm}  p{0.35cm} | p{0.35cm}  p{0.35cm} | p{0.35cm}  p{0.35cm} | p{0.35cm}  p{0.35cm} | p{0.35cm}  p{0.35cm} |}
    \hline
    \textbf{Nome} & \multicolumn{2}{|c|}{\textbf{Res}} & \multicolumn{2}{|c|}{\textbf{An}} & \multicolumn{2}{|c|}{\textbf{Amm}} & \multicolumn{2}{|c|}{\textbf{Pr}} & \multicolumn{2}{|c|}{\textbf{Pt}} & \multicolumn{2}{|c|}{\textbf{Ve}} & \multicolumn{2}{|c|}{\textbf{Tot}}\\
    \hline
    & \textbf{Inv} & \textbf{Ren} & \textbf{Inv} & \textbf{Ren} & \textbf{Inv} & \textbf{Ren} & \textbf{Inv} & \textbf{Ren} & \textbf{Inv} & \textbf{Ren} & \textbf{Inv} & \textbf{Ren} & \textbf{Inv} & \textbf{Ren}\\
    \hline
    Andrea Baldan & & & 8 & 6 & & & 17 & 14 & & & 12 & 11 & 37 & 31\\
    Alberto de Agostini & & & 6 & 4 & 5 & 5 & 20 & 18 & & & & & 31 & 27\\
    Michael Munaro & & & 7 & 5 & & & & & & & 18 & 15 & 25 & 20\\
    Giacomo Vanin & 11 & 10 & & & & & 18 & 16 & & & & & 29 & 26\\
    Marco Boseggia & 8 & 7 & & & 4 & 3 & 6 & 5 & & & 17 & 15 & 35 & 30\\
    Francesco Agostini & & & 8 & 7 & & & 17 & 15 & & & & & 25 & 22\\
    Davide Trevisan & & & & & & & 16 & 13 & & & 9 & 8 & 25 & 21\\
    \hline
  \end{tabular}
\end{center}
I seguenti grafici illustrano le ore investite per Persona suddivise tra ore
investite e ore rendicontate, per l'attività di Progettazione, comprensive delle
ore riservate all'attività \textbf{Analisi di dettaglio}.
\begin{figure}[H]
  \begin{subfigure}[H]{0.47\textwidth}
    \includegraphics[width=1.2\textwidth,keepaspectratio]{ProgettazioneConInvestimento.png}
    \caption{Ore con investimento, attività di Progettazione}
  \end{subfigure}
  \qquad
  \begin{subfigure}[H]{0.47\textwidth}
    \includegraphics[width=1.2\textwidth,keepaspectratio]{ProgettazioneConRendicontazione.png}
    \caption{Ore con rendicontazione, attività di Progettazione}
  \end{subfigure}
\end{figure}

\newpage
\paragraph{Pianificazione temporale}
\textbf{Diagramma di Gantt}
\begin{figure}[H]
  \includegraphics[width=1.0\textwidth, keepaspectratio]{gantt/2-Revisione-Progettazione-gantt-preventivo.png}
  \caption{Gantt predittivo dell'attività di progettazione}
\end{figure}

\newpage
\textbf{Distribuzione oraria}
\scriptsize
\begin{center}
  \begin{tabular}{| c | c | c | c |}
    \hline
    \textbf{Id} & \textbf{Nome} & \textbf{Ruolo} & \textbf{Ore}\\
    \hline
    \textbf{ADPA1} & \textbf{Glossario} & &\\
    \hline
    ADPA1.1 & Incremento &  &\\
    \hline
    ADPA1.2 & Verifica & Verificatore 1 & 4\\
    \hline
    \textbf{AD1} & \textbf{Norme di Progetto} & &\\
    \hline
    AD1.1 & Incremento & \multiLineCell[t]{Amministratore 1\\Amministratore 2} & \multiLineCell[t]{2\\2}\\
    \hline
    AD1.2 & Verifica & Verificatore 2 & 4\\
    \hline
    \textbf{AD2} & \textbf{Piano di Progetto} & &\\
    \hline
    AD2.1 & Correzioni & Responsabile & 5\\
    \hline
    AD2.2 & Consuntivo & Responsabile & 4\\
    \hline
    AD2.3 & Verifica & Verificatore 3 & 6\\
    \hline
    \textbf{AD3} & \textbf{Analisi dei Requisiti} & &\\
    \hline
    AD3.1 & Incremento & \multiLineCell[t]{Analista 1\\Analista 2} & \multiLineCell[t]{4\\4}\\
    \hline
    AD3.2 & Verifica & Verificatore 1 & 4\\
    \hline
    \textbf{PA1} & \textbf{Norme di Progetto} & &\\
    \hline
    PA1.1 & Incremento & \multiLineCell[t]{Amministratore 1\\Amministratore 2} & \multiLineCell[t]{2\\2}\\
    \hline
    PA1.2 & Verifica & Verificatore 2 & 3\\
    \hline
    \textbf{PA2} & \textbf{Analisi dei Requisiti} & &\\
    \hline
    PA2.1 & Incremento & Analista 3 & 6\\
    \hline
    PA2.2 & Verifica & Verificatore 3 & 4\\
    \hline
    \textbf{PA3} & \textbf{Piano di Progetto} & &\\
    PA3.1 & Incremento & Responsabile & 4\\
    \hline
    PA3.2 & Consuntivo & Responsabile & 4\\
    \hline
    PA3.3 & Verifica & Verificatore 1 & 4\\
    \hline
    \textbf{PA4} & \textbf{Piano di Qualifica} & &\\
    \hline
    PA4.1 & Incremento & \multiLineCell[t]{Responsabile\\Progettista 1\\Analista 3} & \multiLineCell[t]{2\\2\\3}\\
    \hline
    PA4.2 & Verifica & \multiLineCell[t]{Verificatore 1\\Verificatore 2} & \multiLineCell[t]{4\\4}\\
    \hline
    \textbf{PA5} & \textbf{Specifica Tecnica} & &\\
    \hline
    PA5.1 & Strumenti e tecnologie usati & \multiLineCell[t]{Amministratore 1\\Progettista 1} & \multiLineCell[t]{1\\6}\\
    \hline
    PA5.2 & Architettura generale & \multiLineCell[t]{Progettista 2\\Progettista 3\\Progettista 4} & \multiLineCell[t]{7\\7\\7}\\
    \hline
    PA5.3 & Design pattern & \multiLineCell[t]{Progettista 1\\Progettista 2\\Progettista 3} & \multiLineCell[t]{8\\6\\7}\\
    \hline
    PA5.4 & Descrizione componenti & \multiLineCell[t]{Progettista 1\\Progettista 2\\Progettista 3\\Progettista 4} & \multiLineCell[t]{8\\8\\8\\8}\\
    \hline
    \textit{PA5.5} & \textit{Tracciamento} &  &\\
    \hline
    PA5.5.1 & Tracciamento componenti-requisiti & \multiLineCell[t]{Analista 1\\Progettista 4} & \multiLineCell[t]{10\\10}\\
    \hline
    PA5.5.2 & Tracciamento requisiti-componenti & \multiLineCell[t]{Analista 2\\Progettista 3} & \multiLineCell[t]{2\\2}\\
    \hline
    PA5.6 & Attività di verifica & \multiLineCell[t]{Verificatore 3\\Verificatore 1} & \multiLineCell[t]{10\\9}\\
    \hline
  \end{tabular}
\end{center}
\normalsize


\subsubsection{Codifica}
Durante l'attività di Codifica, i componenti ricopriranno i ruoli di progetto secondo la distribuzione seguente:
\begin{center}
  \scriptsize
  \begin{tabular}{| c | p{0.35cm}  p{0.35cm} | p{0.35cm}  p{0.35cm} | p{0.35cm}  p{0.35cm} | p{0.35cm}  p{0.35cm} | p{0.35cm}  p{0.35cm} | p{0.35cm}  p{0.35cm} | p{0.35cm}  p{0.35cm} |}
    \hline
    \textbf{Nome} & \multicolumn{2}{|c|}{\textbf{Res}} & \multicolumn{2}{|c|}{\textbf{An}} & \multicolumn{2}{|c|}{\textbf{Amm}} & \multicolumn{2}{|c|}{\textbf{Pr}} & \multicolumn{2}{|c|}{\textbf{Pt}} & \multicolumn{2}{|c|}{\textbf{Ve}} & \multicolumn{2}{|c|}{\textbf{Tot}}\\
    \hline
    & \textbf{Inv} & \textbf{Ren} & \textbf{Inv} & \textbf{Ren} & \textbf{Inv} & \textbf{Ren} & \textbf{Inv} & \textbf{Ren} & \textbf{Inv} & \textbf{Ren} & \textbf{Inv} & \textbf{Ren} & \textbf{Inv} & \textbf{Ren}\\
    \hline
    Andrea Baldan & & & 7 & 6 & & & 17 & 15 & 14 & 12 & & & 38 & 33\\
    Alberto de Agostini & & & & & & & 14 & 11 & 14 & 12 & 16 & 12 & 44 & 35\\
    Michael Munaro & 10 & 9 & & & & & 18 & 15 & 14 & 11 & 15 & 12 & 57 & 47\\
    Giacomo Vanin & & & & & 6 & 5 & & & 20 & 18 & 24 & 18 & 50 & 41\\
    Marco Boseggia & & & & & & & 19 & 16 & 17 & 15 & & & 36 & 31\\
    Francesco Agostini & 8 & 7 & & & & & 18 & 16 & 10 & 9 & 20 & 18 & 56 & 50\\
    Davide Trevisan & & & & & 4 & 3 & 17 & 15 & 16 & 14 & 16 & 14 & 53 & 46\\
    \hline
  \end{tabular}
\end{center}
\normalsize I seguenti grafici illustrano le ore investite per Persona suddivise
tra ore investite e ore rendicontate, per l'attività di Codifica, comprensive
delle ore riservate per l'attività \textbf{Progettazione di Dettaglio}.
\begin{figure}[H]
  \begin{subfigure}[H]{0.47\textwidth}
    \includegraphics[width=1.2\textwidth,keepaspectratio]{CodificaConInvestimento.png}
    \caption{Ore con investimento, attività di Codifica}
  \end{subfigure}
  \qquad
  \begin{subfigure}[H]{0.47\textwidth}
    \includegraphics[width=1.2\textwidth,keepaspectratio]{CodificaConRendicontazione.png}
    \caption{Ore con rendicontazione, attività di Codifica}
  \end{subfigure}
\end{figure}

\newpage
\paragraph{Pianificazione temporale}
\textbf{Diagramma di Gantt}
\begin{figure}[H]
  \includegraphics[width=1.0\textwidth, keepaspectratio]{gantt/3-Revisione-Qualifica-gantt-preventivo.png}
  \caption{Gantt predittivo dell'attività di codifica}
\end{figure}

\newpage
\textbf{Distribuzione oraria}
\scriptsize
\begin{center}
  \begin{tabular}{| c | c | c | c |}
    \hline
    \textbf{Id} & \textbf{Nome} & \textbf{Ruolo} & \textbf{Ore}\\
    \hline
    \textbf{PDC1} & \textbf{Glossario} & &\\
    \hline
    PDC1.1 & Incremento &  &\\
    \hline
    PDC1.2 & Verifica & Verificatore 1 & 1\\
    \hline
    \textbf{PD1} & \textbf{Norme di Progetto} & &\\
    \hline
    PD1.1 & Incremento & Amministratore 1 & 3\\
    \hline
    PD1.2 & Verifica & Verificatore 2 & 2\\
    \hline
    \textbf{PD2} & \textbf{Analisi dei Requisiti} & &\\
    \hline
    PD2.1 & Incremento & Analista 1 & 3\\
    \hline
    AD2.3 & Verifica & Verificatore 3 & 2\\
    \hline
    \textbf{PD3} & \textbf{Piano di Progetto} & &\\
    \hline
    \textit{PD3.1} & \textit{Incremento} & &\\
    \hline
    PD3.1.1 & Correzioni & Responsabile & 2\\
    \hline
    PD3.1.2 & Consuntivo & Responsabile & 2\\
    \hline
    PD3.2 & Attività di Verifica & Verificatore 1 & 2\\
    \hline
    \textbf{PD4} & \textbf{Piano di Qualifica} & &\\
    \hline
    PD4.1 & Incremento & \multiLineCell[t]{Verificatore 3\\Progettista 1} & \multiLineCell[t]{3\\5}\\
    \hline
    PD4.2 & Verifica & \multiLineCell[t]{Verificatore 2\\Verificatore 3} & \multiLineCell[t]{2\\2}\\
    \hline
    \textbf{PD5} & \textbf{Resoconto delle attività di verifica} & &\\
    \hline
    PD5.1 & Analisi statica & Verificatore 1 & 7\\
    \hline
    PD5.2 & Analisi dinamica & Verificatore 2 & 7\\
    \hline
    \textbf{PD6} & \textbf{Specifica Tecnica} & &\\
    \hline
    PD6.1 & Incremento & Progettista 2 & 5\\
    \hline
    PD6.2 & Verifica & Verificatore 3 & 3\\
    \hline
    \textbf{C1} & Definizione del Prodotto & &\\
    \hline
    C1.1 & Standard di progetto & \multiLineCell[t]{Progettista 3\\Progettista 4} & \multiLineCell[t]{14\\14}\\
    \hline
    C1.2 & Specifica delle componenti & \multiLineCell[t]{Progettista 1\\Progettista 2} & \multiLineCell[t]{14\\14}\\
    \hline
    C1.3 & Tracciamento requisiti-componenti & \multiLineCell[t]{Progettista 2\\Progettista 2} & \multiLineCell[t]{14\\14}\\
    \hline
    C1.4 & Verifica & \multiLineCell[t]{Verificatore 1\\Verificatore 2\\Verificatore 3} & \multiLineCell[t]{10\\10\\10}\\
    \hline
    \textbf{C2} & \textbf{Codifica} & \multiLineCell[t]{Programmatore 1\\Programmatore 2\\Programmatore 3\\Programmatore 4\\Verificatore 1\\Verificatore 2\\Verificatore 3} & \multiLineCell[t]{26\\27\\26\\26\\10\\10\\10}\\
    \hline
    \textbf{C3} & \textbf{Manuale Utente} & &\\
    \hline
    C3.1 & Stesura & \multiLineCell[t]{Amministratore\\Analista\\Responsabile} & \multiLineCell[t]{7\\4\\7}\\
    \hline
    C3.2 & Verifica & Responsabile & 16\\
    \hline
  \end{tabular}
\end{center}
\normalsize
\newpage

\subsubsection{Verifica e Validazione}
Durante le attività di verifica e di validazione, i componenti ricopriranno i ruoli di progetto secondo la distribuzione seguente:
\begin{center}
  \scriptsize
  \begin{tabular}{| c | p{0.35cm}  p{0.35cm} | p{0.35cm}  p{0.35cm} | p{0.35cm}  p{0.35cm} | p{0.35cm}  p{0.35cm} | p{0.35cm}  p{0.35cm} | p{0.35cm}  p{0.35cm} | p{0.35cm}  p{0.35cm} |}
    \hline
    \textbf{Nome} & \multicolumn{2}{|c|}{\textbf{Res}} & \multicolumn{2}{|c|}{\textbf{An}} & \multicolumn{2}{|c|}{\textbf{Amm}} & \multicolumn{2}{|c|}{\textbf{Pr}} & \multicolumn{2}{|c|}{\textbf{Pt}} & \multicolumn{2}{|c|}{\textbf{Ve}} & \multicolumn{2}{|c|}{\textbf{Tot}}\\

    \hline
    & \textbf{Inv} & \textbf{Ren} & \textbf{Inv} & \textbf{Ren} & \textbf{Inv} & \textbf{Ren} & \textbf{Inv} & \textbf{Ren} & \textbf{Inv} & \textbf{Ren} & \textbf{Inv} & \textbf{Ren} & \textbf{Inv} & \textbf{Ren}\\
    \hline
    Andrea Baldan & & & & & & & 4 & 4 & 8 & 6 & & & 12 & 10\\
    Alberto de Agostini & & & & & & & 4 & 3 & 3 & 3 & & & 7 & 6\\
    Michael Munaro & & & & & 6 & 5 & & & & & 7 & 6 & 13 & 11\\
    Giacomo Vanin & & & 4 & 3 & & & & & 5 & 4 & & & 9 & 7\\
    Marco Boseggia & & & 6 & 4 & & & 4 & 3 & & & 6 & 5 & 16 & 12\\
    Francesco Agostini & & & 6 & 5 & & & 2 & 2 & & & & & 8 & 7\\
    Davide Trevisan & 6 & 6 & & & & & & & 5 & 4 & & & 11 & 10\\
    \hline
  \end{tabular}
\end{center}
\normalsize I seguenti grafici illustrano le ore investite per Persona suddivise
tra ore investite e ore rendicontate, per le attività di Verifica e Validazione.
\begin{figure}[H]
  \begin{subfigure}[H]{0.47\textwidth}
    \includegraphics[width=1.2\textwidth,keepaspectratio]{VerificaConInvestimento.png}
    \caption{Ore con investimento, attività di Verifica e Validazione}
  \end{subfigure}
  \qquad
  \begin{subfigure}[H]{0.47\textwidth}
    \includegraphics[width=1.2\textwidth,keepaspectratio]{VerificaConRendicontazione.png}
    \caption{Ore con rendicontazione, attività di Verifica e Validazione}
  \end{subfigure}
\end{figure}

\newpage
\paragraph{Pianificazione temporale}
\textbf{Diagramma di Gantt}
\begin{figure}[H]
  \includegraphics[width=1.0\textwidth, keepaspectratio]{gantt/4-Revisione-Accettazione-gantt-preventivo.png}
  \caption{Gantt predittivo dell'attività di verifica e validazione}
\end{figure}

\newpage
\textbf{Distribuzione oraria}
\scriptsize
\begin{center}
  \begin{tabular}{| c | c | c | c |}
    \hline
    \textbf{Id} & \textbf{Nome} & \textbf{Ruolo} & \textbf{Ore}\\
    \hline
    \textbf{VV1} & \textbf{Glossario} & &\\
    \hline
    VV1.1 & Incremento & &\\
    \hline
    VV1.2 & Verifica & Verificatore & 1\\
    \hline
    \textbf{PD2} & \textbf{Analisi dei Requisiti} & &\\
    \hline
    PD2.1 & Incremento & Analista 1 & 5\\
    \hline
    AD2.3 & Verifica & Verificatore 1 & 1\\
    \hline
    \textbf{VV3} & \textbf{Norme di Progetto} & &\\
    \hline
    VV3.1 & Incremento & Amministratore & 4\\
    \hline
    PD3.2 & Verifica & Verificatore 2 & 1\\
    \hline
    \textbf{PD4} & \textbf{Piano di Progetto} & &\\
    \hline
    \textit{PD4.1} & \textit{Incremento} & &\\
    \hline
    PD4.1.1 & Correzioni & Responsabile & 3\\
    \hline
    PD4.1.2 & Consuntivo & Responsabile & 2\\
    \hline
    PD4.2 & Attività di Verifica & Verificatore 2 & 1\\
    \hline
    \textbf{VV5} & \textbf{Piano di Qualifica} & &\\
    \hline
    VV5.1 & Incremento & Analista 2 &3\\
    \hline
    VV5.2 & Verifica & Verificatore 1 &1\\
    \hline
    \textbf{VV6} & \textbf{Resoconto delle attività di verifica} & &\\
    \hline
    VV6.1 & Analisi statica & Verificatore 1 & 2\\
    \hline
    VV6.2 & Analisi dinamica & Verificatore 2 & 1\\
    \hline
    \textbf{VV7} & \textbf{Specifica Tecnica} & &\\
    \hline
    VV7.1 & Incremento & Progettista 2 & 5\\
    \hline
    VV7.2 & Verifica & Verificatore 1 & 1\\
    \hline
    \textbf{VV8} & Definizione del Prodotto & &\\
    \hline
    VV8.1 & Incremento & \multiLineCell[t]{Progettista\\Analista 2} 1 & \multiLineCell[t]{3\\3}\\
    \hline
    VV8.2 & Verifica & Verificatore 2 & 1\\
    \hline
    \textbf{VV9} & \textbf{Manuali} & &\\
    \hline
    VV9.1 & Incremento & \multiLineCell[t]{Analista 2\\Progettista 2} &\multiLineCell[t]{5\\6}\\
    \hline
    VV9.2 & Verifica & Verificatore 2 & 1\\
    \hline
    \textbf{VV10} & Collaudo & \multiLineCell[t]{Programmatore 1\\Programmatore 2\\Amministratore\\Verificatore 1\\Responsabile} & \multiLineCell[t]{11\\10\\2\\2\\1}\\
    \hline
  \end{tabular}
\end{center}
\normalsize
\newpage

\subsubsection{Totale}
Per la durata complessiva del progetto, i componenti ricopriranno i ruoli di progetto secondo la distribuzione seguente:
\begin{center}
  \scriptsize
  \begin{tabular}{| c | p{0.35cm}  p{0.35cm} | p{0.35cm}  p{0.35cm} | p{0.35cm}  p{0.35cm} | p{0.35cm}  p{0.35cm} | p{0.35cm}  p{0.35cm} | p{0.35cm}  p{0.35cm} | p{0.35cm}  p{0.35cm} |}
    \hline
    \textbf{Nome} & \multicolumn{2}{|c|}{\textbf{Res}} & \multicolumn{2}{|c|}{\textbf{An}} & \multicolumn{2}{|c|}{\textbf{Amm}} & \multicolumn{2}{|c|}{\textbf{Pr}} & \multicolumn{2}{|c|}{\textbf{Pt}} & \multicolumn{2}{|c|}{\textbf{Ve}} & \multicolumn{2}{|c|}{\textbf{Tot}}\\
    \hline
    & \textbf{Inv} & \textbf{Ren} & \textbf{Inv} & \textbf{Ren} & \textbf{Inv} & \textbf{Ren} & \textbf{Inv} & \textbf{Ren} & \textbf{Inv} & \textbf{Ren} & \textbf{Inv} & \textbf{Ren} & \textbf{Inv} & \textbf{Ren}\\
    \hline
    Andrea Baldan & 12 & 10 & 15 & 12 & 10 & 8 & 38 & 33 & 22 & 18 & 24 & 20 & 121 & 101\\
    Alberto de Agostini & 12 & 9 & 18 & 14 & 5 & 5 & 38 & 32 & 17 & 15 & 34 & 27 & 124 & 102\\
    Michael Munaro & 10 & 9 & 21 & 17 & 6 & 5 & 18 & 15 & 14 & 11 & 56 & 47 & 125 & 104\\
    Giacomo Vanin & 11 & 10 & 18 & 15 & 6 & 5 & 18 & 16 & 25 & 22 & 43 & 34 & 121 & 102\\
    Marco Boseggia & 8 & 7 & 20 & 16 & 4 & 3 & 29 & 24 & 17 & 15 & 39 & 35 & 117 & 100\\
    Francesco Agostini & 8 & 7 & 19 & 16 & 10 & 8 & 37 & 33 & 10 & 9 & 34 & 30 & 118 & 103\\
    Davide Trevisan & 6 & 6 & 14 & 12 & 4 & 3 & 33 & 28 & 21 & 18 & 41 & 35 & 119 & 102\\
    \hline
  \end{tabular}
\end{center}
\normalsize
I seguenti grafici illustrano le ore investite per Persona suddivise tra ore investite e ore rendicontate durante l'intero progetto:
\begin{figure}[H]
  \begin{subfigure}[H]{0.47\textwidth}
    \includegraphics[width=1.2\textwidth,keepaspectratio]{TotaleConInvestimento.png}
    \caption{Ore con investimento totali}
  \end{subfigure}
  \qquad
  \begin{subfigure}[H]{0.47\textwidth}
    \includegraphics[width=1.2\textwidth,keepaspectratio]{TotaleConRendicontazione.png}
    \caption{Ore con rendicontazione totali}
  \end{subfigure}
\end{figure}

\newpage
\paragraph{Pianificazione temporale}
\textbf{Diagramma di Gantt}
\begin{figure}[H]
  \includegraphics[width=1.0\textwidth, keepaspectratio]{gantt/Gantt-Totale.png}
  \caption{Gantt predittivo di tutto l'arco temporale}
\end{figure}

\newpage
\subsection{Prospetto economico}
In questa sottosezione vengono presentati, per ciascuna attività del progetto identificata
nella sottosezione \ref{sub:fasi}, le ore preventivate di impiego per i ruoli coinvolti.
\subsubsection{Analisi}
Nella fase di Analisi, le ore tra i ruoli sono state divise nel seguente modo:
\begin{center}
  \normalsize
  \begin{tabular}{| c | c | c |}
    \hline
    \multicolumn{3}{|c|}{\textbf{Con investimento}}\\
    \hline
    \textbf{Ruolo} & \textbf{Ore} & \textbf{Costo}\\
    \hline
    Responsabile & 24 & 720\\
    Analista & 73 & 1825\\
    Amministratore & 20 & 400\\
    Progettista & 0 & 0\\
    Programmatore & 0 & 0\\
    Verificatore & 111 & 1665 \\
    \hline
    \textbf{Totale} & 228 & 4610\\
    \hline
  \end{tabular}
  \qquad
  \begin{tabular}{| c | c | c |}
    \hline
    \multicolumn{3}{|c|}{\textbf{Senza investimento}}\\
    \hline
    \textbf{Ruolo} & \textbf{Ore} & \textbf{Costo}\\
    \hline
    Responsabile & 19 & 570\\
    Analista & 62 & 1550\\
    Amministratore & 16 & 320\\
    Progettista & 0 & 0\\
    Programmatore & 0 & 0\\
    Verificatore & 94 & 1410 \\
    \hline
    \textbf{Totale} & 191 & 3850\\
    \hline
  \end{tabular}
\end{center}
I grafici a seguire rappresentano l'influenza di ciascun ruolo sul totale delle ore e dei costi durante l'attività di Analisi.
\begin{figure}[H]
  \begin{subfigure}[H]{0.47\textwidth}
    \includegraphics[width=1.2\textwidth,keepaspectratio]{AnalisiCC.png}
    \caption{Costi con investimento, attività di Analisi}
  \end{subfigure}
  \qquad
  \begin{subfigure}[H]{0.47\textwidth}
    \includegraphics[width=1.2\textwidth,keepaspectratio]{AnalisiSC.png}
    \caption{Costi Senza investimento, attività di Analisi}
  \end{subfigure}
\end{figure}
\newpage
\subsubsection{Progettazione}
Durante l'attività di Progettazione, le ore tra i ruoli sono state divise nel seguente modo:
\begin{center}
  \normalsize
  \begin{tabular}{| c | c | c |}
    \hline
    \multicolumn{3}{|c|}{\textbf{Con investimento}}\\
    \hline
    \textbf{Ruolo} & \textbf{Ore} & \textbf{Costo}\\
    \hline
    Responsabile & 19 & 570 \\
    Analista & 29 & 725 \\
    Amministratore & 9 & 180 \\
    Progettista & 94 & 2068\\
    Programmatore & 0 & 0\\
    Verificatore & 56 & 840 \\
    \hline
    \textbf{Totale} & 207 & 4383\\
    \hline
  \end{tabular}
  \qquad
  \begin{tabular}{| c | c | c |}
    \hline
    \multicolumn{3}{|c|}{\textbf{Senza investimento}}\\
    \hline
    \textbf{Ruolo} & \textbf{Ore} & \textbf{Costo}\\
    \hline
    Responsabile & 17 & 510\\
    Analista & 22 & 550\\
    Amministratore & 8 & 160\\
    Progettista & 81 & 1782\\
    Programmatore & 0 & 0\\
    Verificatore & 49 & 735\\
    \hline
    \textbf{Totale} & 177 & 3737 \\
    \hline
  \end{tabular}
\end{center}
I grafici a seguire rappresentano l'influenza di ciascun ruolo sul totale delle ore e dei costi durante l'attività di Progettazione.
\begin{figure}[H]
  \begin{subfigure}[H]{0.47\textwidth}
    \includegraphics[width=1.2\textwidth,keepaspectratio]{ProgettazioneCC.png}
    \caption{Costi con investimento, attività di Progettazione}
  \end{subfigure}
  \qquad
  \begin{subfigure}[H]{0.47\textwidth}
    \includegraphics[width=1.2\textwidth,keepaspectratio]{ProgettazioneSC.png}
    \caption{Costi con investimento, attività di Progettazione}
  \end{subfigure}
\end{figure}
\newpage
\subsubsection{Codifica}
Durante l'attività di Codifica, le ore tra i ruoli sono state divise nel seguente modo:
\begin{center}
  \normalsize
  \begin{tabular}{| c | c | c |}
    \hline
    \multicolumn{3}{|c|}{\textbf{Con investimento}}\\
    \hline
    \textbf{Ruolo} & \textbf{Ore} & \textbf{Costo}\\
    \hline
    Responsabile & 18 & 540 \\
    Analista & 7 & 175\\
    Amministratore & 10 & 200\\
    Progettista & 103 & 2266\\
    Programmatore & 105 & 1575\\
    Verificatore & 91 & 1365\\
    \hline
    \textbf{Totale} & 334 & 6121\\
    \hline
  \end{tabular}
  \qquad
  \begin{tabular}{| c | c | c |}
    \hline
    \multicolumn{3}{|c|}{\textbf{Senza investimento}}\\
    \hline
    \textbf{Ruolo} & \textbf{Ore} & \textbf{Costo}\\
    \hline
    Responsabile & 16 & 480\\
    Analista & 6 & 150\\
    Amministratore & 8 & 160\\
    Progettista & 88 & 1936\\
    Programmatore & 91 & 1365\\
    Verificatore & 74 & 1110\\
    \hline
    \textbf{Totale} & 283 & 5201\\
    \hline
  \end{tabular}
\end{center}
I grafici a seguire rappresentano l'influenza di ciascun ruolo sul totale delle ore e dei costi durante l'attività di Codifica.
\begin{figure}[H]
  \begin{subfigure}[H]{0.47\textwidth}
    \includegraphics[width=1.2\textwidth,keepaspectratio]{CodificaCC.png}
    \caption{Costi con investimento, attività di Codifica}
  \end{subfigure}
  \qquad
  \begin{subfigure}[H]{0.47\textwidth}
    \includegraphics[width=1.2\textwidth,keepaspectratio]{CodificaSC.png}
    \caption{Costi senza investimento, attività di Codifica}
  \end{subfigure}
\end{figure}
\newpage
\subsubsection{Verifica e Validazione}
Durante le attività di Verifica e Validazione, le ore tra i ruoli sono state divise nel seguente modo:
\begin{center}
  \normalsize
  \begin{tabular}{| c | c | c |}
    \hline
    \multicolumn{3}{|c|}{\textbf{Con investimento}}\\
    \hline
    \textbf{Ruolo} & \textbf{Ore} & \textbf{Costo}\\
    \hline
    Responsabile & 6 & 180 \\
    Analista & 16 & 400\\
    Amministratore & 6 & 120\\
    Progettista & 14 & 308\\
    Programmatore & 21 & 315\\
    Verificatore & 13 & 195\\
    \hline
    \textbf{Totale} & 81 & 1518\\
    \hline
  \end{tabular}
  \qquad
  \begin{tabular}{| c | c | c |}
    \hline
    \multicolumn{3}{|c|}{\textbf{Senza investimento}}\\
    \hline
    \textbf{Ruolo} & \textbf{Ore} & \textbf{Costo}\\
    \hline
    Responsabile & 6 & 180\\
    Analista & 12 & 300\\
    Amministratore & 5 & 100\\
    Progettista & 12 & 264\\
    Programmatore & 17 & 255\\
    Verificatore & 11 & 165\\
    \hline
    \textbf{Totale} & 63 & 1264\\
    \hline
  \end{tabular}
\end{center}
I grafici a seguire rappresentano l'influenza di ciascun ruolo sul totale delle ore e dei costi durante le attività di Verifica e Validazione.
\begin{figure}[H]
  \begin{subfigure}[H]{0.47\textwidth}
    \includegraphics[width=1.2\textwidth,keepaspectratio]{VerificaCC.png}
    \caption{Costi con investimento, attività di Verifica e Validazione}
  \end{subfigure}
  \qquad
  \begin{subfigure}[H]{0.47\textwidth}
    \includegraphics[width=1.2\textwidth,keepaspectratio]{VerificaSC.png}
    \caption{Costi senza investimento, attività di Verifica e Validazione}
  \end{subfigure}
\end{figure}
\newpage
\subsection{Totale}
Le ore rendicontate totali, previste per la realizzazione del prodotto, senza
investimento sono:
\begin{center}
  \normalsize
  \begin{tabular}{| c | c | c |}
    \hline
    \multicolumn{3}{|c|}{\textbf{Con investimento}}\\
    \hline
    \textbf{Ruolo} & \textbf{Ore} & \textbf{Costo}\\
    \hline
    Responsabile & 67 & 2010 \\
    Analista & 125 & 3125\\
    Amministratore & 45 & 900\\
    Progettista & 211 & 4642\\
    Programmatore & 126 & 1890\\
    Verificatore & 271 & 4065\\
    \hline
    \textbf{Totale} & 845 & 16632\\
    \hline
  \end{tabular}
  \qquad
  \begin{tabular}{| c | c | c |}
    \hline
    \multicolumn{3}{|c|}{\textbf{Senza investimento}}\\
    \hline
    \textbf{Ruolo} & \textbf{Ore} & \textbf{Costo}\\
    \hline
    Responsabile & 58 & 1740\\
    Analista & 102 & 2550\\
    Amministratore & 37 & 740\\
    Progettista & 181 & 3982\\
    Programmatore & 108 & 1620\\
    Verificatore & 228 & 3420\\
    \hline
    \textbf{Totale} & 714 & 14052 \\
    \hline
  \end{tabular}
\end{center}
I grafici a seguire rappresentano l'influenza di ciascun ruolo sul totale delle ore e dei costi Complessivi.
\begin{figure}[H]
  \begin{subfigure}[H]{0.47\textwidth}
    \includegraphics[width=1.2\textwidth,keepaspectratio]{TotaliCC.png}
    \caption{Costi Totali con investimento}
  \end{subfigure}
  \qquad
  \begin{subfigure}[H]{0.47\textwidth}
    \includegraphics[width=1.2\textwidth,keepaspectratio]{TotaliSC.png}
    \caption{Costi Totali senza investimento}
  \end{subfigure}
\end{figure}
\section{Meccanismi di controllo e rendicontazione}
\subsection{Meccanismi di controllo}
Durante lo sviluppo del nostro ambiente di lavoro, si è reso necessario un sistema in grado di permettere:
\begin{itemize}
\item{Pianificazione e controllo le attività;}
\item{Flessibilità sulla pianificazione delle attività;}
\item{Rendicontazione delle ore spese nelle attività.}
\end{itemize}
\subsubsection{Controllo attività}
Per organizzare il lavoro in maniera ottimale e tenere traccia di ogni singola evoluzione si è deciso di adottare un \gloss{sistema di ticketing}, già descritto nelle norme di progetto, grazie al quale è
possibile avere un riscontro immediato dello svolgersi di tutte le attività mediante il \gloss{diagramma di Gantt} fornito dal sistema. Esso si aggiorna dinamicamente segnalando graficamente eventuali ritardi nelle attività pianificate, fornendo una visione intuitiva d'insieme dello stato di progetto.
Nello specifico sono evidenziate:
\begin{itemize}
\item{Le attività in ritardo che vengono segnalate in rosso se superata la scadenza prestabilita;}
\item{Lo stato di avanzamento di tutte le attività in corso;}
\item{Le attività già concluse e il tempo reale impiegato.}
\end{itemize}
\subsubsection{Calendario Risorse}
Nel calendario risorse ogni componente del gruppo può segnare impegni e tempo disponibile dedicato
al progetto, in modo che tutto il personale possa organizzare il proprio lavoro in base ai propri
impegni e in sintonia con il lavoro degli altri componenti del gruppo.
\subsubsection{Calendario Attività}
Tutte le attività vengono automaticamente inserite nel calendario dal sistema di ticketing, il quale
indica la data di inizio e la data di fine di ogni attività. Tutti i membri del gruppo vi hanno accesso
e possono monitorare e pianificare il proprio lavoro in base ad esse.
\subsection{Meccanismi di Rendicontazione}
Integrato nel sistema di ticketing vi è un meccanismo di conteggio e misurazione del lavoro svolto
e delle ore impiegate divise per attività e ruolo, esso consente in maniera agevole la rendicontazione
in base a:
\begin{itemize}
\item{Ore di lavoro per attività svolta;}
\item{Ore di lavoro per ruolo.}
\end{itemize}
\newpage
\section{Consuntivo a finire}
In questa sezione vengono descritte le spese effettivamente sostenute e suddivise in base ai ruoli,
inoltre nel dettaglio vengono rappresentate per ogni ruolo le ore realmente impiegate. Con questi
dati è possibile stilare un bilancio, che viene calcolato in base alla differenza tra le ore preventivate
e le ore realmente impiegate.
Calcolato il bilancio è possibile trovarsi in tre stati, nello specifico:
\begin{itemize}
\item{\textbf{Bilancio in  Positivo} Nel caso le ore preventivate siano superiori di quelle impiegate. ;}
\item{\textbf{Bilancio in  Negativo} Nel caso le ore impiegate siano superiori a quelle preventivate. ;}
\item{\textbf{Bilancio in Pari} Nel caso le ore preventivate corrispondano a quelle realmente impiegate. ;}
\end{itemize}
\subsection{Analisi}
Viene di seguito viene riportato il consuntivo relativo alla fase di analisi e riportate per ogni ruolo lo
scarto tra le ore in preventivo e le ore impiegate formando un bilancio per ciascuno di essi.
\begin{center}
  \normalsize
  \begin{tabular}{| c | c | c |}
    \hline
    \textbf{Ruolo} & \textbf{Ore} & \textbf{Costo}\\
    \hline
    Responsabile & -1 & -30 \\
    Analista & -2 & -50\\
    Amministratore & -1 & -20\\
    Progettista & 0 & 0\\
    Programmatore & 0 & 0\\
    Verificatore & +1 & +15\\
    \hline
    \textbf{Totale} & -3 & -85\\
    \hline
  \end{tabular}
\end{center}

\subsubsection{Conclusioni}
Le attività pianificate si sono svolte sforando di alcune ore rispetto a quanto pianificato, con un
bilancio in negativo di \textbf{\euro{} 85}.
\section{Organigramma}
\subsection{Accettazione componenti}
\begin{center}
  \begin{tabular}{|l | l | p{4cm} |}
    \hline
    Nome & Data & Firme \\
    \hline
    Alberto De Agostini & 2015-12-18 &\\
    Andrea Giacomo Baldan & 2015-12-18 &\\
    Marco Boseggia & 2015-12-18 &\\
    Giacomo Vanin & 2015-12-18 &\\
    Michael Munaro & 2015-12-18 &\\
    Davide Trevisan & 2015-12-18 &\\
    Francesco Agostini & 2015-12-18 &\\
    \hline
  \end{tabular}
\end{center}
\subsection{Componenti}
\begin{center}
  \begin{tabular}{|l | l | l |}
    \hline
    Nome & Matricola & Email \\
    \hline
    Alberto De Agostini & 579021 & albertodeagostini88@gmail.com\\
    Andrea Giacomo Baldan & 579117 & a.g.baldan@gmail.com\\
    Marco Boseggia & 1044608 & boseggiam91@gmail.com\\
    Giacomo Vanin & 1026988 & giacomo.vanin92@gmail.com\\
    Michael Munaro & 1049522 & munaro.michael@gmail.com\\
    Davide Trevisan & 1070686 & trevisan.davide94@libero.it\\
    Francesco Agostini & 1051519 & francesco.agostini.93@gmail.com\\
    \hline
  \end{tabular}
\end{center}
\end{document}
