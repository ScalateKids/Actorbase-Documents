\documentclass{scalatekids-article}
\begin{document}
\lfoot{Glossario 1.0.0}
\newgeometry{top=3.5cm}
\begin{titlepage}
  \begin{center}
    \begin{center}
      \includegraphics[width=10cm]{sklogo.png}
    \end{center}
    \vspace{1cm}
    \begin{Huge}
      \begin{center}
        \textbf{Glossario}
      \end{center}
    \end{Huge}
    \vspace{11pt}
    \bgroup
    \def\arraystretch{1.3}
    \begin{tabular}{r|l}
      \multicolumn{2}{c}{\textbf{Informazioni sul documento}} \\
      \hline
      \setbox0=\hbox{0.0.1\unskip}\ifdim\wd0=0pt
      \\
      \else
      \textbf{Versione} & 1.0.0\\
      \fi
      \textbf{Redazione} & \multiLineCell[t]{Redattore}\\
      \textbf{Verifica} & \multiLineCell[t]{Verificatore}\\
      \textbf{Approvazione} & \multiLineCell[t]{Approvatore}\\
      \textbf{Uso} & Esterno\\
      \textbf{Lista di Distribuzione} & \multiLineCell[t]{ScalateKids\\Prof. Tullio Vardanega\\Prof. Riccardo Cardin}\\
    \end{tabular}
    \egroup
    \vspace{22pt}
  \end{center}
\end{titlepage}
\restoregeometry
\clearpage
\pagenumbering{arabic}
\setcounter{page}{1}
\begin{flushleft}
  \vspace{0cm}
         {\large\bfseries Diario delle modifiche \par}
\end{flushleft}
\vspace{0cm}
\begin{center}
  \begin{tabular}{|l | l | l | l | l |}
    \hline
    Versione & Autore & Ruolo & Data & Descrizione \\
    \hline
    1.0.0 & Andrea Giacomo Baldan & Responsabile & 2016-01-20 & Approvazione documento\\
    \hline
    0.3.0 & Francesco Agostini & Verificatore & 2016-01-19 & Verifica definizioni\\
    \hline
    0.2.3 & Michael Munaro & Analista & 2016-01-18 & Aggiunte definizioni\\
    \hline
    0.2.2 & Marco Boseggia & Analista & 2016-01-16 & Aggiunte definizioni\\
    \hline
    0.2.1 & Giacomo Vanin & Analista & 2016-01-10 & Aggiunte definizioni\\
    \hline
    0.2.0 & Alberto De Agostini & Verificatore & 2016-01-03 & Verifica definizioni\\
    \hline
    0.1.2 & Davide Trevisan & Analista & 2016-01-02 & Aggiunte definizioni\\
    \hline
    0.1.1 & Michael Munaro & Analista & 2015-12-28 & Aggiunte definizioni\\
    \hline
    0.1.0 & Francesco Agostini & Verificatore & 2015-12-26 & Verifica definizioni\\
    \hline
    0.0.3 & Giacomo Vanin & Analista & 2015-12-23 & Aggiunte definizioni\\
    \hline
    0.0.2 & Michael Munaro & Analista & 2015-12-16 & Prima stesura voci riscontrate\\
    \hline
    0.0.1 & Andrea Giacomo Baldan & Amministratore & 2015-12-15 & Creazione scheletro del documento\\
    \hline
  \end{tabular}
\end{center}
\subsection{Scopo del documento}
Questo documento ha il compito di raccogliere tutti i termini che possono risultare sconosciuti o di difficile comprensione ad un lettore esterno oppure, semplicemente, concetti specifici che necessitano di essere esposti per poter meglio comprendere quanto viene detto nei documenti.
\subsection{Riferimenti} 
\subsubsection{Normativi}
\begin{itemize}
	\item \href{run:NormeDiProgetto\_v1.0.0.pdf}{Norme di Progetto v1.0.0}.
\end{itemize}
\subsubsection{Informativi}
\begin{itemize}
	\item Wikipedia: \url{https://it.wikipedia.org};
	\item Wordreference: \url{http://www.wordreference.com}.
\end{itemize}
\newpage
\section{Definizioni}

\glossaryLetter{A}

\glossDef{Account} insieme di funzionalità, strumenti e contenuti attribuiti ad un nome utente in determinati contesti operativi, spesso in siti web o per usufruire di determinati servizi su Internet.
\\

\glossDef{Actormodel} Modello matematico per la gestione della concorrenza computazionale proposto nel 1973 da Carl Hewitt.
All'interno del modello, gli "attori" vengono trattati come primitive universali e possono scambiarsi messaggi tra di loro; in risposta ad un messaggio ricevuto, un attore può prendere decisioni, creare nuovi attori, inviare altri messaggi e determinare come rispondere al messaggio successivo.
\\

\glossDef{Akka} È uno strumento per la semplificazione e costruzione di applicazioni concorrenti e distribuite sulla JVM (Java Virtual Machine). Akka supporta diversi modelli per la concorrenza prediligendo, però, l'uso del modello ad attori.
\\

\glossDef{Amazon DynamoDB} Database proprietario di tipo NoSQL e completamente gestito da Amazon.com all'interno del suo portfolio 'Amazon Web Services'.
\\

\glossDef{Attività} Aggregazione di \textit{task} (vedi sotto).
\\

\glossDef{Attore}  vedi \textit{Actormodel}
\\

\glossaryLetter{B}

\glossDef{Beacon} il beacon è un piccolo dispositivo BLE (acronimo per Bluetooth Low Energy, cioè Bluetooth a basso consumo energetico) che trasmette piccole quantità di dati ogni secondo o quasi.  A livello di hardware, i beacon sono dispositivi BLE-compatibile che trasmettono dati sotto il protocollo “beacon”.
A livello software, i beacon sono messaggi inviati da questi dispositivi trasmittenti, che sono poi rilevati e processati da un dispositivo ricevente, come un’app mobile. Nel caso degli iBeacon vengono trasmessi i valori Major e Minor che insieme all’UUID (acronimo di Universally Unique IDentifier), permettono l’identificazione di un iBeacon da parte dell’applicazione.(riferirsi a https://it.wikipedia.org/wiki/IBeacon e http://www.beaconitaly.it/cose-ibeacon/);
\\

\glossDef{Brainstorming} Metodo decisionale in cui la ricerca della soluzione di un problema è effettuata mediante sedute intensive di dibattito e confronto delle idee e delle proposte espresse dai partecipanti.
\\

\glossDef{Branch} In computer science, il branch ('salto' in inglese) è un punto in cui in un processo viene alterato il sequenziale flusso delle istruzioni.
\\

\glossDef{Bundled Software} Si tratta di software distribuito con un altro prodotto come parte di un hardware o come parte di un gruppo di pacchetti software venduti assieme.
\\

\glossDef{Bug} in informatica, identifica un errore nella scrittura di un programma software, impedendone il corretto funzionamento e provocandone un comportamento imprevisto o comunque diverso da quello specificato dal produttore. Meno comunemente, il termine bug può indicare un difetto di progettazione in un componente hardware.
L'origine del nome (che in inglese significa "insetto") deriva da un malfunzionamento del computer Mark II nel 1947 dovuto ad una falena incastratasi tra i circuiti.
\\

\glossaryLetter{C}

\glossDef{CamelCase (Notazione a Cammello)} è la pratica nata durante gli anni settanta di scrivere parole composte o frasi unendo tutte le parole tra loro, ma lasciando le loro iniziali maiuscole. La prima lettera può essere sia maiuscola che minuscola.
Esistono diverse varianti della CamelCase tra le quali ricordiamo la \textbf{UpperCamelCase}, ovvero una notazione a cammello in cui la prima lettera è obbligatoriamente maiuscola, e la \textbf{lowerCamelCase} dove la prima lettera è sempre minuscola.
\\

\glossDef{Ciclo di Deming} vedi \textit{PDCA}.
\\

\glossDef{CLI (Command Line Interface)} detta semplicemente \textit{riga di comando}, indica una tipologia di interfaccia utente caratterizzata da un'interazione di tipo testuale tra utente ed elaboratore.
\\

\glossDef{Codice} Testo di un algoritmo di un programma scritto in un linguaggio di programmazione da un programmatore.
\\

\glossDef{Collaboratore} Nell'ambito di Actorbase, con collaboratore si intende un utente (autenticato) che ha visione in collezioni non create da lui stesso
\\

\glossDef{Collezione} Nell'ambito di Actorbase, una collezione è una struttura dati di tipo chiave-valore
\\

\glossDef{Commit} Nel contesto di computer science e gestione di dati, commit si riferisce all'idea di rendere permanenti i cambiamenti effettuati nel repository locale.
\\

\glossDef{Concorrenza} in informatica, il concetto di concorrenza, contrapposto a quello di sequenzialità, indica una caratteristica dei sistemi di elaborazione nei quali può verificarsi che un insieme di processi o sottoprocessi (thread) computazionali siano in esecuzione nello stesso istante.
L'esecuzione contemporanea può condurre a interazione tra processi quando viene coinvolta una risorsa condivisa.
\\

\glossDef{Container docker} I container docker impacchettano un pezzo di software in un filesystem completo, che contiene tutto ciò che è necessario per farlo funzionare: codice, strumenti di sistema, librerie di sistema e tutto ciò che può essere installato su un server.
Questo garantisce che il software funzionerà sempre allo stesso modo, indipendentemente dall'ambiente in cui si troverà.
\\

\glossaryLetter{D}

\glossDef{Database} In informatica, il termine database (in italiano \textit{base di dati}) indica un insieme organizzato di dati.
Le informazioni contenute in un database sono strutturate e collegate tra loro secondo un particolare modello logico scelto dal progettista del database (ad esempio relazionale, gerarchico, key-value, ...).
Gli utenti si interfacciano con i database attraverso i cosiddetti \textit{query language} e grazie a particolari applicazioni software dedicate.
\\

\glossDef{Database Management System (DBMS)} è un sistema software progettato per consentire la creazione, la manipolazione, da parte di un amministratore DBA (Database Admin) e l'interrogazione efficiente (da parte di uno o più utenti client) di database, per questo detto anche "gestore o motore del database", e ospitato su architettura hardware dedicata(server) oppure su semplice computer.
\\

\glossDef{DBMS (DataBase Management System)} è un sistema software progettato per consentire la creazione, la manipolazione (da parte di un amministratore) e l'interrogazione efficiente di un database.
È ospitato su architettura hardware dedicata (server) oppure su semplice computer.
\\

\glossDef{Design Pattern} In ingegneria del software, un design pattern descrive una soluzione generale a un problema di progettazione ricorrente, gli attribuisce un nome, astrae e identifica gli aspetti principali della struttura utilizzata per la soluzione del problema, identifica le classi e le istanze partecipanti e descrive quando e come può essere applicato. In breve definisce un problema, i contesti tipici in cui si trova e la soluzione ottimale allo stato dell'arte.
\\

\glossDef{Dizionario (o Array Associativo)} Struttura dati che contiene una collezione di oggetti (o records).
Questi oggetti contengono diversi campi, ognuno dei quali contiene dei dati.
È possibile accedere, gestire e salvare i diversi records tramite una chiave che li identifica in maniera univoca.
\\

\glossDef{Diagramma di Gantt} È uno strumento di supporto alla gestione dei progetti.
Si tratta, sostanzialmente, di un diagramma costruito a partire da un asse orizzontale, a rappresentazione delll'arco temporale totale (o parziale) del progetto, e da un asse verticale che rappresenta le mansioni o attività che costituiscono il progetto.
Lo scopo principale del diagramma di Gantt è di dislocare temporalmente le attività per poterne rappresentare la durata, la sequenzialità e il parallelismo e poter confrontare le stime con i progressi.
\\

\glossDef{Docker} è un progetto open-source che automatizza il deployment di applicazioni all'interno di container software, fornendo un'astrazione addizionale grazie alla virtualizzazione a livello di sistema operativo di Linux.
\\

\glossDef{Domain Specific Language (DSL)} È un linguaggio di programmazione o un linguaggio di specifica dedicato a particolari problemi di un dominio, a una particolare tecnica di rappresentazione e/o a una particolare soluzione tecnica.
\\

\glossDef{Drive, Google} Google Drive è un servizio, in ambiente cloud computing, di memorizzazione e sincronizzazione online introdotto da Google il 24 aprile 2012. Il servizio comprende il file hosting, il file sharing e la modifica collaborativa di documenti
\\

\glossDef{Driver} componente attiva fittizia per pilotare i test.
\\

\glossaryLetter{E}

\glossaryLetter{F}

\glossDef{File} In informatica, viene utilizzato per riferirsi a un contenitore di informazioni/dati in formato digitale, tipicamente presenti su un supporto digitale di memorizzazione opportunamente formattato in un determinato file system.
Le informazioni scritte/codificate al suo interno sono leggibili solo tramite uno specifico software in grado di effettuare l'operazione.
\\

\glossDef{form} In informatica, un form è un termine usato per indicare l'interfaccia di un'applicazione che consente all'utente client di inserire e inviare al web server uno o più dati liberamente digitati dallo stesso.
\\

\glossDef{Forum (internet)} Il termine "forum", nell'ambito dell'informatica, è utilizzato per indicare l'insieme delle sezioni di discussione in una piattaforma informatica dove gli utenti intrattengono conversazioni postando messaggi.
Un forum può essere di tipo gerarchico e con struttura ad albero in quanto possono esservi al suo interno diversi subforum, ognuno dei quali con un proprio topic (argomento di discussione).
\\

\glossDef{Funzionale, programmazione} In informatica la programmazione funzionale è un paradigma di programmazione in cui il flusso di esecuzione del programma assume la forma di una serie di valutazioni di funzioni matematiche. Il punto di forza principale di questo paradigma è la mancanza di effetti collaterali (side-effect) delle funzioni, il che comporta una più facile verifica della correttezza e della mancanza di bug del programma e la possibilità di una maggiore ottimizzazione dello stesso.
\\

\glossaryLetter{G}

\glossDef{Git} Git è un sistema software di controllo di versione distribuito, creato da Linus Torvalds nel 2005.
Esso permette, dunque, di tenere traccia delle modifiche e delle versioni apportate al codice sorgente di un software, senza la necessità di dover utilizzare un server centrale.
Con questo sistema gli sviluppatori possono collaborare individualmente e parallelamente offline su di un proprio ramo (\textit{branch}) di sviluppo, registrare le proprie modifiche (\textit{commit}) ed in seguito condividerle o unirle (\textit{merge}) a quelle di altri, il tutto, come detto in precedenza, senza bisogno del supporto di un server.
\\

\glossDef{GitHub} Servizio Web di hosting per lo sviluppo di progetti software, che usa il sistema di controllo  versione Git.
\\

\glossaryLetter{H}

\glossaryLetter{I}

\glossDef{IDE (Integrated development environment)} Un \textit{ambiente di sviluppo integrato} è un software che, in fase di programmazione, aiuta i programmatori nello sviluppo del codice sorgente di un programma. Spesso l'IDE aiuta lo sviluppatore segnalando errori di sintassi del codice direttamente in fase di scrittura, oltre a tutta una serie di strumenti e funzionalità di supporto alla fase di sviluppo e debugging.
\\

\glossDef{IEC (International Electrotechnical Commission)} È un'organizzazione internazionale per la definizione di standard in materia di elettricità, elettronica e tecnologie correlate, fondata a Londra nel 1906.
\\

\glossDef{Instant messaging} È una categoria di sistemi di comunicazione in tempo reale in rete, tipicamente Internet o una rete locale, che permette ai suoi utilizzatori lo scambio di brevi messaggi.
\\

\glossDef{Integrazione Continua} Nell'ingegneria del software, l'integrazione continua (continuous integration in inglese, spesso abbreviato in CI) è una pratica che si applica in contesti in cui lo sviluppo del software avviene attraverso un sistema di versioning. Consiste nell'allineamento frequente dagli ambienti di lavoro degli sviluppatori verso l'ambiente condiviso (mainline).
\\

\glossDef{IoT (Internet of Things)} è un neologismo riferito all'estensione di Internet al mondo degli oggetti e dei luoghi concreti.
L' \textit{Internet delle cose} è vista come una possibile evoluzione dell'uso della Rete: gli oggetti si rendono riconoscibili e acquisiscono intelligenza grazie al fatto di poter comunicare dati su se stessi e accedere ad informazioni aggregate da parte di altri.
\\

  \glossDef{ISO (International Organization for Standardization)} È la più importante organizzazione a livello mondiale per la definizione di norme tecniche fondata nel 1947 a Ginevra, in Svizzera.
  L'ISO coopera strettamente con l'\textit{IEC} (vedi sopra), responsabile per la standardizzazione degli equipaggiamenti elettrici.
  Le norme ISO sono numerate e hanno un formato del tipo ISO nnnn:yyyy - titolo, dove nnnn è il numero della norma, yyyy l'anno di pubblicazione e titolo è una breve descrizione della norma.
  \\
  
  \glossDef{Item} con item si intende una entry di una \textit{collezione} (vedi sopra)
  \\

  \glossaryLetter{J}

  \glossDef{JSON (JavaScript Object Notation)} è un formato adatto all'interscambio di dati fra applicazioni client-server.
  \\

  \glossDef{JVM} La Java Virtual Machine è il componente della piattaforma Java che esegue i programmi tradotti in bytecode dopo una prima compilazione.
  \\

  \glossaryLetter{K}

  \glossDef{Key-value Database} È un paradigma di archiviazione di dati progettato per archiviare, recuperare e gestire array associativi (vedi \textit{Dizionario}).
  \\

  \glossaryLetter{L}
  
  \glossDef{Link} speciale rimando contenuto all'interno di un ipertesto.

  \glossDef{lowerCamelCase} vedi \textit{CamelCase}.
  \\

  \glossaryLetter{M}

  \glossDef{Macro} In informatica indica una procedura, ovvero un insieme di comandi o istruzioni, tipicamente ricorrente durante l'esecuzione di un programma.
  \\

  \glossDef{Mailing List} È un servizio/strumento offribile da una rete di computer verso vari utenti e costituito da un sistema organizzato per la partecipazione di più persone ad una discussione asincrona o per la distribuzione di informazioni utili agli interessati/iscritti attraverso l'invio di email ad una lista di indirizzi di posta elettronica di utenti iscritti.
  \\

  \glossDef{Major Release} con major release di un software si intende una release che non è puramente una revisione o una release atta a correggere dei bug, ma che comunque contiene dei cambiamenti sostanziali.
  \\
  
  \glossDef{Merge} mescolare, unire.
  \\

  \glossDef{Modello ad attori} vedi \textit{Actormodel}.
  \\

  \glossDef{Modello relazionale} modello logico di rappresentazione o strutturazione dei dati di un database implementato su sistemi di 'gestione di basi di dati' (DBMS).
  Si basa sulla teoria degli insiemi e sulla logica del primo ordine ed è strutturato intorno al concetto matematico di relazione (detta anche tabella). Per il suo trattamento ci si avvale di strumenti quali il calcolo relazionale e l'algebra relazionale.
  \\

  \glossaryLetter{N}
  
  \glossDef{nodo} In informatica e telecomunicazioni un nodo è un qualsiasi dispositivo del sistema in grado di comunicare con gli altri dispositivi che fanno parte della rete.
  \\

  \glossDef{NoSQL} È un movimento che promuove sistemi software dove la persistenza dei dati è caratterizzata dal fatto di non utilizzare il \textit{modello relazionale} (vedi sopra), solitamente utilizzato dai database tradizionali.
  L'espressione NoSQL fa riferimento al linguaggio SQL, il più comune tra i linguaggi per l'interrogazione dei database relazionali, qui preso a simbolo dell'intero paradigma relazionale. Molto spesso nei linguaggi NoSQL viene sacrificato in favore delle prestazioni il paradigma ACID (atomicità, consistenza, isolamento, durabilità) in quanto, per il teorema CAP, noto anche come teorema di Brewer, è impossibile per un sistema informatico distribuito, fornire simultaneamente tutte e tre le seguenti garanzie:
  Coerenza (tutti i nodi vedono gli stessi dati nello stesso momento)
  Disponibilità (la garanzia che ogni richiesta riceva una risposta su ciò che sia riuscito o fallito)
  Tolleranza di partizione (il sistema continua a funzionare nonostante arbitrarie perdite di messaggi)
  Secondo il teorema, un sistema distribuito è in grado di soddisfare al massimo due di queste garanzie allo stesso tempo, ma non tutte e tre.
  \\

  \glossaryLetter{O}

  \glossDef{OOP (Object Oriented Programming)} Paradigma di programmazione che prevede di raggruppare in una zona circoscritta del codice sorgente (chiamata classe) la dichiarazione delle strutture dati e delle procedure che operano su di esse. Le classi costituiscono dei modelli astratti, che a tempo di esecuzione vengono invocate per istanziare o creare oggetti software relativi alla classe invocata.
  \\

  \glossDef{Organigramma} In un'organizzazione aziendale, con il termine organigramma si intende una formalizzazione di quali sono gli organi e le loro relazioni gerarchiche o funzionali all'interno di un ambiente di lavoro.
  \\

  \glossaryLetter{P}

  \glossDef{Package} Un pacchetto (o pacchetto di software), in informatica è una serie di programmi che si distribuiscono congiuntamente.
  \\

  \glossDef{PDCA} Il ciclo di Deming (chiamato anche ciclo di PDCA - "plan–do–check–act") è un modello studiato per il miglioramento continuo della qualità in un'ottica a lungo raggio. Serve per promuovere una cultura della qualità che è tesa al miglioramento continuo dei processi e all'utilizzo ottimale delle risorse. Questo strumento parte dall'assunto che per il raggiungimento del massimo della qualità sia necessaria la costante interazione tra ricerca, progettazione, test, produzione e vendita. Per migliorare la qualità e soddisfare il cliente, le quattro fasi devono ruotare costantemente, tenendo come criterio principale la qualità.
  \\

  \glossDef{PDF} Il Portable Document Format, comunemente abbreviato in PDF, è un formato di file basato su un linguaggio di descrizione di pagina sviluppato da Adobe Systems nel 1993 per rappresentare documenti in modo indipendente dall'hardware e dal software utilizzati per generarli o per visualizzarli.
  \\
  
  \glossDef{Peer-to-peer} Peer-to-peer (P2P) o rete paritaria, in informatica, è un'espressione che indica un'architettura logica di rete informatica in cui i nodi non sono gerarchizzati unicamente sotto forma di client o server fissi, ma sotto forma di nodi equivalenti o paritari (in inglese peer) che possono cioè fungere sia da cliente che da servente verso gli altri nodi terminali (host) della rete. Essa dunque è un caso particolare dell'architettura logica di rete client-server.

  \glossDef{Plugin} Il plugin in campo informatico è un programma non autonomo che interagisce con un altro programma per ampliarne o estenderne le funzionalità originarie.
  A seconda dei programmi e delle piattaforme software, i plugin possono esser chiamati con sinonimi diversi: add-in, add-on e estensione.
  \\

  \glossDef{Prototipo} Il modello realizzato nell’ultima fase della progettazione e sperimentazione, e destinato a divenire il punto di partenza della produzione in serie
  \\

  \glossDef{Pull} Comando di git per aggiornare il repository locale alla commit più recente.
  \\

  \glossDef{Push} Comando di git per inviare le modifiche sulla copia locale al repository remoto.
  \\

  \glossaryLetter{Q}

  \glossDef{Query} Termine che viene usato in informatica per indicare l'interrogazione di un database da parte di un utente.
  \\

  \glossaryLetter{R}
  
  \glossDef{Range} estensione, ampiezza.
  \\
  
  \glossDef{Reactive} Un paradigma di programmazione orientato al flusso di dati e al loro continuo cambiamento. 
  Questo significa poter esprimere con facilità, nel linguaggio di programmazione utilizzato, sia flussi di dati statici che dinamici.
  Inoltre, il modello di esecuzione sottostante propagherà automaticamente i cambiamenti nei flussi di dati. 
  \\

  \glossDef{Refactoring} In ingegneria del software, il refactoring (o code refactoring) è una tecnica per modificare la struttura interna di porzioni di codice senza modificarne il comportamento esterno applicata per migliorare alcune caratteristiche non funzionali del software.
  I vantaggi che il refactoring persegue riguardano in genere un miglioramento della leggibilità della manutenibilità, della riusabilità e dell'estendibilità del codice e la riduzione della sua complessità, eventualmente attraverso l'introduzione a posteriori di design pattern.
  \\

  \glossDef{Repository (Software Repository)} È uno spazio di archiviazione da dove è possibile recuperare e scaricare dei pacchetti software.
  \\

  \glossaryLetter{S}

  \glossDef{SBT (Scala Build Tool)} È un software open source per il build di progetti in Scala e Java.

  \glossDef{Scala} È un linguaggio di programmazione di tipo general-purpose studiato per integrare le caratteristiche e funzionalità dei linguaggi orientati agli oggetti e dei linguaggi funzionali. La compilazione di codice sorgente Scala produce Java bytecode per l'esecuzione su una JVM.
  \\
  
  \glossDef{Scalabilità Orizzontale (Scale Out)} Il termine scalabilità, nelle telecomunicazioni, nell'ingegneria del software e in altre discipline, si riferisce, in termini generali, alla capacità di un sistema di "crescere" o diminuire di scala in funzione delle necessità e delle disponibilità. Un sistema che gode di questa proprietà viene detto scalabile.
  In particolare la scalabilità orizzontale fa riferimento all’aggiunta di nuove unità messe tra loro in parallelo perchè funzionino come una sola, unica unità.
  \\

  \glossDef{Scaladoc} È un applicativo incluso nel Scala Development Kit, utilizzato per la generazione automatica della documentazione del codice sorgente scritto in linguaggio Scala.
  \\
  
  \glossDef{Scale Out} vedi \textit{Scalabilità Orizzontale}

  \glossDef{Script} Il termine script, in informatica, designa un tipo particolare di programma, scritto in una particolare classe di linguaggi di programmazione, detti linguaggi di scripting.
  La distinzione tra un programma normale ed uno script non è netta ma solitamente gli script si distinguono per la loro complessità relativamente bassa, la mancanza di una propria interfaccia grafica e il fatto di essere utilizzati per svolgere mansioni accessorie e molto specifiche.
  \\
  
  \glossDef{Section} comando di \LaTeX\xspace che indica l'inizio di una sezione.
  \\

  \glossDef{Shell} in informatica, è la parte di un sistema operativo che permette agli utenti di interagire con il sistema stesso, impartendo comandi e richiedendo l'avvio di altri programmi.
  La shell può essere di tipo \textit{testuale} (vedi \textit{terminale}) o \textit{grafico}.
  \\

  \glossDef{Single Responsability Principle} il principio di singola responsabilità (single responsibility principle, abbreviato con SRP) afferma che ogni elemento di un programma (classe, metodo, variabile) deve avere una sola responsabilità, e che tale responsabilità deve essere interamente incapsulata dall'elemento stesso. Tutti i servizi offerti dall'elemento dovrebbero essere strettamente allineati a tale responsabilità.
  \\
  
  \glossDef{Sistema di Ticketing} Un sistema di ticketing è uno strumento web, appoggiato a un database accessibile dagli utenti, che permette una facile introduzione e descrizione della propria necessità attraverso l’apertura di un ticket.
  \\

  \glossDef{Smartphone} Lo smartphone ("telefono intelligente") è un telefono cellulare con capacità di calcolo, di memoria e di connessione dati molto più avanzate rispetto ai normali telefoni cellulari, basato su un sistema operativo per dispositivi mobili.
  \\

  \glossDef{SQL (Structured Query Language)} è un linguaggio standardizzato per database basati sul modello relazionale progettato per:
  \begin{itemize}
  	\item creare e modificare schemi di database (DDL - Data Definition Language);
  	\item inserire, modificare e gestire dati memorizzati (DML - Data Manipulation Language);
  	\item interrogare i dati memorizzati (DQL - Data Query Language);
  	\item creare e gestire strumenti di controllo ed accesso ai dati (DCL - Data Control Language).
  \end{itemize}

  \glossDef{Storekeeper} Nel contesto del capitolato da noi considerato, con storekeeper si intende un particolare tipo di attore che mantiene fisicamente al proprio interno le coppie chiave-valore.
  \\

  \glossDef{SVN} Subversion (noto anche come svn, che è il nome del suo client a riga di comando) è un sistema di controllo versione per software distribuito come free software sotto licenza Apache.
  Gli sviluppatori software utilizzano Subversion per mantenere le versioni correnti e storiche di file quali codice sorgente, pagine web e documentazione.
  \\

  \glossaryLetter{T}

  \glossDef{Task} Compito assegnabile ad un singolo componente del gruppo di lavoro che dovrebbe essere ultimato entro un periodo di tempo prefissato o entro una deadline.
  \\

  \glossDef{Template} Trattasi di un modello che ha il compito di fornire la struttura di un oggetto (documento, sito web, ...) in cui si trovano spazi temporaneamente "bianchi" da riempire successivamente.
  \\

  \glossDef{Topic} Tema di discussione, argomento.
  \\

  \glossDef{Travis-CI} Servizio distribuito per l'integrazione continua usato per testare i progetti hostati su GitHub.
  \\

  \glossaryLetter{U}

  \glossDef{UI (User Interface)} L'interfaccia utente è ciò che si frappone tra una macchina e un utente, consentendo l'interazione tra i due. In ambito informatico sono tipicamente utilizzati i seguenti tipi di interfaccia utente:

  \begin{itemize}
  \item Interfaccia a riga di commando (CLI);
  \item Interfaccia grafica (GUI).
  \end{itemize}

  \glossDef{UML (Unified Modeling Language)} In ingegneria del software, UML è un linguaggio di modellizzazione e specifica basato sul paradigma orientato agli oggetti.
  L'UML svolge un'importantissima funzione di "lingua franca" nella comunità della progettazione e programmazione a oggetti. Gran parte della letteratura di settore usa UML per descrivere soluzioni analitiche e progettuali in modo sintetico e comprensibile a un vasto pubblico.
  \\

   \glossDef{Unicode} sistema di codifica che assegna un numero univoco ad ogni carattere usato per la scrittura di testi, in maniera indipendente dalla lingua, dalla piattaforma informatica e dal programma utilizzato
   \\

  \glossDef{UpperCamelCase} vedi \textit{CamelCase}.
  \\

  \glossDef{URL (Uniform Resource Locator)} nella terminologia delle telecomunicazioni e dell'informatica è una sequenza di caratteri che identifica univocamente l'indirizzo di una risorsa in Internet, tipicamente presente su un host server, come ad esempio un documento, un'immagine, un video, rendendola accessibile ad un client che ne faccia richiesta attraverso l'utilizzo di un web browser.
  \\

  \glossDef{UTF-8 (Unicode Transformation Format, 8 bit)} È una codifica dei caratteri \textit{Unicode} (vedi sopra) in sequenze di lunghezza variabile di byte, creata da Rob Pike e Ken Thompson. UTF-8 usa gruppi di byte per rappresentare i caratteri Unicode, ed è particolarmente utile per il trasferimento tramite sistemi di posta elettronica a 8-bit.
  In particolare, UTF-8 usa da 1 a 4 byte per rappresentare un carattere Unicode.
  \\

  \glossaryLetter{V}

  \glossDef{Versionamento} Il versionamento (versioning), in informatica, è la gestione di versioni multiple di un insieme di informazioni.
  Gli strumenti software per il controllo versione sono ritenuti molto spesso necessari per la maggior parte dei progetti di sviluppo software.
  \\

  \glossDef{Virtual Machine} Il termine macchina virtuale (VM) indica un software che, attraverso un processo di virtualizzazione, crea un ambiente virtuale che emula tipicamente il comportamento di una macchina fisica grazie all'assegnazione di risorse hardware (porzioni di disco rigido, RAM e risorse di processamento) ed in cui alcune applicazioni possono essere eseguite come se interagissero con tale macchina.
  Tra i vantaggi vi è il fatto di poter offrire contemporaneamente ed efficientemente a più utenti diversi ambienti operativi separati, ciascuno attivabile su effettiva richiesta, senza sporcare il sistema fisico reale con il partizionamento del disco rigido oppure fornire ambienti clusterizzati (connessi tra loro) su sistemi server.
  \\
  
  \glossDef{Visibilità} Nell'ambito di Actorbase, gli utenti registrati hanno la possibilità di avere diversi gradi di visibilità sulla base di dati, in base al fatto che questi siano amministratori o semplici utenti.
  La visibilità garantisce privilegi sia in lettura che in scrittura; un amministratore ha visibilità sull'intero sistema mentre ogni utente generico solo sulle proprie collezioni e/o su collezioni di proprietà altrui su cui è in collaborazione.
  \\

  \glossDef{VoIP (Voice over IP)} tecnologia che rende possibile effettuare una conversazione telefonica sfruttando una connessione Internet o una qualsiasi altra rete dedicata a commutazione di pacchetto che utilizzi il protocollo IP senza connessione per il trasporto dati.
  \\

  \glossaryLetter{W}

  \glossDef{Warehousemen} Nel contesto del capitolato da noi considerato, con warehousemen si intende un particolare tipo di attore che si interfaccia con gli \textit{storekeeper} (vedi sopra) e persiste su disco le rispettive mappe.
  \\

  \glossDef{Whiteboarding} È una tecnica di condivisione di un notebook o di una whiteboard (Lavagna) virtuali nei quali più persone possono lavorare contemporaneamente per scambiarsi idee, bozze di grafici e schemi.
  Tutto questo viene fatto online così da permettere ai singoli utenti di vedere in tempo reale le modifiche fatte al documento.
  \\

  \glossDef{Wiki} Il wiki è un sito web, costruito appoggiandosi su una piattaforma o software collaborativo detto software wiki o semplicemente Wiki, che permette ai propri utenti di aggiungere, modificare o cancellare contenuti apertamente attraverso un browser web, in genere utilizzando un linguaggio di markup semplificato o un editor di testo online.
  Si tratta in altre parole di una raccolta di documenti ipertestuali che viene aggiornata dai suoi stessi utilizzatori e i cui contenuti sono sviluppati in collaborazione da tutti coloro che vi hanno accesso (contenuto generato dagli utenti), memorizzati normalmente su un database o un repository.
  \\

  \glossaryLetter{X}

  \glossaryLetter{Y}

  \glossaryLetter{Z}

\end{document}
