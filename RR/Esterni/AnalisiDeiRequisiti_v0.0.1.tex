% Generated with genpdf
% Domain Interno
% Title Analisi dei Requisiti
\documentclass{scalatekids-article}
\begin{document}
\lfoot{Analisi dei Requisiti 0.0.1}
\newgeometry{top=3.5cm}
\begin{titlepage}
  \begin{center}
    \begin{center}
      \includegraphics[width=10cm]{sklogo.png}
    \end{center}
    \vspace{1cm}
    \begin{Huge}
      \begin{center}
        \textbf{Analisi dei Requisiti}
      \end{center}
    \end{Huge}
    \vspace{11pt}
    \bgroup
    \def\arraystretch{1.3}
    \begin{tabular}{r|l}
      \multicolumn{2}{c}{\textbf{Informazioni sul documento}} \\
      \hline
      \setbox0=\hbox{0.0.1\unskip}\ifdim\wd0=0pt
      \\
      \else
      \textbf{Versione} & 0.0.3\\
      \fi
      \textbf{Redazione} & \multiLineCell[t]{Redattore}\\
      \textbf{Verifica} & \multiLineCell[t]{Verificatore}\\
      \textbf{Approvazione} & \multiLineCell[t]{Approvatore}\\
      \textbf{Uso} & Esterno\\
      \textbf{Lista di Distribuzione} & \multiLineCell[t]{ScalateKids\\Prof. Tullio Vardanega\\Prof. Riccardo Cardin}\\
    \end{tabular}
    \egroup
    \vspace{22pt}
  \end{center}
\end{titlepage}
\restoregeometry
\clearpage
\setcounter{page}{1}
\begin{flushleft}
  \vspace{0cm}
         {\large\bfseries Diario delle modifiche \par}
\end{flushleft}
\vspace{0cm}
\begin{center}
  \begin{tabular}{| l | l | l | l | l |}
    \hline
    Versione & Autore & Ruolo & Data & Descrizione \\
    \hline
    0.0.1 & Andrea Giacomo Baldan & & 2015-12-27 & Creazione scheletro del documento\\
    \hline
  \end{tabular}
\end{center}
\tableofcontents
\newpage
\prodPurpose
\glossExpl
\section{Descrizione generale}
\subsection{Prospettive}
\subsection{Funzioni}
\subsection{Caratteristiche Utenza}
\subsection{Vincoli}
\section{Casi d'uso}
\subsection{UC1: Interazioni Basilari}
\textbf{Attori primari:} Utente\\ \\
\textbf{Scopo e descrizione:} Il database è installato correttamente e il server è in ascolto.
L'utente ha avviato un terminale ed è pronto ad inserire comandi.\\ \\
\textbf{Precondizione:} Il database è installato correttamente, e il server è in ascolto.\\ \\
\textbf{Flusso eventi:}
\begin{itemize}
\item{UC1.1} Utente digita comando
\item{UC1.2} Utente conferma il comando
\end{itemize}
\textbf{Postcondizione:} Il sistema ha prodotto l'output corrispondente al comando immesso dall'utente.
\subsection{UC1.1: Scrittura comando}
\textbf{Attori primari:} Utente\\ \\
\textbf{Scopo e descrizione:} L’utente scrive il comando sul terminale.\\ \\
\textbf{Precondizione:} Il database è installato correttamente, il server è in ascolto e l'utente intende scrivere un comando.\\ \\
\textbf{Flusso eventi:}
\begin{itemize}
\item UC1.1.1 Creazione di una collezione
\item UC1.1.2 Elencazione di collezioni
\item UC1.1.3 Cancellazione di una o più collezioni
\item UC1.1.4 Modifica di una collezione
\item UC1.1.5 Ricerca generale
\item UC1.1.6 Inserimento di un item
\item UC1.1.7 Cancellazione di un item
\item UC1.1.8 Modifica di un item
\item UC1.1.9 Richiesta di supporto
\end{itemize}
\textbf{Estensioni:}
\begin{itemize}
\item Nel caso in cui l'utente utilizzi un comando sconosciuto:
  \begin{itemize}
  \item viene visualizzato un messaggio di errore esplicativo (e rimando all'aiuto)
  \end{itemize}
\item Nel caso in cui l'utente utilizzi un comando conosciuto ma con parametri errati:
  \begin{itemize}
    viene visualizzato un messaggio di supporto che indica come usare correttamente il comando
  \end{itemize}
\item Nel caso in cui l'utente voglia modificare dati non presenti nel database:
  \begin{itemize}
    viene visualizzato un messaggio di errore esplicativo
  \end{itemize}
\end{itemize}
\textbf{Postcondizione:} L'utente ha digitato un comando
\subsection{UC1.1.1: Creazione di una collezione}
\textbf{Attore primario:} Utente\\ \\
\textbf{Scopo e descrizione:} L’utente intende creare una nuova collezione, deve inserire il nome della collezione.\\ \\
\textbf{Precodizione:} Il database è pronto a ricevere comandi e l’utente vuole creare una nuova collezione\\ \\
\textbf{Flusso di eventi:}
\begin{itemize}
\item l'utente digita il comando di creazione con parametro nome della nuova collezione
\end{itemize}
\textbf{Postcondizione:} L'utente ha digitato il comando per creare una collezione
\subsection{UC1.1.2: Elencazione di collezioni}
\textbf{Attore primario:} Utente\\ \\
\textbf{Scopo e descrizione:} L’utente intende elencare tutte le collezioni presenti nel database.\\ \\
\textbf{Precodizione:} Il database è pronto a ricevere comandi e l’utente vuole visualizzare un elenco delle collezioni\\ \\
\textbf{Flusso di eventi:}
\begin{itemize}
\item l'utente digita il comando di elencazione di collezioni
\end{itemize}
\textbf{Postcondizione:} L'utente ha digitato il comando per elencare le collezioni
\subsection{UC1.1.3: Cancellazione di una o più collezioni}
\textbf{Attore primario:} Utente\\ \\
\textbf{Scopo e descrizione:} L’utente intende cancellare una o più collezioni presenti nel database.\\ \\
\textbf{Precodizione:} Il database è pronto a ricevere comandi e l’utente intende cancellare una o più collezioni\\ \\
\textbf{Flusso di eventi:}
\begin{itemize}
\item l'utente digita il comando di cancellazione di una o più collezioni
\end{itemize}
\textbf{Postcondizione:} L’utente ha digitato il comando per cancellare una o più collezioni
\subsection{UC1.1.4: Modifica di una collezione}
\textbf{Attore primario:} Utente \\ \\
\textbf{Scopo e descrizione:} L’utente intende modificare una collezione, deve inserire i parametri per la modifica della collezione\\ \\
\textbf{Precondizione:} Il database è pronto a ricevere comandi e l’utente vuole modificare una collezione\\ \\
\textbf{Flusso di eventi:}
\begin{itemize}
\item l’utente digita il comando di modifica con parametri nome della collezione da modificare e valori dei campi da modificare
\end{itemize}
\textbf{Postcondizione:} L’utente ha digitato il comando per modificare una collezione
\subsection{UC1.1.5: Ricerca generale}
\textbf{Attore primario:} Utente \\ \\
\textbf{Scopo e descrizione:} L’utente intende effettuare una ricerca\\ \\
\textbf{Precondizione:} Il database è pronto a ricevere comandi e l’utente intende effettuare una ricerca\\ \\
\textbf{Flusso di eventi:}
\begin{itemize}
\item UC1.1.5.1 Utente può digitare il comando per effettuare una ricerca sull’intera base di dati
\item UC1.1.5.2 Utente può digitare il comando per effettuare una ricerca all’interno di una o più collezioni specifiche
\end{itemize}
\textbf{Postcondizione:} L’utente ha digitato il comando per effettuare una ricerca
\subsection{UC1.1.5.1: Ricerca all'interno della base di dati}
\textbf{Attore primario:} Utente \\ \\
\textbf{Scopo e descrizione:} L’utente intende effettuare una ricerca sull’intera base di dati\\ \\
\textbf{Precondizione:} Il database è pronto a ricevere comandi e l’utente intende effettuare una ricerca generale\\ \\
\textbf{Flusso di eventi:}
\begin{itemize}
\item l’utente digita il comando di ricerca generale
\end{itemize}
\textbf{Postcondizione:} L’utente ha digitato il comando per effettuare una ricerca di aiutanza generale
\subsection{UC1.1.5.2: Ricerca all'interno di una specifica collezione(o più)}
\textbf{Attore primario:} Utente \\ \\
\textbf{Scopo e descrizione:} L’utente intende effettuare una ricerca all’interno di una o più collezioni specifiche\\ \\
\textbf{Precondizione:} Il database è pronto a ricevere comandi e l’utente intende effettuare una ricerca dettagliata\\ \\
\textbf{Flusso di eventi:}
\begin{itemize}
\item l’utente digita il comando di ricerca con parametro i nomi delle collezioni sulle quali vuole effettuare la ricerca
\end{itemize}
\textbf{Postcondizione:} L’utente ha digitato il comando per effettuare una ricerca specifica
\end{document}
