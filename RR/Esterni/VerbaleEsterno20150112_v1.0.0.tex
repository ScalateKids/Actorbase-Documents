% Generated with genpdf
% Domain Esterno
% Title Verbale riunione con proponente
\documentclass{scalatekids-article}
\def\changemargin#1#2{\list{}{\rightmargin#2\leftmargin#1}\item[]}
\let\endchangemargin=\endlist
\begin{document}
\lfoot{Verbale esterno 2016-01-12 0.1.1}
\begin{titlepage}
  \centering
  \includegraphics[width=0.45\textwidth]{logo.png}\par\vspace{1cm}
  \vspace{1.5cm}
         {\Huge\bfseries Verbale esterno 2016-01-12  \par}
         \begin{center}
           \vspace{1.0cm}
                  {\large\bfseries Informazioni sul documento \par}
         \end{center}
         \vspace{-1cm}
         \begin{center}
           \line(1,0){300}
         \end{center}
         \vspace{0cm}
         \begin{tabular}[c]{l|l}
           \textbf{Versione} & 1.0.0\\
           \textbf{Redazione} & Alberto De Agostini\\ & Andrea Giacomo Baldan\\
           \textbf{Verifica} & Marco Boseggia\\
           \textbf{Responsabile} & Andrea Giacomo Baldan\\
           \textbf{Uso} & Esterno\\
           \textbf{Lista di distribuzione} & ScalateKids\\ &Riccardo Cardin\\ & Tullio Vardanega\\
         \end{tabular}
\end{titlepage}
\clearpage
\setcounter{page}{1}
\begin{flushleft}
  \vspace{0cm}
         {\large\bfseries Diario delle modifiche \par}
\end{flushleft}
\vspace{0cm}
\begin{center}
  \begin{tabular}{| l | l | l | l | l |}
    \hline
    Versione & Autore & Ruolo & Data & Descrizione \\
    \hline
    1.0.0 & Andrea Giacomo Baldan & Responsabile & 2016-01-13 & Validazione documento\\
    \hline
    0.1.1 & Marco Boseggia & Verificatore & 2016-01-12 & Verifica documento\\
    \hline
    0.1.0 & Alberto De Agostini & Analista & 2016-01-12 & Prima stesura documento\\
    \hline
    0.0.1 & Alberto De Agostini & Analista & 2016-01-12 & Creazione scheletro del documento\\
    \hline
  \end{tabular}
\end{center}
\tableofcontents
\newpage
\section{Informazioni sulla riunione}
\begin{itemize}
  \item \textbf{Data}: 2015-01-12
  \item \textbf{Luogo}: Torre Archimede aula 1C150
  \item \textbf{Orario d'inizio}: 13:15 
  \item \textbf{Durata}: 25'
  \item \textbf{Partecipanti interni}: \textit{ScalateKids}:
  \begin{itemize}
    \item Andrea Giacomo Baldan
    \item Alberto De Agostini
    \item Francesco Agostini
    \item Giacomo Vanin
    \item Marco Boseggia
    \item Michael Munaro
  \end{itemize}
  \item \textbf{Partecipanti esterni}:
  \begin{itemize}
    \item Riccardo Cardin
  \end{itemize}
\end{itemize}
\section{Domande e risposte}
In questo capitolo sono elencate le domande fatte dal gruppo \textit{ScalateKids} in grassetto e le risposte del proponente \textit{Riccardo Cardin} in corsivo.
\textbf{\\ \\Nel capitolato non c'è una chiara divisione tra requisiti desiderabili e requisiti opzionali, c'è semplicemente un elenco di requisiti opzionali. Le chiedevamo se c'è una preferenza tra questi.}
\begin{quote}
  \textit{Avendoli lasciati tutti opzionali nel capitolato, rimando a voi la scelta su quali pensate siano più importanti, cercando di implementare prima gli opzionali che al meglio completano quelli obbligatori.\\ \\}
\end{quote}
\textbf{Dovremo istanziare un terminale isolato? (come ad esempio una \gloss{shell} MySql)}
\begin{quote}
  \textit{No, la \gloss{CLI} da creare sarà composta da un gruppo di semplici comandi da usare da terminale.\\ \\}
\end{quote}
\textbf{Nel capitolato sono presenti solo i comandi per inserimento, cancellazione e modifica dati. I database consentono più operazioni di queste, alcune delle quali riteniamo essenziali (es. \gloss{query}), sono queste considerate requisiti opzionali?}
\begin{quote}
  \textit{Sicuramente dev'essere possibile interrogare il database, gli altri comandi sono dei requisiti obbligatori impliciti. La lista delle funzionalità tuttavia è una cosa che voi dovete decidere e scrivere nei vostri documenti.\\ \\}
\end{quote}
\textbf{Nel capitolato c'è scritto che gli attori di tipo \textit{Manager} sono responsabili della gestione del numero massimo di coppie chiave / valore contenute in un’istanza di un \gloss{attore storekeeper}. Con che criterio deve essere scelto questo numero massimo?}
\begin{quote}
  \textit{Può essere scelto in diversi modi; il più facile di tutti è senz'altro scegliere un numero prefissato, altrimenti si potrebbe scegliere di cambiare questo numero dinamicamente in base al carico. Tuttavia questo non è un compito semplice, quindi anche in questo caso la decisione spetta a voi.\\ \\}
\end{quote}
\textbf{Abbiamo deciso di utilizzare il linguaggio \textit{Scala} per sviluppare il prodotto. Le ultime versioni di questo linguaggio non sono compatibili tra loro, quale versione ci consiglia di utilizzare? (2.10 o 2.11)}
\begin{quote}
  \textit{Sempre meglio utilizzare le ultime versioni, dovrete quindi utilizzare di conseguenza la \gloss{JVM} versione 8\\ \\}
\end{quote}
\textbf{Cosa dovrà arrivare all'utente finale visto come utilizzatore?}
\begin{quote}
  \textit{Dovrete consegnare un eseguibile comprensivo di librerie \gloss{bundled} per semplificare l'installazione all'utente.}
\end{quote}
\end{document}
