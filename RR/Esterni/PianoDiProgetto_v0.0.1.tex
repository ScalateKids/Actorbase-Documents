\documentclass{scalatekids-article}
\begin{document}
\lhead{ScalateKids}
\rhead{Piano Di Progetto \thepage}
\lfoot{Piano Di Progetto 0.0.1}
\begin{titlepage}
  \centering
  \includegraphics[width=0.45\textwidth]{logo.png}\par\vspace{1cm}
  \vspace{1.5cm}
         {\Huge\bfseries Piano Di Progetto \par}
         \begin{center}
           \vspace{1.0cm}
                  {\large\bfseries Informazioni sul documento \par}
         \end{center}
         \vspace{-1cm}
         \begin{center}
           \line(1,0){300}
         \end{center}
         \vspace{0cm}
         \begin{tabular}[c]{l|l}
           \textbf{Versione} & 0.0.1\\
           \textbf{Redazione} & Redattore1\\ & Redattore2\\ & Redattore3..\\
           \textbf{Verifica} & Verificatore1\\ & Verificatore2..\\
           \textbf{Responsabile} & Responsabile\\
           \textbf{Uso} & Esterno
         \end{tabular}
\end{titlepage}
\clearpage
\setcounter{page}{1}
\begin{flushleft}
  \vspace{0cm}
         {\large\bfseries Diario delle modifiche \par}
\end{flushleft}
\vspace{0cm}
\begin{center}
  \begin{tabular}{|l | l | l | l |}
    \hline
    Versione & Autore & Data & Descrizione \\
    \hline
    0.0.1 & Andrea Giacomo Baldan & 2015-12-21 & Creazione scheletro del documento\\
    \hline
  \end{tabular}
\end{center}
\tableofcontents
\section{Sommario}
\section{Analisi dei rischi}
L'analisi dei rischi ha costituito una fase critica della pianificazione, in
quanto l'azienda, alla sua prima esperienza, ha dovuto pensare agli ipotetici
scenari negativi che possono formarsi col procedere della fase di sviluppo.
\subsection{Identificazione del rischio}
Per l'identificazione dei rischi il primo passo è stato un' attività di
\summ{Brainstorming} suddiviso per categorie di pertinenza, in modo da esplorare
in maniera quanto più generale e ampia possibile i vari aspetti di ogni
possibile scenario, seguito da una fase di analisi di ogni rischio emerso e
l'impatto che questo avrebbe portato al raggiungimento degli obiettivi preposti
per la conclusione del progetto. Le categorie prese in considerazione sono:
\begin{itemize}
\item\textbf{Logistica}
\item\textbf{Competenza tecnica}
\item\textbf{Infrastruttura}
\end{itemize}
\subsubsection{Parametri di quantificazione dei rischi}
Per ogni possibile rischio previsto l'analisi ha fornito i seguenti parametri:
\begin{itemize}
\item\textbf{Categoria}: Indica la categoria di appartenenza in cui ricade il
  rischio preso in considerazione
\item\textbf{Probabilità}: Probabilità statistica che lo scenario indesiderato
  si presenti, può assumere i seguenti valori:
  \begin{itemize}
  \item\textbf{Basso}
  \item\textbf{Medio}
  \item\textbf{Alto}
  \end{itemize}
\item\textbf{Impatto}: Grado di pericolosità dello scenario indesiderato, può
  assumere i seguenti valori:
  \begin{itemize}
  \item\textbf{Forte}
  \item\textbf{Debole}
  \end{itemize}
\item\textbf{Descrizione}: Una descrizione del caso preso in considerazione.
\item\textbf{Contromisure di mitigazione}: Provvedimenti da attuare in
  previsione del rischio, e/o di mitigazione in caso di bisogno
\end{itemize}
\paragraph{Impegni personali}
\textbf{Categoria}: Logistica\\
\textbf{Probabilità}: Alta\\
\textbf{Impatto}: Basso\\
\textbf{Descrizione}: Il gruppo può avere problemi a riunirsi fisicamente con la maggior parte dei membri presenti.
\textbf{Contromisure di mitigazione}: Il Responsabile di Progetto ha stabilito una frequenza di incontri fissa,
l'azienda ha inoltre creato alcuni canali di comunicazione remota in modo da poter lavorare quanto più possibile in
coordinazione reciproca.
\paragraph{Inesperienza}
\textbf{Categoria}: Competenza\\
\textbf{Probabilità}: Alta\\
\textbf{Impatto}: Medio\\
\textbf{Descrizione}: Il gruppo può rimanere spiazzato dal metodo di lavoro da seguire, l'inesperienza del gruppo nell'attuazione
di competenze di pianificazione e analisi può portare a rallentamenti del perseguimento degli obiettivi.
\textbf{Contromisure di mitigazione}: I componenti del gruppo si impegneranno a studiare e applicarsi il più possibile
nella pratica delle competenze richieste.
\end{document}
