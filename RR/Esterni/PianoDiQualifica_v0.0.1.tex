% Generated with genpdf
% Domain Interno
% Title Piano di Qualifica
\documentclass{scalatekids-article}
\begin{document}
\lfoot{Piano Di Qualifica 0.0.1}
\begin{titlepage}
  \centering
  \includegraphics[width=0.45\textwidth]{logo.png}\par\vspace{1cm}
  \vspace{1.5cm}
         {\Huge\bfseries Piano Di Qualifica \par}
         \begin{center}
           \vspace{1.0cm}
                  {\large\bfseries Informazioni sul documento \par}
         \end{center}
         \vspace{-1cm}
         \begin{center}
           \line(1,0){300}
         \end{center}
         \vspace{0cm}
         \begin{tabular}[c]{l|l}
           \textbf{Versione} & 0.0.1\\
           \textbf{Redazione} & Redattore1\\ & Redattore2\\ & Redattore3..\\
           \textbf{Verifica} & Verificatore1\\ & Verificatore2..\\
           \textbf{Responsabile} & Responsabile\\
           \textbf{Uso} & Esterno\\
           \textbf{Lista di distribuzione} & ScalateKids
         \end{tabular}
\end{titlepage}
\clearpage
\setcounter{page}{1}
\begin{flushleft}
  \vspace{0cm}
         {\large\bfseries Diario delle modifiche \par}
\end{flushleft}
\vspace{0cm}
\begin{center}
  \begin{tabular}{| l | l | l | l | l |}
    \hline
    Versione & Autore & Ruolo & Data & Descrizione \\
    \hline
    0.0.1 & Andrea Giacomo Baldan & & 2015-12-27 & Creazione scheletro del documento\\
    \hline
  \end{tabular}
\end{center}
\tableofcontents
\newpage
\section{Sommario}
\subsection{Scopo del documento}
Il seguente documento ha lo scopo di delineare le strategie che il gruppo \textit{ScalateKids} ha deciso di adottare per garantire degli obiettivi qualitativi da applicare ai processi sfruttati per lo sviluppo del progetto \textbf{ActorBase}. Per raggiungere tali obiettivi è necessario svolgere un'attività di verifica continua in modo da rilevare e correggere errori o malfunzionamenti minimizzando lo spreco di risorse.
\prodPurpose
\glossExpl
\subsection{Riferimenti}
%TODO
\newpage
\section{Visione generale della strategia di verifica}
\subsection{Organizzazione}
Per garantire la qualità del prodotto finale verranno effettuate attività di verifica su ogni processo attuato. I controlli su ogni processo avranno come obiettivo garantire la qualità sia del processo stesso sia dell'eventuale prodotto da esso ricavato.\\
Ogni fase descritta in \href{run:./PianoDiProgetto\_v0.1.1.pdf}{Piano Di Progetto v0.1.1} necessità di attività di verifica diverse a causa della differenza natura del loro output. Di seguito sono elencate le strategie di verifica usate per ciascuna fase:
\begin{itemize}
\item\textbf{Analisi:} durante questa fase verrà verificato che ogni documento creato sia conforme a quanto descritto nel documento \href{run:../Interni/NormeDiProgetto\_v0.0.1.pdf}{Norme Di Progetto v0.0.1}, inoltre sarà necessario controllare che ogni requisito sia associato a un caso d'uso e viceversa;
\item\textbf{Progettazione:} durante questa fase l'attività di verifica si concentrerà sul processo incrementale dei documenti prodotti in fase di Analisi e sui processi attuati per l'attività di progettazione assicurandosi che ogni documento e ogni processo sia conforme a quanto descritto nelle \href{run:../Interni/NormeDiProgetto\_v0.0.1.pdf}{Norme Di Progetto v0.0.1};
\item\textbf{Codifica:}durante questa fase l'attività di verifica si concentrerà sul processo incrementale dei documenti prodotti in fase di Analisi e sui processi attuati per l'attività di codifica assicurandosi che ogni documento e ogni processo sia conforme a quanto descritto nelle \href{run:../Interni/NormeDiProgetto\_v0.0.1.pdf}{Norme Di Progetto v0.0.1}.
\end{itemize}
In ogni documento viene inoltre tenuto un diario delle modifiche per permettere di mantenere uno storico delle attività svolte su di esso e delle responsabilità per ogni attività.
\subsection{Pianificazione strategica e temporale}
L'attività di redazione di ogni documento sarà sempre anticipata da uno studio sulla struttura e sui contenuti del documento interessato. In questo modo si riduce la possibilità di errori concettuali e il tempo effettivo di scrittura del documento viene ridotto.\\
Il processo di verifica deve essere ben organizzato e pianificato per rispettare le scadenze fissate nel \href{run:./PianoDiProgetto\_v0.1.1.pdf}{Piano Di Progetto v0.1.1}.\\
L'attività di verifica segue le metodologie descritte nelle \href{run:../Interni/NormeDiProgetto\_v0.0.1.pdf}{Norme Di Progetto v0.0.1}. Esse sono state pensate in maniera tale da consentire l'individuazione e la correzione degli errori il più presto possibile evitando che si diffondano.
\end{document}
