% Generated with genpdf
% Domain Interno
% Title Piano di Qualifica
\documentclass{scalatekids-article}
\begin{document}
\lfoot{Piano Di Qualifica 0.0.1}
\newgeometry{top=3.5cm}
\begin{titlepage}
  \begin{center}
    \begin{center}
      \includegraphics[width=10cm]{sklogo.png}
    \end{center}
    \vspace{1cm}
    \begin{Huge}
      \begin{center}
        \textbf{Piano di Qualifica}
      \end{center}
    \end{Huge}
    \vspace{11pt}
    \bgroup
    \def\arraystretch{1.3}
    \begin{tabular}{r|l}
      \multicolumn{2}{c}{\textbf{Informazioni sul documento}} \\
      \hline
      \setbox0=\hbox{0.0.1\unskip}\ifdim\wd0=0pt
      \\
      \else
      \textbf{Versione} & 0.0.3\\
      \fi
      \textbf{Redazione} & \multiLineCell[t]{Redattore}\\
      \textbf{Verifica} & \multiLineCell[t]{Verificatore}\\
      \textbf{Approvazione} & \multiLineCell[t]{Approvatore}\\
      \textbf{Uso} & Esterno\\
      \textbf{Lista di Distribuzione} & \multiLineCell[t]{ScalateKids\\Prof. Tullio Vardanega\\Prof. Riccardo Cardin}\\
    \end{tabular}
    \egroup
    \vspace{22pt}
  \end{center}
\end{titlepage}
\restoregeometry
\clearpage
\setcounter{page}{1}
\begin{flushleft}
  \vspace{0cm}
         {\large\bfseries Diario delle modifiche \par}
\end{flushleft}
\vspace{0cm}
\begin{center}
  \begin{tabular}{| l | l | l | l | l |}
    \hline
    Versione & Autore & Ruolo & Data & Descrizione \\
    \hline
    0.0.1 & Andrea Giacomo Baldan & & 2015-12-27 & Creazione scheletro del documento\\
    \hline
  \end{tabular}
\end{center}
\tableofcontents
\newpage
\section{Sommario}
\subsection{Scopo del documento}
Il seguente documento ha lo scopo di delineare le strategie che il gruppo \textit{ScalateKids} ha deciso di adottare per garantire degli obiettivi qualitativi da applicare ai processi sfruttati per lo sviluppo del progetto \textbf{ActorBase}. Per raggiungere tali obiettivi è necessario svolgere un'attività di verifica continua in modo da rilevare e correggere errori o malfunzionamenti minimizzando lo spreco di risorse.
\prodPurpose
\glossExpl
\subsection{Riferimenti}
%TODO
\newpage
\section{Visione generale della strategia di verifica}
\subsection{Definizione obiettivi}
\subsubsection{Qualità di processo}
Per perseguire la qualità del prodotto è necessario garantire la qualità dei processi attuati per la sua creazione. Per raggiungere questo scopo i \textit{Verificatori} con approvazione del \textit{Responsabile} hanno trovato utile adottare lo standard \gloss{ISO}/\gloss{IEC} 15504 denominato SPICE (Software Process Improvement and Capability dEtermination). Il \gloss{ciclo di Deming} (noto anche come ciclo \gloss{PDCA}) è stato scelto per applicare correttamente questo standard. Esso definisce un metodo di controllo costante sui processi consentendo in questo modo un miglioramento incrementale della qualità continuo.
\subsubsection{Qualità di prodotto}
I \textit{Verificatori} con approvazione del \textit{Responsabile} hanno deciso di seguire lo standard \gloss{ISO}/\gloss{IEC} 9126 che ha lo scopo di delineare delle metriche per la misurazione della qualità di prodotto. Questa è fondamentale per garantire che il prodotto \gloss{software} finale funzioni correttamente e rispetti degli obiettivi di qualità prefissati.
\subsection{Procedure di controllo di qualità di processo}
La qualità dei processi è assicurata dall'applicazione del ciclo \gloss{PDCA} a ogni processo. Tale metodo consente di migliorare continuamente ogni processo coinvolto nell'attività di sviluppo. Per applicare correttamente il ciclo \gloss{PDCA} è necessario avere una pianificazione chiara di processi e risorse. La descrizione dettagliata della pianificazione si trova nel \href{run:./PianoDiProgetto\_v0.1.1.pdf}{Piano Di Progetto v0.1.1}.\\
Le metriche usate per quantificare la qualità verranno descritte nella sezione *heh, quando sarà fatta metto il numero*.
\subsection{Procedure di controllo di qualità di prodotto}
Il controllo di qualità del prodotto si suddividerà in tre parti:
\begin{itemize}
\item{Attuazione di tecniche di analisi statica e dinamica descritte nella sezione *heh, quando sarà fatta metto il numero*;}
\item{Verifica costante dell'output di ogni attività, i cui risultati verranno riportati nell'appendice *B?*;}
\item{Validazione, ovvero la conferma che il prodotto risponda correttamente ai requisiti attesi.}
\end{itemize}
\subsection{Organizzazione}
Per garantire la qualità del prodotto finale verranno effettuate attività di verifica su ogni processo attuato. I controlli su ogni processo avranno come obiettivo garantire la qualità sia del processo stesso sia dell'eventuale prodotto da esso ricavato.\\
Ogni fase descritta in \href{run:./PianoDiProgetto\_v0.1.1.pdf}{Piano Di Progetto v0.1.1} necessità di attività di verifica diverse a causa della differenza natura del loro output. Di seguito sono elencate le strategie di verifica usate per ciascuna fase:
\begin{itemize}
\item\textbf{Analisi:} durante questa fase verrà verificato che ogni documento creato sia conforme a quanto descritto nel documento \href{run:../Interni/NormeDiProgetto\_v0.0.1.pdf}{Norme Di Progetto v0.0.1}, inoltre sarà necessario controllare che ogni requisito sia associato a un caso d'uso e viceversa;
\item\textbf{Progettazione:} durante questa fase l'attività di verifica si concentrerà sul processo incrementale dei documenti prodotti in fase di Analisi e sui processi attuati per l'attività di progettazione assicurandosi che ogni documento e ogni processo sia conforme a quanto descritto nelle \href{run:../Interni/NormeDiProgetto\_v0.0.1.pdf}{Norme Di Progetto v0.0.1};
\item\textbf{Codifica:}durante questa fase l'attività di verifica si concentrerà sul processo incrementale dei documenti prodotti in fase di Analisi e sui processi attuati per l'attività di codifica assicurandosi che ogni documento e ogni processo sia conforme a quanto descritto nelle \href{run:../Interni/NormeDiProgetto\_v0.0.1.pdf}{Norme Di Progetto v0.0.1}.
\end{itemize}
In ogni documento viene inoltre tenuto un diario delle modifiche per permettere di mantenere uno storico delle attività svolte su di esso e delle responsabilità per ogni attività.
\subsection{Pianificazione strategica e temporale}
L'attività di redazione di ogni documento sarà sempre anticipata da uno studio sulla struttura e sui contenuti del documento interessato. In questo modo si riduce la possibilità di errori concettuali e il tempo effettivo di scrittura del documento viene ridotto.\\
Il processo di verifica deve essere ben organizzato e pianificato per rispettare le scadenze fissate nel \href{run:./PianoDiProgetto\_v0.1.1.pdf}{Piano Di Progetto v0.1.1}.\\
L'attività di verifica segue le metodologie descritte nelle \href{run:../Interni/NormeDiProgetto\_v0.0.1.pdf}{Norme Di Progetto v0.0.1}. Esse sono state pensate in maniera tale da consentire l'individuazione e la correzione degli errori il più presto possibile evitando che si diffondano.
%TODO
\end{document}
