% Generated with genpdf
% Domain Esterno
% Title Verbale riunione con proponente
\documentclass{scalatekids-article}
\def\changemargin#1#2{\list{}{\rightmargin#2\leftmargin#1}\item[]}
\let\endchangemargin=\endlist
\begin{document}
\lfoot{Verbale esterno 2016-01-19 1.0.0}
\newgeometry{top=3.5cm}
\begin{titlepage}
  \begin{center}
    \begin{center}
      \includegraphics[width=10cm]{sklogo.png}
    \end{center}
    \vspace{1cm}
    \begin{Huge}
      \begin{center}
        \textbf{Verbale Esterno 2016-01-19}
      \end{center}
    \end{Huge}
    \vspace{11pt}
    \bgroup
    \def\arraystretch{1.3}
    \begin{tabular}{r|l}
      \multicolumn{2}{c}{\textbf{Informazioni sul documento}} \\
      \hline
      \setbox0=\hbox{0.0.1\unskip}\ifdim\wd0=0pt
      \\
      \else
      \textbf{Versione} & 1.0.0\\
      \fi
      \textbf{Redazione} & \multiLineCell[t]{Giacomo Vanin}\\
      \textbf{Verifica} & \multiLineCell[t]{Marco Boseggia}\\
      \textbf{Approvazione} & \multiLineCell[t]{Andrea Giacomo Baldan}\\
      \textbf{Uso} & Esterno\\
      \textbf{Lista di Distribuzione} & \multiLineCell[t]{ScalateKids\\Prof. Tullio Vardanega\\Prof. Riccardo Cardin}\\
    \end{tabular}
    \egroup
    \vspace{22pt}
  \end{center}
\end{titlepage}
\restoregeometry
\clearpage
\setcounter{page}{1}
\begin{flushleft}
  \vspace{0cm}
         {\large\bfseries Diario delle modifiche \par}
\end{flushleft}
\vspace{0cm}
\begin{center}
  \begin{tabular}{| l | l | l | l | l |}
    \hline
    Versione & Autore & Ruolo & Data & Descrizione \\
    \hline
    1.0.0 & Andrea Giacomo Baldan & Responsabile & 2016-01-21 & Validazione documento\\
    \hline
    0.1.1 & Marco Boseggia & Verificatore & 2016-01-20 & Verifica documento\\
    \hline
    0.1.0 & Giacomo Vanin & Analista & 2016-01-20 & Prima stesura documento\\
    \hline
    0.0.1 & Alberto De Agostini & Analista & 2016-01-12 & Creazione scheletro del documento\\
    \hline
  \end{tabular}
\end{center}
\tableofcontents
\newpage
\section{Informazioni sulla riunione}
\begin{itemize}
\item \textbf{Data}: 2015-01-19
\item \textbf{Luogo}: Torre Archimede aula 1C150
\item \textbf{Orario d'inizio}: 13:15
\item \textbf{Durata}: 15'
\item \textbf{Partecipanti interni}: \textit{ScalateKids}
  \begin{itemize}
  \item Andrea Giacomo Baldan
  \item Alberto De Agostini
  \item Francesco Agostini
  \item Giacomo Vanin
  \item Marco Boseggia
  \item Michael Munaro
  \end{itemize}
\item \textbf{Partecipanti esterni}:
  \begin{itemize}
  \item Riccardo Cardin
  \end{itemize}
\end{itemize}
\section{Domande e risposte}
In questo capitolo sono elencate le domande fatte dal gruppo \textit{ScalateKids} in grassetto e le risposte del proponente \textit{Riccardo Cardin} in corsivo.
\textbf{\\ \\E' necessario fornire un sistema di autenticazione per il database?}
\begin{quote}
  \textit{Nel capitolato non è espressamente scritto ma è una feature che però hanno tutti i database.\\ \\}
\end{quote}
\textbf{Il prodotto finale sarà un client e server allo stesso tempo in modo da garantire scalabilità orizzontale?}
\begin{quote}
  \textit{Il prodotto finale sarà un'applicazione distribuita ma saranno tutti peer, cioè parti che interoperano tra di loro per dare a un client finale il risultato di un'operazione.\\ \\}
\end{quote}
\textbf{E' ragionevole quindi che con il prodotto finale forniamo un eseguibile con due istanze, una per la modalità server e una per la modalità client?}
\begin{quote}
  \textit{Dovrà esserci un eseguibile che sarà il server che avrà vari componenti e l'interfaccia riga di comando che sarà la parte client.\\ \\}
\end{quote}
\textbf{Nei casi d'uso come dovrà essere rappresentata l'interazione tra i vari "server" e il client?}
\begin{quote}
  \textit{Dovranno essere rappresentati come attori secondari.\\ \\}
\end{quote}
\textbf{Quanto dovranno andare nel dettaglio i casi d'uso? Dovrà esserci la shell che parla con il DBMS,il DBMS che parla con la base di dati effettiva?}
\begin{quote}
  \textit{Basterà considerarlo come un'unico blocco. Dovrete però chiarire il funzionamento nella parte architetturale.\\ \\}
\end{quote}
\textbf{Abbiamo visto il funzionamento di Amazon Dynamo per prendere spunto; Dobbiamo puntare a qualcosa del genere? Cioè dobbiamo avere un key-schema per ogni collection?}
\begin{quote}
  \textit{Scegliete voi il modello da seguire. L'importante è che il sistema funzioni. Ovviamente fare una copia di un sistema già esistente sarebbe poco sfidante, non sarebbe un'opportunità.}
\end{quote}
\end{document}
