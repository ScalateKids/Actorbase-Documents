\documentclass{scalatekids-article}
\begin{document}
\lhead{ScalateKids}
\rhead{Glossario \thepage}
\lfoot{Glossario 0.0.1}
\begin{titlepage}
  \centering
  \includegraphics[width=0.45\textwidth]{logo.png}\par\vspace{1cm}
  \vspace{1.5cm}
         {\Huge\bfseries Glossario \par}
         \begin{center}
           \vspace{1.0cm}
                  {\large\bfseries Informazioni sul documento \par}
         \end{center}
         \vspace{-1cm}
         \begin{center}
           \line(1,0){300}
         \end{center}
         \vspace{0cm}
         \begin{tabular}[c]{l|l}
           \textbf{Versione} & 0.0.1\\
           \textbf{Redazione} & Redattore1\\ & Redattore2\\ & Redattore3..\\
           \textbf{Verifica} & Verificatore1\\ & Verificatore2..\\
           \textbf{Responsabile} & Responsabile\\
           \textbf{Uso} & Esterno
         \end{tabular}
\end{titlepage}
\clearpage
\setcounter{page}{1}
\begin{flushleft}
  \vspace{0cm}
         {\large\bfseries Diario delle modifiche \par}
\end{flushleft}
\vspace{0cm}
\begin{center}
  \begin{tabular}{|l | l | l | l | l |}
    \hline
    Versione & Autore & Ruolo & Data & Descrizione \\
    \hline
    0.0.1 & Marco Boseggia & & 2015-12-22 & Creazione scheletro del documento\\
    \hline
  \end{tabular}
\end{center}
\subsection{Scopo del documento}
Questo documento ha il compito di raccogliere tutti i termini che possono risultare sconosciuti o di difficile comprensione ad un lettore esterno oppure, semplicemente, concetti specifici che necessitano di essere esposti per poter meglio comprendere quanto viene detto nei documenti.
\subsection{Riferimenti} //da fare
\newpage
\section{Definizioni}

\summLett{A}

\textbf{Actormodel:} Modello matematico per la gestione della concorrenza computazionale proposto nel 1973 da Carl Hewitt.
All'interno del modello, gli "attori" vengono trattati come primitive universali e possono scmabiarsi messaggi tra di loro; in risposta ad un messaggio ricevuto, un attore può prendere decisioni, creare nuovi attori, inviare altri messaggi e determinare come rispondere al messaggio successivo.

\textbf{Akka:} E' uno strumento per la semplificazione e costruzione di applicazioni concorrenti e distribuite sulla JVM (Java Virtual Machine). Akka supporta diversi modelli per la concorrenza prediligendo, però, l'uso del modello ad attori.

\textbf{Amazon DynamoDB:} Database proprietario di tipo NoSQL e completamente gestito da Amazon.com all'interno del suo portfolio 'Amazon Web Services'.

\summLett{B}

\textbf{Beacon:} il beacon è un piccolo dispositivo BLE (acronimo per Bluetooth Low Energy, cioè bluethooth a basso consumo energetico) che trasmette piccole quantità di dati ogni secondo o quasi.  A livello di hardware, i beacon sono dispositivi BLE-compatible che trasmettono dati sotto il protocollo “beacon”.
A livello software, i beacon sono messaggi inviati da questi dispositivi trasmittenti, che sono poi rilevati e processati da un dispositivo ricevente, come un’app mobile. Nel caso degli iBeacon vengono trasmessi i valori Major e Minor che insieme all’UUID (acronimo di Universally Unique IDentifier), permettono l’identificazione di un iBeacon da parte dell’applicazione.(riferirsi a https://it.wikipedia.org/wiki/IBeacon e http://www.beaconitaly.it/cose-ibeacon/);

\textbf{Brainstorming:} Metodo decisionale in cui la ricerca della soluzione di un problema è effettuata mediante sedute intensive di dibattito e confronto delle idee e delle proposte espresse dai partecipanti.

\summLett{C}

\summLett{D}

\textbf{Dizionario (o Array Associativo):} Struttura dati che contiene una collezione di oggetti (o records).
Questi oggetti contengono diversi campi, ognuno dei quali contiene dei dati.
E' possibile accedere, gestire e salvare i diversi records tramite una chiave che li identifica in maniera univoca.

\textbf{Domain Specific Language (DSL):} E' un linguaggio di programmazione o un linguaggio di specifica dedicato a particolari problemi di un dominio, a una particolare tecnica di rappresentazione e/o a una particolare soluzione tecnica.

\summLett{E}

\summLett{F}

\summLett{G}

\summLett{H}

\summLett{I}

\textbf{Instant messaging: } E' una categoria di sistemi di comunicazione in tempo reale in rete, tipicamente Internet o una rete locale, che permette ai suoi utilizzatori lo scambio di brevi messaggi.

\textbf{IEC (International Electrotechnical Commission ):} E' un'organizzazione internazionale per la definizione di standard in materia di elettricità, elettronica e tecnologie correlate, fondata a Londra nel 1906.

\textbf{ISO (International Organization for Standardization):} E' la più importante organizzazione a livello mondiale per la definizione di norme tecniche fondata nel 1947 a Ginevra, in Svizzera.
L'ISO coopera strettamente con l'IEC (vedi sopra), responsabile per la standardizzazione degli equipaggiamenti elettrici.
Le norme ISO sono numerate e hanno un formato del tipo ISO nnnn:yyyy - titolo, dove nnnn è il numero della norma, yyyy l'anno di pubblicazione e titolo è una breve descrizione della norma.

\summLett{J}

\summLett{K}

\textbf{Key-value Database:} E' un paradigma di archivazione di dati progettato per archiviare, recuperare e gestire array associativi (vedi "Dizionario").

\summLett{L}

\summLett{M}

\textbf{Mailing List:} E' un servizio/strumento offribile da una rete di computer verso vari utenti e costituito da un sistema organizzato per la partecipazione di più persone ad una discussione asincrona o per la distribuzione di informazioni utili agli interessati/iscritti attraverso l'invio di email ad una lista di indirizzi di posta elettronica di utenti iscritti.

\textbf{Modello ad attori:} vedi "Actormodel".

\textbf{Modello relazionale:} modello logico di rappresentazione o strutturazione dei dati di un database implementato su sistemi di 'gestione di basi di dati' (DBMS).
Si basa sulla teoria degli insiemi e sulla logica del primo ordine ed è strutturato intorno al concetto matematico di relazione (detta anche tabella). Per il suo trattamento ci si avvale di strumenti quali il calcolo relazionale e l'algebra relazionale.

\summLett{N}

\textbf{NoSQL:} E' un movimento che promuove sistemi software dove la persistenza dei dati è caratterizzata dal fatto di non utilizzare il modello relazionale (vedi sopra), solitamente utilizzato dai database tradizionali.
L'espressione NoSQL fa riferimento al linguaggio SQL, il più comune tra i linguaggi per l'interrogazione dei database relazionali, qui preso a simbolo dell'intero paradigma relazionale. Molto spesso nei linguaggi NoSQL viene sacrificato in favore dele prestazioni il paradigma ACID (atomicità, consistenza, isolamento, durabilità) in quanto, per il teorema CAP, noto anche come teorema di Brewer, è impossibile per un sistema informatico distribuito, fornire simultaneamente tutte e tre le seguenti garanzie:
Coerenza (tutti i nodi vedono gli stessi dati nello stesso momento)
Disponibilità (la garanzia che ogni richiesta riceva una risposta su ciò che sia riuscito o fallito)
Tolleranza di partizione (il sistema continua a funzionare nonostante arbitrarie perdite di messaggi)
Secondo il teorema, un sistema distribuito è in grado di soddisfare al massimo due di queste garanzie allo stesso tempo, ma non tutte e tre.

\summLett{O}

\textbf{Organigramma:} In un'organizzazione aziendale, con il termine organigramma si intende una formalizzazione di quali sono gli organi e le loro relazioni gerarchiche o funzionali all'interno di un smbiente di lavoro.

\summLett{P}

\textbf{Prototipo:} Il modello realizzato nell’ultima fase della progettazione e sperimentazione, e destinato a divenire il punto di partenza della produzione in serie

\summLett{Q}

\summLett{R}

\textbf{Repository (Software Repository):} E' uno spazio di archiviazione da dove è possibile recuperare e scaricare dei pacchetti software.

\summLett{S}

\textbf{Scala:} E' un linguaggio di programmazione di tipo general-purpose studiato per integrare le caratteristiche e funzionalità dei linguaggi orientati agli oggetti e dei linguaggi funzionali. La compilazione di codice sorgente Scala produce Java bytecode per l'esecuzione su una JVM.

\summLett{T}

\textbf{Template:} Trattasi di un modello che ha il compito di fornire la struttura di un oggetto (documento, sito web, ...) in cui si trovano spazi temporaneamente "bianchi" da riempire successivamente.

\summLett{U}

\summLett{V}

\textbf{VoIP (Voice over IP):} tecnologia che rende possibile effettuare una conversazione telefonica sfruttando una connessione Internet o una qualsiasi altra rete dedicata a commutazione di pacchetto che utilizzi il protocollo IP senza connessione per il trasporto dati.

\summLett{W}

\textbf{Whiteboarding:} E' una tecnica di condivisione di un notebook o di una whiteboard (Lavagna) virtuali nei quali più persone possono lavorare contemporaneamente per scambiarsi idee, bozze di grafici e schemi.
Tutto questo viene fatto online così da permettere ai singoli utenti di vedere in tempo reale le modifiche fatte al documento.

\summLett{X}

\summLett{Y}

\summLett{Z}

\end{document}
