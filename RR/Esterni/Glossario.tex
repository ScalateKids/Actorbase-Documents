\documentclass{scalatekids-article}
\begin{document}
\lhead{ScalateKids}
\rhead{Glossario \thepage}
\lfoot{Glossario 0.0.1}
\begin{titlepage}
  \centering
  \includegraphics[width=0.45\textwidth]{logo.png}\par\vspace{1cm}
  \vspace{1.5cm}
         {\Huge\bfseries Glossario \par}
         \begin{center}
           \vspace{1.0cm}
                  {\large\bfseries Informazioni sul documento \par}
         \end{center}
         \vspace{-1cm}
         \begin{center}
           \line(1,0){300}
         \end{center}
         \vspace{0cm}
         \begin{tabular}[c]{l|l}
           \textbf{Versione} & 0.0.1\\
           \textbf{Redazione} & Redattore1\\ & Redattore2\\ & Redattore3..\\
           \textbf{Verifica} & Verificatore1\\ & Verificatore2..\\
           \textbf{Responsabile} & Responsabile\\
           \textbf{Uso} & Esterno
         \end{tabular}
\end{titlepage}
\clearpage
\setcounter{page}{1}
\begin{flushleft}
  \vspace{0cm}
         {\large\bfseries Diario delle modifiche \par}
\end{flushleft}
\vspace{0cm}
\begin{center}
  \begin{tabular}{|l | l | l | l |}
    \hline
    Versione & Autore & Data & Descrizione \\
    \hline
    0.0.1 & Marco Boseggia & 2015-12-22 & Creazione scheletro del documento\\
    \hline
  \end{tabular}
\end{center}
\tableofcontents
\newpage
\section{Introduzione}
\subsection{Scopo del documento}
Questo documento ha il compito di raccogliere tutti i termini che possono risultare sconosciuti o di difficile comprensione ad un lettore esterno oppure, semplicemente, concetti specifici che necessitano di essere esposti per poter meglio comprendere quanto viene detto nei documenti.
\subsection{Riferimenti} //da fare
\newpage
\section{Definizioni}
\textbf{Actormodel:} Modello matematico per la gestione della concorrenza computazionale proposto nel 1973 da Carl Hewitt.
All'interno del modello, gli "attori" vengono trattati come primitive universali e possono scmabiarsi messaggi tra di loro; in risposta ad un messaggio ricevuto, un attore può prendere decisioni, creare nuovi attori, inviare altri messaggi e determinare come rispondere al messaggio successivo.

\textbf{Amazon DynamoDB:} Database proprietario di tipo NoSQL e completamente gestito da Amazon.com all'interno del suo portfolio 'Amazon Web Services'.

\textbf{Beacon:} il beacon è un piccolo dispositivo BLE (acronimo per Bluetooth Low Energy, cioè bluethooth a basso consumo energetico) che trasmette piccole quantità di dati ogni secondo o quasi.  A livello di hardware, i beacon sono dispositivi BLE-compatible che trasmettono dati sotto il protocollo “beacon”.
A livello software, i beacon sono messaggi inviati da questi dispositivi trasmittenti, che sono poi rilevati e processati da un dispositivo ricevente, come un’app mobile. Nel caso degli iBeacon vengono trasmessi i valori Major e Minor che insieme all’UUID (acronimo di Universally Unique IDentifier), permettono l’identificazione di un iBeacon da parte dell’applicazione.(riferirsi a https://it.wikipedia.org/wiki/IBeacon e http://www.beaconitaly.it/cose-ibeacon/);

\textbf{Dizionario (o Array Associativo):} Struttura dati che contiene una collezione di oggetti (o records).
Questi oggetti contengono diversi campi, ognuno dei quali contiene dei dati.
E' possibile accedere, gestire e salvare i diversi records tramite una chiave che li identifica in maniera univoca.

\textbf{Domain Specific Language (DSL):} E' un linguaggio di programmazione o un linguaggio di specifica dedicato a particolari problemi di un dominio, a una particolare tecnica di rappresentazione e/o a una particolare soluzione tecnica.

\textbf{Key-value Database:} E' un paradigma di archivazione di dati progettato per archiviare, recuperare e gestire array associativi (vedi "Dizionario").

\textbf{Modello ad attori:} vedi "Actormodel".

\textbf{Modello relazionale:} modello logico di rappresentazione o strutturazione dei dati di un database implementato su sistemi di 'gestione di basi di dati' (DBMS).
Si basa sulla teoria degli insiemi e sulla logica del primo ordine ed è strutturato intorno al concetto matematico di relazione (detta anche tabella). Per il suo trattamento ci si avvale di strumenti quali il calcolo relazionale e l'algebra relazionale.

\textbf{NoSQL:} E' un movimento che promuove sistemi software dove la persistenza dei dati è caratterizzata dal fatto di non utilizzare il modello relazionale (vedi sopra), solitamente utilizzato dai database tradizionali.
L'espressione NoSQL fa riferimento al linguaggio SQL, il più comune tra i linguaggi per l'interrogazione dei database relazionali, qui preso a simbolo dell'intero paradigma relazionale. Molto spesso nei linguaggi NoSQL viene sacrificato in favore dele prestazioni il paradigma ACID (atomicità, consistenza, isolamento, durabilità) in quanto, per il teorema CAP, noto anche come teorema di Brewer, è impossibile per un sistema informatico distribuito, fornire simultaneamente tutte e tre le seguenti garanzie:
Coerenza (tutti i nodi vedono gli stessi dati nello stesso momento)
Disponibilità (la garanzia che ogni richiesta riceva una risposta su ciò che sia riuscito o fallito)
Tolleranza di partizione (il sistema continua a funzionare nonostante arbitrarie perdite di messaggi)
Secondo il teorema, un sistema distribuito è in grado di soddisfare al massimo due di queste garanzie allo stesso tempo, ma non tutte e tre.
\end{document}