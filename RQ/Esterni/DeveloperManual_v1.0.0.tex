\documentclass{scalatekids-article}
\usepackage[official]{eurosym}
\begin{document}
\lfoot{Developer Manual 1.0.0}
\newgeometry{top=3.5cm}
\begin{titlepage}
  \begin{center}
    \begin{center}
      \includegraphics[width=10cm]{sklogo.png}
    \end{center}
    \vspace{1cm}
    \begin{Huge}
      \begin{center}
        \textbf{Developer Manual}
      \end{center}
    \end{Huge}
    \vspace{11pt}
    \bgroup
    \def\arraystretch{1.3}
    \begin{tabular}{r|l}
      \multicolumn{2}{c}{\textbf{Informazioni sul documento}} \\
      \hline
      \setbox0=\hbox{0.0.1\unskip}\ifdim\wd0=0pt
      \\
      \else
      \textbf{Versione} & 1.0.0\\
      \fi
      \textbf{Editing} & \multiLineCell[t]{\\}\\
      \textbf{Verification} & \multiLineCell[t]{\\}\\
      \textbf{Approvation} & \multiLineCell[t]{}\\
      \textbf{Use} & External\\
      \textbf{Distribution List} & \multiLineCell[t]{ScalateKids\\Prof. Tullio Vardanega\\Prof. Riccardo Cardin}\\
    \end{tabular}
    \egroup
    \vspace{22pt}
  \end{center}
\end{titlepage}
\restoregeometry
\clearpage
\pagenumbering{Roman}
\setcounter{page}{1}
\begin{flushleft}
  \vspace{0cm}
         {\large\bfseries History log}
\end{flushleft}
\vspace{0cm}
\begin{center}
  \begin{tabular}{| l | l | l | l | p{5cm} |}
    \hline
    Version & Author & Role & Date & Description \\
  \end{tabular}
\end{center}
\tableofcontents
\newpage
\pagenumbering{arabic}
\section{Summary}
\subsection{Document purpose}
This document aim to describe the procedures and methods to use in order to take advantage of \gloss{Actorbase}.

\subsection{References}

\subsubsection{Normative} % TODO check, translated with wordreference

\subsubsection{Informational} % TODO check, translated with wordreference

% TODO parte server: installazione, configurazione, accensione, spegnimento

% TODO mettere una sezione in cui si spiega cosa è una collection, un item, tutto quel che bisogna descrivere

% TODO bisogna spiegare come devnoon essere formati i file pel l'import

%%%%%%%%%%%%%%%%%%%%%%%%%%%%%%%%%%%%%%%%%%%%%%%%%%%%%%%%%%%%%%%%%%%%
%%%							  INSTALLATION PART					 %%%
%%%%%%%%%%%%%%%%%%%%%%%%%%%%%%%%%%%%%%%%%%%%%%%%%%%%%%%%%%%%%%%%%%%%


%%%%%%%%%%%%%%%%%%%%%%%%%%%%%%%%%%%%%%%%%%%%%%%%%%%%%%%%%%%%%%%%%%%%
%%%%%%%%%%%%                  MACHETE                   %%%%%%%%%%%%
%%%%%%%%%%%%%%%%%%%%%%%%%%%%%%%%%%%%%%%%%%%%%%%%%%%%%%%%%%%%%%%%%%%%


%%%%%%%%%%%%%%%%%%%%%%%%%%%%%%%%%%%%%%%%%%%%%%%%%%%%%%%%%%%%%%%%%%%%
%%%							  DRIVER PART  						 %%%
%%%%%%%%%%%%%%%%%%%%%%%%%%%%%%%%%%%%%%%%%%%%%%%%%%%%%%%%%%%%%%%%%%%%

\section{Actorbase usage}

\subsection{ActorbaseDriver: Using Actorbase from inside a Scala source}

Actorbase is essentially an actor system built upon a server component that
exposes some API forming a RESTful web services. To facilitate the use from
inside a Scala source, it has been developed a driver in order to allow the
usage of all features that Actorbase provides.\\
The most common operations include:
\begin{itemize}
\item Authentication;
\item Profile management operations;
\item Collections operations;
\item Items operations;
\item Administrative operations.
\end{itemize}

\end{document}