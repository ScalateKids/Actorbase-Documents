\documentclass{scalatekids-article}
\usepackage[official]{eurosym}
\usepackage[italian]{babel}
\begin{document}
\lfoot{User Manual 1.0.0}
\newgeometry{top=3.5cm}
\begin{titlepage}
  \begin{center}
    \begin{center}
      \includegraphics[width=10cm]{sklogo.png}
    \end{center}
    \vspace{1cm}
    \begin{Huge}
      \begin{center}
        \textbf{User Manual}
      \end{center}
    \end{Huge}
    \vspace{11pt}
    \bgroup
    \def\arraystretch{1.3}
    \begin{tabular}{r|l}
      \multicolumn{2}{c}{\textbf{Document Information}} \\
      \hline
      \setbox0=\hbox{0.0.1\unskip}\ifdim\wd0=0pt
      \\
      \else
      \textbf{Version} & 1.0.0\\
      \fi
      \textbf{Editing} & \multiLineCell[t]{}\\
      \textbf{Verification} & \multiLineCell[t]{}\\
      \textbf{Approvation} & \multiLineCell[t]{Giacomo Vanin}\\
      \textbf{Use} & External\\
      \textbf{Distribution list} & \multiLineCell[t]{ScalateKids\\Prof. Tullio Vardanega\\Prof. Riccardo Cardin}\\
    \end{tabular}
    \egroup
    \vspace{22pt}
  \end{center}
\end{titlepage}
\restoregeometry
\clearpage
\pagenumbering{Roman}
\setcounter{page}{1}
\begin{flushleft}
  \vspace{0cm}
         {\large\bfseries History log}
\end{flushleft}
\vspace{0cm}
\begin{center}
  \begin{longtable}{| l | l | l | l | p{5cm} |}
  	\hline
    Version & Author & Role & Date & Description \\
    \hline
    0.0.2 & Alberto De Agostini & Programmer & 2016-05-04 & Added CLI section\\
    \hline
    0.0.1 & Alberto De Agostini & Programmer & 2016-05-04 & Created document structure\\
    \hline
  \end{longtable}
\end{center}
\tableofcontents
\newpage
\pagenumbering{arabic}
\section{Summary}
\subsection{Document purpose}
This document represents the user manual for the \textit{actorbase} application and describes in detail every single feature of it.
Actorbase is a database that comes with two different software modules capable of interacting with it; the user can use both these modules, so this document will present them in two different sections.

\prodPurpose

\subsection{References}

\subsubsection{Regular} % TODO non so tradurre

\subsubsection{Information} % TODO non so tradurre

% TODO parte server: installazione, configurazione, accensione, spegnimento

% TODO mettere una sezione in cui si spiega cosa è una collection, un item, tutto quel che bisogna descrivere

% TODO bisogna spiegare come devnoon essere formati i file pel l'import

%%%%%%%%%%%%%%%%%%%%%%%%%%%%%%%%%%%%%%%%%%%%%%%%%%%%%%%%%%%%%%%%%%%%
%%%							  CLI PART  						 %%%
%%%%%%%%%%%%%%%%%%%%%%%%%%%%%%%%%%%%%%%%%%%%%%%%%%%%%%%%%%%%%%%%%%%%
\section{Actorbase by Command Line Interface}

\subsection{DSL}

%TODO scrivere qualcosa sul dsl

%TODO spiegare anche che nei capitoli prossimi dove c'è scritto write login <username> si intende dire login e il tuo username

%TODO scrivere in caso di errore cosa succede.. tipo errori di typo e connessione

%TODO scrievere anche i tipi di dato possibili da inserire? tipo le stringhe devono essere in " ", permission su addcontributor può essere vuoto, r o rw ecc

\subsection{Configuration of the actorbase CLI}
\label{sec:configurationcli}
The \textit{actorbase} CLI is designed specifically to operate 
with the actorbase server software.\\
To work properly an actorbase server instance must be runnning 
(anywhere) and the user must configurate the access point on the configuration file.
%TODO configurazione cli? da fare 

\subsection{Operations}

The following sections requires the actorbase CLI to be configured and 
running (see \hyperref[sec:configurationcli]{Configure CLI}).\\
The features described are divided by the type of the user.

\subsubsection{Standard user operations}
\label{sec:everyuser}
\paragraph{Exit from CLI}

\textbf{Description:} every user can exit from the actorbase CLI 
using this procedure.\\
\textbf{How to:} to exit from the actorbase CLI the user must simply type \texttt{exit} on the CLI and 
press Enter.

\subsubsection{Unauthenticated user}
\label{sec:unauthenticateduser}

\paragraph{Login}

\textbf{Description:} this is the procedure an unauthenticated user must follow to authenticate.\\
\textbf{Requirements:} this feature requires the user to have the credentials 
to authenticate in the actorbase server.\\
Otherwise the user must contact an actorbase server 
administrator to obtain one.\\
\textbf{How to:} 
\begin{itemize}
	\item Type \texttt{login <your username>}
	\item The CLI will ask you to insert your password (it won't be	displayed)
	\item Type \texttt{<your password>} 
	\item The CLI will display a message notifing the user if the login was successfull or not
\end{itemize}
\textbf{Possible errors:} the CLI will answer with self explanatory error messages
\begin{itemize}
	\item If the credentials you entered are not recognized by the server
\end{itemize}

\paragraph{Help - generic help}
\label{sec:generichelp}
\textbf{Description:} this feature allows the user to ask for help.\\
The CLI will answer with a list of all the possible operations alongside
their right syntax.\\
The list of functionalities differs with the user type (e.g. authenticated user has more features than an unauthenticated one).\\
\textbf{How to:} 
\begin{itemize}
	\item Type \texttt{help} 
	\item The CLI will display a help message
\end{itemize}

\subsubsection{Authenticated user}
\label{sec:authenticateduser}

\paragraph{Help - generic help}

See hyperref[sec:generichelp]{Generic help}

\paragraph{Help - specific help}
\label{sec:specifichelp}
\textbf{Description:} this feature allows the authenticated user to ask for help for 
a specific command.\\
The CLI will answer with a message explaining how to use that command and 
which parameters it needs.\\
\textbf{How to:} 
\begin{itemize}
	\item Type \texttt{help <command name>}
	\item The CLI will display a message explaining how to use the command and his syntax
\end{itemize}
\textbf{Possible errors:} the CLI will answer with self explanatory error messages 
\begin{itemize}
	\item If the command name the user typed is not a command of the actorbase CLI
\end{itemize}

\paragraph{Create a collection}
\label{sec:createcollection}
\textbf{Description:} this feature allows the user to create a 
collection into the database.\\
\textbf{How to:} 
\begin{itemize}
	\item Type createCollection ``<collection name>''
	\item The CLI will display a message explaining how to use the command and what parameters it needs
\end{itemize}
\textbf{Possible errors:} the CLI will answer with self explanatory error messages 
\begin{itemize}
	\item If the collection you want to create is already inside the server the CLI will answer with a message
\end{itemize}

\paragraph{List collections}
\label{sec:listcollection}
\textbf{Description:} this functionality allow the user to see a list of 
the collections he has permission on.\\
\textbf{How to:} 
\begin{itemize}
	\item Type listCollection
	\item The CLI will display a list of all the collections the user has 
	permission on
\end{itemize}

\paragraph{Rename a collection}
\label{sec:renamecollection}
\textbf{Description:} this functionality allow the user to rename a 
collection of his own.\\
\textbf{How to:} 
\begin{itemize}
	\item Type renameCollection ``<old collection name>'' to ``<new collection name>''
\end{itemize}
\textbf{Possible errors:} the CLI will answer with self explanatory error messages 
\begin{itemize}
	\item If the old collection name you typed corrispond to no collection inside the actorbase server
	\item If the new collection name you typed is already taken by a collection inside the actorbase server 
\end{itemize}

\paragraph{Delete a collection}
\label{sec:deletecollection}
\textbf{Description:} this functionality allow the user to remove a collection and all of the items inside it from the database.\\
\textbf{How to:} 
\begin{itemize}
	\item Type deleteCollection ``<collection name>''
\end{itemize}

\paragraph{Add a contributor to a collection}
\label{sec:addcontributor}
\textbf{Description:} this functionality allow the user to add a contributor 
to a collection of his own. The contributor will have the permission to operate on the selected collection depending on the permission type the user inserted.\\
\textbf{How to:} 
\begin{itemize}
	\item Type addContributor <username> <permissions> to ``<collection name>''
\end{itemize}
\textbf{Possible errors:} the CLI will answer with self explanatory error messages 
\begin{itemize}
	\item If the collection you want to add the contributor on does not exists on the server or it's not a collection you own
	\item If the username you typed does not corrispond with a user registrated in actorbase
	\item If the permissions you typed are not a possible permission
\end{itemize}

\paragraph{Remove contributor from a collection}
\label{sec:removecontributor}
\textbf{Description:} this functionality allow the user to remove a contributor from a collection of his own.\\
\textbf{How to:} 
\begin{itemize}
	\item Type removeContributor <username> from ``<collection name>''
\end{itemize}
\textbf{Possible errors:} the CLI will answer with self explanatory error messages 
\begin{itemize}
	\item If the collection you want to remove the contributor from does not exists on the server or it's not a collection you own
	\item If the username you typed does not corrispond with a user registrated in actorbase
\end{itemize}

\paragraph{Export one or more collections to file}
\label{sec:export}
\textbf{Description:} this functionality allow the user to export to a file one or more collections you have access at.\\
The filename you insert can have have a path prefix, if path is not present the file will be created at your local work directory.\\
The collection list can contain zero or more collection name. If empty it will export ALL the collection the user have access (this can be slow).
\textbf{How to:} 
\begin{itemize}
	\item Type export ``<collection name1>'', ``<collection name2>'', ... , to ``<filename>''
\end{itemize}
\textbf{Possible errors:} the CLI will answer with self explanatory error messages 
\begin{itemize}
	\item If at least one of the collection name you inserted does not exist on the server
	\item If the filename you inserted contain a path that's not valid
\end{itemize}

\paragraph{Insert item}
\label{sec:insertitem}
\textbf{Description:} this functionality allow the user to insert a item in a collection.\\
The last parameter ( o ) can be omitted, if omitted the insert will be without overwrite, otherwise the insert will overwrite if another item is present with the same key in the collection.\\
The <key> is a String representing the key of the item you want to insert.\\
The <value> can be anything you want to insert and it represent the value of the item to be inserted.\\
If the collection you want to insert the item into does not exist the server 
will automaticly create that collection.\\
\textbf{How to:} 
\begin{itemize}
	\item Type insert ``<key>'' -> <value> to ``<collection name>'' o
\end{itemize}
\textbf{Possible errors:} the CLI will answer with self explanatory error messages 
\begin{itemize}
	\item If the collection you want to insert the item into does not exist or you don't have the permission to write on it
	\item If the key of the item you want to insert is already used in the collection and you didn't chose to overwrite  
\end{itemize}

\paragraph{Import from file}
\label{sec:import}
\textbf{Description:} this functionality allow the user to insert a list of collections and items from a file.\\
The file has to be formatted as specified in \\%TODO mettere link to sezione giusta
\textbf{How to:} 
\begin{itemize}
	\item Type import ``nomefile''
\end{itemize}
\textbf{Possible errors:} the CLI will answer with self explanatory error messages 
\begin{itemize}
	\item If the file is not JSON file.
	\item If the file is not formatted as specified in %TODO link a sezione
	\item If the file contains collection to be created that you cannot create (see \hyperref[sec:createcollection]{create collection})
	\item If the file contains items to be inserted that you cannot insert (see \hyperref[sec:insertitem]{insert item}) 
\end{itemize}

\paragraph{Remove item}
\label{sec:removeitem}
\textbf{Description:} this functionality allow the user to remove a item from a collection.\\
Key represent the key of the item you want to remove from the collection
\textbf{How to:} 
\begin{itemize}
	\item Type remove ``<key>'' from ``<collection name>''
\end{itemize}
\textbf{Possible errors:} the CLI will answer with self explanatory error messages 
\begin{itemize}
	\item If the collection you typed does not exists in the server or you don't have permission to remove items from that collection
\end{itemize}

\paragraph{Find item}
\label{sec:find}
\textbf{Description:} this functionality allow the user to search for items in the database. There are different types of search based on the parameters typed.\\
To make the different types of find more easy to understand they will be 
explained in differents sections below.

\subparagraph{Find item in one or more collections}
\textbf{Description:} this functionality let the user search for an item between one or more collections in which he has the permission to read.\\
\textbf{How to:} 
\begin{itemize}
	\item Type find ``<key>'' from ``<collection name 1>'', ``<collection name 2>'', ... , ``<collection name N>''
\end{itemize}
\textbf{Possible errors:} the CLI will answer with self explanatory error messages 
\begin{itemize}
	\item If at least one of the collection names does not exists on the server
\end{itemize}

\subparagraph{Find items in database}
\textbf{Description:} this functionality let the user search for an item in the entire database.\\
\textbf{How to:} 
\begin{itemize}
	\item Type find ``<key>''
\end{itemize}

\subparagraph{Get all items of one or more collections}
\textbf{Description:} this functionality let the user to ask for all the items contained in one or more collections.\\
\textbf{How to:} 
\begin{itemize}
	\item Type find from ``<collection name 1>'', ``<collection name 2>'', ... , ``<collection name N>''
\end{itemize}
\textbf{Possible errors:} the CLI will answer with self explanatory error messages 
\begin{itemize}
	\item If at least one of the collection names does not exists on the server
\end{itemize}

\subparagraph{Get all database items grouped by collection}
\textbf{Description:} this functionality let the user to ask for all the items contained in the database grouped by collection.\\
This operation can be expensive.\\
\textbf{How to:} 
\begin{itemize}
	\item Type find
\end{itemize}

% TODO finire comandi possibili da CLI (occhio a differenziare quelli dell'admin)






\subsubsection{Administrator user}
\label{sec:administratoruser}

%%%%%%%%%%%%%%%%%%%%%%%%%%%%%%%%%%%%%%%%%%%%%%%%%%%%%%%%%%%%%%%%%%%%
%%%							DRIVER PART 						 %%%
%%%%%%%%%%%%%%%%%%%%%%%%%%%%%%%%%%%%%%%%%%%%%%%%%%%%%%%%%%%%%%%%%%%%
\section{Use throught Driver}

\subsection{Introduction to Driver}

\listoftables

\listoffigures

\end{document}
