\documentclass{scalatekids-article}
\usepackage[official]{eurosym}
\usepackage[italian]{babel}
\begin{document}
\lfoot{User Manual 1.0.0}
\newgeometry{top=3.5cm}
\begin{titlepage}
  \begin{center}
    \begin{center}
      \includegraphics[width=10cm]{sklogo.png}
    \end{center}
    \vspace{1cm}
    \begin{Huge}
      \begin{center}
        \textbf{User Manual}
      \end{center}
    \end{Huge}
    \vspace{11pt}
    \bgroup
    \def\arraystretch{1.3}
    \begin{tabular}{r|l}
      \multicolumn{2}{c}{\textbf{Document Information}} \\
      \hline
      \setbox0=\hbox{0.0.1\unskip}\ifdim\wd0=0pt
      \\
      \else
      \textbf{Version} & 1.0.0\\
      \fi
      \textbf{Editing} & \multiLineCell[t]{}\\
      \textbf{Verification} & \multiLineCell[t]{}\\
      \textbf{Approvation} & \multiLineCell[t]{Giacomo Vanin}\\
      \textbf{Use} & External\\
      \textbf{Distribution list} & \multiLineCell[t]{ScalateKids\\Prof. Tullio Vardanega\\Prof. Riccardo Cardin}\\
    \end{tabular}
    \egroup
    \vspace{22pt}
  \end{center}
\end{titlepage}
\restoregeometry
\clearpage
\pagenumbering{Roman}
\setcounter{page}{1}
\begin{flushleft}
  \vspace{0cm}
         {\large\bfseries History log}
\end{flushleft}
\vspace{0cm}
\begin{center}
  \begin{longtable}{| l | l | l | l | p{5cm} |}
  	\hline
    Version & Author & Role & Date & Description \\
    \hline
    0.0.2 & Alberto De Agostini & Programmer & 2016-05-04 & Added CLI section\\
    \hline
    0.0.1 & Alberto De Agostini & Programmer & 2016-05-04 & Created document structure\\
    \hline
  \end{longtable}
\end{center}
\tableofcontents
\newpage
\pagenumbering{arabic}
\section{Summary}
\subsection{Document purpose}
This document represents the user manual for the \textit{actorbase} application and describes in detail every single feature of it.
Actorbase is a database that comes with two different software modules capable of interacting with it; the user can use both these modules, so this document will present them in two different sections.

\prodPurpose

\subsection{References}

\subsubsection{Normative} % TODO check, translated with wordreference

\subsubsection{Informational} % TODO check, translated with wordreference

% TODO parte server: installazione, configurazione, accensione, spegnimento

% TODO mettere una sezione in cui si spiega cosa è una collection, un item, tutto quel che bisogna descrivere

% TODO bisogna spiegare come devnoon essere formati i file pel l'import

%%%%%%%%%%%%%%%%%%%%%%%%%%%%%%%%%%%%%%%%%%%%%%%%%%%%%%%%%%%%%%%%%%%%
%%%							  CLI PART  						 %%%
%%%%%%%%%%%%%%%%%%%%%%%%%%%%%%%%%%%%%%%%%%%%%%%%%%%%%%%%%%%%%%%%%%%%
\section{Actorbase by Command Line Interface}

\subsection{DSL} %TODO vedete voi se aggiungerci qualcosa, best english in the document atm z.z

The DSL, upon which the \textit{actorbase} CLI is based, has been developed
trying to emulate a SQL language. By doing so we achieved an easily
understandable language that's close to spoken English.

%TODO scrivere in caso di errore cosa succede.. tipo errori di typo e connessione, best english in the document atm z.z
\subsection{Error management}

There are two kinds of errors that are shared by every command which will be
now explained:
\begin{itemize}
\item In case of an unrecognized command, or in case of a typographical error,
	the CLI will show an unrecognized command error followed by a list of all
	the commands available to the user;
\item In case of a connection error, which can be caused by not receiving a
	response from the server in a predetermined time, the CLI will show a
	timeout connection error.
\end{itemize}
There are some errors that are specific to some commands, these will be explained
in the section dedicated to the commands that cause them.

%TODO scrivere anche i tipi di dato possibili da inserire? tipo le stringhe devono essere in " ", permission su addcontributor può essere vuoto, r o rw ecc

%TODO scrivere i requisiti delle password di actorbase

\subsection{Configuration of the actorbase CLI}
\label{sec:configurationcli}
The \textit{actorbase} CLI is designed specifically to operate 
with the actorbase server software.\\
To work properly an actorbase server instance must be running 
(anywhere) and the user must configurate the access point on the configuration file.
%TODO configurazione cli? da fare 

\subsection{Operations}

The following sections requires the actorbase CLI to be configured and 
running (see \hyperref[sec:configurationcli]{Configure CLI}).\\
The features described are divided by the type of the user.

\subsubsection{Standard user operations}
\label{sec:everyuser}
\paragraph{Exit from CLI}

\textbf{Description:} every user can exit from the actorbase CLI 
using this procedure.\\
\textbf{How to:} type \texttt{exit} on the CLI

\subsubsection{Unauthenticated user}
\label{sec:unauthenticateduser}

\paragraph{Login}

\textbf{Description:} this is the procedure an unauthenticated user must follow to authenticate.\\
\textbf{Requirements:} this feature requires the user to have the credentials 
to authenticate in the actorbase server.\\
Otherwise the user must contact an actorbase server 
administrator to obtain one.\\
\textbf{How to:} 
\begin{itemize}
	\item Type \texttt{login <your username>}
	\item The CLI will ask you to insert your password (it will be masked during the typing)
	\item Type \texttt{<your password>} 
	\item The CLI will display a message notifing the user if the login was successfull or not
\end{itemize}
\textbf{Possible errors:} the CLI will answer with self explanatory error messages
\begin{itemize}
	\item If the credentials you entered are not recognized by the server
\end{itemize}

\paragraph{Help - generic help}
\label{sec:generichelp}
\textbf{Description:} this feature allows the user to ask for help.\\
The CLI will answer with a list of all the possible operations alongside
their right syntax.\\
The list of functionalities differs with the user type (e.g. authenticated user has more features than an unauthenticated one).\\
\textbf{How to:} 
\begin{itemize}
	\item Type \texttt{help} 
	\item The CLI will display a help message
\end{itemize}

\subsubsection{Authenticated user}
\label{sec:authenticateduser}

\paragraph{Help - generic help}

See hyperref[sec:generichelp]{Generic help}

\paragraph{Help - specific help}
\label{sec:specifichelp}
\textbf{Description:} this feature allows the authenticated user to ask for help for 
a specific command.\\
The CLI will answer with a message explaining how to use that command and 
which parameters it needs.\\
\textbf{How to:} 
\begin{itemize}
	\item Type \texttt{help <command name>}
	\item The CLI will display a message explaining how to use the command and his syntax
\end{itemize}
\textbf{Possible errors:} the CLI will answer with self explanatory error messages 
\begin{itemize}
	\item If the command name the user typed is not a command of the actorbase CLI
\end{itemize}

\paragraph{Create a collection}
\label{sec:createcollection}
\textbf{Description:} this feature allows the user to create a 
collection into the database.\\
\textbf{How to:} 
\begin{itemize}
	\item Type \texttt{createCollection <collection name>}
	\item The CLI will display a message explaining how to use the command and what parameters it needs
\end{itemize}
\textbf{Possible errors:} the CLI will answer with self explanatory error messages 
\begin{itemize}
	\item If the collection the user wants to create is already in the server the CLI will answer with a message
\end{itemize}

\paragraph{List collections}
\label{sec:listcollection}
\textbf{Description:} this feature allows the user to see a list of 
the collections he has permission on.\\
\textbf{How to:} 
\begin{itemize}
	\item Type \texttt{listCollection}
	\item The CLI will display a list of all the collections the user has 
	permission on
\end{itemize}

\paragraph{Rename a collection}
\label{sec:renamecollection}
\textbf{Description:} this feature allows the user to rename a 
collection of his own.\\
\textbf{How to:} 
\begin{itemize}
	\item Type \texttt{renameCollection <old collection name> to <new collection name>}
\end{itemize}
\textbf{Possible errors:} the CLI will answer with self explanatory error messages 
\begin{itemize}
	\item If the old collection name the user typed correspond with no collection inside the actorbase server
	\item If the new collection name the user typed is already taken by a collection inside the actorbase server 
\end{itemize}

\paragraph{Delete a collection}
\label{sec:deletecollection}
\textbf{Description:} this feature allows the user to remove a collection and all the items inside it from the database.\\
\textbf{How to:} 
\begin{itemize}
	\item Type \texttt{deleteCollection <collection name>}
\end{itemize}

\paragraph{Add a contributor to a collection}
\label{sec:addcontributor}
\textbf{Description:} this feature allows the user to add a contributor 
to a collection of his own. The contributor will have the permission to operate on the selected collection depending on the what permissions the user decide to give him.\\
\textbf{How to:} 
\begin{itemize}
	\item Type \texttt{addContributor <username> <permissions> to <collection name>}
\end{itemize}
\textbf{Possible errors:} the CLI will answer with self explanatory error messages 
\begin{itemize}
	\item If the collection the user wants to add the contributor to does not exists on the server or it is not a collection the user own
	\item If the username the user typed does not correspond with a user registered in actorbase
	\item If the permissions the user typed are not possible permissions
\end{itemize}

\paragraph{Remove contributor from a collection}
\label{sec:removecontributor}
\textbf{Description:} this feature allows the user to remove a contributor from a collection of his own.\\
\textbf{How to:} 
\begin{itemize}
	\item Type \texttt{removeContributor <username> from <collection name>}
\end{itemize}
\textbf{Possible errors:} the CLI will answer with self explanatory error messages 
\begin{itemize}
	\item If the collection the user wants to remove the contributor from does not exists on the server or it is not a collection the user owns
	\item If the username the user typed does not correspond with a user registered in actorbase
\end{itemize}

\paragraph{Export one or more collections to file}
\label{sec:export}
\textbf{Description:} this feature allows the user to export one or more collections he has access to.\\
The filename the user insert can be a relative or absolute path; if the path is not present, the file will be created at the user local work directory.\\
The collection list can contain zero or more collections' name, if empty it will export all the collections the user has access to (possibly resulting in an expensive operation).
\textbf{How to:} 
\begin{itemize}
	\item Type \texttt{export <collection name1>, <collection name2>, ... , to <filename>}
\end{itemize}
\textbf{Possible errors:} the CLI will answer with self explanatory error messages 
\begin{itemize}
	\item If at least one of the collections' name the user inserted does not exist on the server
	\item If the filename the user inserted contains a path that is not valid
\end{itemize}

\paragraph{Insert item}
\label{sec:insertitem}
\textbf{Description:} this feature allows the user to insert an item in a particular collection.\\
The last parameter ( o ) can be omitted; if omitted the insert operation will not overwrite existing items with the same key. It will overwrite them otherwise. \\
The <key> is a String representing the key of the item the user wants to insert.\\
The <value> can be anything the user wants to insert and it represent the value of the item itself.\\
If the user wants to insert the item in a non existing collection the server 
will automatically create that collection.\\
\textbf{How to:} 
\begin{itemize}
	\item Type \texttt{insert <key> -> <value> to <collection name> o} %TODO WHAT?
\end{itemize}
\textbf{Possible errors:} the CLI will answer with self explanatory error messages 
\begin{itemize}
	\item If the collection the user wants to insert the item in does not exist or user has not permission to write on it
	\item If the user decides to insert an item with an already existing key in that collection and the overwrite option has not been chosen.  
\end{itemize}

\paragraph{Import from file}
\label{sec:import}
\textbf{Description:} this feature allows the user to insert a list of collections and items from a file.\\
The file has to be formatted as specified in \\%TODO mettere link to sezione giusta
\textbf{How to:} 
\begin{itemize}
	\item Type \texttt{import <nomefile>}
\end{itemize}
\textbf{Possible errors:} the CLI will answer with self explanatory error messages 
\begin{itemize}
	\item If the file is not a JSON file
	\item If the file is not formatted as specified in %TODO link a sezione
	\item If the file specified the creation of collection that the user can not create (see \hyperref[sec:createcollection]{create collection})
	\item If the file specified the insertion of items that the user can not insert (see \hyperref[sec:insertitem]{insert item}) 
\end{itemize}

\paragraph{Remove item}
\label{sec:removeitem}
\textbf{Description:} this feature allows the user to remove an item from a collection.\\
<key> represents the key of the item the user wants to remove from the collection
\textbf{How to:} 
\begin{itemize}
	\item Type \texttt{remove <key> from <collection name>}
\end{itemize}
\textbf{Possible errors:} the CLI will answer with self explanatory error messages 
\begin{itemize}
	\item If the collection the user typed does not exist in the server or if the user does not have permission to remove items from
\end{itemize}

\paragraph{Find item}
\label{sec:find}
\textbf{Description:} this feature allows the user to search for particular items in the database. There are different types of search based on the parameters typed.\\

\subparagraph{Search items in one or more collections}
\textbf{Description:} this feature lets the user search for an item in one or more collections; the user must have the permission to read these collections.\\
\textbf{How to:} 
\begin{itemize}
	\item Type \texttt{find <key> from <collection name 1>, <collection name 2>, ... , <collection name N>}
\end{itemize}
\textbf{Possible errors:} the CLI will answer with self explanatory error messages 
\begin{itemize}
	\item If at least one of the collections does not exist on the server
\end{itemize}

\subparagraph{Search items in the entire database}
\textbf{Description:} this feature lets the user search for an item in the entire database visible to him.\\
\textbf{How to:} 
\begin{itemize}
	\item Type \texttt{find <key>}
\end{itemize}

\subparagraph{Search for all the items of one or more collections}
\textbf{Description:} this feature lets the user retrieve all the items contained in one or more collections.\\
\textbf{How to:} 
\begin{itemize}
	\item Type \texttt{find from <collection name 1>, <collection name 2>, ... , <collection name N>}
\end{itemize}
\textbf{Possible errors:} the CLI will answer with self explanatory error messages 
\begin{itemize}
	\item If at least one of the collection does not exist on the server or if the user doesn't have access to it
\end{itemize}

\subparagraph{Get all database items grouped by collection}
\textbf{Description:} this feature lets the user to retrieve all the items contained in the section of the database that is visible to him grouped by collection.\\
This operation could be expensive.\\
\textbf{How to:} 
\begin{itemize}
	\item Type \texttt{find}
\end{itemize}

\paragraph{Change password}
\label{sec:changepassword}
\textbf{Description:} this feature allows the user to change his password.\\
\textbf{How to:} 
\begin{itemize}
	\item Type \texttt{changePassword ``<old password>'' ``<new password>''}
\end{itemize}
\textbf{Possible errors:} the CLI will answer with self explanatory error messages 
\begin{itemize}
	\item If the old password typed was wrong
	\item If the new password typed doesn't follow the rules %TODO linkare rules
\end{itemize}

\paragraph{Logout}
\label{sec:logout}
\textbf{Description:} this feature allows the user to logout from the server becoming an unauthenticated user.\\
\textbf{How to:} 
\begin{itemize}
	\item Type \texttt{logout}
\end{itemize}

\subsubsection{Administrator user}
\label{sec:administratoruser}



%%%%%%%%%%%%%%%%%%%%%%%%%%%%%%%%%%%%%%%%%%%%%%%%%%%%%%%%%%%%%%%%%%%%
%%%							DRIVER PART 						 %%%
%%%%%%%%%%%%%%%%%%%%%%%%%%%%%%%%%%%%%%%%%%%%%%%%%%%%%%%%%%%%%%%%%%%%
\section{Use through Driver}

\subsection{Introduction to Driver}

\listoftables

\listoffigures

\end{document}