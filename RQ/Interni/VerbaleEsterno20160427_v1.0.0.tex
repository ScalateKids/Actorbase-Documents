\documentclass{scalatekids-article}
\def\changemargin#1#2{\list{}{\rightmargin#2\leftmargin#1}\item[]}
\let\endchangemargin=\endlist
\begin{document}
\lfoot{Verbale esterno 2016-04-27 1.0.0}
\newgeometry{top=3.5cm}
\begin{titlepage}
  \begin{center}
    \begin{center}
      \includegraphics[width=10cm]{sklogo.png}
    \end{center}
    \vspace{1cm}
    \begin{Huge}
      \begin{center}
        \textbf{Verbale Esterno 2016-04-27}
      \end{center}
    \end{Huge}
    \vspace{11pt}
    \bgroup
    \def\arraystretch{1.3}
    \begin{tabular}{r|l}
      \multicolumn{2}{c}{\textbf{Informazioni sul documento}} \\
      \hline
      \setbox0=\hbox{0.0.1\unskip}\ifdim\wd0=0pt
      \\
      \else
      \textbf{Versione} & 1.0.0\\
      \fi
      \textbf{Redazione} & \multiLineCell[t]{Marco Boseggia}\\
      \textbf{Verifica} & \multiLineCell[t]{Andrea Giacomo Baldan}\\
      \textbf{Approvazione} & \multiLineCell[t]{Michael Munaro}\\
      \textbf{Uso} & Interno\\
      \textbf{Lista di Distribuzione} & \multiLineCell[t]{ScalateKids\\Prof. Tullio Vardanega\\Prof. Riccardo Cardin}\\
    \end{tabular}
    \egroup
    \vspace{22pt}
  \end{center}
\end{titlepage}
\restoregeometry
\clearpage
\pagenumbering{arabic}
\setcounter{page}{1}
\begin{flushleft}
  \vspace{0cm}
         {\large\bfseries Diario delle modifiche \par}
\end{flushleft}
\vspace{0cm}
\begin{center}
  \begin{tabular}{| l | l | l | l | l |}
    \hline
    Versione & Autore & Ruolo & Data & Descrizione \\
    \hline
    1.0.0 & Michael Munaro & Responsabile & 2016-04-28 & Validazione documento\\
    \hline
    0.1.0 & Andrea Giacomo Baldan & Verificatore & 2016-04-28 & Verifica documento\\
    \hline
    0.0.2 & Marco Boseggia & Progettista & 2016-04-27 & Prima stesura documento\\
    \hline
    0.0.1 & Marco Boseggia & Progettista & 2016-04-27 & Creazione scheletro del documento\\
    \hline
  \end{tabular}
\end{center}
\tableofcontents
\newpage
\section{Informazioni sulla riunione}
\begin{itemize}
\item \textbf{Data}: 2016-04-27
\item \textbf{Luogo}: Torre Archimede aula 1C150
\item \textbf{Orario d'inizio}: 13:00
\item \textbf{Durata}: 19'
\item \textbf{Partecipanti interni}: \textit{ScalateKids}
  \begin{itemize}
  \item Andrea Giacomo Baldan 
  \item Marco Boseggia
  \item Michael Munaro
  \item Francesco Agostini
  \item Giacomo Vanin
  \item Alberto De Agostini
  \end{itemize}
\item \textbf{Partecipanti esterni}:
  \begin{itemize}
  \item Riccardo Cardin
  \end{itemize}
\end{itemize}
\section{Domande e risposte}
In questo capitolo sono elencate le domande fatte dal gruppo \textit{ScalateKids} in grassetto e le risposte del proponente \textit{Riccardo Cardin} in corsivo.
\subsection{Il professor Vardanega ci ha fatto notare che implementare una connessione TCP per interfacciarsi con il server risulta essere antiquata. Preferisce anche lei
un'interfaccia RESTful?}
\begin{quote}
  \textit{Sì, mi sembra la soluzione migliore. Per quanto riguarda il driver vi consiglio di dare un'occhiata al driver JDBC e, se non vi causa troppi cambiamenti, fate un driver simile a quest'ultimo.\\}
\end{quote}
\subsection{Facendo un \gloss{driver} che fa richieste HTTP al server è meglio implementare una connessione stateless o statefull?}
\begin{quote}
  \textit{Preferisco la connessione stateless.\\}
\end{quote}
\subsection{Dobbiamo usare SSL per l'invio di dati al server?}
\begin{quote}
  \textit{Sì, almeno nel caso in cui si passino le password. Non sarebbe accettabile nel caso di invio di password.\\}
\end{quote}
\subsection{Abbiamo trovato diverse librerie per implementare il driver, scalaj sarebbe la più leggera tuttavia il driver diventerebbe sincrono. Volevamo sapere cosa ne pensa a riguardo.}
\begin{quote}
  \textit{La maggior parte dei \gloss{driver} per i database in Java sono sincroni. Se riuscite a implementare entrambe le versioni sarebbe meglio, altrimenti sceglietene una.\\}
\end{quote}
\subsection{Come ci propone di salvare i dati?}
\begin{quote}
  \textit{Io li salverei come array di byte.\\}
\end{quote}
\section{Decisioni prese}
In seguito alla riunione sono state prese le seguenti decisioni:
\begin{itemize}
\item Per il lato server si è scelto di usare la connessione tramite HTTP;
\item Si è scelto di usare una connessione di tipo stateless;
\item Verrà usato SSL per la comunicazione con il server;
\item Si è scelto di implementare il driver in maniera sincrona;
\item Si è scelto di salvare i dati come array di byte.
\end{itemize}
\end{document}