
\documentclass{scalatekids-article}
\def\changemargin#1#2{\list{}{\rightmargin#2\leftmargin#1}\item[]}
\let\endchangemargin=\endlist
\begin{document}
\lfoot{Verbale esterno 2016-03-08 1.0.0}
\newgeometry{top=3.5cm}
\begin{titlepage}
  \begin{center}
    \begin{center}
      \includegraphics[width=10cm]{sklogo.png}
    \end{center}
    \vspace{1cm}
    \begin{Huge}
      \begin{center}
        \textbf{Verbale Interno 2016-03-19}
      \end{center}
    \end{Huge}
    \vspace{11pt}
    \bgroup
    \def\arraystretch{1.3}
    \begin{tabular}{r|l}
      \multicolumn{2}{c}{\textbf{Informazioni sul documento}} \\
      \hline
      \setbox0=\hbox{0.0.1\unskip}\ifdim\wd0=0pt
      \\
      \else
      \textbf{Versione} & 1.0.0\\
      \fi
      \textbf{Redazione} & \multiLineCell[t]{Alberto De Agostini}\\
      \textbf{Verifica} & \multiLineCell[t]{Michael Munaro}\\
      \textbf{Approvazione} & \multiLineCell[t]{Marco Boseggia}\\
      \textbf{Uso} & Interno\\
      \textbf{Lista di Distribuzione} & \multiLineCell[t]{ScalateKids\\Prof. Tullio Vardanega\\Prof. Riccardo Cardin}\\
    \end{tabular}
    \egroup
    \vspace{22pt}
  \end{center}
\end{titlepage}
\restoregeometry
\clearpage
\pagenumbering{arabic}
\setcounter{page}{1}
\begin{flushleft}
  \vspace{0cm}
         {\large\bfseries Diario delle modifiche \par}
\end{flushleft}
\vspace{0cm}
\begin{center}
  \begin{tabular}{| l | l | l | l | l |}
    \hline
    Versione & Autore & Ruolo & Data & Descrizione \\
    \hline
    1.0.0 & Marco Boseggia & Responsabile & 2016-03-19 & Validazione documento\\
    \hline
    0.1.1 & Michael Munaro & Verificatore & 2016-03-19 & Verifica documento\\
    \hline
    0.1.0 & Alberto De Agostini & Progettista & 2016-03-19 & Prima stesura documento\\
    \hline
    0.0.1 & Alberto De Agostini & Progettista & 2016-03-19 & Creazione scheletro del documento\\
    \hline
  \end{tabular}
\end{center}
\tableofcontents
\newpage
\section{Informazioni sulla riunione}
\begin{itemize}
\item \textbf{Data}: 2016-03-19
\item \textbf{Luogo}: Conferenza in remoto
\item \textbf{Orario d'inizio}: 15:00
\item \textbf{Durata}: 80'
\item \textbf{Partecipanti interni}: \textit{ScalateKids}
  \begin{itemize}
  \item Andrea Giacomo Baldan 
  \item Alberto De Agostini
  \item Marco Boseggia
  \item Michael Munaro
  \item Francesco Agostini
  \item Giacomo Vanin
  \item Davide Trevisan
  \end{itemize}
\end{itemize}
\section{Decisioni prese}
\begin{itemize}
  \item Abbandonata l'idea di utilizzo del framework \gloss{Spring} per la realizzazione di un \gloss{DSL}, si è deciso di utilizzare \gloss{Scala}.
  \item Per il protocollo di comunicazione client-server si è optato per l'utilizzo del protocollo \gloss{RESP}
  \item Gestione del ciclo di vita della connessione da parte di un client esterno mediante \gloss{attore} dedicato (\textit{\gloss{Clientactor}}).
  \item Rendere la distribuzione dei main tra i vari nodi configurabile.
\end{itemize}

\end{document}